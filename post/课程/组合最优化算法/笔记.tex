\documentclass[a4paper,UTF8,fontset=windows]{ctexart}
\pagestyle{headings}
\title{\textbf{组合最优化算法\ 笔记}}
\author{原生生物}
\date{}
\setcounter{tocdepth}{2}
\setlength{\parindent}{0pt}
\usepackage{amsmath,amssymb,amsthm,enumerate,geometry,paralist}
\geometry{left = 2.0cm, right = 2.0cm, top = 2.0cm, bottom = 2.0cm}
\ctexset{section={number=\zhnum{section}}}
\ctexset{subsection={name={\S},number=\arabic{section}.\arabic{subsection}}}

\newcommand{\exce}[2]{\S\textbf{练习}(#1):{\kaishu #2}}
\let\itemize\compactitem
\newcommand{\proo}[1]{{\kaishu $\bullet$证明:
\begin{itemize}
    \item[] #1
\end{itemize}
}}
\DeclareMathOperator*{\opt}{opt}
\newcommand{\er}{\mathrm{e}}

\begin{document}
\maketitle

*邵嗣烘老师《组合最优化算法》课程笔记,练习解答见对应的作业文件。

*对集合以绝对值符号表集合大小。

\tableofcontents

\newpage
\section{背包问题}
\subsection{组合优化绪论}
\exce{1.1}{描述P与NP问题。}

组合优化:\textbf{有限}个对象集合(可行解集,或所有可能解)中找到最优的(有清晰的数学表示);该集合\textbf{元素数目巨大},往往随问题规模(用某种编码方式输入问题所需要的存储空间)指数提升,不可能遍历。

\exce{1.2}{找三个组合优化问题(NP问题)使用遍历法,记录规模与计算时间。}

1960年代:认为算法能被问题规模的多项式出发控制住,则称为有效。所有存在有效算法的问题可以称为一类,即P类。

1970年代,发现一些“最难”的问题,称为NPC问题,它们具有等价性:只要某一个存在有效算法,则其他所有均存在有效算法。

*目前为止,几乎任何组合优化问题(NP问题)要么是P的,要么是NPC的,或不知道属于哪类。

组合优化主要讨论\textbf{有效算法},也即讨论P问题,而计算复杂性理论会讨论NPC问题。

*对P问题主要讨论最低复杂度的有效算法,而对NPC问题主要讨论有近似比保证的算法,两者共同目标为\textbf{避免穷举}找到最好(或较好)的解。

下面以背包问题为例进行说明。

\subsection{背包问题精确解}
任给$n$个物品$I_1,\dots,I_n$,每个物品$I_i$体积$s_i$、价值$c_i$,要求$s_i$、$c_i$为正整数。另有容积$S$的背包,需要挑选物品集合$A$,使得其中物品体积和不超过$S$,且价值最高。

\textbf{整数规划}:考虑$x_i$为0-1变量,为1代表放入,为0代表不放入,则问题变为$n$维0-1变量寻找可行的最优解,输入为$2n+1$个正整数,输出为$n$位,进一步变为求(下标默认为1到$n$):
$$\max c(\vec{x})=\sum_ic_ix_i$$
使得
$$\sum_is_ix_i\le S,\quad x_i\in\{0,1\}$$

用opt表示上述问题的最优值,并进一步假定$s_i\le S$均成立,则至少有$\opt\ge c_i$均成立。

\

\textbf{精确算法}[动态规划]

对编号集合$A$,令$s_A$表示其中物品体积和,$c_A$表示价值和。定义二元组$(i,j)$,其中$i$为1到$n$,$j$为0到$\sum_ic_i$间的整数。

若存在$A\subset[1,i]$使得$c_A=j$且$s_A\le S$,则定义$c(i,j)$为使$s_A$最小的$A$,否则认为$c(i,j)$为nil,由此有
$$\opt=\max_{c(n,j)\ne\text{nil}}j$$

初始化:$c(1,j)$当且仅当$j=0$时为$\varnothing$,$j=c_1$时为$\{1\}$,否则为nil。

循环计算:对$i$从2到$n$、$j$从0到$\sum_ic_i$,若
$$c(i-1,j-c_i)\ne\text{nil}$$
$$S_{c(i-1,j-c_i)}\le S-s_i$$
且$c(i-1,j)=\text{nil}$或非空时$S_{c(i-1,j)}>S_{c(i-1,j-c_i)}+s_i$,则$c(i,j)=c(i-1,j-c_i)\cup\{i\}$,否则$c(i,j)=c(i-1,j)$。

\exce{1.3}{证明此算法可以求出最优结果。}

问题:复杂度为$O(n^3M)$,其中$M$为$c_i$中最大值,其为\textbf{伪多项式时间}算法,与输出内容有关。

\subsection{性能比与贪婪算法}
\textbf{近似算法}[贪婪算法]

考虑价容比$c_i/s_i$,按大小递减排列,不妨设编号小的价容比更大,则直接按照从1到$n$的顺序放入$A$。若能全部放入则直接输出。否则设第$k+1$个加入时体积超过$S$,输出可行选择
$$c_G=\max\bigg\{\sum_{i=1}^kc_k,c_{k+1}\bigg\}$$

*只需要$O(n\ln n)$复杂度,远比精确算法快。贪婪算法可行的原因:模型可以排列出某种\textbf{单调性}结构。

*这门课研究的主题是\textbf{有理论保障的近似算法},此处性能比即为所需的理论保障[近似算法主要指最优值接近,可能与最优解相差甚远]。

*并非所有离散算法中都存在可以满足需求的近似解,有时必须寻找最优解,例如密码学中,也不在这门课讨论的范畴内。

性能比:$\opt<2c_G$,意味着结果不会太差。

\proo{
    若能全部放入则$\opt=c_G$,得证。否则,至少有
    $$\opt\ge\sum_{i=1}^kc_i$$
    考虑松弛后的连续线性规划问题,求
    $$\max c(\vec{x})=\sum_ic_ix_i$$
    使得
    $$\sum_is_ix_i\le S,\quad x_i\in[0,1]$$
    求出最优值$\hat{c}$\ (见下方练习)后,由定义有$\opt\le\hat{c}$,直接计算可知(不等号由无法放入第$k+1$个即得)
    $$\hat{c}=\sum_{i=1}^kc_i+c_{k+1}\frac{S-\sum_{i=1}^ks_i}{s_{k+1}}<\sum_{i=1}^kc_i+c_{k+1}$$
    由此即有$\hat{c}<2c_G$,从而$\opt<2c_G$,得证。
}

\exce{1.4}{证明上述线性规划问题最优解为
$$x_j=\begin{cases}1&j=1,\dots,k\\\frac{1}{s_{k+1}}\big(S-\sum_{i=1}^ks_i\big)&j=k+1\\0&j>k+1\end{cases}$$
}

\

*希望进一步提升性能比(即希望比2更小的倍数)。

想法:\textbf{按价值分组}之后再按贪婪方法选择。

将物品按照价值阈值$\alpha$分为两组,价值不超过$\alpha$的物品集合记为$A_\alpha$,其余为价值大于$\alpha$的物品,记为$B_\alpha$。

为总价值大,应尽量多选出价值大于$\alpha$的物品,而最优解中最多能选择$\frac{\opt}{\alpha}\le\frac{2c_G}{\alpha}$个$B_\alpha$中的物品。

\subsection{更好的近似算法}

\textbf{分组贪婪}

做法:对某$\alpha$,先利用贪婪找到$c_G$,并分出$A_\alpha$、$B_\alpha$,设$|A_\alpha|=m$,且按照价容比排序,计算出$\varepsilon=\frac{\alpha}{c_G}$。

对满足$|I|\le2/\varepsilon$的集合$I\subset\{m+1,\dots,n\}$,若$\sum_{i\in I}s_i>S$则置$c(I)$为0,否则记$c_0(I)$为将体积去除$S-\sum_{i\in I}s_i$后对$A_\alpha$应用贪婪算法得到的结果,记
$$c(I)=c_0(I)+\sum_{i\in I}c_i$$
对所有$I$取最大值作为此算法最优结果$c_{GG}$。

*分析可知计算代价最高的步骤为遍历过程,也即$O(n^{1+\varepsilon/2})$。

性能比:$\opt\le(1+\varepsilon)c_{GG}$。

\proo{
    设最优解包含编号集合为$I^*$,设$\bar{I}=I^*\cap A_\alpha$,则根据之前分析可知$|\bar{I}|\le\frac{2}{\varepsilon}$,且满足(右侧因$A_\alpha$中所有物品价值不超过$\alpha$)
    $$C(\bar{I})\le\opt\le C(\bar{I})+\alpha$$
    而$c_{GG}\ge C(\bar{I})$,于是即得
    $$\opt\le c_{GG}+\varepsilon c_G$$
    只需证明$c_{GG}\ge c_G$即得结论,而通过考虑$c_G$中选择的$A_\alpha$中物品集合$I_G$,则$|I_G|\le\frac{c_G}{\alpha}$,于是$I_G$被遍历了,从而$c_G=C(\bar{I}_G)\le c_{GG}$,得证。
}

\

另一种近似思路:\textbf{权衡算法}[复杂度出发改进]。

设$M=\max_ic_i$,并令$c_k'=\lfloor\frac{nc_k(1+h)}{M}\rfloor$,这里$h$为某给定正整数。考虑价值变为$c_k'$后,用精确法求解最优解,利用精确算法复杂度可知其复杂度为$O(n^4h)$,记此最优解对应的原问题结果为$c_{GGG}$。

性能比:$\opt\le\big(1+\frac{1}{h}\big)c_{GGG}$。

\exce{2.1}{证明权衡算法的性能比结论。}

\subsection{背包问题等价形式}

*组合问题一般都有三种形式:\textbf{整数规划}形式、\textbf{判定}形式与\textbf{图}形式。

背包问题判定形式:任给$2n+2$个正整数$S$、$s_1,\dots,s_n$、$c_1,\dots,c_n$与$K$,判定是否存在$x_1,\dots,x_n\in[0,1]$使得
$$\sum_{i=1}^nx_is_i\le S,\quad \sum_{i=1}^nx_ic_i\ge K$$

\exce{2.2}{证明背包问题存在有效算法,当且仅当其判定形式存在有效算法。}

\

背包问题图形式:构造网格所有顶点为$(i,j)$,其中$0\le i\le n+1$,$0\le j\le S$。

当$i<n$时,连接$(i-1,j)\to(i,k)$当且仅当$j=k$,此时边权为0;或$k=j+s_i$,此时边权为$-c_i$。

当$i=n$时,将所有$(n,j)$连接至$(n+1,S)$,边权为0。

等价性:从$(0,0)$到$(n+1,S)$的最短路径即对应背包问题最优解的相反数。

\exce{2.3}{验证图形式的等价性结论。}

*非负权重最短路为P问题,但可能为负时为NPC问题。不过,由于此图为无环图,事实上仍然为P问题,但由于出现了$S$,算法事实上是伪多项式时间的。

\section{最短路径问题}
\subsection{定义与基础版本}
\textbf{基本定义}

考虑有向图$D=(V,A)$,其中$V$为顶点集,$A$为边集,每边可写为$(u,v)$或$u\to v$。

\textbf{途}[walk]:顶点、边相间的序列$(v_0,a_1,v_1,a_2,v_2,\dots,a_m,v_m)$,其中$a_i$为$v_{i-1}\to v_i$的边($m$可以等于0)。

\textbf{路}[pass]:顶点不重复的途。

\textbf{长度}:给定每边$a$的权$l(a)$,途/路的长度定义为$\sum_il(a_i)$。

\textbf{s-t图}:固定某两点为源点$s$与汇点$t$后的图。

\

\textbf{最短路问题-基础版本}

假设所有边权均为1,求$s$为起点$t$为终点的最短(长度最小)路,定义长度为$s$与$t$间的\textbf{距离},若不存在这样的路则称距离为$\infty$。

\textbf{广度优先}遍历:记$V_i$为到$s$距离为$i$的节点集合,通过
$$V_{i+1}=\bigg\{v\in V\backslash\bigcup_{k=0}^iV_i\ \bigg|\ \exists u\in V_i,(u,v)\in A\bigg\}$$

由$V_0=\{s\}$开始遍历,直到找到$V_{i+1}=\varnothing$,可找到最短路径,且为多项式量级。

\textbf{s-t割}:称$A'\subset A$为s-t割,若存$U\subset V$使得$s\in U,t\notin U$,且$A'$为所有以$U$为起点,$U$外的点为终点的边的集合(可记为$\delta^{out}(U)$)。

\exce{2.4}{证明基础版本最短路问题的最优值等于不相交s-t割的最大个数。}

\subsection{一般情况}
\textbf{最短路问题-正权版本}

若所有边权均为正,则可以通过$O(|V|^2)$量级的算法得到$s$为起点$t$为终点的最短路。

\textbf{Dijkstra最短路径算法}:
\begin{enumerate}
    \item 取$U$,$f(s)=0$,其余点处$f(s)=\infty$。
    \item 令$u$为满足$f(u)$最低的$u$;
    \item 对每条边$a=(u,v)\in A$,若$f(v)>f(u)+l(a)$,则$f(v)=f(u)+l(a)$;
    \item 从$U$中去除$u$,回到第二步,直到$U$为空集或其中所有点$f(u)=\infty$。
\end{enumerate}

命题:这样的迭代给出的$f(v)$即为$v$到$s$距离。

\proo{
    记实际距离为$d(v)$,根据计算过程可发现$f(v)\ge d(v)$,只需证明$v\in V\backslash U$时一直有$f(v)=d(v)$即可。

    利用归纳,只要证明每步去除的$u$满足$f(u)=d(u)$即可。若否,考虑实际最短路中$U$中节点最小下标$i$,可发现$s$到$v_i$的路径长度已经$\ge f(u)$,矛盾。
}

*对稠密图,上述复杂度可以接受,但对稀疏图($|E|\ll|V|^2$),希望有更快算法。 

\textbf{优化}:考虑到主要复杂度在于求解最小顶点,算法可以达到$O((|E|+|V|)\log|V|)$量级。

\exce{2.5}{利用合适的数据结构构造$O((|E|+|V|)\log|V|)$复杂度的正权最短路径算法。}

\

\textbf{最短路问题-无负环版本}

若边权可以为负,一般无P算法,但不存在负长度[所有边权和为负]的有向环时,存在P算法。

*满足上述条件的有向环,若存在s-t途,则最短s-t路存在。

\proo{
    由于不存在负长度有向环,有重复顶点时一定长度超过无重复顶点时,由此考虑所有途中长度最短的即为最短路。
}

\exce{2.6}{构造无负环最短路的Bellman-Ford算法并证明其正确性。}

\subsection{优化问题算法设计概述}
考虑可行域$\Omega$上函数$f(x)=f(x_1,\dots,x_n)$的最小/最大值问题,其取值范围离散。

若原问题难以处理[可由\textbf{图灵机}严谨定义],设计近似算法一般分为三步:
\begin{enumerate}
    \item 将原始问题输入参数、目标函数或可行域作一定\textbf{扰动}[实质上是更换问题,如背包问题时将离散取值修改为连续取值],得到容易处理的问题。
    \item 设计求解新问题的有精度[性能比]保证的算法。若所得解不是原问题的可行解,需要处理为某个接近的可行解。
    \item 估计性能并得到保证。以值域为正的最小值问题为例,假设$f(x^*)$为$f(x)$在$\Omega$中的最优值,对应最优解$x^*$,在第一步中讲可行域限制到子集$\Gamma$上,且限制后的最优解为$y^*$,可认为$y^*$是$x^*$的近似。对$x^*$作一定变换成为$\Gamma$上的元素$y$,则有
    $$\frac{f(y^*)}{f(x^*)}\le\frac{f(y)}{f(x^*)}$$
    得到性能比的一个上界。
\end{enumerate}

\section{贪婪算法}
\subsection{相关定义}
贪婪算法一般流程:
\begin{enumerate}
    \item 定义\textbf{可能解集}[一般比可行域更大]上的\textbf{势函数}$f$;
    
    *这里假定可能解集是一些集合的集合。

    \item 从$A=\varnothing$开始,每次添加元素$x_0$,使得$f(A\cup\{x_0\})$为所有$A\cup\{x\}$中的最优。
    \item 当$f(A)$不能再改变时停止。
\end{enumerate}

\textbf{边际效应}:随着$A$中元素越来越多,增加单个元素带来的势函数增量在减小[数学上称为\textbf{次模性质}]。
    
*问题:如何刻画$A$不断增加元素的过程?

\

\textbf{独立系统}[世袭系统]:考虑元素个数有限的底集$E$,其子集族$\mathcal{I}$构成独立系统,若对任何$I\in\mathcal{I}$,有$I'\subset I\Rightarrow\in\mathcal{I}$。并称其中元素为独立集。

*如图上无环子集构成的子集族。

\textbf{基}:\textbf{个数最多}的独立子集。

\textbf{环}:个数最少的\textbf{依赖集}[不独立的集合]。

\textbf{秩}:独立集的最大元素个数。

\textbf{最大独立子集问题}:设非负函数$c:E\to\mathbb{R}^+$,记$$c(I)=\sum_{e\in I}c(e)$$
任给独立系统$(E,\mathcal{I})$与权函数$c$,求$\max_{I\in\mathcal{I}}c(I)$。


\

通过最大独立子集设计贪婪算法:
\begin{itemize}
    \item \textbf{最长哈密顿圈}问题:对某完全图,给定每边正整数权值,求权值最大的哈密顿圈。
    \item 转化为最大独立子集问题。$E$为完全图边集,$\mathcal{I}$为$E$的某子集,其或构成哈密顿圈,或为若干不相交路的并,$c$即为边权。
    \item 贪婪算法:势函数$f$即为此处的$c(I)$。将所有边按权值从大到小排序,从$A=\varnothing$开始,每次选出$c(x)$最大的满足$A\cup\{x\}\subset\mathcal{I}$的$x$加入,直到无法加入。
\end{itemize}

设这样选出的边为$I_G$,真实最优哈密顿圈为$I^*$。

\textbf{性能比}:
$$1\le\frac{c(I^*)}{c(I_G)}\le\max_{F\subset E}\frac{v(F)}{u(F)}$$
这里$u(F)$为$F$的极大独立子集(不存在真包含它的独立子集)的个数最小值,$v(F)$为$F$的独立子集个数最大值(利用定义等价于极大独立子集个数最大值)。将不等号最右端记为$\rho$,其只依赖独立系统,与权函数无关。

\proo{
    不妨设$c(e_1)\ge c(e_2)\ge\dots\ge c(e_n)$,记$E_i$为$E$中前$i$条边构成的集合,则$E_i\cap I_G$一定是$E_i$的极大独立子集,否则能添入的边一定会被贪婪算法选到,矛盾。从而有$|E_i\cap I_G|\ge u(E_i)$。

    另一方面,由$I^*$为极大独立子集,$E_i\cap I^*$也应为$E_i$的独立子集,从而$|E_i\cap I^*|\le v(E_i)$。
    
    注意到,当$e_i\in I_G$时$|E_i\cap I_G|-|E_{i-1}\cap I_G|=1$,否则为0,因此$c(I_G)$为
    $$c(e_1)|E_1\cap I_G|+\sum_{i=2}^ne_i(|E_2\cap I_G|-|E_{i-1}\cap I_G|)=\sum_{i=1}^{n-1}|E_i\cap I_G|(c(e_i)-c(e_{i+1}))+|E_n\cap I_G|c(e_n)$$
    同理
    $$c(I^*)=\sum_{i=1}^{n-1}|E_i\cap I^*|(c(e_i)-c(e_{i+1}))+|E_n\cap I^*|c(e_n)$$
    由于已经假设了$c(e_i)\ge c(e_{i+1})$,且$c(e_n)\ge0$,有
    $$c(I^*)\le\sum_{i=1}^{n-1}v(E_i)(c(e_i)-c(e_{i+1}))+v(E_n)c(e_n)\le\sum_{i=1}^{n-1}\rho u(E_i)(c(e_i)-c(e_{i+1}))+\rho u(E_n)c(e_n)$$
    而右侧即不超过$\rho c(I_G)$。
}

上述证明对任何最大独立子集问题的非负权函数贪婪算法均成立。事实上,最长哈密顿圈问题中必然有$\rho\le2$,从而性能比可以控制。

\proo{
    利用$u,v$定义,只需证明若$I,J$为$F$的两个极大独立子集,则$|J|\le 2|I|$。

    若$F$中有哈密顿圈,或无哈密顿圈但有哈密顿路,则由定义则已经相等,同为顶点数/顶点数减一。

    记$V_i$为$I$中度数为$i$的顶点集合,则$i$只能为1,2,$V_1$为端点集合,$V_2$为中间点集合,于是
    $$|I|=|V_2|+\frac{1}{2}|V_1|$$

    由于$I$为$F$的极大独立子集,$F$中每条边或至少有一个端点在$V_2$中,或连接$I$中同一条路的两个端点。

    记$J_2\subset J$为满足至少有一个端点在$V_2$中的边集合,$J_1=J\backslash J_2$,利用$J$为极大独立集合,$J_2$最多只有两条边与$V_2$中的每一个顶点相连,因此$|J_2|\le 2|V_2|$。另一方面,$J_1$中每条边最多与$V_1$中两个端点相连,于是$|J_1|\le\frac{1}{2}|V_1|$,由此求和可知
    $$|J|=|J_1|+|J_2|\le\frac{1}{2}|V_1|+2|V_2|\le 2|I|$$
}

\exce{3.1}{将最长有向哈密顿路问题转化为最大独立子集问题,并对应定义贪婪算法,计算性能比。}

\exce{3.2}{设$(E,\mathcal{I})$是一个独立系统,且假设$E$的所有极大独立集都含有$k$个元素。考虑非负权函数$c$,仍类似前文定义$\rho$,并考虑权和最小的极大独立子集问题,设真实最优解$I'$,对应的贪婪算法选出的集合为$I_G$,证明
$$c(I')\le c(I_G)\le\frac{1}{\rho}c(I')+\frac{\rho-1}{\rho}kM,\quad M=\max_ec(e)$$}

\subsection{拟阵}
对于任何独立系统,都可以定义相应的$\rho$,$\rho=1$时的独立系统称为\textbf{拟阵}。根据定义即可知拟阵等价于其任何子集的极大独立子集个数均相等。

*上节所说的图上无环子集构成的子集族事实上构成拟阵,称为\textbf{图拟阵}。若图连通,则拟阵的基为图的生成树,结点个数$|V|-1$。

\exce{3.3}{验证图拟阵构成拟阵。}

*有限向量组的线性无关组构成拟阵,称为\textbf{线性拟阵}。

利用拟阵的定义,若权函数\textbf{非负},贪婪算法可以给出最大独立子集问题的最优解(设计方法类似最长哈密顿圈问题中)。

反之,只要独立系统不为拟阵,对某个权函数,贪婪算法无法给出最优解。

\proo{
    存在子集合$F\subset E$使得$F$有两个大小不同的极大独立子集$I$、$I'$,不妨设$|I|>|I'|$,由此可定义如下的非负权函数
    $$c(e)=\begin{cases}1+\varepsilon&e\in I'\\1&e\in I\backslash I'\\0&e\in E\backslash(I\cap I')\end{cases}$$
    且$0<\varepsilon<\frac{1}{|I'|}$。进一步计算即得贪婪算法会取出$I'$,但$c(I)>c(I')$,并非最优解。
}

\

若$E$的子集族$\mathcal{I}$构成独立系统,则$E$上存在$k$个拟阵$\mathcal{G}_1,\dots,\mathcal{G}_k$,使得
$$\mathcal{I}=\bigcap_{i=1}^k\mathcal{G}_i$$

\proo{
    记$c_1,\dots,c_k$是$\mathcal{I}$的全部极小相关集(也即极小的非独立子集的集合),下面构造对应的$\mathcal{G}_1,\dots,\mathcal{G}_k$。对每个$i=1,\dots,k$定义
    $$\mathcal{G}_i=\{F\subset E\mid c_i\not\subset F\}$$

    利用它们为全部极小相关集可验证$\mathcal{G}_i$为独立系统且交为$\mathcal{G}$,只需证明$\mathcal{G}_i$为拟阵。
    
    考虑$\mathcal{G}_i$中,对任何$F\subset E$,若$c_1\not\subset F$,则$F$自身(且只有自身)为极大独立子集,否则,其每一个极大独立子集为去掉$c_i$中任一个元素,由此个数均为$|F|-1$。
}

\exce{3.4}{验证证明中构造的$\mathcal{G}_i$为独立系统,且交为$\mathcal{G}$。}

反之,设$\mathcal{I}=\cap_{i=1}^k\mathcal{G}_i$,且$\mathcal{G}_i$为拟阵,则$\mathcal{I}$为独立系统,且$\rho\le k$。

\proo{
    可验证独立系统的交仍为独立系统,从而得证其为独立系统。

    对任何$E$的子集$F$,只需证明$v(F)\le ku(F)$,也即$F$的两个极大独立子集$I,J$大小至多相差$k$倍($|J|\le k|I|$)。

    设$I_i$为$\mathcal{G}_i$中$I\cup J$的极大独立子集,且其包含$I$\ (考虑从$\mathcal{G}_i$的独立子集$I$开始添加元素,直到极大)。任给元素$e\in J\backslash I$,下证其最多出现在$k-1$个$I_i\backslash I$中。若否,其出现在全部$k$个$I_i$中,则$e\cup\{I\}$为$\mathcal{G}$的独立子集,与$I$的极大性矛盾。
    
    由此即得
    $$\sum_{i=1}^k|I_i|-k|I|=\sum_{i=1}^k|I_i\backslash I|\le(k-1)|J\backslash I|\le (k-1)|J|$$
    同理构造$J_i$可知(等号利用了拟阵的性质)
    $$k|J|\le\sum_{i=1}^k|J_i|=\sum_{i=1}^k|I_i|\le k|I|+(k-1)|J|$$
    从而得证。
}

*\textbf{最大三维匹配问题}:任给三个不相交的集合$X,Y,Z$,给定$X\times Y\times Z$上的非负权函数$c$,求$X\times Y\times Z$的一个子集,使得任意两三元组无共同元素,且三元组权和最大。

给定拟阵$(E,\mathcal{G})$,对任何$A\subset E$,定义$A$的\textbf{秩}$r(A)$为其中极大独立子集的最大个数。

\subsection{次模函数}
\textbf{次模函数}:函数$f:2^E\to\mathbb{R}$\ (或要求值域非负)满足对$E$任意两子集$A,B$有
$$f(A\cup B)+f(A\cap B)\le f(A)+f(B)$$

\exce{3.5}{证明$r(A)$是次模函数。}

*直接定义$f(A)=|A|$可发现其也为次模函数。

\textbf{单调增函数}:满足$A\subset B$时$f(A)\le f(B)$的函数$f:2^E\to\mathbb{R}$。

\textbf{边际效应}:对$2^E$上的次模函数$f$,则对任何$A,C\subset E$,有
$$\Delta_Cf(A)\le\sum_{x\in C}\Delta_xf(A),\quad \Delta_Cf(A)=f(A\cup C)-f(A),\quad\Delta_xf(A)=f(A\cup\{x\})-f(A)$$

\proo{
    引理:$f$为次模函数等价于对任何$A\subset B\subset E,x\notin B$有
    $$\Delta_xf(A)\ge\Delta_xf(B)$$

    左推右:上式即为$f(A\cup\{x\})-f(A)\ge f(B\cup\{x\})-f(B)$,而
    $$B\cup\{x\}=(A\cup\{x\})\cup B,\quad A=(A\cup\{x\})\cap B$$
    从而由次模函数定义得证。

    右推左:从$\Delta_xf(A)\ge\Delta_xf(B)$可以归纳得到对任何$C\cap B=\varnothing$有$\Delta_Cf(A)\ge\Delta_Cf(B)$,也即
    $$f(A\cup C)-f(A)\ge f(B\cup C)-f(B)$$
    对任何集合$D,E$考虑
    $$A=D\cap E,\quad B=E,\quad C=D\backslash E$$
    即可发现
    $$f(D)+f(E)\ge f(D\cup E)+f(D\cap E)$$

    \

    原命题证明:由于交集部分不影响可不妨设$A\cap C=\varnothing$。设$C=\{x_1,\dots,x_n\}$,则利用引理可知
    $$\Delta_Cf(A)=\sum_{i=1}^n\Delta_{x_i}f(A\cup\{x_1,\dots,x_{i-1}\})\ge\sum_{i=1}^n\Delta_{x_i}f(A)$$
    从而得证。

}

\

\textbf{最小集合覆盖}

任给集合$S$与$S$的子集构成的子集族$\mathcal{C}$,满足其中所有子集并为$c$,求$\mathcal{C}$中个数最少的并为$S$的子集。对$\mathcal{C}$的一个子集族,定义函数
$$f(\mathcal{A})=\bigg|\bigcup_{A\in\mathcal{A}}A\bigg|$$
由于$f(\mathcal{A})+f(\mathcal{B})-f(\mathcal{A}\cup\mathcal{B})$表示既在$\cup_{A\in\mathcal{A}}A$中又在$\cup_{B\in\mathcal{B}}B$中的元素个数,由此利用定义其为次模函数,

贪婪算法构造:输入$S$与$\mathcal{C}$,初始化$\mathcal{A}=\varnothing$,只要$f(\mathcal{A})<|S|$,选取$C\in\mathcal{C}$使得$\Delta_Cf(\mathcal{A})$最大,并加入$\mathcal{A}$,在$f(\mathcal{A})=|S|$时输出。

\textbf{性能比}:若上述算法选出的集合个数为$g$,真实最优中元素个数为$m$,设$\gamma$为$\mathcal{C}$中最大子集的元素个数,则
$$1\le\frac{g}{m}\le 1+\ln\gamma$$

\proo{
    设$\mathcal{A}_G=\{A_1,\dots,A_g\}$为贪婪法给出的解,且按照选出顺序排列,也即$A_{i+1}$可以覆盖最多未被$A_i$覆盖的元素,将后者构成的集合记为$U_i$,,并记$\mathcal{A}_i=\{A_1,\dots,A_i\}$,则$|U_i|=|S|-f(\mathcal{A}_i)$。
    
    设最优解为$\mathcal{A}^*=\{C_1,\dots,C_m\}$。由于$U_i$一定可被最优解覆盖,一定存在$C_j$至少覆盖了$U_i$中$\frac{|S|-f(\mathcal{A}_i)}{m}$个元素。

    由此,利用定义可知
    $$f(\mathcal{A}_{i+1})-f(\mathcal{A}_i)\ge\frac{|S|-f(\mathcal{A}_i)}{m}$$
    也即
    $$|S|-f(\mathcal{A}_{i+1})\le(|S|-f(\mathcal{A}_i))\bigg(1-\frac{1}{m}\bigg)$$
    利用$\er$的定义归纳得
    $$|U_i|\le|S|\er^{-i/m}$$
    由于$|U_i|$从$|S|$递减到0,一定存在$i_0\le g$使得$|U_{i_0+1}|<m\le|U_{i_0}|$,之后最多迭代$m-1$次,由此
    $$g\le i_0+m\le m\bigg(1+\ln\frac{|S|}{m}\bigg)\le m(1+\ln\gamma)$$
}

*事实上$f(\mathcal{A}_{i+1})-f(\mathcal{A}_i)$的估算可以由次模性推出,规避更具体的抽屉原理运用。
\proo{
    利用贪婪算法定义可知
    $$f(\mathcal{A}_{i+1})-f(\mathcal{A}_i)=\Delta_{\mathcal{A}_{i+1}}f(\mathcal{A}_i)\ge\Delta_{C_j}f(\mathcal{A}_i)$$
    对$j$求和即得
    $$m\big(f(\mathcal{A}_{i+1})-f(\mathcal{A}_i)\big)\ge\sum_{j=1}^m\Delta_{C_j}f(\mathcal{A}_i)$$
    而
    $$|S|-f(\mathcal{A}_i)=f(\mathcal{A}_i\cup\mathcal{A}^*)-f(\mathcal{A}_i)=\sum_{j=1}^m\Delta_{C_j}f(\mathcal{A}_i\cup\mathcal{A}^*_{j-1})$$
    再通过次模性可得成立。
}

\

\textbf{最小次模覆盖}:给定$E$,定义$2^E$上单调增非负正规(即$f(\varnothing)=0$)次模函数$f$,求
$$\min c(A)=\sum_{x\in A}c(x)$$
这里$A\in\Omega_f=\{A\subset E\mid\forall x\in E,\Delta_xf(A)=0\}$。

\exce{4.1}{给出能化为最小次模覆盖问题的实例,并结合实例解释下方性能比结论的含义。}

贪婪算法构造:只要存在$x\in E$使得$\Delta_xf(A)>0$,就选取$\frac{\Delta_xf(A)}{c(x)}$最大的$x$,并置$A=A\cup\{x\}$。

\textbf{性能比}:
$$\frac{c(A_G)}{c(A^*)}\le H(\gamma),\quad\gamma=\max_{x\in E}f(\{x\}),\quad H(t)=\sum_{i=1}^t\frac{1}{i}\le1+\ln t$$

\proo{
    引理:$f$为单调增次模函数的充要条件为
    $$\forall A\subset B\subset E,\quad\forall x\in E,\quad\Delta_xf(A)\ge\Delta_xf(B)$$
    
    引理证明:考虑一个一个添入元素可知单调增等价于
    $$\forall A\subset E,\quad\forall x\in E,\quad\Delta_xf(A)\ge0$$
    由此结合次模函数的等价定义(见之前证明中引理)即得证。

    \

    引理2:$f$为单调增次模函数,则
    $$\Omega_f=\{A\subset E\mid f(A)=f(E)\}$$

    证明:留作习题。

    \

    命题证明:记
    $$A^*=\{y_1,\dots,y_h\},\quad r_i=\Delta_{x_i}f(A_{i-1}),\quad\zeta_{y,i}=\Delta_yf(A_{i-1}),\quad\zeta_{y,k+1}=\Delta_yf(A_G)=0$$
    定义$A^*$上的权函数$w$,希望将$A_G$的权分配给$A^*$中元素,且其中每个$y$得到的权不超过$H(\gamma)c(y)$,也即想要
    $$c(A_G)\le\sum_{y\in A^*}w(y),\quad w(y)\le c(y)H(\gamma)$$
    将上方条件称为第一式与第二式,我们验证如下的定义符合要求:
    $$w(y)=\sum_{i=1}^k(\zeta_{y,i}-\zeta_{y,i+1})\frac{c(x_i)}{\gamma_i}$$
    利用次模性与正规性计算可发现
    $$\sum_{y\in A_*}\sum_{i=1}^k(\zeta_{y,i}-\zeta_{y,i+1})=\sum_{y\in A^*}f(\{y\})\ge\sum_{j=1}^n\Delta_{y_j}f(A^*_{j-1})=f(A^*)=f(A_G)=\sum_{i=1}^kr_i$$

    \

    第一式:配凑得
    $$w(y)=\frac{c(x_1)}{r_1}\zeta_{y,1}+\sum_{i=2}^k\bigg(\frac{c(x_i)}{r_i}-\frac{c(x_{i-1})}{r_{i-1}}\bigg)\zeta_{y,i}$$
    $$c(A_G)=\sum_{i=1}^k\bigg(\sum_{j=i}^kr_j-\sum_{j=i+1}^kr_j\bigg)\frac{c(x_i)}{r_i}=\frac{c(x_1)}{r_1}\sum_{j=1}^kr_j+\sum_{i=2}^k\bigg(\frac{c(x_i)}{r_i}-\frac{c(x_{i-1})}{r_{i-1}}\bigg)\sum_{j=i}^kr_j$$
    因此只需证
    $$\sum_{j=i}^kr_j\le\sum_{y\in A^*}\zeta_{y,i}$$
    而左侧为
    $$f(A_G)-f(A_{i-1})=f(A^*)-f(A_{i-1})=f(A^*\cup A_{i-1})-f(A_{i-1})=\Delta_{A^*}(A_{i-1})\le\sum_{y\in A_k}\Delta_yf(A_{i-1})$$
    从而得证。

    \

    第二式:对$\zeta_{y,i}$进行讨论。若其大于0,利用贪婪算法定义可知
    $$\frac{c(x_i)}{r_i}\le\frac{c(y)}{\zeta_{y,i}}$$
    再由定义可知$\zeta_{y,i}\ge\zeta_{y,i+1}$,于是可记$l_y$为使得$\zeta_{y,i}>0$的最大$i$,有
    $$w(y)=\sum_{i=1}^{l_y}(\zeta_{y,i}-\zeta_{y,i+1})\frac{c(x_i)}{r_i}\le\sum_{i=1}^{l_y}(\zeta_{y,i}-\zeta_{y,i+1})\frac{c(y)}{\zeta_{y,i}}$$
    利用放缩可估算出前方系数不超过$H(f(\{y\}))\le c(y)H(\gamma)$。
}

*这里的证明思路为微调法,通过权重分配估计出结果。

\exce{4.2}{证明$f$为单调增次模函数时
$$\Omega_f=\{A\subset E\mid f(A)=f(E)\}$$}

*练习(4.3)见附录。

*记$A_G=\{x_1,\dots,x_k\}$,$A_0=\varnothing$,$A_i=\{x_1,\dots,x_i\}$,若对每个$i=1,\dots,k$有$\Delta_{x_i}f(A_{i-1})\ge c(x_i)$,则
$$c(A_G)\le\bigg(1+\ln\frac{f(A^*)}{c(A^*)}\bigg)c(A^*)$$

\proo{
    仍设$A^*=\{y_1,\dots,y_h\}$,记$a_i=f(A^*)-f(A_i)$,且$a_0=f(A^*)$,则有
    $$\Delta_{x_i}f(A_{i-1})=a_{i-1}-a_i$$
    由选择$x_j$时的策略并放缩可得
    $$\frac{a_{j-1}-a_j}{c(x_j)}\ge\max_{1\le i\le h}\frac{\Delta_{y_i}f(A_{j-1})}{c(y_i)}\ge\frac{\sum_{i=1}^h\Delta_{y_i}f(A_{j-1})}{c(A^*)}\ge\frac{\Delta_{A^*}f(A_{j-1})}{c(A^*)}=\frac{f(A^*)-f(A_{j-1})}{c(A^*)}$$
    倒数第二个等号是由于真解在$\Omega_f$中。由于分母即为$a_{j-1}$,整理即得对任何$j=1,\dots,k$有
    $$a_j\le a_{j-1}\bigg(1-\frac{c(x_j)}{c(A^*)}\bigg)$$
    另一方面,对$a_0$,利用假设有
    $$a_0=f(A^*)=f(A_k)\ge\sum_{i=1}^kc(x_i)=c(A_k)\ge c(A^*)$$
    且$a_k=0$,

    由于$f$单调增,$a_i$单调减,由此一定存在某个非负整数$r\le k$使得
    $$a_{r+1}<c(A^*)\le a_r$$
    而通过之前的估算可知
    $$\frac{a_r-a_{r+1}}{c(A_{r+1})}\ge\frac{a_r}{c(A^*)}$$
    为了结合以上两式,设$a''=a_r-c(A^*)\ge0$、$a'=c(A^*)-a_{r+1}>0$,且将$c(A_{r+1})$拆分成$c'$与$c''$的和,使得$c'a''=a'c''$\ (即比例相同),则有
    $$\frac{a'}{c'}=\frac{a_r-a_{r+1}}{c(A_{r+1})}\ge\frac{a_r}{c(A^*)}$$
    于是再利用$c'=c(A_{r+1})-c''$可计算得放缩
    $$c(A^*)=a_{r+1}+a'\le a_r\bigg(1-\frac{c''}{c(A^*)}\bigg)$$
    进一步,反复利用之前估算放缩$a_r$可知
    $$c(A^*)\le a_0\prod_{i=1}^r\bigg(1-\frac{c(x_1)}{c(A^*)}\bigg)\bigg(1-\frac{c''}{c(A^*)}\bigg)$$
    再由$1+x\le\er^x$可知
    $$\frac{c(A^*)}{a_0}\le\exp\bigg(-\frac{c''+\sum_{i=1}^rc(x_i)}{c(A^*)}\bigg)$$
    取$\ln$即
    $$c''+\sum_{i=1}^rc(x_i)\le c(A^*)\ln\frac{a_0}{c(A^*)}$$
    而
    $$\sum_{i=r+2}^kc(x_i)\le\sum_{i=r+2}^k\Delta_{x_i}f(A_{i-1})=a_{r+1}$$
    将上两式的左侧求和加$c'$,并将$c'$放大为$a'$\ (利用之前得到的估计与$a_r\ge c(A^*)$),即可得到最终估算
    $$c(A_k)\le c(A^*)\ln\frac{a_0}{c(A^*)}+a'+a_{r+1}$$
    这就是结论。

}

\subsection{最大割问题}
最大割问题:对无向无权图$G=(V,E)$,对于$A,B\subset V$,用$E(A,B)$表示所有连接$A,B$中顶点的边集,最大割即为要找
$$\max_{A\subset V}|E(A,A^c)|$$

*利用蒙特卡洛方法可以做到期望意义下近似比$\alpha_{GW}\approx 0.878$。

\exce{4.4}{设计近似比$\ge1/2$的最大割贪婪算法。}

另一个近似比:$\alpha_T\approx0.531$,人类历史上第一个大于0.5下界的最大割贪婪算法,称为\textbf{递归谱分解}。

思路:给图上的顶点赋予$\{-1,0,1\}$中的取值以逐步确定(某种\textbf{松弛}想法),总体想法即,给定每点处$-1,0,1$中的某个值,记$G_0=G$,以某种规则定义$G$上每个顶点的取值,$L_1$、$U_1$、$R_1$表示$G_0$中取值为$-1,0,1$的顶点集合,取$G_1=G_0[U_1]$代表$U_1$在$G$上的\textbf{诱导子图}(只保留$U_1$中的顶点与连接$U_1$两点的边),并重新按规则定义取值、作分解,直到某次分解后不再存在$U_N$,设此次对应的为$G_{N-1}$,其分解出$L_N$与$R_N$。

从后往前,得到全图划分为一系列的顶点子集
$$(L_N,R_N),\quad(L_{N-1},R_{N-1}),\quad\dots,\quad(L_1,R_1)$$

我们只需要定义合适的\textbf{划分规则}与\textbf{拼合规则}即可。

\begin{itemize}
    \item \textbf{拼合规则}
    
    希望每次拼出来的是$G_N$中的较大割。
    
    记第$t$轮中
    $$C_t=|E(L_t,R_t)|,\quad X_t=|E(L_t\cup R_t,U_t)|,\quad M_t=C_t+X_t+|E(L_t,L_t)|+|E(R_t,R_t)|$$
    这里$M_t$即为总边数。

    第一步,考虑$G_{N-1}$的分割,有$(L_{N-1}\cup L_N,R_{N-1}\cup R_N)$与$(R_{N-1}\cup L_N,L_{N-1}\cup R_N)$两种备选,前者的割值为
    $$C_N+C_{N-1}+|E(L_N,R_{N-1})|+|E(L_{N-1},R_N)|$$
    后者为
    $$C_N+C_{N-1}+|E(L_N,L_{N-1})|+|E(R_{N-1},R_N)|$$
    而又注意到
    $$X_{N-1}=|E(L_N,R_{N-1})|+|E(L_{N-1},R_N)|+|E(L_N,L_{N-1})|+|E(R_N,R_{N-1})|$$
    由此选取较大的结果必然满足割值至少为
    $$C_N+C_{N-1}+\frac{1}{2}X_{N-1}$$

    不断重复上述方法合并可以得到最终的两分割。

    *由此,只要分的某一步
    $$C_t+\frac{1}{2}X_t\le\frac{1}{2}M_t$$
    新产生的割值过小,可直接采用近似比$1/2$的贪婪算法,会有更好的效果。

    \item \textbf{划分规则}
    
    0.531近似比对应方案:每步最大化
    $$\frac{C_t}{M_t-X_t/2}$$

    0.614近似比对应方案:每步最大化
    $$\frac{C_t+X_t/2}{M_t}$$

    *后者更符合之前的割值估算,而事实上计算有
    $$1\ge\frac{C_t+X_t/2}{M_t}\ge\frac{C_t}{M_t-X_t/2}$$
    这也是后者相较前者更好的原因。

    前者的理由:计算可发现
    $$\frac{C_t}{M_t-X_t/2}=\frac{2|E(L_t,R_t)|}{\mathrm{vol}(L_t\cup R_t)}$$
    这里vol指度数之和,也即其在选择
    $$\max\bigg\{\frac{2|E(A,B)|}{\mathrm{vol}(A\cup B)},\quad A\cap B=\varnothing,\mathrm{vol}(A\cup B)\ne0\bigg\}$$
    
    *相当于寻找某个子图中的最大割,称为对偶Cheeger问题。虽然直接求解此问题比最大割问题更难,但其更易于估算。

    针对图$G$中的对偶Cheeger问题,将其写为三值向量形式,也即找
    $$\max\bigg\{1-\frac{\sum_{\{i,j\}\in E}|y_i+y_j|}{\sum_{i\in V}d_i|y_i|},\quad\vec{y}\in\{-1,0,1\}^n,\sum_{i\in V}d_i|y_i|\ne0\bigg\}$$

    \exce{4.5}{验证对偶Cheeger问题可以等价为三值向量形式。}

    将其写成
    $$\min\bigg\{\frac{\sum_{\{i,j\}\in E}|y_i+y_j|}{\sum_{i\in V}d_i|y_i|},\quad\vec{y}\in\{-1,0,1\}^n,\sum_{i\in V}d_i|y_i|\ne0\bigg\}$$
    只需对此问题进行近似算法设计即可。

    *不妨假设\textbf{所有点的度数均非零}(否则其不会在此问题中有影响),这样$\vec{y}$只要所有分量不全为0即可。
\end{itemize}
为进行近似比分析,设$G_N=G[U_t]=(V_t,E_t)$,并记$\rho_t=|E_t|/|E|$,其应随$t$单调下降,且$\rho_0=1$,记$\rho_N=0$。可发现根据定义有
$$M_t=(\rho_t-\rho_{t+1})|E|$$
设并图的最大割值为(最大割一定超过原图一半的证明见贪婪算法)
$$(1-\varepsilon)|E|,\quad\varepsilon\in\bigg(0,\frac{1}{2}\bigg)$$

\

待解决问题:给出三值向量化后的\textbf{对偶Cheeger问题}的近似算法。

思路:谱方法[Spectral],从比较好的连续解$\vec{x}$产生离散解$\vec{y}$,并保证近似比(这里$r$为$\varepsilon$某函数)
$$\frac{\sum_{\{i,j\}\in E}|y_i+y_j|}{\sum_{i\in V}d_i|y_i|}\le r(\varepsilon)$$

\textbf{连续解构造}:取$\vec{x}$为图的Laplace矩阵$L=D-A$的广义特征值问题$L\vec{x}=\lambda D\vec{x}$属于最大特征值的特征向量(这里$D$为各点度数排成的对角阵,$A$为邻接矩阵,在无向图中对称),下先证明有
$$\vec{x}^T(D+A)\vec{x}\le2\varepsilon\vec{x}^TD\vec{x}$$
\proo{
    省略向量符号。原式可等价于
    $$x^T(D-A)x\le2(1-\varepsilon)x^TDx$$
    其即等价于
    $$\frac{x^T(D-A)x}{x^TDx}\ge2(2-\varepsilon)$$
    利用矩阵特征值理论可知
    $$x=\arg\max_{x\ne\mathbf{0}}\frac{x^T(D-A)x}{x^TDx}$$
    于是有(中间的等号是由于可验证$y^T(D-A)y$即是最大割问题的矩阵表示)
    $$\frac{x^T(D-A)x}{x^TDx}\ge\max_{y_i\in\{-1,1\}}\frac{y^T(D-A)y}{y^TDy}=\frac{4(1-\varepsilon)|E|}{2|E|}=2(1-\varepsilon)$$
}

\exce{5.1}{证明$L$对应的广义特征值问题$L\vec{x}=\lambda D\vec{x}$属于最大特征值的特征向量$\vec{x}$满足
$$\vec{x}=\arg\max_{\vec{x}\ne\mathbf{0}}\frac{\vec{x}^TL\vec{x}}{\vec{x}^TD\vec{x}}$$}

由上方结论,变形可发现有
$$\frac{\sum_{\{i,j\}\in E}(x_i+x_j)^2}{\sum_{i\in V}d_ix_i^2}\le 2\varepsilon$$

\textbf{离散解构造}:产生$n$个三值向量$\vec{y}^k\in\{-1,0,1\}^n$,满足
$$y_i^k=\begin{cases}-1&x_i<-|x_k|\\1&x_i>|x_k|\\0&|x_i|\le|x_k|\end{cases}$$
并取使得结果最小者为$\vec{y}$,则应有
$$\frac{\sum_{\{i,j\}\in E}|y_i+y_j|}{\sum_{i\in V}d_i|y_i|}\le2\sqrt\varepsilon$$

*称为双阈值谱分解[2TSC],这里双阈值即为$\pm|x_k|$。

\proo{
    不妨设$|x_1|\le|x_2|\le\cdots\le|x_n|=1$\ (所有过程均齐次,乘比例不影响)。考虑如下随机:
    \begin{enumerate}
        \item 按均匀分布取$t\in[0,1]$;
        \item 向量$\vec{Y}\in\{-1,0,1\}^n$定义为
        $$Y_i=\begin{cases}-1&x_i<-\sqrt{t}\\1&x_i>\sqrt{t}\\0&|x_i|\le\sqrt{t}\end{cases}$$
    \end{enumerate}
    下证
    $$\frac{\sum_{\{i,j\}\in E}|y_i+y_j|}{\sum_{i\in V}d_i|y_i|}\le\frac{\mathbb{E}\big(\sum_{\{i,j\}\in E}|Y_i+Y_j|\big)}{\mathbb{E}\big(\sum_{i\in V}d_i|Y_i|\big)}$$
    左侧为(讨论$t\in[|x_i|,|x_{i+1}|]$的情况,这时$\vec{Y}=\vec{y}^i$)
    $$\min_{1\le k\le n}\frac{\sum_{\{i,j\}\in E}|y_i^k+y_j^k|}{\sum_{i\in V}d_i|y_i^k|}=\min_{t\in[0,1]}\frac{\sum_{\{i,j\}\in E}|Y_i+Y_j|}{\sum_{i\in V}d_i|Y_i|}$$
    根据上方讨论,将期望展开为求和,利用
    $$\min_i\frac{a_i}{b_i}\le\frac{\sum_ia_i}{\sum_ib_i}$$
    即可发现成立。

    直接计算可知
    $$\mathbb{E}(|Y_i|)=\int_0^{x_i^2}1\mathrm{d}t=x_i^2$$
    为计算$\mathbb{E}(|Y_i+Y_j|)$,按$x_i$、$x_j$是否同号讨论,即可得到
    $$\mathbb{E}(|Y_i+Y_j|)=\begin{cases}|x_i^2-x_j^2|&x_ix_j<0\\x_i^2+x_j^2&x_ix_j>0\end{cases}$$
    无论何种情况均有
    $$\mathbb{E}(|Y_i+Y_j|)\le|x_i+x_j|(|x_i|+|x_j|)$$
    最终得到(利用Cauchy不等式后将$(|x_i|+|x_j|)^2$放为$2x_i^2+2x_j^2$)
    $$\frac{\sum_{\{i,j\}\in E}|y_i+y_j|}{\sum_{i\in V}d_i|y_i|}\le\frac{\sum_{\{i,j\}\in E}|x_i+x_j|(|x_i|+|x_j|)}{\sum_id_ix_i^2}\le\frac{\sqrt{\sum_{\{i,j\}\in E}(x_i+x_j)^2\sum_{\{i,j\}\in E}(2x_i^2+2x_j^2)}}{\sum_id_ix_i^2}$$
    注意到分母第二项即为$2d_ix_i^2$即可得结论。
}

\

\textbf{最终近似比}:在$0<\varepsilon<\frac{1}{16}$时,设上述递归谱分解得到的割值为$c$,有
$$c\ge(1-4\sqrt\varepsilon+8\varepsilon)|E|$$

*这里$\frac{1}{16}$来自$\frac{1}{2}=2\sqrt\varepsilon$,若$\varepsilon>\frac{1}{16}$,求解对偶Cheeger问题得到的近似比不如直接使用贪婪算法,此时计算可得近似比至少
$$\frac{1/2|E|}{(1-\varepsilon)|E|}>0.533$$
否则有近似比至少
$$\frac{1-4\sqrt\varepsilon+8\varepsilon}{1-\varepsilon}\ge\frac{1}{2}(\sqrt{65}-7)\approx0.531$$

\exce{5.2}{从割值的估算出发验算最终的近似比结论。}

\proo{
    先证明$G_t$的最大割值至少为$\big(1-\frac{\varepsilon}{\rho_t}\big)|E_t|$。

    由于$\varepsilon|E|$的边不在$G$的最大割中,对最大割在$G_t$上诱导的而分割,进行二分割后剩余的边至多$\varepsilon|E|$,也即$G_t$的割值至少为
    $$\rho_t|E|-\varepsilon|E|=\bigg(1-\frac{\varepsilon}{\rho_t}\bigg)|E_t|$$
    对$G_t$进行递归谱分解后,至少会得到$C_t+\frac{1}{2}X_t$的边,因此所得比例至少为
    $$\frac{C_t+X_t/2}{M_t}\ge\frac{C_t}{M_t-X_t/2}=\frac{2|E(L_t,R_t)|}{\mathrm{vol}(L_t\cup R_t)}\ge1-2\sqrt\frac{\varepsilon}{\rho_t}$$
    由此RSC算法至少得到边数为
    $$\bigg(1-2\sqrt\frac{\varepsilon}{\rho_t}\bigg)(\rho_t-\rho_{t+1})|E|$$
    若$\rho_t\ge16\varepsilon$且$\rho_{t+1}\ge16\varepsilon$,使用双阈值方法生成分割,此时数量即为
    $$|E|\int_{\rho_{t+1}}^{\rho_t}\bigg(1-2\sqrt{\frac{\varepsilon}{\rho_t}}\bigg)\mathrm{d}r\ge|E|\int_{\rho_{t+1}}^{\rho_t}\bigg(1-2\sqrt{\frac{\varepsilon}{r}}\bigg)\mathrm{d}r$$

    若$\rho_t\ge16\varepsilon\ge\rho_{t+1}$,则部分用双阈值方法, 部分采用贪婪,分界点在$r=16\varepsilon$,此时界为
    $$|E|(\rho_t-16\varepsilon)\bigg(1-2\sqrt{\frac{\varepsilon}{\rho_t}}\bigg)+|E|(16\varepsilon-\rho_{t+1})\frac{1}{2}$$
    与上类似积分放缩为
    $$|E|\int_{16\varepsilon}^{\rho_t}\bigg(1-2\sqrt{\frac{\varepsilon}{r}}\bigg)\mathrm{d}r+|E|\int_{\rho_{t+1}}^{16\varepsilon}\frac{1}{2}\mathrm{d}r$$
    若$\rho_t<16\varepsilon$且$\rho_{t+1}<16\varepsilon$,全部使用贪婪算法,界放为
    $$|E|\int_{\rho_{t+1}}^{\rho_t}\frac{1}{2}\mathrm{d}r$$

    综上,全部拼接后可得最终切割后至少得到的边数为
    $$|E|\bigg(\int_{16\varepsilon}^1\bigg(1-2\sqrt\frac{\varepsilon}{r}\bigg)\mathrm{d}r+\int_0^{16\varepsilon}\frac{1}{2}\mathrm{d}r\bigg)$$
    直接积分得到结果。
}

\section{图与线性规划}
\subsection{图论问题}
\textbf{团}:即完全子图,图中最大团的大小称为\textbf{团数}$\omega(G)$。

\textbf{独立集}:两两互不相邻的顶点构成的集合,图中最大独立集的大小称为\textbf{独立数}$\alpha(G)$。

\textbf{Ramsey数}:任意正整数$r,s$,存在最小正整数$n$,使得$n$阶完全图二染色必然出现颜色$A$的$r$阶团或颜色$B$的$s$阶团,记为$R(r,s)$。

*可以构造出$R(3,3)>5$,讨论可进一步证明$R(3,3)=6$。

*求解精确值非常困难,相关工作基本为上下界估计。

\

\textbf{minor}:$H$可以通过原图$G$进行若干次删点、删边、收缩边(将一条边与相邻两顶点合为一个)得到。

\textbf{着色}:使得相邻顶点不同色的对顶点的染色,至少需要的颜色数称为图的\textbf{色数}$\chi(G)$。$k$-染色为使用$k$种颜色的满足条件的染色方法。

\textbf{Hadwiger猜想}:若$G$的是无环图且不包含$K_t$-minor,则其色数小于$t$。

*其$t=5$的情况可以推出四色定理。

*由于判断是否存在指定minor的算法是低复杂度的,其被彻底解决可以带来图染色算法的飞跃发展。

\

\textbf{顶点覆盖}:与图中所有边都相交的顶点集。最小顶点覆盖的大小称为\textbf{顶点覆盖数}$\tau(G)$。

\textbf{团覆盖}:一组团$C_1,\dots,C_K$使得每个$C_i$是图的一个团,且它们的并集覆盖了$V$。最少的团覆盖的大小称为图的\textbf{团覆盖数}$\bar{\chi}(G)$。

\

\textbf{补图}:顶点集与$G$相同,与$G$的边之并为完全图,且与$G$无公共边的图,记为$\bar{G}$。

基本关系:
\begin{itemize}
    \item $\alpha(G)=\omega(\bar{G})$;
    \item $\bar{\chi}(G)=\chi(\bar{G})$;
    \item $\omega(G)\le\chi(G)$;
    \item $\alpha(G)\le\bar{\chi}(G)$;
    \item $\tau(G)=|V|-\alpha(G)$。
\end{itemize}

\exce{5.3}{证明上述图的基本量间的关系。}

\textbf{k-部图}:能将$n$个顶点划分为$k$个非空子集,使得仅当顶点属于不同子集时存在边。

\textbf{Tur\'an图}:能将$n$个顶点划分为$k$个非空子集,使得当且仅当属于不同子集时存在边。这样的图记作$T_{k,n}$。

Tur\'an定理:若$G$是简单图且其中不包含大于等于2阶的完全图,则$e(G)\le e(T_{k-1,n})$,这里$e$代表边的个数。

\

\textbf{一些P问题}

*\textbf{独立数计算}是NPC问题(可以约化为SAT问题)。此外,判断$\alpha(G)\ge k$是否成立或$\alpha(G)\le4$是否成立仍为NPC问题;对3-正则(每个顶点度数为3)平面图$G$,计算独立数仍为NPC问题。

*根据之前基本关系,团数与顶点覆盖数也为NPC问题。

*独立集问题可以约化为点染色问题,由此色数也为NPC问题。对3-正则平面图$G$,计算色数为NPC问题;判定3-正则图是否色数为3是NPC问题。但\textbf{判定色数是否为2是P问题}。

\

\textbf{平面图}:一个图是平面图当且仅当$K_5$和$K_{3,3}$不为其minor(这里$K_{3,3}$代表两部分均三个顶点且互相完全连接的二部图)。

*也可描述为不包含$K_5$或$K_{3,3}$的剖分,剖分指任何边上可以随便加点。

*\textbf{四色定理}:平面图色数至多为4。

\textbf{AGC问题}:找到色数不超过$c$的图的$d$-染色,其中$3\le c\le d$。

*即使对$c=3$、$d=6$的情况,复杂度仍然未知。

\subsection{线性规划}
*本节及之后主要讨论线性规划问题[Linear Programing, LP]与利用线性规划构造的近似算法。

\

\textbf{标准形式}
$$\min_\Omega cx,\quad\Omega=\{x\in\mathbb{R}^n\mid Ax=b,\ x\ge0\}\quad A\in\mathbb{R}^{m\times n},\quad c\in\mathbb{R}^{1\times n},\quad b\in\mathbb{R}^{m\times 1}$$

*几何:可行域$\Omega$为多面体,称\textbf{顶点}(只要$x$为$y,z$的中点,且$x,y,z\in\Omega$,则必须$x=y=z$的$x$)为极点。

若最优解存在,则\textbf{至少一个最优解在顶点上}。

\proo{
    设最优解中含有零分量最多的一个(未必唯一)为$x^*$,下证$x^*$为顶点。若否,存在$y$、$z$使得$x^*=\frac{y+z}{2}$,且三者互不相同。

    由于$x^*$为最优解,从$cx^*=\frac{1}{2}(cy+cz)$可知只能$cy=cz=cx^*$,从而三者均为最优解,于是$x^*$与$y$连线上的任何点$x^*+\alpha(y-x^*),\alpha\in\mathbb{R}$直接计算可知均为最优解,记此集合为$l$。

    验证可知$l$上任何点满足$Ax=b$,其参数方程为
    $$(x^*_1+\alpha(y_1-x^*_1),x^*_2+\alpha(y_2-x^*_2),\dots,x^*_2+\alpha(y_n-x^*_n))$$
    对$x^*$为0的分量,若$y$大于0则$z$小于0,反之亦然,于是从$y,z\in\Omega$可知$y$对应分量为0,由此直线上任何点在$x^*$为0的分量为0。但是,由于$y\ne x^*$,一定存在$y_j-x^*_j\ne0$。对所有满足此条件的分量$j_1,\dots,j_r$,选取出其中$\frac{x^*_j}{|y_j-x^*_j|}$最小的一个,记为$j_0$,并取$\alpha=-\frac{x^*_{j_0}}{y_{j_0}-x^*_{j_0}}$。计算可发现这样的$\alpha$可以保证结果仍在$\Omega$中,且第$j_0$个分量为0,而$x^*$的第$j_0$个分量为0,又已知$x^*$为0的分量在$l$中均为0,即与为0分量最多矛盾。
}

设$A$第$i$列为$a_i$,若$x\in\Omega$,则$x$是顶点当且仅当满足$x_j\ne0$的$x_j$\ (记为$J=\{j_1,\dots,j_k\}$)对应的$a_j$线性无关。

\proo{
    右推左:仍然反证,若结论不成立,设$x=\frac{y+z}{2}$,且$y\ne x\ne z$,与上个证明相同得$x$为0的分量$y,z$亦为0,因此用$x_J$表示$x$在$J$中的分量构成的向量,代入可知$x_J$、$y_J$、$z_J$均满足关于$u$的方程
    $$a_{j_1}u_1+a_{j_2}u_2+\dots+a_{j_k}u_k=b$$
    但根据线性无关性,此方程组的解应至多唯一,矛盾。

    左推右:若线性相关可发现上述方程组的解不止一个,由于$x_J$必然为解,且$x_J>0$,由解空间连续性必然存在解$x'_J$使得其各分量与$x_J$的差距小于$x_J$的分量,由此即可验证$x'_J$、$2x_J-2x'_J$扩充而成的向量$y,z$满足$x=\frac{y+z}{2}$,且$y,z\in\Omega$,$y\ne z\ne x$,矛盾。
}

\

\textbf{对偶理论}

标准形的对偶问题为
$$\min_{\Omega'} yb,\quad\Omega'=\{y\in\mathbb{R}^{1\times n}\mid yA\le c\}$$
有如下结论:
\begin{enumerate}
    \item 对$x\in\Omega$、$y\in\Omega'$有$cx\ge yb$。
    \item 原问题与对偶问题的解一定属于如下四种情况之一:
    \begin{itemize}
        \item $\Omega=\Omega'=\varnothing$;
        \item $\Omega\ne\varnothing$,但无最优解,$\Omega'=\varnothing$;
        \item $\Omega'\ne\varnothing$,但无最优解,$\Omega=\varnothing$;
        \item 两问题均有最优解。
    \end{itemize}
    \item 若\textbf{互补松弛条件}$cx=yb$成立,则$x,y$分别为原问题、对偶问题的最优解。
    \item 若$x,y$分别为原问题、对偶问题的最优解,则$cx=yb$,从而两问题最优值相同。
\end{enumerate}

*标准形对偶问题的结论事实上来自一般的线性规划对偶定义,也即下方练习题。

\exce{5.4}{设$c\in\mathbb{R}^n$、$b\in\mathbb{R}^m$、$A\in\mathbb{R}^{m\times n}$,定义原始问题与对偶问题分别为
$$\min_\Omega c^Tx,\quad\Omega=\{x\in\mathbb{R}^n\mid Ax\ge b,\ x\ge0\}$$
$$\min_{\Omega'} b^Tw,\quad\Omega'=\{w\in\mathbb{R}^m\mid w^TA\ge c^T,\ w\ge0\}$$
考察它们的解的性质。}

\

\textbf{应用:最小顶点覆盖}

*也即给定每点权重,寻找顶点覆盖中权和最小的一个。

设顶点为1到$n$,原问题转化为优化问题:求$c^Tx=c_1x_1+\dots+c_nx_n$最小值,满足
$$x_i+x_j\ge1,\quad\forall\{i,j\}\in E$$
$$x_i\in\{0,1\},\quad\forall i=1,\dots,n$$

将其松弛为,求$c_1x_1+\dots+c_nx_n$最小值,满足
$$x_i+x_j\ge1,\quad\forall\{i,j\}\in E$$
$$x_i\in[0,1],\quad\forall i=1,\dots,n$$
成为线性规划问题。

*此问题的可行域形式为$Ax\ge b$、$x\in[0,1]$,设辅助变量$y=Ax-b$、$z=1-x$,即有
$$Ax-y=b,\quad x+z=1,\quad x\ge0,\quad y\ge0,\quad z\ge0$$
成为标准形式,从而此的确为线性规划。

为设计近似算法,已知线性规划可以求解,并设最优解$x^*$。对$x^*$进行四舍五入,即大于等于1/2时输出1,否则为0,成为原问题的一个近似解$x^A$。

此近似算法可以达到2的近似比,也即$c^Tx^A$不超过真实最优解的两倍。

\proo{
    对任何一条边$\{i,j\}\in E$,有$x_i^*+x_j^*\ge1$,于是至少一个$\ge1/2$,四舍五入后$x_i^A+x_j^A\ge1$,由此可知$x^A$为可行解。

    记$\sum_ic_ix_i=c^Tx$,由于$x_i^A\le 2x_i^*$,可知$c^Tx^A\le2c^Tx^*$。然而,松弛后的问题的最优解一定比原问题最优解更小,从而得到结论。
}

由于只需要四舍五入的结果,我们希望对线性规划也可采用充分快的近似算法求解。

简单起见,考虑$c$所有分量为1的情况,此时即为计算$\tau(G)$。我们考虑另一种松弛思路:求$\sum_ix_i$最小值,使得
$$x_i+x_j\ge1,\quad\forall\{i,j\}\in E$$
$$x_i\ge0,\quad\forall i=1,\dots,n$$

分析可知,此问题的对偶问题为,求$\sum_{\{i,j\}\in E}y_{ij}$的最大值,使得
$$\sum_{j\mid\{i,j\}\in E}y_{ij}\le1,\quad\forall i=1,\dots,n$$
$$y_{ij}\ge0,\quad\forall\{i,j\}\in E$$

将原始问题称为问题I、线性规划原问题称为问题II、对偶问题称为问题III。

对问题III的任何一个可行解,假设其有0-1形式,则第一个约束表示每个顶点连接的边中至多能选取一条,此时的最优问题即称为\textbf{最大匹配}问题(每个顶点至多与一条相连的边集合称为图的\textbf{匹配},由此每个0-1可行解为一个匹配)。

从匹配出发给出问题I的可行解:若与顶点$i$连接的边有在匹配中的,则令$x_i=1$,否则为0。

*若找到某个\textbf{极大匹配}作为$y$的近似解,可证明其对应的$x$构成原问题的一个2-近似解。

\subsection{舍入方法}
*都只能针对某类问题,一般没有通用性。

\

\textbf{基础可行解}

不妨设$m<n$且$\mathrm{rank}(A)=m$以保证可行域非空。由已证,$x$为$\Omega$的顶点当且仅当$A$的对应列向量线性无关。由此,顶点$x$最多有$m$个非零分量。

设$A$的列的某极大线性无关组$a_{j_1},\dots,a_{j_m}$,指标集称为\textbf{可行基}$J$,当$j\ne J$时$x_j=0$的解称为\textbf{基础可行解}。直接利用线性方程组知识计算可知可行基$J$对应的唯一基础可行解为$x_J=A_J^{-1}b$、其余分量为0。

*若$x_J$所有分量非零,称为\textbf{非退化}的,否则称为退化的。对于退化的基础可行解,其可能成为不同基对应的基础可行解。

若一个线性规划问题所有基础可行解都非退化,称其满足\textbf{非退化假设}。

*可证明基础可行解与极点等价,一定存在基础可行解为最优解。

\exce{6.1}{给出一个不满足非退化假设的线性规划问题,使其存在基础可行解到可行基的双射。}

\

\textbf{管道舍入:最大权命中集}

给定有限集$E$与其子集族$\mathcal{C}$,$w$是定义在$\mathcal{C}$上的非负权函数,任给正整数$p$,求含有$E$中$p$个元素的子集$A$使得$\mathcal{C}$中与$A$有交的子集的权值之和最大。

\textbf{整数规划}形式:元素为1到$n$,$\mathcal{C}=\{S_1,\dots,S_m\}$,$w(S_i)=w_i$,求最大值
$$L(x)=\sum_{j=1}^mw_j\min\bigg\{1,\sum_{i\in S_j}x_i\bigg\}$$
使得$\sum_{i=1}^nx_i=p$,且$x_i\in\{0,1\}$。

*最大值可等价于
$$F(x)=\sum_{j=1}^mw_j\bigg(1-\prod_{i\in S_j}(1-x_i)\bigg)$$

$L(x)$松弛到$x_i\in[0,1]$后实际上是线性规划(见下方),而$F(x)$并非线性规划,但$F(x)$事实上是更易于设计舍入算法。

\textbf{松弛问题}比较:$x_i\in[0,1]$时$F(x)\ge\big(1-\frac{1}{\er}\big)L(x)$。

\proo{
    考虑某$\mathcal{C}$中子集$S_j$,且$|S_j|=k$。

    将几何平均数放大为算术平均数可知
    $$1-\prod_{i\in S_j}(1-x_i)\ge1-\bigg(1-\frac{\sum_{i\in S_j}x_i}{k}\bigg)^k$$

    对$f(z)=1-\big(1-\frac{z}{k}\big)^k$,求导估算可发现其为单调增凹函数,且$f(0)=0$,当$z\in[0,1]$时,有$f(z)/z\ge f(1)$,即$f(z)\ge zf(1)$,而$z>1$时$f(z)>f(1)$,于是$f(z)\ge f(1)\min(1,z)$。由此再利用$\er$的极限形式得证。
}

为将原问题松弛为线性规划,引入$z_1,\dots,z_m$,并将问题变为求$\sum_jw_jz_j$最大值使得
$$\sum_{i\in S_j}x_i\ge z_j,\quad\forall j=1,\dots,m$$
$$0\le z_j\le1,\quad\forall j=1,\dots,m$$
$$0\le x_i\le1,\quad\forall i=1,\dots,n$$
$$\sum_{i=1}^nx_i=p$$

此问题的解$x^*$中若有不属于$\{0,1\}$分量,至少有两个(否则和不可能为整数),将$x^*$赋值给$x$,重复执行,对两个$x_k,x_j\in(0,1)$,定义
$$x(\varepsilon)=\begin{cases}x_i&i\notin\{j,k\}\\x_j+\varepsilon&i=j\\x_k-\varepsilon&i=k\end{cases}$$
将$\varepsilon$分别取为$\varepsilon_1=-\min\{x_j,1-x_k\}$与$\varepsilon_2=\min\{1-x_j,x_k\}$,比较$F(x(\varepsilon_1))$与$F(x(\varepsilon_2))$,以较大者进行这两个分量的舍入,直到所有分量被舍入。

*由于舍入过程保证了和不变,结果恒为可行解。

\textbf{近似比}:舍入过程中$F$不下降,设舍入结果为$x^A$,从而有
$$L(x^A)=F(x^A)\ge F(x^*)\ge\bigg(1-\frac{1}{\er}\bigg)L(x^*)\ge\bigg(1-\frac{1}{\er}\bigg)\opt$$

\proo{
    只需证明$F(x(\varepsilon))$关于$\varepsilon$是凸函数(可发现其事实上是二次函数)即可。

    由于$x,j,k$已固定,对每个$l=1,\dots,m$,只需考虑三种情况,分类讨论。$S_l$不含$j,k$时对应的求和中为常数,凸;$S_l$含$j,k$中一个时对应的求和中为线性函数,凸;$S_l$含$j,k$时对应的求和中为二次函数,观察二次项系数可知凸。
}

*由此管道舍入的核心为某种将乘积放为求和进行估算,并通过乘积进行舍入。

\exce{6.2}{用管道舍入给出最大可满足性问题的$\frac{\er}{\er-1}$近似算法。}

\

\textbf{迭代舍入:广义生成网络问题}

*基本方案,通过多次求解线性规划问题进行更好的舍入。

给定图$G=(V,E)$、边上的非负权函数$c$与正整数$k>0$,求一个$k$-边连通的子图,使得其中边权值之和最小。

先给出基本的求解方法:
\begin{itemize}
    \item \textbf{连通性质}
    
    $F$为$k$-边连通的当且仅当对图的顶点集合任何划分$S,S^c$\ (要求它们均非空),其中都至少存在$F$中的$k$条边。
    \item \textbf{整数规划}
    
    对$x_e\in\{0,1\},\forall e$,求
    $$\min\sum_{e\in E}c_ex_e$$
    使得
    $$\forall S\subset V,S\notin\{\varnothing,V\},\quad\sum_{e\in\delta_G(S)}x_e\ge k$$
    这里$\delta_G(S)$为$G$中恰有一个端点在$S$中的所有边组成的集合。

    *可验证其符合要求,但此时不等式约束为指数量级个,看似线性规划也无法方便求解。

    \item \textbf{连续松弛}

    若线性规划问题具有性质:对任何不可行解$x$,可以在多项式时间找到其不满足的约束条件(称为\textbf{分离神谕}[separation oracle]),则即使约束条件个数为指数个,也能用\textbf{椭球法}在多项式时间找到其最优解(省略过程细节)。

    于是,将$x_e$松弛到$[0,1]$后,我们希望找到上述的分离神谕。

    \item \textbf{网络流转化}
    
    将松弛问题的可行域设为$\Omega$,给定$x_e$事实上相当于给$G$的每条边赋予$[0,1]$中的值,将其看作每边的流量上界。

    由此,约束条件事实上可以转化为,对任何顶点对$s$与$t$,$s$到$t$的最大流至少是$k$,由此只需要计算$s$到$t$的最大流。

    \proo{
        根据网络流的知识,最大流问题与最小割问题互为对偶,于是等价。也即,任两点$s,t$之间的最大流值等于$s\in S$、$t\in S^c$的最小割值$\min E(S,S^c)$。
        
        原问题条件可以看作对任何图割,割值$E(S,S^c)$至少为$k$,而任两点最大流为图割最多为$k$;反之,若存在$E(S,S^c)<k$,任取$s\in S$、$t\in S^c$可得矛盾。
    }

    由于最大流/最小割问题是P问题,对任何$x\notin\Omega$,若有某个分量不在$[0,1]$中则已经在$|E|$量级找到其不满足的条件,否则只需对任何两点(这是$|V|^2$量级的)求解最大流问题,若发现最大流低于$k$,将其对应的最小割找到即得到不满足的条件。

    *上述算法也是判断一个子图是否$k$-边连通的多项式算法。
\end{itemize}

迭代舍入的算法为:
\begin{enumerate}
    \item 输入$G$、$k$与$c$,初始$F=\varnothing$\ (可将其看作边集合);
    \item 构造对应的迭代LP问题
    $$\min\sum_{e\in E\backslash F}c_ex_e$$
    $$\forall e\in E,\quad x_e\in[0,1]$$
    $$\forall S\subset V,\quad\sum_{e\in\delta_{G\backslash F}(S)}x_e\ge f_0(S)-|\delta_F(S)|$$
    $$f_0(S)=\begin{cases}0&S\in\{\varnothing,V\}\\k&S\notin\{\varnothing,V\}\end{cases}$$
    并求最优解$x^*$;
    \item 将$F$扩充为$F\cup\{e\mid x_e^*\ge 1/3\}$;
    \item 若$F$已经$k$-边连通,输出,否则回到第二步。
\end{enumerate}

我们需要先证明算法的有效性。

\textbf{弱超模函数}:$f$为$V$的子集到$\mathbb{Z}$的函数,满足$f(V)=0$,且对任何$A,B\subset V$,以下两式至少一个成立(这里$A-B$表示属于$A$不属于$B$的函数):
$$f(A)+f(B)\le f(A-B)+f(B-A),\quad f(A)+f(B)\le f(A\cap B)+f(A\cup B)$$

*\textbf{超模函数}:其相反数是次模函数,也即第二式恒成立。

*\textbf{强超模函数}:两式均恒成立。

引理:对弱超模函数$f$,线性规划问题
$$\min\sum_{e\in E}c_ex_e$$
$$\forall e\in E,\quad x_e\in[0,1]$$
$$\forall S\subset V,\quad\sum_{e\in\delta_G(S)}x_e\ge f(S)$$
的每个基础可行解都包含至少一个$x_e\ge\frac{1}{3}$的分量。

\proo{
    若$A,B\subset V$满足互不为子集且交非空,则称它们是公平相交的。若$V$的一个子集族不包含两个公平相交的集合,则称它是层状的。

    用$a_S$表示$S\subset V$对应的不等式约束的行向量,也即将每个约束写为$a_Sx\ge f(S)$。若某个约束$a_Sx=f(S)$对$x$取等,称其对$x$是积极的,也称对应的$S$对$x$是积极的。

    设$x$的基础可行解非零分量为$k$\ (也即存在$k$条边$e$使得$x_e>0$),则利用线性规划理论可知至少存在$k$个积极约束,对应的$a_S$线性无关。

    \

    引理:若$x$为一个基础可行解,且每个$x_e\in(0,1)$,则图$G$中存在一个$V$的层状积极子集族$\mathcal{F}$,满足:
    \begin{enumerate}
        \item $|\mathcal{F}|=|E|$;
        \item $\{a_S\mid S\in\mathcal{F}\}$线性无关;
        \item 任何$S\in\mathcal{F}$有$f(S)\ge1$;
        \item 存在$S\in\mathcal{F}$使得$\delta_G(S)\le3$。
    \end{enumerate}

    引理证明:
    \begin{enumerate}
        \item 秩性质
        
        考虑任何一个$V$的极大的层状积极子集族$\mathcal{L}$,先证明$\{a_S\mid S\in\mathcal{L}\}$的秩为$|E|$。

        若否,根据列数可知其秩应小于$|E|$,用$\mathrm{span}\{L\}$表示$\{a_S\mid S\in\mathcal{L}\}$生成的线性空间。利用线性规划基础可行解的结论,所有积极约束对应的$a_S$集合秩应为$|E|$,于是存在积极集$A$满足$a_A\notin\mathrm{span}\{L\}$。

        由极大性,$A$无法添进$\mathcal{L}$,于是其与$\mathcal{L}$中某个集合公平相交。取所有符合要求的$A$中,与$\mathcal{L}$中公平相交个数最少的集合,仍记为$A$,并设$B\in\mathcal{L}$与$A$公平相交。利用$f$为弱超模函数可知
        $$f(A)+f(B)\le f(A-B)+f(B-A),\quad f(A)+f(B)\le f(A\cap B)+f(A\cup B)$$
        至少成立一个。

        若第一式成立,记$S_1=A-B$、$S_2=B\cap A$、$S_3=B-A$、$S_4=(A\cup B)^c$,第一式可写为
        $$f(A)+f(B)\le f(S_1)+f(S_3)$$
        再记
        $$m_{ij}=\sum_{e\in E(S_i,S_j)}x_e$$
        由$A,B$积极可知
        $$f(A)=a_Ax=m_{13}+m_{14}+m_{23}+m_{24},\quad f(B)=m_{31}+m_{34}+m_{21}+m_{24}$$
        $$f(S_1)\le a_{S_1}x=m_{12}+m_{13}+m_{14},\quad f(S_3)\le m_{31}+m_{32}+m_{34}$$
        由无向图,$m_{ij}=m_{ji}$,对比即可得到
        $$f(S_1)+f(S_3)+2m_{24}\le f(A)+f(B)$$
        由此第一式成立只能$m_{24}=0$\ (根据$x_e>0$知只能$E(S_2,S_4)=\varnothing$),且$S_1,S_3$积极。

        然而,由$a_A\notin\mathrm{span}(\mathcal{L})$可知$a_{S_1},a_{S_3}$至少有一个不在$\mathrm{span}(\mathcal{L})$中:
        \begin{itemize}
            \item 若$a_{S_1}\notin\mathrm{span}(\mathcal{L})$,由于$B$与$A$公平相交但$B$不与$S_1$公平相交,只需证明$C\in\mathcal{L}$且$C$与$S_1$公平相交则$A$与$C$公平相交,即与个数最少性矛盾,而这只需要证明$C$不包含于$A$,即$C-A$非空。
            
            利用公平相交性,$S_1\cap C\ne\varnothing$,于是$C-B$非空,但由它们都在$\mathcal{L}$中可知$C$包含$B$\ (此时$B-A\subset C-A$非空)或$C\cap B=\varnothing$\ (此时$C-A=C-S_1$),均矛盾。

            \item 若$a_{S_3}\notin\mathrm{span}(\mathcal{L})$同上可证。
        \end{itemize}

        若第二式成立可完全类似考虑$S_1,\dots,S_4$证明矛盾,从而推出原结论成立。

        \item 前三条件存在性
        
        由上述,考虑$\{a_S\mid S\in\mathcal{L}\}$的极大线性无关组,即可满足前两个条件。

        此外,由积极集定义可知$f(S)=a_Sx$,根据$a_S$非负且有1分量、每个$x_e\in(0,1)$可知$f(S)>0$,而由$f(S)$为整数,其至少为1。

        \item 森林转化
        
        下面证明条件4成立。反证,若这样的$S$不存在,也即对任何$S\in\mathcal{F}$有$|\delta_G(S)|\ge4$。

        由$\mathcal{F}$为层状积极子集族,根据定义可发现$\mathcal{F}$中的集合以包含关系连边($A,B$有边当且仅当$A\supset B$且不存在$C$使得$A\supset C\supset B$)可以构造一个森林$T$,设其顶点$V'$、边$E'$。

        对每个$G$中顶点$u\in V$、$e\in E$,且$u$为$e$的一个端点,称$(u,e)$为$G$中一个端点。若$(u,e)$对$S\in\mathcal{F}$满足,$u\in S$且对任何$S$的真子集$S'\in\mathcal{F}$有$u\notin S'$,则记$(u,e)\in P(S)$,下面利用此映射计数。

        对$T$的子树$T'$,记
        $$P(T')=\bigcup_{S\in V'(T')}P(S)$$
        下面证明$|P(T')|\ge2|V(T')|+2$,则$|P(T)|\ge 2|\mathcal{F}|+2=2|E|+2$,但总端点数至多$2|E|$,矛盾。

        \item 端点计数

        利用归纳法。首先,由假设,$T$的每个叶子结点$S$应有$|P(S)|\ge4$。
        
        对任何森林$T'$,只需说明其为子树的情况成立,利用层状子集族特性可知不交子树对应的端点集合不交,从而求和得证。下假设对$T'$的任何孩子结论均成立。
        
        若$T'$的根$R$至少有两个孩子节点,其每个孩子对应的子树$T_i$,利用层状子集族性质可知不同子树对应的端点不会重复,因此
        $$|P(T')|\ge|P(T_1)|+\dots+|P(T_k)|\ge2(|V(T_1)|+\dots+|V(T_k)|)+2k=2|V(T')|+2k-2$$
        从而成立。

        若$R$只包含一个孩子节点$S$,记$T_1$为以$S$为根的子树,由归纳假设
        $$|P(T_1)|\ge 2|V(T_1)|+2$$

        若$P(R\backslash S)$中至少有两个端点,则由定义可知$|P(T)|-|P(T_1)|\ge2$,已经得证。否则,只要证明$\delta_G(R)$与$\delta_G(S)$恰好相差一条边$e$,即可从$a_Rx=f(R)$、$a_Sx=f(S)$得到$x_e=|f(R)-f(S)|$。但左侧不为整数,右侧为整数,矛盾。

        由线性无关性,$\delta_G(R)$与$\delta_G(S)$不可能完全相同(否则$a_R=a_S$),从而也可得到$P(R\backslash S)$不可能为空。由此,$P(R\backslash S)$中恰包含一个端点$(u,e)$,讨论可发现$\delta_G(R)$与$\delta_G(S)$至多相差$e$,得证。

    \end{enumerate}

    \

    利用此引理,解中$x_e=0$的边可以直接去掉,只考虑子图,而$x_e=1$的边已经符合$\ge\frac{1}{3}$的要求。又由于$\mathcal{F}$中一定有$S$使得$|\delta_G(S)|\le3$,利用
    $$\sum_{e\in\delta_G(S)}x_e=f(S)\ge1$$
    即可得到结论。
}

\exce{6.3}{证明对弱超模函数$f$与$G$的子图$F$,$f(S)-|\delta_F(S)|$仍为弱超模函数。}

*可验证$f_0(S)$是弱超模函数,由此根据上方练习(本质上是由于$\delta_F(S)$是强次模函数)可知算法的确可以在多项式次迭代后结束。

\exce{6.4}{证明迭代舍入算法给出了原问题的一个3-近似解。}

\subsection{最小顶点覆盖}
*继续研究\textbf{无需求解线性规划问题}的近似算法。

回顾之前对非加权的最小顶点覆盖问题的讨论,考虑其加权形式:求$c_1x_1+\dots+c_nx_n$最小值,满足
$$x_i+x_j\ge1,\quad\forall\{i,j\}\in E$$
$$x_i\in\{0,1\},\quad\forall i=1,\dots,n$$
\begin{itemize}
    \item 对偶构造
    
    如之前,对应的LP为将$x_i$松弛到$[0,1]$。其对偶问题DP为
    $$\max\sum_{\{i,j\}\in E}y_{ij}$$
    $$\sum_{j\mid\{i,j\}\in E}y_{ij}\le c_i,\quad y_{ij}\ge0$$

    \item 对偶问题近似求解
    
    从全0出发,不断选择$\{i,j\}\in E$,将$y_{ij}$增加到不超过约束的最大可能值,直到不能再选择边(对于非加权形式,这样即得到一个极大匹配)。

    \item 原问题近似解构造
    
    $x_i=1$当且仅当$x_i$对应的约束是积极的,即
    $$\sum_{j\mid\{i,j\}\in E}y_{ij}=c_i$$

    将此解称为$x_A$。

    *若某条边两端点的约束均不积极,可以增加这条边的$y_{ij}$的值,矛盾,因此$x^A$确为可行解。
\end{itemize}

\

近似比结论:
$$\sum_{i=1}^nc_ix_i^A\le2\opt$$
\proo{
    将$x^A$中$x_i^A=1$的下标集合记为$S$,即有(每条边最多被求和了两次)
    $$\sum_{i\in S}c_i=\sum_{i\in S}\sum_{j\mid\{i,j\}\in E}y_{ij}\le2\sum_{j\mid\{i,j\}\in E}y_{ij}$$
    由线性规划问题与对偶问题最优值相同,右侧求和不超过$2\opt$,从而得证。
}

\subsection{多胞体理论}
想法:考虑\textbf{多胞体},即线性规划问题的可行域。既然可以将整数规划后松弛为线性规划在对应可行域找解后舍入,是否能利用多胞体中直接寻找整数规划近似解?

\

\textbf{二部图最优匹配}:给定二部图$G=(V,E)$,每边存在权重。若边集的子集$M$互相无公共顶点,则称其为$G$的一个匹配。求使权值和最大的匹配。

记边集的权和
$$\omega(E')=\sum_{e\in E'}\omega_e$$
问题即变为求匹配$M\subset E$使得$\omega(M)$最大。

记示性向量
$$x_e^{E'}=\begin{cases}1&e\in E'\\0&e\notin E'\end{cases}$$
则可得到整数规划形式为(这里$e\ni v$表示$v$是$e$的一个端点)
$$\max\ \omega\cdot x$$
$$x\in\{0,1\}^{|E|},\quad\sum_{e\ni v}x_e\le1,\quad\forall v\in V$$

\

将$\{0,1\}$松弛为$[0,1]$即对应线性规划问题。由于第二个条件可保证任何$x_e\le1$,事实上只需要$x\ge0$即可,其对偶问题为
$$\min\sum_vy_v$$
$$y\ge0,\quad y_{v_1}+y_{v_2}\ge\omega_e,\quad\forall e=(v_1,v_2)$$

利用对偶理论可发现原问题对偶问题最优解相同,且计算可发现\textbf{原问题最优解可在0-1向量取到}。

考虑$\omega$全为1的情况,此时同样可计算证明\textbf{对偶问题的最优解在0-1向量取到},此时其对偶问题的最优解可以取为最小顶点覆盖问题的解。

\textbf{K\"onig定理}:\textbf{二部图}最大匹配数等于其最小顶点覆盖数(一般图未必成立)。

*多胞体理论的发展即来源于二部图最大匹配的特殊性。

\

仍回到整数规划问题,考虑另一种松弛方式:将可行域\textbf{松弛到凸包}。利用目标函数是线性函数,其必然是凸函数,由此松弛到凸包后解不变。

由此,若原问题凸包恰好为松弛后线性规划问题的可行域,则两者解等价,二部图恰好满足此性质,从而成立。

*多胞体理论:通过别的途径得到多胞体,判断和松弛后的线性规划可行域是否吻合。
\newpage
\appendix
\section{报告}
\begin{enumerate}
    \item 组合优化的硬件应用
    \item 半定规划与蒙特卡洛方法
    
    \exce{4.3}{证明半定规划的原问题与对偶问题都有内点可行解时,原问题与对偶问题必存最优解,且满足互补条件。}
    \item 后量子密码的数学基础
    \item 基本图割问题的等价谱定理
    \item 组合优化与博弈论
\end{enumerate}
\end{document}