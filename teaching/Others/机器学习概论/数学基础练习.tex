\documentclass[a4paper,UTF8,fontset=windows]{ctexart}
\pagestyle{headings}
\title{Problem Set}
\author{原生生物}
\date{}
\setcounter{tocdepth}{2}
\setlength{\parindent}{0pt}
\usepackage{amsmath,amssymb,amsthm,enumerate,geometry}
\geometry{left = 2.0cm, right = 2.0cm, top = 2.0cm, bottom = 2.0cm}
\ctexset{section={number=\zhnum{section}}}
\ctexset{subsection={name={\S},number=\arabic{section}.\arabic{subsection}}}
\DeclareMathOperator*{\tr}{tr}

\begin{document}
\maketitle

\section{二次型}
\subsection{基本结果}
\begin{enumerate}
    \item 记$Q(x)=\dfrac{1}{2}x^TAx+b^Tx$,其中$A$为$n$阶对称方阵,$x$为$n$维向量。
    \begin{enumerate}
        \item 设$A=(a_{ij})$,试说明$\dfrac{\partial x^TAx}{\partial x_1}=2\sum_{i=1}^na_{1i}x_i$。
        \item 证明$\nabla_x Q(x)=Ax+b$。
        \item 当$A$正定时,证明$Q(x)$存在唯一最小值点$x=A^{-1}b$。[注意梯度为0只能得到驻点]
    \end{enumerate}
    \item 记$Q(x)=x^TAx$,其中$A$为$n$阶对称半正定方阵[由半正定阵定义,$Q(x)$的最小值为0]。
    \begin{enumerate}
        \item 证明$Q(x)=0\Leftrightarrow Ax=0$。
        \item 设$A=P^TDP$,其中$P$正交,$D$是对角阵,非零对角元为$\lambda_1,\dots,\lambda_r$,求$Q(x)$全部最小值点。
        \item 举例:$A$半正定,$\frac{1}{2}x^TAx+b^Tx$不存在最小值。
        \item 对一般对称阵$A$,何时$\frac{1}{2}x^TAx+b^Tx$存在最小值?
    \end{enumerate}
\end{enumerate}

\subsection{最小二乘}
\begin{enumerate}
    \item 记$Q(x)=\|Ax-b\|$,其中$A$为$m\times n$阶矩阵,$x,b$为$n$维向量。[这里$\|\alpha\|$代表二范数,即$\sqrt{\alpha^T\alpha}$]
    \begin{enumerate}
        \item 计算$\nabla_x Q(x)$。[先计算$\nabla_x Q(x)^2$]
        \item 利用梯度结果证明,$Q(x)$取到最小值时必有$A^TAx=A^Tb$。
        \item 证明$A^TAx=A^Tb$时$Q(x)$取到最小值。[假设满足此条件时为$x_0$,考虑$Q(x)^2-Q(x_0)^2$]
    \end{enumerate}
    \item 考虑方程$A^TAx=A^Tb$[由上一题,求解此方程可直接解出最小二乘,此方程称最小二乘问题的正则化方程组]。
    
    此处广义逆的定义[不同情境下广义逆定义未必相同]:满足$AA^+A=A,(AA^+)^T=AA^+$的矩阵$A^+$称为$A$的广义逆。值得注意的是,当$A$未必为方阵时,广义逆仍然可以存在,若$A_{m\times n}$,则$A^+$为$n\times m$阶矩阵。
    \begin{enumerate}
        \item 若$A^TA$可逆,验证$(A^TA)^{-1}A^T$是$A$的广义逆。
        \item 证明:当$x=A^+b$时,$A^TAx=A^Tb$。[注意这里可以是任何一个广义逆]
        \item 若存在$A'$使得$\forall b$,$A'b$是$\|Ax-b\|$的最小值点,证明$A^TAA'=A^T$。
        \item 用上一问的式子说明,$A'$一定满足$(AA')^T=AA',AA'A=A$,从而$A'$是$A$的广义逆。
    \end{enumerate}
\end{enumerate}

\section{范数}
\begin{enumerate}
    \item 在线性空间$\mathbb{R}^n$中,考虑函数$p:\mathbb{R}^n\to\mathbb{R}$。若其满足:
    $$\forall x,p(x)\ge0;p(x)=0\Leftrightarrow x=\mathbf{0}$$
    $$p(\mu x)=|\mu|p(x)$$
    $$p(x)+p(y)\ge p(x+y)$$
    则称为$n$维向量的一个范数。
    \begin{enumerate}
        \item 当$A$为对称正定阵时,验证$p(x)=\sqrt{x^TAx}$是一个范数,此范数记作$\|x\|_A$。
        \item 当$A$半正定时,证明$p(x)$不是一个范数。
    \end{enumerate}

    \item 设$p(x)$是一个向量范数,考虑函数$g(A)=\max_{p(x)=1}p(Ax)$,它是$n$阶方阵映射到一个数的函数。
    \begin{enumerate}
        \item 证明$g(A)\ge0$,且其为0当且仅当$A=O$,即各个分量全为0。
        \item 证明$g(\mu A)=|\mu|g(A)$。
        \item 证明$g(A)+g(B)\ge g(A+B)$。[由这三问,我们已经说明了它是所有$n$阶方阵中的一个范数。]
        \item 证明$g(A)p(x)\ge p(Ax)$。[由此,矩阵范数与向量范数具有某种相容性,例如,若向量范数为$\|x\|_P$,可以将矩阵范数也记作$\|A\|_P$,其中$P$是对称正定阵]
        \item 证明$g(A)g(B)\ge g(AB)$。[这一问是矩阵范数的额外性质] 
        \item 证明$g(I)=1$[而单位阵的Frobenius范数为$\sqrt{n}$],且$g(A)g(A^{-1})\ge1$。[这个$g(A)g(A^{-1})$一般称为$A$在范数$g$下的条件数]
    \end{enumerate}

    \item 这里假设如上题所述的$p,g$记为向量范数$\|x\|_p$和矩阵范数$\|A\|_p$,并将矩阵范数下的条件数记作$\sigma_p(A)$。对线性方程组$Ax=b$,我们试着用矩阵范数考察解的扰动。以下假设$A$可逆。
    \begin{enumerate}
        \item 证明$\|A^{-1}b\|_p\ge\frac{\|b\|_p}{\|A\|_p}$。
        \item 假设$b$变为$b+e_b$时解从$x$变为$x+e_x$,求证$\dfrac{\|e_x\|_p}{\|x\|_p}\le\sigma_p(A)\dfrac{\|e_b\|_p}{\|b\|_p}$。
        \item 证明式中的逆都存在时$(A+E_A)^{-1}-A^{-1}=-(A+E_A)^{-1}E_AA^{-1}$。
        \item 证明式中的逆都存在时$\|I-(A+E_A)^{-1}E_A\|_p\|A^{-1}\|_p\|E_A\|_p\ge\|(A+E_A)^{-1}E_A\|_p$
        \item 假设$A$变为$A+E_A$,保证$A+E_A$仍可逆且$\|A^{-1}\|_p\|E_A\|_p<1$。若解从$x$变为$x+e_x$,求证$\dfrac{\|e_x\|_p}{\|x\|_p}\le\dfrac{\|A^{-1}\|_p\|E_A\|_p}{1-\|A^{-1}\|_p\|E_A\|_p}$。
    \end{enumerate}
\end{enumerate}

\section{矩阵求导}
\subsection{基本定义}
\begin{enumerate}
    \item 若$A$的每个分量都是$x$的函数[这里$x$为一维变量],定义$\frac{\partial A}{\partial x}$的第$i$行第$j$列为$\frac{\partial a_{ij}}{\partial x}$。
    \begin{enumerate}
        \item 计算说明$\frac{\partial AB}{\partial x}=\frac{\partial A}{\partial x}B+A\frac{\partial B}{\partial x}$。[注意矩阵乘法顺序不可交换]
        \item 计算说明$\sum_{i,j}a_{ij}b_{ij}=\tr(A^TB)$,其中$A,B$为同阶矩阵。
        \item 证明$\frac{\partial\tr(A^TB)}{\partial x}=\tr(\frac{\partial A^TB}{\partial x})$。[注意$\tr(A^TB)$是$x$的一维函数]
    \end{enumerate}

    \item 若$f(A)$是矩阵映射到数的函数,定义$\frac{\partial f(A)}{\partial A}$的第$i$行第$j$列为$\frac{\partial f(A)}{\partial a_{ij}}$。[当$A$是列向量$a$的时候,$\frac{\partial f(a)}{\partial a}$就是梯度]
    \begin{enumerate}
        \item 计算说明$\frac{\partial f(A)g(A)}{\partial A}=f(A)\frac{\partial g(A)}{\partial A}+g(A)\frac{\partial f(A)}{\partial A}$。[注意$f(A),g(A)$是数乘]
        \item 计算说明$\frac{\partial f(A)}{\partial x}=\tr\big(\big(\frac{\partial f(A)}{\partial A}\big)^T\frac{\partial A}{\partial x}\big)$。[这里$A$的每个位置都是$x$的函数,而最后又综合成了一个$x$的一维函数]
        \item 计算$\frac{\partial \tr(A)}{\partial A}$,并由此重新证明上一题的c。
    \end{enumerate}
\end{enumerate}

\subsection{更多计算}
\begin{enumerate}
    \item 有关行列式的导数。
    \begin{enumerate}
        \item 用Laplace展开证明$\frac{\partial\det A}{\partial a_{ij}}$是$A$的第$ij$个代数余子式。
        \item 证明$\frac{\partial\det A}{\partial A}=(A^*)^T$,其中$A^*$为$A$的伴随方阵。
        \item 利用上题的b计算$\frac{\partial\det A}{\partial x}$,并进一步计算$\frac{\partial\ln\det A}{\partial x}$。
    \end{enumerate}
    \item 极值问题:以下$w$是向量,$W$是$d\times d'$矩阵,$X$是$d\times m$矩阵,且$n>d>d'$。
    \begin{enumerate}
        \item 回顾$\frac{\partial w^TAw}{\partial w}$的结果,并计算$\frac{\partial\tr(W^TXX^TW)}{\partial W}$。
        \item 从计算结果分析怎样的$W$可以使$\tr(W^TXX^TW)$取到极值。
        
        下面限定$W$满足$W^TW=I_{d'}$,求解怎样的$W$使得$\tr(W^TXX^TW)$取到最大值。
        \item 利用拉格朗日乘数法,假设$W^TW$的第$i$行第$j$列对应乘数$\lambda_{ij}$,且其拼成矩阵$\Lambda$,证明Lagrange函数$L(W,\Lambda)=\tr(W^TXX^TW)-\tr(\Lambda^T(W^TW-I))$。
        \item 计算乘子部分对$W$的导数$\dfrac{\partial \tr(\Lambda^T(W^TW-I))}{\partial W}$。
        \item 注意到$W^TW$对称,其$ij$位置与$ji$位置恒相同,因此乘子$\lambda_{ij}=\lambda_{ji}$,由此可知$\Lambda$也为对称阵。证明$\dfrac{\partial L(W,\Lambda)}{\partial W}=O\Leftrightarrow XX^TW=W\Lambda$。
        
        \item [附加]对$XX^T$作正交相似对角化$P^TDP$,使得$D$的对角元从大到小排列。这时,$P$的前$d'$列构成的矩阵就是最优的$W$,此时$\Lambda$为对角阵,对角元是$D$的前$d'$个对角元。
        
        [注:这就是主成分分析的数学表达]
    \end{enumerate}
\end{enumerate}

\end{document}