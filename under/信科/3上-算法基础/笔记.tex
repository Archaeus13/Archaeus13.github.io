\documentclass[a4paper,UTF8,fontset=windows]{ctexart}
\pagestyle{headings}
\title{\textbf{算法基础\ 笔记}}
\author{原生生物}
\date{}
\setcounter{tocdepth}{2}
\setlength{\parindent}{0pt}
\usepackage{amsmath,amssymb,enumerate,geometry}
\geometry{left = 2.0cm, right = 2.0cm, top = 2.0cm, bottom = 2.0cm}
\ctexset{section={number=\zhnum{section}}}
\ctexset{subsection={name={\S},number=\arabic{section}.\arabic{subsection}}}
\setmainfont{Consolas}

\DeclareMathOperator{\DFT}{DFT}

\newenvironment{code}{\rule{36em}{0.1em}\setlength{\parindent}{1em}

}{

\setlength{\parindent}{0em}\rule{36em}{0.1em}}

\begin{document}
\maketitle

*\hspace{0em}顾乃杰老师算法基础课堂笔记[以*开头的行代表动机与注释等,**开头的行代表补充内容]

\tableofcontents

\newpage

\section{算法概念与数学基础}
定义:输入->算法->输出

可计算问题[computational problem]:对输入输出要求的说明

问题的\textbf{实例}[instance]:某个求解需要的具体输入

(排序算法中,输入数组、输出排好的数组是问题,而一个具体的乱序数组则是实例)

正确性:要求对任何输入能正确给出输出(\textbf{随机算法}可能一定程度违反)

学习算法的意义:确定解决方式的正确性、选择容易实现的算法,比较时空复杂度

*\hspace{0em}\textbf{算法效率比电脑速度更重要}

\subsection{算法简介与分析}

伪代码书写:注重表达清楚,不关注语言细节

约定:\textbf{缩进}表示块结构、循环结构与C类似解释、//代表注释、变量如无特殊说明表示局部变量

\

举例:插入排序(就地[in-place]排序,需要的额外空间为$O(1)$,即常量)

\begin{code}
def INTERSECTION-SORT(A):

\ \ for (j <- 2) to length(A)

\ \ \ \ key <- A[j]

\ \ \ \ i <- j-1

\ \ \ \ while (i > 0 and A[i] > key)

\ \ \ \ \ \ A[i+1] <- A[i]

\ \ \ \ \ \ i <- i - 1

\ \ \ \  A[i+1] <- key
\end{code}

*\hspace{0em}如何证明正确性?

可选方法:循环不变式[loop invariant],保证\textbf{初始化}、\textbf{迭代}时均为真(类似归纳),循环\textbf{终止}时能提供有用性质(局限性:只能用于判断循环,无法判断分支等)

插入排序中不变式:A[1..j-1]在循环后有序

*\hspace{0em}现实:理论证明几乎无法做到,通过大量测试假定正确

**\hspace{0em}关于正确性[correctness]:正确性指算法在程序规范下被认定为正确的判定,其中功能[functiconal]正确针对输入输出的行为,一般分为部分[partial]正确与完全[total]正确,前者指输出结果时结果正确,后者还额外要求必须能输出结果。

\

\textbf{算法分析}

含义:预测算法需要的资源,通常关心时间[computational time]与内存空间[memory],偶尔会涉及通信带宽、硬件使用等

为了统一评价,需要使用\textbf{统一}、\textbf{简洁}的计算模型

*\hspace{0em}对串行算法一般使用RAM模型[Random-access machine],并行则为PRAM模型[Parallel RAM]

\

RAM模型特点:
\begin{enumerate}
    \item 指令一条条执行(不存在并发)
    \item 包含常见算数指令、数据移动指令、控制指令,且指令所需时间为常量
    
        *\hspace{0em}按照真实计算机设定,不应滥用

        *\hspace{0em}特殊情况:如一般指数运算不是常量时间,但$2^k$通过左移可常量时间
    
    \item 数据类型有整数与浮点
        
        *\hspace{0em}不关心精度

    \item 假定数据字的规模存在范围,字长不能任意长
    
        *\hspace{0em}不关心内存层次(高速缓存、虚拟内存)
\end{enumerate}

影响运行时间的主要因素:
\begin{enumerate}
    \item 输入规模
    \item 输入数据的分布
    
        *\hspace{0em}将算法运行时间描述成输入规模的函数

    \item 算法实现所选用的底层数据结构
\end{enumerate}

*\hspace{0em}思考:RAM模型中其他影响因素?

\

*\hspace{0em}\textbf{输入规模}与\textbf{运行时间}如何严谨定义?

输入规模[input size]:对许多问题为输入项的个数(如排序),但有时(如整数相乘)关心的是二进制表示的总位数,有时则用两个数表示更合适(图的顶点数与边数)

*\hspace{0em}必须描述清楚输入规模的量度

运行时间[running time]:执行的基本操作数或步数,与机器无关,一般假设每行伪代码恒定时间

*\hspace{0em}实际计算时,由于循环嵌套所需的步数不同,很可能较为复杂,于是需要二次抽象[最终将系数抽象为独立于数据规模的常数,只关心\textbf{量级}]

最好/最坏运行时间:最快/最慢情况的运行时间(插入排序的例子中,最好为$O(n)$,最坏为$O(n^2)$)

平均运行时间:运行时间在所有输入下的期望值(\textbf{与数据的概率分布有关},一般默认均匀一致分布,插入排序的例子中为$O(n^2)$)

*\hspace{0em}平均vs最坏:最好运行时间的参考意义不大,而平均运行时间往往非常难以计算,因此一般采取最坏运行时间[事实上平均运行时间往往和最坏运行时间量级相同]。最坏运行时间给定了运行时间的\textbf{上界},课程中主要讨论最坏,偶尔讨论平均。

\subsection{设计算法}

\textbf{分治}、\textbf{贪婪}、\textbf{动态规划}、线性规划、回溯、分支定界……

\

分治[divide-and-conquer]法:分解[Divide]、解决[Conquer]、合并[Combine]

举例:归并排序(分解成子序列,对子序列排序后合成)

\begin{code}
def MergeSort(A,p,r):

\ \ if(p < r)

\ \ \ \ q <- (p + r) / 2

\ \ \ \ MergeSort(A,p,q)

\ \ \ \ MergeSort(A,q+1,r)

\ \ \ \ Merge(A,p,q,r)

def Merge(A,p,q,r):

\ \ n1 = q - p + 1;

\ \ n2 = r - q

\ \ Let L[1..n1+1],R[1..n2+1] be new arrays

\ \ for i <- 1 to n1

\ \ \ \ L[i] <- A[p+i-1]

\ \ for j <- 1 to n2

\ \ \ \ R[j] <- A[q+j]

\ \ L[n1+1] <- R[n2+1] <- INFTY [监视哨]

\ \ i <- j <- 1

\ \ for k <- p to r

\ \ \ \ if L[i] <= R[j]

\ \ \ \ \ \ A[k] <- L[i]

\ \ \ \ \ \ i++

\ \ \ \ else

\ \ \ \ \ \ A[k] <- R[j]

\ \ \ \ \ \ j++
\end{code}

*\hspace{0em}采用无穷大作\textbf{监视哨}避免过多判断(注意:一定要保证充分大)

*Merge算法正确性:迭代时子数组A[p..k-1]按从小到大的顺序包含B[1..n1+1]与C[1..n2+1]中的k-p个最小元素

\

分析基于分治法的算法:递归式、\textbf{递归方程}

设T(n)是规模为n的运行时间(当n在某个常数之下时可当作常数)

假设分解为a个子问题,每个的规模是原本1/b,分解所需时间为D(n),合并所需时间为C(n),则总时间为:

$$T(n)=\begin{cases}\Theta(1)&n<c\\aT(n/b)+D(n)+C(n)&otherwise.\end{cases}$$

对归并排序:$T(n)=\begin{cases}\Theta(1)&n<c\\2T(n/2)+\Theta(n)&otherwise.\end{cases}$,事实上复杂度为$\Theta(n\log n)$

\

思考:

(1) 自底向上[buttom-up]通过两两归并也可实现归并排序

(2) 如何使数组A[0..n-1]循环左移k位?

法一:颠倒0到k-1、颠倒k到n-1、颠倒0到n-1\ (约3n次内容读写)

法二:思路:把置换拆分成轮换进行(约n次内容读写)

\begin{code}
d <- gcd(n,k)

for i <- 0 to d-1

\ \ x <- A[i]

\ \ t <- i

\ \ for j <- 1 to n/d-1

\ \ \ \ A[t] <- A[(t+k)%n]

\ \ \ \ t <- (t+k) mod n

\ \ A[t] <- x
\end{code}

\subsection{渐进记号与递归}

$f(n)=\Theta(g(n))$代表存在正常数$C_1,C_2,n_0$使得$n\ge n_0$时$0\le C_1g(n)\le f(n)\le C_2g(n)$。

*\hspace{0em}即趋于无穷时\textbf{阶相同},$g(n)=\Theta(f(n))$时亦有$f(n)=\Theta(g(n))$

*\hspace{0em}一般证明中可以取较粗糙的$C_1,C_2,n_0$,不用解出精确的点

$f(n)=O(g(n))$代表存在正常数$C,n_0$使得$n\ge n_0$时$0\le f(n)\le Cg(n)$。

*\hspace{0em}即趋于无穷时$f$的阶\textbf{不超过}$g$,$g(n)=\Theta(f(n))$时必有$g(n)=O(f(n))$

$f(n)=\Omega(g(n))$代表存在正常数$C,n_0$使得$n\ge n_0$时$0\le C g(n)\le f(n)$。

*\hspace{0em}即趋于无穷时$g$的阶\textbf{不超过}$f$,$g(n)=\Theta(f(n))$时必有$g(n)=\Omega(f(n))$

$f(n)=o(g(n))$代表任意正常数$c$存在正常数$n_0$使得$n\ge n_0$时$0\le f(n)< cg(n)$。

$f(n)=\omega(g(n))$代表任意正常数$c$存在正常数$n_0$使得$n\ge n_0$时$0\le cg(n) < f(n)$。

*\hspace{0em}大小写区别在存在与任意,也可以理解为大写剔除$\Theta$

*\hspace{0em}实际使用时,$O(f(n))$可以代表某个满足$g(n)=O(f(n))$的$g(n)$,这样的匿名[anonymous]函数可以正常参与运算,但不应出现存在歧义的情况

\

渐进关系的性质:

*\hspace{0em}五种关系都具有传递性

*\hspace{0em}$\Theta,O,\Omega$具有自反性

*\hspace{0em}$\Theta$与$\Theta$、$O$与$\Omega$、$o$与$\omega$互相置换对称

**\hspace{0em}不是任何两个函数都渐进可比,例如$n$与$n^{1+\sin n}$,不存在$O$或$\Omega$关系

\

\textbf{上/下取整}

取整性质:对正整数$a,b,n$,$\lceil\frac{\lceil n/a\rceil}{b}\rceil=\lceil\frac{n}{ab}\rceil,\lfloor\frac{\lfloor n/a\rfloor}{b}\rfloor=\lfloor\frac{n}{ab}\rfloor$

引理:$f(x)$是连续单调上升函数,且整点处才取整值,则$\lceil f(x)\rceil=\lceil f(\lceil x\rceil)\rceil,\lfloor f(x)\rfloor=\lfloor f(\lfloor x\rfloor)\rfloor$

引理证明:对第一个式子,若$\lceil f(x)\rceil\ne\lceil f(\lceil x\rceil)\rceil$,由单调增可知$f(x)\le f(\lceil x\rceil)$,从而$\lceil f(x)\rceil<\lceil f(\lceil x\rceil)\rceil$。由向上取整定义可知$\lceil f(x)\rceil<f(\lceil x\rceil)$,而$f(x)\le\lceil f(x)\rceil$,从而$\lceil f(x)\rceil$在$f(x)$与$f(\lceil x\rceil)$之间。由连续定义知存在$x_0$使得$f(x_0)=\lceil f(x)\rceil$,由条件知$x_0$必然为整数,但$x\le x_0\le\lceil x\rceil$,其由向上取整定义只能为$\lceil x\rceil$,与$\lceil f(x)\rceil<\lceil f(\lceil x\rceil)\rceil$矛盾。另一个式子同理。

定理证明:取$f(x)=\frac{x}{b}$,$x=\frac{n}{a}$即可。

\

其他数学:模、指数、\textbf{对数}、阶乘(Stirling公式)、函数迭代($f^{(n)}(x)$)

*\hspace{0em}斐波那契[Fibonacci]数$a_1=a_2=1,a_n=a_{n-1}+a_{n-2}$,通项为$\frac{1}{\sqrt5}\left[\left(\frac{1+\sqrt5}{2}\right)^n-\left(\frac{1-\sqrt5}{2}\right)^n\right]$

*\hspace{0em}思考:服务器每隔g秒送包,拥有k个端口(同时发k个),收到的服务器需要花L秒时间解压,自此每隔g秒给新服务器发送信息,时间与接受的服务器总数关系?(起初1对大兔子,大兔子每个隔g个月可生k对小兔子,小兔子需要L个月成熟,求第n个月的兔子对数?)[广义斐波那契数]

\textbf{递归介绍}

*\hspace{0em}递归可能分解为规模不等的子问题

求解递归方程的方法:替代法(猜测上界后证明)、递归树法(转化成树,结点表示不同层次产生的代价,再采用边界求和)、主方法(求解$T(n)=aT(n/b)+f(n)$,$a\ge1,b>1$,$f(n)$是某给定函数(并非对任何都可解))

**\hspace{0em}技术细节:
\begin{enumerate}
    \item 假定自变量整数,忽略上下取整。
    \item 对足够小的$n$假设$T(n)$为常数,忽略边界。
        
        (一些特殊情况可能导致技术细节非常重要,上面两种条件仍然值得重视)

    \item 有时会存在不等式情况,如$T(n)\le2T(n/2)+O(n)$,此时一般用$O$描述上界,反之对大于等于可用$\Omega$描述下界。
\end{enumerate}

\

例:最大子数组问题(给定数组,求和最大的连续子数组)

分治策略:找到数组中央位置,任何连续子数组必然在其左侧、右侧,或包含它。左侧与右侧可直接通过递归,由此只需要找到包含中间位置的最大子数组后取最大值即可。

\begin{code}
def find\_max\_crossing\_subarray(A,low,mid,high):

\ \ left\_sum = -INFTY

\ \ sum = 0

\ \ for i = mid downto low

\ \ \ \ sum += A[i]

\ \ \ \ if (sum > left\_sum)

\ \ \ \ \ \ left\_sum = sum

\ \ \ \ \ \ max\_left = i

\ \ right\_sum = -INFTY

\ \ sum = 0

\ \ for i = mid -> low

\ \ \ \ sum += A[i]

\ \ \ \ if (sum > right\_sum)

\ \ \ \ \ \ right\_sum = sum

\ \ \ \ \ \ max\_right = i

\ \ return (max\_left, max\_right, left\_sum+right\_sum)

\end{code}

整体算法:利用上方的算法进行递归,low=high即为终止条件。

复杂度分析:包含中间位置的部分的复杂度为$\Theta(n)$,递归方程为$T(n)=\begin{cases}\Theta(1)&n=1\\2T(n/2)+\Theta(n)&otherwise.\end{cases}$,可发现复杂度与归并排序相同,为$\Theta(n\log n)$。

\

*\hspace{0em}算法改进($\Theta(n)$算法):从左侧开始,找到第一个大于0的位置i1开始,依次求和(sum+=A[j]),max1记录目前的最大值,并记录当前的j1。直到sum<0时中止,然后继续向右找到下一个大于0的位置i2,清空sum,重复此过程,在比较中得到maxk中的最大值即可(证明思路:反证,若否则可以拼接为更大)。

\begin{code}
def find\_max\_subarray(A, low, high):

\ \ now <- low

\ \ sum <- 0

\ \ max <- -INFTY

\ \ for now <- low to high

\ \ \ \ if (sum > 0)

\ \ \ \ \ \ sum += A[now]

\ \ \ \ \ \ if (sum > max\_now)

\ \ \ \ \ \ \ \ j\_now <- now

\ \ \ \ \ \ \ \ max\_now <- sum

\ \ \ \ \ \ if (sum <= 0 or now == high)

\ \ \ \ \ \ \ \ if (max\_now > max)

\ \ \ \ \ \ \ \ \ \ i\_max <- i\_now

\ \ \ \ \ \ \ \ \ \ j\_max <- j\_now

\ \ \ \ \ \ \ \ \ \ max <- max\_now

\ \ \ \ \ \ \ \ sum <- 0

\ \ \ \ else if (A[now] > 0)

\ \ \ \ \ \ max\_now <- sum <- A[now]

\ \ \ \ \ \ i\_now <- j\_now <- i

\ \ return (max,i\_max,j\_max)

\end{code}


\subsection{判断方法}

\textbf{替代法}

包含两个步骤:猜测解的形式、归纳常数(直接替代)并证明解正确(要求易于猜得)

例:针对$T(n)=2T(\lfloor n/2\rfloor)+n$,先猜测解为$T(n)=O(n\log n)$,选取常数$C>T(2)+1$即可。

更复杂的例子:求$T(n)=2T(\lfloor n/2\rfloor + 17)+n$的解的上界。

解法(平移的思路):令$n=m+34$,可发现$T(m+34)=2T(\lfloor m/2\rfloor+34)+m+34$,由此$T(m+34)+34=2(T(\lfloor m/2\rfloor+34)+34)+m$类似上一种情况可直接估算出上界。

\

*\hspace{0em}猜测出渐近界未必能归纳成功,有时是因为归纳假设偏弱,可以尝试加强假设、调整低阶项、初等变换等,如$T(n)=2T(\lfloor n/2\rfloor)+1$,归纳假设$cn$无法继续,假设为$cn-2$即可。

*\hspace{0em}不应将猜测无理由放大

*\hspace{0em}变量代换:对$T(n)=2T(\lfloor\sqrt{n}\rfloor)+\log n$,取$m=\log n$可发现最终结果为$O(\log n\log\log n)$。

**\hspace{0em}弊端:猜测、证明都可能困难

\

\hspace{0em}\textbf{递归树}

每个结点代表相应子问题代价,行和代表某层代价,总和得总代价

*\hspace{0em}一般用来获得猜测解再替代,省略大部分细节。也可细化直接解出结果。

例:$T(n)=3T(\lfloor n/4\rfloor)+\Theta(n^2)\Longrightarrow T(n)=3T(n/4)+cn^2$简化,接着画出递归树求得复杂度大约为$\displaystyle\sum_{i=0}^\infty \frac{3^i}{16^i}cn^2=c'n^2$,因此为$\Theta(n^2)$量级。

代价不相同:$T(n)=T(n/3)+T(2n/3)+O(n)$,作出递归树可发现每层的和为$cn$,再由行数(通过解方程可计算出层数具体值,不过估算中没有精确必要)为$O(\log n)$可知复杂度$O(n\log n)$。同理,对于$T(n)=T(n/4)+T(n/2)+n^2$,可类似算得结果为等比级数的$n^2$倍,仍为$n^2$量级。

*\hspace{0em}关键为求行和与总代价,需要上下行和的关联

\

\textbf{主方法}

主要适用范围:$T(n)=aT(n/b)+f(n)$,其中$a\ge1,b>1$,$f$渐近非负。

主定理[The master theorem]:若$f(n)=O(n^{\log_ba-\varepsilon}),\varepsilon>0$,则$T(n)=\Theta(n^{\log_ba})$;若$f(n)=O(n^{\log_ba}\log^kn)$,则$T(n)=\Theta(n^{\log_ba}\log^{k+1}n)$;若$f(n)=\Omega(n^{\log_ba-\varepsilon}),\varepsilon>0$,且$af(n/b)\le cf(n)$,则$T(n)=\Theta(f(n))$。

*\hspace{0em}三种情况存在间隙,可能无法判断(如$T(n)=3T(n/3)+\log\log^2n$),并不代表递归方程无解

\subsection{摊还分析}
思路:考虑\textbf{一系列操作}的可能的\textbf{最坏情况}的\textbf{平均}复杂度为

*\hspace{0em}不涉及概率问题,与平均复杂度分析不同

\

例:栈操作

只有PUSH和POP时,视二者代价为1,则操作的平均代价必然为1。

在PUSH和POP上增添一个MULTIPOP(k),一次允许弹出多个元素(最多到栈底),其代价实际上为min(s,k),其中s代表栈中元素。

*\hspace{0em}加入此操作后,n次操作的平均代价?

虽然MULTIPOP的上限代价为$O(n)$,但由于最多入栈n次,出栈也最多n次,实际上平均代价仍为$O(1)$。

\

例:二进制计数器

k位二进制计数器,将每位的修改作为代价1。在允许加法时,一次加法最高需要修改k位,但是n次的总代价可以估算为$\frac{n}{2}+2\frac{n}{4}+3\frac{n}{8}+\dots=O(n)$,于是平均代价仍为$O(1)$。

\

\textbf{核算法}[accounting method]

给每个操作赋予\textbf{摊还代价},保证每次操作摊还代价的总和不小于实际代价(而当某步摊还代价比实际代价大时,将其称为\textbf{信用},用来支付之后的差额)。

*\hspace{0em}栈操作的例子中,PUSH的摊还代价为2,其余全为0;对二进制计数器,假设0到1操作[置位]的摊还代价为2,1到0操作[复位]的摊还代价为0,由于每次增加恰好产生一次置位,而复位时每个为1的位都保留信用支付,因此可得到结论。

\

\textbf{势能法}[potential method]

对每个状态定义\textbf{势函数}[映射到实数],一般初始状态为0。将摊还代价定义为\textbf{代价+操作后的势-操作前的势}。

*\hspace{0em}栈操作的例子中,势为栈中元素个数;对二进制计数器,势为数中1的个数。可发现得到的摊还代价与核算法中一致。

*\hspace{0em}通过势能法分析,当二进制计数器计数器不是从零开始时,这样的定义仍然合理,于是n个操作的实际代价为$2n+b_n-b_0$,$b$为势函数。于是,充分大时平均复杂度仍然$O(1)$。

\section{排序与顺序统计}
\textbf{一些概念}

稳定性:不论输入数据如何分布,关键字相同的数据对象(如两个3)在整个过程中保持相对位置不变

内排序/外排序:是/否在内存中进行

时间开销:通过比较次数与移动次数衡量

评价标准:所需时间[主要因素]、附加空间[一般都不大,矛盾不突出]、实现复杂程度

\subsection{简单排序与希尔排序}
直接插入、简单选择、冒泡

希尔排序:调用直接插入,不稳定,复杂度更低。

\

*\hspace{0em}直接插入见第一章

简单选择:n-1遍处理,第i遍将a[i..n]中的最小元与a[i]交换。比较次数$O(n^2)$,移动次数最多$3(n-1)$,平均时间复杂度$O(n^2)$。需要常数额外空间,就地排序。存在跨格跳跃,分析可发现不稳定。

冒泡:至多n-1遍处理,第i遍在a[i..n]中依次比较,相邻交换,某次发现没有需要交换时结束。比较次数$O(n^2)$,移动次数亦为最多$O(n^2)$,平均时间复杂度$O(n^2)$。稳定就地排序。

希尔[shell]排序:又称缩小增量排序,看作有增量的子列进行插入排序,对位置间隔较大的结点进行比较以跨过较大距离。性能与增量序列有关,尽量避免序列中的值互为倍数,优于直接插入。

*\hspace{0em}最后一趟必须以1作为增量以保证正确

*\hspace{0em}由于几趟后接近分块有序,希尔排序的实际效率大大优于直接插入排序。复杂度分析非常困难,统计结论一般时间复杂度在$O(n^{1.25})$左右,而空间效率也很高。虽然如此,其理论最有复杂度亦无法达到$O(n\log n)$。

\begin{code}
def ShellPass(A,d):

\ \ for i <- d+1 to n

\ \ \ \ if (A[i]<A[i-d])

\ \ \ \ \ \ A[0] <- A[i]; j <- i-d;

\ \ \ \ \ \ while (j>0 and A[0] < A[j])

\ \ \ \ \ \ \ \ A[j+d] <- A[j]

\ \ \ \ \ \ \ \ j <- j-d

\ \ \ \ \ \ A[j+d] <- A[0]

def ShellSort(A,D):

\ \ for i <- 1 to length(D)

\ \ \ \ ShellPass(A,D[i])
\end{code}

\subsection{堆排序}

堆[heap]排序:集中了归并排序与插入排序的优点,时间复杂度$O(n\log n)$,空间复杂度$\Theta(1)$。使用堆[heap]数据结构,不止可以用在堆排序中,还可以构造另一种有效的数据结构:优先队列。

**\hspace{0em}和内存的“堆”并不是同一个东西

(二叉)堆定义:树上每个结点对应数组中一个元素的完全二叉树,分为大根堆[父结点大于等于其任何子结点]与小根堆[父结点小于等于其任何子结点],排序算法中使用大根堆。

*\hspace{0em}表示堆的数组A有两个属性:数组元素个数[length]与数组中属于堆的元素的个数[heapsize],heapsize<=length,数组的第一个元素代表根结点。

下标1开始的数组中,由于其为完全二叉树,可各层顺序排列,A[i]的父结点是[i/2],左孩子2i,右孩子2i+1。

*[标准定义]满[full]二叉树:每层均为满;完全[complete]二叉树:最后一层的结点都在左侧,未必满;严格[strict]二叉树:每个结点的孩子都为零个或两个

*0开始:A[i]的父结点是[(i-1)/2],左孩子2i+1,右孩子2i+2。

结点高度:该结点到叶结点最长简单路径上边的数目(堆的高度:根结点的高度,为$\lceil\log_2(n+1)\rceil$)。

\

\textbf{基本操作}

维护堆:假定A[i]的左右子树都为大根堆,但A[i]有可能小于其孩子,则通过逐级下降使得下标为i的根结点子树重新遵循大根堆。

\begin{code}
def MAX\_HEAPIFY(A, i):

\ \ l = LEFT(i)

\ \ r = RIGHT(i)

\ \ if l <= A.heap\_size and A[l] > A[i]

\ \ \ \ Largest = l

\ \ else

\ \ \ \ Largest = i

\ \ if r <= A.heap\_size and A[r] > A[Largest]

\ \ \ \ Largest = r

\ \ if (Largest != i)

\ \ \ \ swap A[i], A[Largest]

\ \ \ \ MAX\_HEAPIFY(A, Largest)
\end{code}

*\hspace{0em}每次至少下移一层,时间复杂度$O(\log n)$。

\

建堆:从后往前进行转化,一半处开始即可。

\begin{code}
def BUILD\_MAX\_HEAP(A):

\ \ A.Heap\_size <- A.length

\ \ for i <- [A.length/2] downto 1

\ \ \ \ MAX\_HEAPIFY(A, i)
\end{code}

*\hspace{0em}求和估算可知时间复杂度$O(n)$。

\

堆排序:建好堆后,每次将根结点[当前最大]与最后一个元素互换,堆长度减小1,然后调整堆。

\begin{code}
def HEAPSORT(A):

\ \ BUILD\_MAX\_HEAP(A);

\ \ for i <- length(A) downto 2

\ \ \ \ swap A[1], A[i]

\ \ \ \ A.Heap\_size -= 1

\ \ \ \ MAX\_HEAPIFY(A, 1)
\end{code}

\

二叉堆的扩展:\textbf{优先队列}\ [\textbf{并不是堆的基本操作!}]

*\hspace{0em}用来维护一组元素构成的集合S的数据结构,每个元素都有一个相关值,称\textbf{关键字}[key]

最大优先队列[可用大根堆构造]基本操作:

\begin{enumerate}
    \item insert(S,x) 插入元素进S
    \item maximum(S) 返回S中最大关键字元素
    \item extract\_max(S) 去掉并返回S中最大关键字元素
    \item increase\_key(S, x, k) 将x的关键字增加到k\ [只允许增加]
\end{enumerate}

*\hspace{0em}最小优先队列类似相反

*\hspace{0em}大根堆实现时,最大关键字元素即为第一个元素[复杂度$O(1)$];删除最大值直接将最大值与最后一个元素交换,堆长度减小1并调整即可[复杂度$O(\log n)$];增大某元素值通过不断和父结点比较交换实现[复杂度$O(\log n)$];加入结点堆长度增加1,先将末尾元素赋值为$-\infty$再增大值到目标即可[复杂度$O(\log n)$]。

\subsection{快速排序}

对于包含$n$的数的输入数组,时间复杂度最坏情况$O(n^2)$,不稳定、就地,期望时间$\Theta(n\log n)$且常数因子很小。

*\hspace{0em}分治思想

划分:划分为左右两个子数组与中间,使得左侧均小于等于中间,右侧均大于等于中间,中间下标q在划分中确定。

解决:递归调用,两边分别排序。

合并:由于子数组有序,合并直接已经排好。

\begin{code}
def QUICKSORT(A, p, r):

\ \ if (p < r)

\ \ \ \ q = Partition(A, p, r)

\ \ \ \ QUICKSORT(A, p, q-1)

\ \ \ \ QUICKSORT(A, q+1, r)

def PARTITION(A, p, r):

\ \ x <- A[r]

\ \ i <- p - 1

\ \ for j <- p to r-1

\ \ \ \ if A[j] <= x

\ \ \ \ \ \ i <- i + 1

\ \ \ \ \ \ swap A[i], A[j]

\ \ swap A[i+1], A[r]

\ \ return i + 1
\end{code}

另一种PARTITION:

\begin{code}
def PARTITION(A, p, r):

\ \ i <- p; j <- r; temp <- A[i];

\ \ while (i != j)

\ \ \ \ while (A[j] >= temp and i < j)

\ \ \ \ \ \ j <- j - 1

\ \ \ \ if (i < j)

\ \ \ \ \ \ A[i] <- A[j]

\ \ \ \ \ \ i <- i + 1

\ \ \ \ while (A[i] <= temp and i < j)

\ \ \ \ \ \ i <- i + 1

\ \ \ \ if (i < j)

\ \ \ \ \ \ A[j] <- A[i]

\ \ \ \ \ \ j <- j - 1

\ \ A[i] <- temp

\ \ return i
\end{code}

*\hspace{0em}看似双方向进行,但由于\textbf{必须检查越界},需要额外比较

\

复杂度:分解均匀时接近归并,不均匀时最坏情况,为$O(n^2)$

*\hspace{0em}解递推式$T(n)=\max_q(T(q)+T(n-1-q))+Cn$可知最坏情况上界,而每次最不均匀划分能取到$cn^2$时间,从而可知最坏情况复杂度

平均情况:可以发现,对任何不超过固定比例(如每次少比多不超过1:9)的划分,复杂度都是$O(n\log n)$,利用此估算,设平均复杂度$T(n)$,有$T(n)=\frac{1}{n}\sum_{k=0}^{n-1}(T(k)+T(n-1-k)+cn)=\frac{2}{n}\sum_{k=0}^{n-1}T(k)+cn$。

考虑$nT(n)-(n-1)T(n-1)$可估算出$\frac{T(n)}{n+1}\le\frac{T(n-1)}{n}+\frac{2c}{n}$,利用$1+\dots+\frac{1}{n}\sim \log n$可知平均复杂度。

\

随机快速排序:每次PARTITION并非将最右元素作为分割,而是\textbf{选取随机元素}分割以优化性能。

\

**\hspace{0em}堆排序与快速排序比较:按照通常RAM模型会发现堆排序的复杂度更低,而引进一级缓存(缓存的量级在$n^{1/3}$左右)时即会有快速排序的复杂度更低。

\subsection{线性时间排序}
\textbf{比较排序的时间下界}

比较排序的算法可以用\textbf{决策树}的方式表示,边代表判定过程,而结点代表已经确定顺序的部分。

*\hspace{0em}决策树中至少需要$n!$个叶结点表示$n!$种判定结果

定理:比较排序的最坏情况时间$\Omega(n\log n)$。

证明:由于一次比较产生一层,比较$h$次的高度为$h$,而此时最多容纳$2^h$个叶结点,因此$2^h\ge n!$,从而$h\ge\log(n!)=\Omega(n\log n)$。

*\hspace{0em}线性时间排序一定不能直接基于比较

\

\textbf{计数[counting]排序}

思路:对0到k中的整数组成的数列,只要知道每个整数出现了几次就会知道结果所在的位置。

\begin{code}
def COUNTING\_SORT(A, B, k):

\ \ for i <- 0 to k

\ \ \ \ C[i] <- 0

\ \ for i <- 1 to length(A)

\ \ \ \ C[A[j]] <- C[A[j]] + 1   //C[t] = sum(A==t)

\ \ for i <- 1 to k

\ \ \ \ C[i] <- C[i] + C[i-1]   //C[t] = sum(A<=t)

\ \ for i <- length(A) downto 1

\ \ \ \ B[C[A[j]]] <- A[j]

\ \ \ \ C[A[j]] <- C[A[j]] - 1
\end{code}

*\hspace{0em}考虑四个循环可发现其复杂度为$\Theta(n+k)$,且为稳定排序。

\

\textbf{基数[radix]排序}

思路:对整数,按位数从末位向前排序(对每位采用稳定排序算法,如计数排序)。

时间复杂度:n个d位数,每位k种取值,计数排序时耗时$O(d(n+k))$

*\hspace{0em}由上方可知,给定n个b位二进制数与正整数$r\le b$,时间复杂度可以控制在$\Theta(b(n+2^r)/r)$内。

\

\textbf{桶[bucket]排序}

思路:对0到1之间,先划分所有数据到n个不同的“桶”内再进行排序。

\begin{code}
def BUCKET\_SORT(A):

\ \ n <- length(A)

\ \ for i <- 1 to n

\ \ \ \ insert A[i] into list B[floor(nA[i])]

\ \ for i <- 0 to n-1

\ \ \ \ sort list B[i] with insertion sort

\ \ Print B[i] in order
\end{code}

时间复杂度:针对\textbf{均匀一致分布}才能达到较好效果,最坏情况$O(n^2)$。平均性能较好,为$\Theta(n)$。

\subsection{中位数与顺序统计}
定义:第i小的元素称为第i个顺序统计量,n个元素的低中位数为第$\lfloor\frac{n+1}{2}\rfloor$个顺序统计量,高中位数为第$\lceil\frac{n+1}{2}\rceil$个,一般默认为低中位数。

*\hspace{0em}寻找最大或最小值时间复杂度必然为$\Theta(n)$,但同时找最大最小值问题至少可通过$\lceil\frac{3n}{2}\rceil-2$次比较完成。

证明:记N状态为从未参与过比较,L状态为参与过但只大不小,S为参与过但只小不大,M为参与过且小大都成为过,则状态间的转换关系为N->L/S->M。所有元素只有一个能保证L/S不变,即为最大/最小,剩下的都会成为M,总状态转换数为$2(n-2)+1+1=2n-2$。

下面考虑一次比较能造成的状态转换:只有当N与N比较时一定会有两次状态转换,其余在最坏情况下至多一次状态转换[严谨性问题:整体最坏未必每次都能遇到最坏]。由此最坏情况至少$\frac{n}{2}+\big(2n-2-(2\frac{n}{2})\big)=\frac{3n}{2}-2$,而由于比较次数一定为整数,需要取上整,即为结果。

算法:先每两位进行比较,将小的放在奇数位置。在偶数位中找到最大值,奇数位中找到最小值即可。

思考:找第二大元素需要的最少比较次数?

\

寻找任意第i小元素:类似快速排序,通过PARTITION后左右元素数量确定应在哪一侧找。

\begin{code}
def RANDOMIZED\_SELECT(A, p, r, i):

\ \ if p == r:

\ \ \ \ return A[p]

\ \ q = RANDOMIZED\_PARTITION(A, p, r)

\ \ k = q - p + 1

\ \ if i == k:

\ \ \ \ return A[q]

\ \ if i < k:

\ \ \ \ return RANDOMIZED\_SELECT(A, p, q-1, i)

\ \ else:

\ \ \ \ return RANDOMIZED\_SELECT(A, q+1, r, i-k)
\end{code}

*\hspace{0em}采用随机PARTITION增加效率

复杂度:最坏情况是$\Theta(n^2)$,平均性能为$\Theta(n)$\ (假设均匀一致分布,可通过随机变量计算得结果)。

*\hspace{0em}最坏情况$O(n)$的算法:针对划分不均匀情况改进,使划分均匀。

想法:将数组每五个分组,插入排序找到中值,再从排好序的组列表中提出中值。递归调用,利用中值的中值作PARTITION。

证明:3.3节提到,PARTITION对任何\textbf{不超过固定比例}的划分都是较好的,而估算可知这样的取中值方式可以保证两侧的比例不超过3:7。计算时间复杂度:
$$T(n)\le T(n/5)+an+\max\{T(left),T(right)\}\le an+T(n/5)+T(7n/10)$$
猜测可解出$T(n)\le10an$,从而为$O(n)$。

\section{算法设计基本策略}
\subsection{动态规划}
Richard Bellman, 1950s: 最优性原理、动态规划[dynamic programming]

与分治法相似,但保存已求解的子问题,\textbf{不需要重复求解}。

最优化问题:寻找符合\textbf{约束条件}时\textbf{优化函数}的最值

*\hspace{0em}可行解(满足约束条件的解)、最优解(获得最佳值的可行解)

例子(Thirsty baby): 有n种不同饮料,每种最多有$a_i$,满意度为$x_i$,总共需要t,求满意度最高的饮用方法。

*\hspace{0em}评价函数为$\sum_{i=1}^nx_is_i$,$x_i$总和为$t$且范围为$[0,a_i]$。

\textbf{最优性原理}[过程的最优决策序列的性质]:无论过程的初始状态和初始决策是什么,其余的决策必须相对初始决策产生的状态构成最优决策序列(也即从任何一步开始看都是最优的)。

*\hspace{0em}全局最优具有一定的局部最优性

刻画最优解的结构特征->递归定义最优解值->自底向上计算最优解值->通过计算信息构造最优解

\

能用动态规划求解的条件:\textbf{最优子结构}[optimal substructure]、\textbf{子问题覆盖}

*\hspace{0em}这要求了全局最优存在某种局部最优,且局部最优可以导出全局最优

运行时间估计:子问题个数乘子问题最多需要考察的选择数得到上界

常用方法:自底向上,先计算子问题再计算原问题[或带记忆的递归方法]

**\hspace{0em}不满足最优性原理的例子:对有向图,寻找两点间的最短路径满足最优性原理,但最长简单[无环]路径不满足最优性原理。

思考:如何求解最长简单路径问题?

子问题覆盖:会涉及重复求解子问题,于是可以存储、利用

*\hspace{0em}利用表格存储辅助信息,从而递归\textbf{构造}最优解

\

例:\textbf{钢条切割}

已知长度为i的钢条价格$p_i$,长度均为整数,总长固定,求最大收益的分割方案。

思路:每次只需要考虑第一次切割(或不切割),剩下的部分可以由更低长度时的最优值[最优子结构]确定

定义$r_n$为长为n的钢条的最优切割方案,递推为$r_n=\max_{0\le i\le n/2}(r_i+r_{n-i})$。

*\hspace{0em}事实上可以将$i$的范围提到$n$,$r_i$写为$p_i$,因为长度已经确定

*\hspace{0em}自顶向下的递归写法会导致大量重复计算,因此需要\textbf{自底向上}

[递归若检测是否已经计算过子问题,仍可做到多项式复杂度]

\begin{code}

def BOTTOM\_UP\_CUT\_ROD(p, n):

\ \ let r[1..n] be array

\ \ r[0] = 0

\ \ for j = 1 to n

\ \ \ \ q = - INFTY

\ \ \ \ for i = 1 to j

\ \ \ \ \ \ q= max(q, p[i] + r[j-i])

\ \ \ \ r[j] = q

\ \ return r[n]
\end{code}

获得最优解的方法:将每个r[i]第一次切割的位置保存为c[i],反复回看以确定最优解

打印方式:

\begin{code}
while n > 0:

\ \ print(c[n])

\ \ n -= c[n]
\end{code}

*\hspace{0em}时间复杂度:$\Theta(n^2)$

\

\textbf{矩阵链乘}

*\hspace{0em}直接计算的方法:A(1:p,1:q)与B(1:q,1:r)相乘的复杂度是pqr

给定n个矩阵组成的序列$A_1,\dots,A_n$,且前一个的行数等于后一个的列数,要计算它们的乘积,求计算速度最快的运算顺序(由结合律可任意加括号)。

*\hspace{0em}总运算顺序可能:通过递推可知为\textbf{卡特兰数}[Catalan Number]

自顶向下分析:在最优顺序中,考虑最后一次运算,划分出的左右两块应各自为最优运算顺序。

于是,记$m_{ij}$为$A_i\dots A_j$所需的最小乘法次数,$p_i$为$A_i$的列数,$p_0$为$A_1$的行数,则有$m_{ij}=\begin{cases}0&i=j\\\min_k\{m_{ik}+m_{k+1,j}+p_{i-1}p_kp_j\}&i<j\end{cases}$

自底向上的方式:按照j-i的大小逐步向上生成$m_{ij}$,需要三重循环(j-i、j、比较),并用额外数组s记录分割点。

\begin{code}
def Matrix\_Chain\_Order(p):

\ \ for i = 1 to n

\ \ \ \ m[i][i] = 0

\ \ for l = 1 to n-1

\ \ \ \ for i = 1 to n-l

\ \ \ \ \ \ j = i + l

\ \ \ \ \ \ m[i][j] = INFTY

\ \ \ \ \ \ for k = i to j-1

\ \ \ \ \ \ \ \ q = m[i][k] + m[k+1][j] +p[i-1]*p[k]*p[j]

\ \ \ \ \ \ \ \ if q < m[i][j]

\ \ \ \ \ \ \ \ \ \ m[i][j] = q

\ \ \ \ \ \ \ \ \ \ s[i][j] = k

return (m, s)
\end{code}

*\hspace{0em}时间复杂度:$O(n^3)$

计算最优解:

\begin{code}
PRINT\_RESULT(s, i, j):

\ \ if i == j:

\ \ \ \ print('A'i)

\ \ else:

\ \ \ \ print('(')

\ \ \ \ PRINT\_RESULT(s, i, s[i][j])

\ \ \ \ PRINT\_RESULT(s,s[i][j]+1,j)

\ \ \ \ print(')')
\end{code}

思考:如果要求顺序打出,如何打印?

\subsection{更多例子}
\textbf{最长公共子序列}

子序列定义:从左到右取出的一列\textbf{未必连续}的元素。

问题:已知两序列X、Y,求最长公共子序列Z。

\

1、最优子结构特性:

*\hspace{0em}记X(i)为X前i个元素构成的子串,<A,B>为A与B的最长公共子序列,X为X[1..i],Y为Y[1..j],Z为Z[1..k]

当X[i]==Y[j]时,Z必为<X(i-1),Y(j-1)>加上X[i]。

否则,当Z[k]!=X[i]时,Z为<X(i-1),Y(j)>;当Z[k]!=Y[j]时,Z为<X(i),Y(j-1)>。

于是,为进行求解,需要对每个a、b考虑<X(a),Y(b)>,记其长度为$e(a,b)$。

\

2、构造递推解法:

$e(a,b)=\begin{cases}e(i-1,j-1)+1&x_a=y_b\\\max\{e(a-1,b),e(a,b-1)\}&x_a\ne y_b\end{cases}$

边界条件:$ab=0$时$e(a,b)=0$。

\

3、自底向上构造:

\begin{code}
def LCS\_LENGTH(X, Y):

\ \ m = length(X)

\ \ n = length(Y)

\ \ c = zeros(m+1,n+1)

\ \ for i = 1 to m, j = 1 to n:

\ \ \ \ if X[i] == Y[j]:

\ \ \ \ \ \ c[i][j] = c[i-1][j-1] + 1

\ \ \ \ \ \ b[i][j] = "upleft"

\ \ \ \ elif c[i-1][j] > c[i][j-1]:

\ \ \ \ \ \ c[i][j] = c[i-1][j]

\ \ \ \ \ \ b[i][j] = "up"

\ \ \ \ else:

\ \ \ \ \ \ c[i][j] = c[i][j-1]

\ \ \ \ \ \ b[i][j] = "left"

\ \ return b, c
\end{code}

\

4、递归寻找最优解:

最优解的值为c右下角元素。从右下角开始按照b中指示行动,直到到达点的c为0,每次向左上走的对应的X或Y下标构成最优解[注意得到的序列需要进行reverse]。

\

\textbf{最优二分检索树}

假设有n个互不相同的关键字k[1..n]从小到大排列,此外要查找的关键字在k[i]与k[i+1]之间(查找不成功)的情况记作d[i],可能出现d[0..n]。

假定p[1..n]为查找k[1..n]的概率,q[0..n]为结果为d[0..n]的概率,使得\textbf{期望查找次数}[即每个结果所在的深度加一与概率相乘求和]最小的二分检索树称为\textbf{最优二分检索树}。

**\hspace{0em}事实上查找失败时所需要的比较次数即为深度,只有查找成功时才需要加一,教材为方便而统一为深度加一

将子问题定义为K(i,j),代表对b[i..j+1]与k[i..j]构造的二分检索树。假定某树是最优二分检索树,其左子树与右子树一定是对应子问题的最优二分检索树[类似矩阵链乘,运算顺序本就相似于构造二叉树]。

于是,计算可发现代价$m[i,j]=\min_r\{m[i,r-1]+m[r+1,j]+\sum_{s=i}^jp_s+\sum_{s=i-1}^jq_s\}$,且边界条件$m[i,i-1]=q_{i-1}$。

*\hspace{0em}可先用$w(i,j)$保存$\sum_{s=i}^jp_s+\sum_{s=i-1}^jq_s$,从而不用每次都计算

\begin{code}
def Optical\_BST(p, q, n):

\ \ for i = 1 to n+1

\ \ \ \ e[i][j-1]=q[i-1]

\ \ \ \ w[i][j-1]=q[i-1]

\ \ for l = 0 to n-1, i = 1 to n-l

\ \ \ \ j = i + l

\ \ \ \ e[i][j] = INFTY

\ \ \ \ w[i][j] = w[i][j-1] + p[j] + q[j]

\ \ \ \ for r = i to j

\ \ \ \ \ \ t = e[i][r-1] + e[r+1][j] + w[i][j]

\ \ \ \ \ \ if t < e[i][j]

\ \ \ \ \ \ \ \ e[i][j] = t

\ \ \ \ \ \ \ \ root[i][j] = r

return e, root
\end{code}

*\hspace{0em}复杂度$\Theta(n^3)$

*\hspace{0em}事实上可以做到$n^2$复杂度:将第三重循环变为root[i][j-1]到root[i+1][j],利用抵消可以发现总和复杂度为$\Theta(n^2)$,合理性需要严格的数学证明

\subsection{贪心算法}

基本思路:每一步按照一定准则做出看上去最优的决策,不会再行修改

贪心准则:做出贪心决策的依据

实现过程:从初始解出发,每一步求出可行解的\textbf{一个解元素}。由所有元素\textbf{组合}成可行解

**\hspace{0em}贪心算法能求解时动态规划必然能求解,但贪心算法效率更高

\

例:\textbf{活动安排问题}

给定S由n个活动a[1..n]构成,每个有开始时间s[i]与结束时间f[i],开始时间小于结束时间,求尽量多的不会冲突的活动选择。

*\hspace{0em}不妨设按结束时间f[i]从小到大排序,增添开始结束都在0时间的与无穷时间的活动,以使下标范围[0..n+1]。记$S_{ij}$为所有在第i个结束后开始,第j个开始前结束的活动,容易发现只在i<j时可能不为空。

利用动态规划的求解思路,假设$S_{ij}$中的最佳安排包含a[k],则其必然为$A_{ik},A_{kj}$与a[k]之并(由此,考虑$A_{ij}$的元素个数c[i][j],可通过尝试$S_{ij}$中的所有a[k]取最大值自底向上构造)。

贪心算法假设:只需要寻找使得$A_{ik}$为空的a[k]即可[取$S_{ij}$中下标最小的活动]。

证明思路:通过替换可说明其一定可以包含此元素。

递归写法:

\begin{code}
def RECURSIVE\_ACTIVITY\_SELECTOR(s, f, i, j)

\ \ m = i + 1

\ \ while m < j and s[m] < f[i]

\ \ \ \ m++

\ \ if m < j

\ \ \ \ return \{a[m]\} $\cup$ RECURSIVE\_ACTIVITY\_SELECTOR(s, f, m, j)

\ \ return $\varnothing$

\end{code}

*\hspace{0em}也即每次找到最早完成的活动,并去掉与之冲突的

非递归写法:

\begin{code}
def ACTIVITY\_SELECTOR(s, f):

\ \ n = length(s)

\ \ A = \{a[1]\}

\ \ i = 1

\ \ for m = 2 to n

\ \ \ \ if s[m] >= f[i]

\ \ \ \ \ \ A = A $\cup$ \{a[m]\}

\ \ \ \ \ \ i = m

\ \ return A
\end{code}

*\hspace{0em}复杂度为$\Theta(n)$,而动态规划方法的复杂度为$\Theta(n^2)$

\

\textbf{设计贪心算法}

*\hspace{0em}可先考虑递归问题的情况,找到其中正确的贪心选择后构造

\begin{enumerate}
\item 
将最优子问题简化为做出选择后只剩一个子问题需要求解的形式。

\item
证明贪心选择后原问题存在最优解(即选择\textbf{安全})。

\item
证明贪心选择后剩余子问题满足性质。
\end{enumerate}

于是,贪心算法需要最优化问题具有\textbf{最优子结构}与\textbf{贪心选择性质},后者即可以通过\textbf{局部}最优解构造出\textbf{全局}最优解。

\

\textbf{贪心算法与动态规划对比}

0/1背包问题:已知每件物品的重量与价值,背包有承重上限,求最大价值装法。

分数背包问题:条件同上,但每个物品可以拿取一部分。

*\hspace{0em}第二个问题可以直接每次选择单位重量价值最高的物品,于是具有贪心结构,但第一个问题如此可能导致最终剩余空间不是最优(类似钢条切割时),只能采用动态规划[其可以被拿取和不拿的子问题覆盖,于是动态规划是可行的]。

\

例:\textbf{哈夫曼编码}

要求:进行\textbf{变长}编码以缩减储存空间,但需要\textbf{前缀编码}(两个不同的编码具有不同前缀,于是可以区分)以避免歧义。已知每个字符出现的频率,求最优编码方式。

前缀码实现方式:构造\textbf{满}二叉树,需要编码的字符在叶结点,从根结点出发,向左表示0,向右表示1,靠到达每个叶结点的方式进行编码。

*\hspace{0em}于是问题转化为最优二叉树的构建

算法[\textbf{自底向上}]:将每个结点的权值记作其左右孩子的权值之和,叶结点的权值即为频率。从每个字符作为根结点出发,每次新建根结点,将权值最小的两棵树作为其左右子树,重复n-1次。

*\hspace{0em}利用\textbf{优先队列}进行存储,方便每次取出最小值

\begin{code}
def HUFFMAN(C):

\ \ n = length(C)

\ \ Q = C

\ \ for i = 1 to n-1

\ \ \ \ z = new node

\ \ \ \ left[z] = x = EXTRACT\_MIN(Q)

\ \ \ \ right[z] = y = EXTRACT\_MIN(Q)

\ \ \ \ f[z] = f[x] + f[y]

\ \ \ \ INSERT(Q, z)

\ \ return EXTRACT\_MIN(Q) \# return the root

\end{code}

*\hspace{0em}根据之前的优先队列分析可发现复杂度为$O(n\log n)$

\

\textbf{证明最优性}

引理:若x, y是出现最少的两个字符,则存在一种最优编码使得x, y对应的编码长度相同且只有最后一位不同。

证明:由于满二叉树性质,深度最深处必然会有一对叶结点,将x, y与这对结点替换后情况不会变差。

利用引理,每次将建立的根下的叶结点\textbf{合并}为一个字符即可得到结果。

\subsection{分治策略案例}

\textbf{硬币问题}

16个硬币,找到其中比其他轻的假币。

直接思路:分为八组比较得结果。

分治思路:每次分两组比较,四次得到结果。

\

\textbf{大整数乘法}

*假设位数为$n$

直接相乘:$\Theta(n^2)$次,按照每位相乘后相加。

分治:

将其分为前n/2与后n/2位[记$2^{n/2}=m$],但直接计算导致$T(n)=4T(n/2)+O(n)$,无法改进。

改进方式:设$X=am+b,Y=cm+d$,可发现$XY=acm^2-((a-b)(c-d)-ac-bd)m+bd$,于是如此计算可以达到$T(n)=3T(n/2)+O(n)$,复杂度$O(n^{\log 3})$。

*\hspace{0em}中间采用减法而非加法是为了避免溢出

**\hspace{0em}更快方法?利用\textbf{快速傅里叶变换}[FFT],$O(n\log n)$解决

\

\textbf{Strassen矩阵乘法}

*\hspace{0em}方阵相乘问题

分块矩阵可得$\begin{pmatrix}A_1&A_2\\A_3&A_4\end{pmatrix}\begin{pmatrix}B_1&B_2\\B_3&B_4\end{pmatrix}$与直接看为每位相乘的形式一致,但是直接计算会导致$T(n)=8T(n/2)+O(n^2)$,无法改进。

改进方式:\textbf{利用更好的组合策略}

考虑
$$D=A_1(B_2-B_4),E=A_4(B_3-B-1),F=(A_3+A_4)B_1,G=(A_1+A_2)B_4$$
$$H=(A_3-A_1)(B_1+B_2),I=(A_2-A_4)(B_3+B_4),J=(A_1+A_4)(B_1+B_4)$$

则
$$C_1=E+I+J-G,C_2=D+G,C_3=E+F,C_4=D+H+J-F$$

于是递推式变为$T(n)=T(n/2)+cn^2$,复杂度降至$n^{\log7}$。

**\hspace{0em}更快方法?利用其他划分策略,目前最好上界约$O(n^{2.376})$

\

\textbf{残缺棋盘问题}

$2^n\times2^n$差一块的棋盘,要求用3小块的L形木板覆盖,求覆盖策略。

分治思路:分为四个,考虑残缺方格所在的区域,其他三个趋于用一个小L形盖住中间使得剩下每区域恰好残缺一个。

复杂度位为$t(k)=4t(k-1)+c$,于是复杂度与格子数一致。由于这和L形板数目相同量级,这必然是\textbf{最优}的算法。

\

\textbf{距离最近点对}

平面中n个点,求最近点对。

直接法:挨个比较,复杂度$\Theta(n^2)$。

分治思路:分为两组,每组内比较并组间比较。

优化方式:优化组间比较。先按一个平行x轴的直线分割,假设两边最近点对较小距离为d,则组间比较只需要考虑到直线距离小于d的点。

*\hspace{0em}复杂度分析:由于最复杂的部分是对单坐标轴进行若干次排序,复杂度只需$O(n\log n)$。

\subsection{快速傅里叶变换}

考虑范围:代数域[可以考虑实数/复数]上\textbf{次数小于n}的多项式

相加:直接系数相加,$O(n)$。

相乘:直接计算的简单方法$\Theta(n^2)$(实际上是计算向量的\textbf{卷积})。

*\hspace{0em}如何达到$\Theta(n\log n)$?

\

多项式表示方式:\textbf{系数}表示法(直接表示)、\textbf{点值}表示法(n个不同点的取值确定)

点值表示好处:假设知道足够多点,多项式相乘只需要$\Theta(n)$次

\textbf{表示方法转化}

*\hspace{0em}点值到系数?

做法:计算多项式\textbf{插值}

(考虑\textbf{范德蒙德行列式}可以知解的唯一性,从而点值唯一确定系数)

利用拉格朗日插值公式可以$\Theta(n^2)$可以得到结果

系数到点值直接方法:$a_0+a_1(x+a_2(x+\dots))$计算不同点,复杂度$\Theta(n^2)$。

*\hspace{0em}FFT思路:选择合适的点让结果为$O(n\log n)$?

\

\textbf{对数复杂度方法}

数学基础:n次单位根$\omega_n^k=\mathrm{e}^{2\pi k\mathrm{i}/n}$,$k$取1时称为单位原根,其他为原根的方幂。所有n次单位根构成循环群,单位原根为其生成元之一。

*\hspace{0em}一些等式:$\omega_{dn}^{dk}=\omega_n^k,2\mid n\Rightarrow\omega_n^{n/2}=-1,\sum_{k=0}^{n-1}\omega_n^{dk}=\begin{cases}n&n\mid d\\0&n\nmid d\end{cases}$

*\hspace{0em}利用\textbf{单位根}作为系数到点值的特殊点

对多项式$\mathrm{a}(x)=a_0+\dots+a_rx^r$,可不妨设其次数在$n=2^t$以下,则令$y_i=f(\omega_n^i)$,$i=0,\dots,n-1$,称$\mathbf{y}$为$\mathbf{a}$的傅里叶变换,记作$\mathbf{y}=\DFT_n(\mathbf{a})$。

*\hspace{0em}计算DFT的复杂度是$\Theta(n\log n)$的:

记$a_0,a_2,\dots$构成$\mathbf{b}$,$a_1,a_3,\dots$构成$\mathbf{c}$,则$\mathbf{a}(x)=\mathbf{b}(x^2)+x\mathbf{c}(x^2)$。

$n$为偶数时$(\omega_n^k)^2=\omega_{n/2}^k$,于是通过分治可知DFT复杂度$T(n)=2T(n/2)+\Theta(n)$,由此为$\Theta(n\log n)$。

\begin{code}

def Recursive\_FFT(a):

\ \ n = length(a)

\ \ wn = exp(2PI*i/n)

\ \ w = 1

\ \ aeven = \{a[0],a[2],...,a[n-2]\}

\ \ aodd = \{a[1],a[3],...,a[n-1]\}

\ \ yeven = Recursive\_FFT(aeven)

\ \ yodd = Recursive\_FFT(aodd)

\ \ for k = 0 to n/2-1

\ \ \ \ y[k] = yeven[k] + w * yodd[k]

\ \ \ \ y[k+n/2] = yeven[k] + w * yodd[k]

\ \ \ \ w = w * wn

\ \ return y
\end{code}

*\hspace{0em}如何计算逆变换?

由于变换可以看作$y=V_na$,其中$(v_{ij})_n=\omega_n^{ij},i,j=0,\dots,n-1$,计算可得$V^{-1}_n$的$(i,j)$位置为$\frac{1}{n}\omega_{n}^{-ij}$

于是,$a_j=\frac{1}{n}\sum_{k=1}^{n-1}y_k\omega_n^{-kj}$,可以完全类似利用分治得到$\Theta(n\log n)$的解法。

*\hspace{0em}优化:利用\textbf{蝴蝶操作}[butterfly operation]计算公用子表达式,归并方式计算,从而减少计算次数

*\hspace{0em}下方算法中BIT\_REVERSE\_COPY为按二进制位逆序

\begin{code}
def ITERATIVE\_FFT(a):

\ \ BIT\_REVERSE\_COPY(a, A)

\ \ n = length(a)

\ \ for s = 1 to log n

\ \ \ \ m = 2**s

\ \ \ \ omegam = exp(2PI*i/m)

\ \ \ \ for k = 0 to n-1 by m

\ \ \ \ \ \ omega = 1

\ \ \ \ \ \ for j = 0 to m/2-1

\ \ \ \ \ \ \ \ t = omega * A[k+j+m/2]

\ \ \ \ \ \ \ \ u = A[k+j]

\ \ \ \ \ \ \ \ A[k+j] = u + t

\ \ \ \ \ \ \ \ A[k+j+m/2] = u - t

\ \ \ \ \ \ \ \ omega *= omegam

\ \ return A
\end{code}

\section{数据结构}
\subsection{二分检索树}
要求:左子树$\le$根$\le$右子树

检索算法:按照与根结点的比较向左/右走

最小/最大:根结点不断寻找左/右孩子,不存在时即为最小/最大,复杂度$O(h)$。

后继:分为有右孩子与无右孩子讨论,复杂度$O(h)$

\begin{code}
def TREE\_SUCCESSOR(x):

\ \ if x.right

\ \ \ \ return TREE\_MINIMUM(x.right)

\ \ y = x.p

\ \ while y and x == y.right

\ \ \ \ x = y

\ \ \ \ y = y.p

\ \ return y
\end{code}

*\hspace{0em}前驱算法与后继对称

\

二分检索问题:对有序数组,元素按关键字从小到大排列,任给关键字,查找出位置(或不存在)并返回。

许多具体实现方法(折半查找、\textbf{Fibonacci查找}等)

\begin{code}
def Binary\_Search(A, K, low, high):

\ \ if high < low

\ \ \ \ return 0

\ \ else mid = (low + high) / 2

\ \ \ \ if K = A[mid]

\ \ \ \ \ \ return Binary\_Search(A, K, low, mid-1)

\ \ \ \ else

\ \ \ \ \ \ return Binary\_Search(A, K, mid+1, high)
\end{code}

*\hspace{0em}正态分布数据,不对半分:将mid每次更新为low * t + high * (1-t)

**\hspace{0em}平衡二叉树与平衡旋转(思考:要使LL、LR、RR、RL旋转各发生一次,至少需要在空二叉树上插入多少个结点?)

\

\textbf{插入结点}

\begin{code}
def TREE\_INSERT(T, z):

\ \ y = NIL; x = T.root

\ \ while x

\ \ \ \ y = x

\ \ \ \ if z.key < x.key

\ \ \ \ \ \ x = x.left

\ \ \ \ else

\ \ \ \ \ \ x = x.right

\ \ z.p = y

\ \ if !y

\ \ \ \ T.root = z

\ \ else if z.key < y.key

\ \ \ \ y.left = z

\ \ else

\ \ \ \ y.right = z
\end{code}

\

\textbf{删除结点}

(数据结构书)基本思路:无孩子直接删除,单孩子连接后删除,两个孩子\textbf{复制为后继}后删除后继结点

(算法书)具体实现分为四种情形:

\begin{enumerate}
\item 
无左孩子

直接用右孩子替换z\ (无论是否为NIL)

\item
有左孩子无右孩子

直接用左孩子替换z

\item
有两个孩子,后继y是z的右孩子

用y替换z,并仅留下y的右孩子(此时y不可能有左孩子)

\item
有两个孩子,右孩子y不是z的右孩子

用y的右孩子替换y,并用y替换z
\end{enumerate}

子算法:替换

\begin{code}
def TRANSPLANT(T, u, v):

\ \ if !u.p

\ \ \ \ T.root = v

\ \ else if u == u.p.left

\ \ \ \ u.p.left = v

\ \ else u.p.right = v

\ \ if v

\ \ \ \ v.p = u.p
\end{code}

(主算法直接利用替换与上方逻辑构造即可)

*\hspace{0em}插入与删除的时间复杂度亦为$O(n)$

**\hspace{0em}数据随机分布时高度期望为$O(\log n)$

\subsection{红黑树}

\textbf{二分检索树},此外每个结点另外保存一位结点的颜色,\textbf{红}或\textbf{黑}

*\hspace{0em}没有一条路径比另一条长出两倍,因此\textbf{近似平衡}

结点包含五个属性:color、key、left、right、p\ [三叉链表]

[left或right的NIL视为叶结点,其余结点为内部结点]

\begin{enumerate}
    \item 每个结点红色或黑色
    \item 根结点黑色
    \item 叶结点(NIL)为黑色
    \item 红结点的子结点为黑色
    \item 每个结点到所有后代叶结点的简单路径上黑结点数目相同
    
    *\hspace{0em}此数目称为结点的\textbf{黑高度}[bh],包含叶结点,不包含自身

    *\hspace{0em}树的黑高度为根结点黑高度
\end{enumerate}

性质:n个内部结点的红黑树高度不会超过$2\log(n+1)$

证明思路:归纳说明x为根的子树至少$2^{\mathrm{bh}(x)}-1$个结点,又由红结点子结点为黑可知树高度不超过黑高度两倍,从而得证。

*\hspace{0em}由此之前对二分检索树的静态操作在红黑树上时间复杂度$O(\log n)$,但插入删除可能需要调整颜色[不过事实上仍可限制在$O(\log n)$]

\

\textbf{旋转}

*\hspace{0em}保持红黑性质的操作,用于插入、删除

y左孩子x,右子树c,x左右子树为ab,则y为中心右旋后x顶替y,且x左子树a,右孩子y,y左右子树bc,左旋则将此操作反向。

\begin{code}
def LEFT\_ROTATE(T, x):

\ \ y = x.right

\ \ x.right = y.left

\ \ if y.left != T.nil:

\ \ \ \ y.left.p = x

\ \ y.p = x.p

\ \ if x.p == T.nil:

\ \ \ \ T.root = y

\ \ elif x == x.p.left:

\ \ \ \ x.p.left = y

\ \ else:

\ \ \ \ x.p.right = y

\ \ y.left = x

\ \ x.p = y
\end{code}

右旋与左旋类似,复杂度$O(1)$

\

\textbf{插入}

*\hspace{0em}需要在正常插入后进行调整

正常插入过程与普通二叉搜索树INSERT完全相同,且置新结点颜色为红色。但是插入后需要调用RB\_INSERT\_FIXUP(因为此时可能出现新结点与父结点均为红):

\begin{code}
def RB\_INSERT\_FIXUP(T, z):

\ \ while z.p.color == RED

\ \ \ \ if z.p == z.p.p.left:

\ \ \ \ \ \ y = z.p.p.right

\ \ \ \ \ \ if y.color == RED:

\ \ \ \ \ \ \ \ z.p.color = BLACK

\ \ \ \ \ \ \ \ y.color = BLACK

\ \ \ \ \ \ \ \ z.p.p.color = RED

\ \ \ \ \ \ \ \ z = z.p.p

\ \ \ \ \ \ else:

\ \ \ \ \ \ \ \ if z == z.p.right:

\ \ \ \ \ \ \ \ \ \ z = z.p

\ \ \ \ \ \ \ \ \ \ LEFT\_ROTATE(T, z)

\ \ \ \ \ \ \ \ z.p.color = BLACK

\ \ \ \ \ \ \ \  z.p.p.color = RED

\ \ \ \ \ \ \ \  RIGHT\_ROTATE(T, z.p.p)

\ \ \ \ else:

\ \ \ \ \ \ y = z.p.p.left

\ \ \ \ \ \ if y.color == RED:

\ \ \ \ \ \ \ \ z.p.color = BLACK

\ \ \ \ \ \ \ \ y.color = BLACK

\ \ \ \ \ \ \ \ z.p.p.color = RED

\ \ \ \ \ \ \ \ z = z.p.p

\ \ \ \ \ \ else:

\ \ \ \ \ \ \ \ if z == z.p.left:

\ \ \ \ \ \ \ \ \ \ z = z.p

\ \ \ \ \ \ \ \ \ \ RIGHT\_ROTATE(T, z)

\ \ \ \ \ \ \ \ z.p.color = BLACK

\ \ \ \ \ \ \ \ z.p.p.color = RED

\ \ \ \ \ \ \ \ LEFT\_ROTATE(T, z.p.p)

\ \ T.root.color = BLACK
\end{code}

*\hspace{0em}通过复杂的分情况讨论验证正确性,由于每层最多$O(1)$,最终复杂度$O(\log n)$

\

\textbf{删除}

*\hspace{0em}先调用二分检索树删除,类似并进行替换[此处需要注意颜色的传递与二分检索树NIL与红黑树T.nil的区别]

*\hspace{0em}只有在真正删除的结点y是黑色结点时才需要进行fixup[由于y的颜色可能改变,需要记录原始颜色]

删除黑色结点时,若其为原来的根结点,而红色孩子成为根结点,则违反性质2;若x与x.p均红,违反性质4;由于y的任何祖先经过y的路径上黑结点个数减小1,违反性质5\ [做法:先认为占有y原来位置的结点x有两重黑色,再去掉一层]。由此构造算法:

\begin{code}
def RB\_DELETE\_FIXUP(T, x):

\ \ while x != T.root and x.color == BLACK

\ \ \ \ if x == x.p.left

\ \ \ \ \ \ w = x.p.right

\ \ \ \ \ \ if w.color == RED

\ \ \ \ \ \ \ \ w.color = BLACK

\ \ \ \ \ \ \ \ LEFT\_ROTATE(T, x.p)

\ \ \ \ \ \ \ \ w = x.p.right

\ \ \ \ \ \ if w.left.color == BLACK and w.right.color == BLACK

\ \ \ \ \ \ \ \ w.color = RED

\ \ \ \ \ \ \ \ x = x.p

\ \ \ \ \ \ else

\ \ \ \ \ \ \ \ if w.right.color == BLACK

\ \ \ \ \ \ \ \ \ \ w.left.color = BLACK

\ \ \ \ \ \ \ \ \ \ w.color = RED

\ \ \ \ \ \ \ \ \ \ RIGHT\_ROTATE(T, w)

\ \ \ \ \ \ \ \ \ \ w = x.p.right

\ \ \ \ \ \ \ \ w.color = x.p.color

\ \ \ \ \ \ \ \ x.p.color = BLACK

\ \ \ \ \ \ \ \ w.right.color = BLACK

\ \ \ \ \ \ \ \ LEFT\_ROTATE(T, x.p)

\ \ \ \ \ \ \ \ x = T.root

\ \ \ \ else

\ \ \ \ \ \ w = x.p.left

\ \ \ \ \ \ if w.color == RED

\ \ \ \ \ \ \ \ w.color = BLACK

\ \ \ \ \ \ \ \ RIGHT\_ROTATE(T, x.p)

\ \ \ \ \ \ \ \ w = x.p.left

\ \ \ \ \ \ if w.right.color == BLACK and w.left.color == BLACK

\ \ \ \ \ \ \ \ w.color = RED

\ \ \ \ \ \ \ \ x = x.p

\ \ \ \ \ \ else

\ \ \ \ \ \ \ \ if w.left.color == BLACK

\ \ \ \ \ \ \ \ \ \ w.right.color = BLACK

\ \ \ \ \ \ \ \ \ \ w.color = RED

\ \ \ \ \ \ \ \ \ \ LEFT\_ROTATE(T, w)

\ \ \ \ \ \ \ \ \ \ w = x.p.left

\ \ \ \ \ \ \ \ w.color = x.p.color

\ \ \ \ \ \ \ \ x.p.color = BLACK

\ \ \ \ \ \ \ \ w.left.color = BLACK

\ \ \ \ \ \ \ \ RIGHT\_ROTATE(T, x.p)

\ \ \ \ \ \ \ \ x = T.root

\ \ x.color = BLACK

\end{code}

*\hspace{0em}类似插入算法,分析可知复杂度$O(\log n)$

\subsection{动态顺序统计}

*\hspace{0em}数据结构扩张一般会增加维护成本,而希望维护成本可控

*\hspace{0em}例:扩张红黑树为\textbf{顺序统计树}以快速查找

在红黑树中添加域size,代表子树中内部结点个数(含自身),而空结点大小视为0

查找第i小元素:

\begin{code}
def OS\_SELECT(T, i, x):

\ \ r = x.left.size + 1

\ \ if i == r

\ \ \ \ return x

\ \ if i < r

\ \ \ \ return OS\_SELECT(x.left, i)

\ \ return OS\_SELECT(x.right, i - r)
\end{code}

判断结点序号:

\begin{code}
def OS\_RANK(T, x)

\ \ r = x.left.size + 1

\ \ y = x

\ \ while y != T.root

\ \ \ \ if y = y.p.right

\ \ \ \ \ \ r = r + y.p.left.size + 1

\ \ \ \ y = y.p

\ \ return r
\end{code}

*\hspace{0em}插入、删除方式:实际改变的结点递归向上更新size

\

一般扩张方式:选择基础数据结构->添加所需域->检验基本修改操作\textbf{维护附加信息}->添加新操作

*\hspace{0em}扩张红黑树,若新增域的值只依赖左右孩子,则插入删除可以保证在$O(\log n)$完成

思路:每次插入/删除最多向上更新至根

\

\textbf{区间树}

扩张红黑树使之支持\textbf{区间构成的动态集合}上的操作

*\hspace{0em}区间用有序实数对表示,均视为闭区间,相交当且仅当i.high >= j.low and j.high >= i.low,记为Overlap(i, j)

*\hspace{0em}每个结点包含一个区间,查找操作为查找任意一个与给定区间相交的区间(或不存在)

排序依据:结点对应的区间[int]的左端点

添加域:max,代表以某结点为根的子树中各区间右端点最大值

区间维护:x.max=max(x.int.high, x.left.max, x.right.max)

\begin{code}
def INTERVAL\_SEARCH(T, i):

\ \ x = T.root

\ \ while x != T.nil and !Overlap(i, x.int)

\ \ \ \ if x.left != T.nil and x.left.max >= i.low

\ \ \ \ \ \ x = x.left

\ \ \ \ else

\ \ \ \ \ \ x = x.right

\ \ return x
\end{code}

*\hspace{0em}证明思路:考虑结点处向左向右都不会错过区间

\subsection{斐波那契堆}

\textbf{可合并堆}[最小堆]:包含优先队列操作,并且允许合并两个建立新堆[基本操作建堆、插入、返回最小、删除最小、合并]

\textbf{斐波那契堆}:额外允许减小关键字、删除

**\hspace{0em}[真正实用]二项堆实现可合并堆操作(斐波那契堆程序设计复杂、常数项大,事实上一般不如二项堆)

斐波那契堆定义:一系列具有最小堆序的有根树集合,每个结点属性:

\begin{enumerate}
    \item x.p 指向父结点的指针
    \item x.child 指向某孩子的指针[所有孩子连成双向循环列表,称为child list]
    \item x.degree 孩子数目
    \item x.mark 上一次成为另一个结点孩子后是否失去过孩子[初始为false]
\end{enumerate}

*\hspace{0em}所有的根也组织成环形双向列表[root list],H.min指向所有根中最小的,H.n表示结点数目

*\hspace{0em}为进行摊还分析,定义堆H的势函数$\Phi(H)=t(H)+2m(H)$,t为树的数目,m为被标记结点的数目

\

建堆直接建空,H.min为NIL,插入结点:

\begin{code}
def FIB\_HEAP\_INSERT(H, x):

\ \ x.degree = 0; x.p = x.child = NIL; x.mark = FALSE

\ \ if !H.min:

\ \ \ \ create a root list for H containing just x

\ \ \ \ H.min = x

\ \ else:

\ \ \ \ insert x in H's root list

\ \ \ \ if x.key < H.min.key:

\ \ \ \ \ \ H.min = x

\ \ H.n++
\end{code}

*\hspace{0em}复杂度$O(1)$,势函数增加1

\

\textbf{合并}

\begin{code}
def FIB\_HEAP\_UNION(H1, H2):

\ \ H = MAKE\_FIB\_HEAP()

\ \ H.min = H1.min

\ \ concatenate the root list of H2 with the root list of H

\ \ if !H1.min or (H2.min and H2.min.key < H1.min.key):

\ \ \ \ H.min = H2.min

\ \ H.n = H1.n + H2.n

\ \ return H
\end{code}

*\hspace{0em}势函数为H1与H2直接求和,不变

\

\textbf{提取最小}

\begin{code}
def FIB\_HEAP\_EXTRACT\_MIN(H):

\ \ z = H.min

\ \ if z != NIL

\ \ \ \ for each child x of z:

\ \ \ \ \ \ add x to the root list of H

\ \ \ \ \ \ x.p = NIL

\ \ \ \ remove z from the root list of H

\ \ \ \ if z == z.right:

\ \ \ \ \ \ H.min = NIL

\ \ \ \ else:

\ \ \ \ \ \ H.min = z.right

\ \ \ \ \ \ CONSOLIDATE(H)

\ \ \ \ H.n -= 1

\ \ return z
\end{code}

*\hspace{0em}最复杂处:CONSOLIDATE,不断将度相同的根结点合并,且维护H.min

CONSOLIDATE需要先计算出最大度数上界$D(n)$,此上界为$\log_\phi n,\phi=\frac{\sqrt5+1}{2}$。

证明思路:记$F_i$为斐波那契数,有$F_{k+2}=1+\sum_{i=0}^kF_i$,而利用此性质可说明度为$k$时结点个数至少$F_{k+2}$,再有斐波那契数列递推式可知结果。

\begin{code}
def CONSOLIDATE(H):

\ \ D = D(H.n)

\ \ let A[0..D] be a new arrays initialed by NIL

\ \ for w in root list of H:

\ \ \ \ x = w; d = x.degree

\ \ \ \ while A[d]:

\ \ \ \ \ \ y = A[d]

\ \ \ \ \ \ if x.key > y.key: swap(x, y)

\ \ \ \ \ \ FIB\_HEAP\_LINK(H, y, x)

\ \ \ \ \ \ A[d] = NIL

\ \ \ \ \ \ d += 1

\ \ \ \ A[d] = x

\ \ H.min = NIL

\ \ for i = 0 to D if A[i]:

\ \ \ \ if !H.min:

\ \ \ \ \ \ create a root list for H containing just A[i]

\ \ \ \ \ \ H.min = A[i]

\ \ \ \ else:

\ \ \ \ \ \ insert A[i] into rootlist and set H.min if need
\end{code}

*\hspace{0em}FIB\_HEAP\_LINK将y从根移到x的孩子并将标记设为false

*\hspace{0em}复杂度分析:抽取最小结点的实际工作量为$O(D(n)+t(H))$,而摊还后势函数$m(H)$部分不变,$t(H)$最多$D(n)+1$,摊还代价$O(\log n)$

\

\textbf{关键字减值}

\begin{code}
def FIB\_HEAP\_DECREASE\_KEY(H, x, k):

\ \ if k > x.key: ERROR

\ \ x.key = k

\ \ y = x.p

\ \ if y and x.key < y.key:

\ \ \ \ CUT(H, x, y)

\ \ \ \ CASCADING\_CUT(H, y)

\ \ if x.key < H.min.key

\ \ \ \ H.min = x
\end{code}

*\hspace{0em}CUT为将结点放置到根结点并设置对应属性(p = NIL, mark = FALSE);CASCADING\_CUT为不断向上寻找,只要结点mark为TRUE且不是根结点就进行切割,若终止不是在根结点就将mark设置为TRUE。

*\hspace{0em}摊还代价为\textbf{常数}(因为改变标记会减小势),而删除结点只需要将关键字减为\textbf{负无穷}再提取最小

**\hspace{0em}红黑树与斐波那契堆无论最坏还是摊还,至少由一项重要操作时间复杂度$O(\log n)$,又由于其基于关键字比较决定,根据排序下界可知必然有$\Omega(\log n)$复杂度操作。而若关键字具有性质,如有界整数,则可以利用如\textbf{van Emde Boas树}来加快操作,达到$O(\log\log n)$复杂度。

\subsection{分离集合}

分离集合定义:将$n$个元素分成若干个\textbf{不相交}的集合,需要能允许查找属于哪个、合并集合。数据结构由一些不交的动态集合组成,每个集合需要一个\textbf{代表元}[保证未修改时代表元不变]。

操作:建立包含单元素的集合[MAKE\_SET]、合并[UNION]、确定在何处[FIND\_SET]。

摊还:假设建立单元素集合$n$次[元素个数],总操作$m$次,\textbf{以下的时间复杂度以此分析}。

\

*\hspace{0em}作用举例:图的连通分量[思考:输出连通分支]

\begin{code}
def CONNECTED\_COMPONENTS(G):

\ \ for each vertex v in G.V:

\ \ \ \ MAKE\_SET(v)

\ \ for each edge (u, v) in G.E:

\ \ \ \ if FIND\_SET(u) != FIND\_SET(v):

\ \ \ \ \ \ UNION(u, v)
\end{code}

(判断是否在同一个连通分支只需要确定FIND\_SET是否相等)

\

\textbf{链表实现}:头结点作为代表,每个元素有指针指向头结点

MAKE\_SET与FIND\_SET均为$O(1)$,只需要创建单结点链表/返回结点对应的头结点

UNION较为复杂:由于需要更新指向头结点指针,对$n$个元素的集合复杂度为$\Theta(n)$,考虑若干次MAKE\_SET之后若干次合并,代价为$O(n)$

若每次用短合并长,复杂度可控制为$O(m+n\log n)$。

\

\textbf{森林实现}:根结点作为代表,每个元素仅指向父结点,根节点指向自己

关键:通过\textbf{按秩合并}[对每个结点维护高度上界,使较小秩的根指向较大秩的根]与\textbf{路径压缩}[在\textbf{查找过程中}让路径上的结点父结点直接为根]改进结果

\begin{code}
def MAKE\_SET(x):

\ \ x.p = x; x.rank = 0

def UNION(x, y):

\ \ LINK(FIND\_SET(x), FIND\_SET(y))

def LINK(x, y):

\ \ if x.rank > y.rank:

\ \ \ \ y.p = x

\ \ else:

\ \ \ \ x.p = y

\ \ \ \ if x.rank == y.rank

\ \ \ \ \ \ x.rank += 1

def FIND\_SET(x):

\ \ if x != x.p:

\ \ \ \ x.p = FIND\_SET(x.p)

\ \ return x.p
\end{code}

*\hspace{0em}注意FIND\_SET中\textbf{递归}两趟法思想的应用

**\hspace{0em}复杂度分析:最坏运行时间为$O(m\alpha(n))$,其中$\alpha$增长极慢,因此可以视为对元素个数线性。

\section{图论算法与串匹配}
\subsection{图的表示与遍历}
图的表示:邻接矩阵[顶点作为维度的01方阵,相连的有向边为1]、邻接表[每个顶点通过链表存储与其相连的顶点]

**\hspace{0em}邻接表存放顶点序号而不是内容

\

\textbf{广度优先搜索}[BFS]

*\hspace{0em}利用队列,颜色标记是否访问过

\begin{code}
def BFS(G, s):

\ \ for each vertex u in G.V - \{s\}

\ \ \ \ u.color = WHITE; u.d = INFTY; u.pi = NIL

\ \ s.color = GRAY; s.d = 0; s.pi = NIL

\ \ Q = EMPTYSET

\ \ ENQUEUE(Q, s)

\ \ while Q is not empty

\ \ \ \ u = DEQUEUE(Q)

\ \ \ \ for each v in G.Adj[u]:

\ \ \ \ \ \ if v.color == WHITE:

\ \ \ \ \ \ \ \ v.color = GRAY; v.d = u.d + 1; v.pi = u

\ \ \ \ \ \ \ \ ENQUEUE(Q, v)

\ \ \ \ \ \ u.color = BLACK
\end{code}

*\hspace{0em}复杂度$O(|V|+|E|)$

作用:最短路径(其中d存储的即是s到某个结点的最短路径长度)

\textbf{广度优先树}:每个结点的pi属性成为广度优先树中的父结点

\

\textbf{深度优先搜索}[DFS]

*\hspace{0em}以下的算法是\textbf{遍历},而广度优先若需要遍历需要在未搜索到的结点中继续寻找源结点

*\hspace{0em}利用\textbf{时间戳}记录发现与离开结点的时间u.d与u.f

\begin{code}
def DFS(S):

\ \ for each vertex u in G.V

\ \ \ \ u.color = WHITE; u.pi = NIL

\ \ time = 0

\ \ for each vertex u in G.V

\ \ \ \ if u.color == WHITE:

\ \ \ \ \ \ DFS\_VISIT(G, u)

def DFS\_VISIT(G, u):

\ \ time += 1; u.d = time; u.color = GRAY;

\ \ for each v in G.Adj[u]:

\ \ \ \ if v.color == WHITE:

\ \ \ \ \ \ v.pi = u

\ \ \ \ \ \ DFS\_VISIT(G, v)

\ \ u.color = BLACK; time += 1; u.f = time
\end{code}

*\hspace{0em}复杂度$O(|V|+|E|)$

*\hspace{0em}每个结点的pi属性构成\textbf{深度优先树森林}

\textbf{括号化定理}:每个结点的d域与f域形成的区间不会互相交叉。

证明:不妨设其中某个结点先被发现,则后发现的结点要么在先发现结点探索完后还未开始,要么包含在先发现结点遍历完的过程中,从而讨论得证。

*\hspace{0em}深度优先森林中v为u的后代当且仅当发现u时有从u到v的白结点路径(利用过程容易说明)

\

\textbf{边分类}:
\begin{enumerate}
    \item 树边\ 深度优先森林中的边
    \item 后向边\ 将结点连到祖先结点的边(包括自环)
    \item 前向边\ 将结点连到后代(非孩子)结点的边
    \item 横向边\ 其他所有边
\end{enumerate}

*\hspace{0em}考虑结点处理先后可发现,\textbf{无向图}深度优先遍历的过程中只会出现树边与后向边

思考:设计算法寻找无向连通图中所有\textbf{关键结点}[去掉就不连通的结点]

\

\textbf{拓扑排序}

定义:\textbf{有向无环图}中的一个满足存在边(u,v)则有u在v前的全序关系

算法:DFS中设置u.f之前将u压入栈,最后自顶向底输出栈即可。

*\hspace{0em}利用结点访问结束顺序与拓扑排序顺序相反可以实现,时间复杂度仍为$O(|V|+|E|)$

*\hspace{0em}有向图无环当且仅当深度搜索不产生后向边

\

\textbf{强连通分量}

定义:有向图中的一个极大子集C,满足其中任何两点互相可到达。

*\hspace{0em}定义图的\textbf{转置}为邻接矩阵转置对应的图,即顶点不变,边的方向变为反向

算法:对G进行DFS,在离开结点时将结点压入栈。将图变为G转置[并清空上一轮计算结果],按照从栈顶到栈底的顺序,对白色结点进行DFS,得到的深度优先森林中每一棵树对应一个强连通分量。

*\hspace{0em}事实上是以拓扑排序的次序进行访问

*\hspace{0em}时间复杂度仍为$O(|V|+|E|)$

**\hspace{0em}利用合并强连通分量得到的\textbf{分支图}可以证明正确性

\subsection{图论问题}
\textbf{最小生成树}

定义:所有边权值之和最小的生成树。

基本思路:从空集开始不断添加边,直到成为生成树。

性质:最小生成树部分边集合A中涉及的顶点记为S,其余为V-S,取一条最短的连接两部分的边加入,添加后A必然还是最小生成树的部分边。

[此性质为所有算法的基本思路]

算法:
\begin{enumerate}
    \item 破圈法\ 起初包含所有边,只要有圈,就去掉其最大边,直到不存在圈。
    \item Kruskal\hspace{0em}算法\ 将所有边排序,每次从剩下的中选最小代价的,只要不会产生环路就加入。
    \item Prim\hspace{0em}算法\ 每次选择使入选的边\textbf{保持为一棵树}的最小的边。
    \item Sollin\hspace{0em}算法:起初每个树为独立顶点,边集合为空,每步每个顶点集各自选择从其连接到外部的最短的边,连接成新的顶点集。[时间复杂度很大,适合并行]
\end{enumerate}

利用分离集合实现Kruskal算法:

\begin{code}
def Kruskal(G):

\ \ T = EMPTYSET

\ \ for each v in G.V:

\ \ \ \ MAKE\_SET(v)

\ \ sort edges of G.E by increasing edge weight w

\ \ for each (u,v) in G.E:

\ \ \ \ if FIND\_SET(u) != FIND\_SET(v):

\ \ \ \ \ \ add (u,v) into T

\ \ \ \ \ \ Union(u, v)

\ \ return T
\end{code}

*\hspace{0em}边集合可以用二叉堆实现,时间复杂度为$O(|V|+|E|\log|E|)$,思考:可不可能为复杂度$O(|V|^2)$?

Prim算法:

\begin{code}
def Prim(G, w, r):

\ \ for each u in G.V

\ \ \ \ u.key = INFTY; u.pi = NIL

\ \ r.key = 0

\ \ Q = G.V

\ \ while Q != EMPTYSET:

\ \ \ \ u = EXTRACT\_MIN(Q)

\ \ \ \ for each v in G.Adj[u]:

\ \ \ \ \ \ if v in Q and w(u,v) < v.key:

\ \ \ \ \ \ \ \ v.pi = u

\ \ \ \ \ \ \ \ v.key = w(u,v)
\end{code}

*\hspace{0em}不断更新当前的最小距离,最终所有的pi看作父结点形成的树即为最小生成树

*\hspace{0em}时间复杂度为$O(|E|\log|V|)$,事实上由于$|E|\in[|V|-1,|V|^2]$,时间复杂度也可看作$O(|E|\log|E|)$,与Kruskal算法量级相同

**\hspace{0em}目前最小生成树算法已经能做到$O(|E|)$复杂度

\

\textbf{单源最短路径问题}

单源最短路径问题:给定带权有向图,求从图中一个特定源点出发到其余各个顶点的最短路径。

单源\textbf{简单}最短路径问题:给定带权有向图,求从图中一个特定源点出发到其余各个顶点的简单最短路径。

*\hspace{0em}对前者,有权值小于0回路时无意义,对后者仍然有意义。只有前者满足\textbf{最优性原理}。

\textbf{松弛}操作:

\begin{code}
def Relax(u, v, w):

\ \ if (d[v] > d[u] + w):

\ \ \ \ d[v] = d[u] + w
\end{code}

*\hspace{0em}Dijkstra算法\textbf{不能对有负边的图使用}

Bellman Ford算法:

\begin{code}
def Bellman\_Ford(G, s)

\ \ for each v in G.V:

\ \ \ \ d[v] = INFTY

\ \ d[s] = 0

\ \ for i = 1 to length(G.V)-1:

\ \ \ \ for each edge (u,v) in G.E:

\ \ \ \ \ \ Relax(u, v, w(u,v))

\ \ for each edge (u,v) in G.E

\ \ \ \ if(d[v] > d[u] + w(u,v)) return NO\_SOLUTION

\ \ return TRUE
\end{code}

*\hspace{0em}复杂度$O(|V||E|)$,$|V|-1$次松弛保证任何边的影响扩展到了所有路径上

*\hspace{0em}优化思路:通过合适的次序减少松弛的趟数

*\hspace{0em}对带权有向无环图,按拓扑排序结果,\textbf{从源点开始依次对其后各个顶点引出的边做松弛},则一趟松弛即可完成。

优先队列Dijkstra:

\begin{code}
def Dijkstra(G):

\ \ for each v in G.V:

\ \ \ \ d[v] = INFTY

\ \ d[s] = 0; S = EMPTYSET; Q = V

\ \ while Q != EMPTYSET:

\ \ \ \ u = EXTRACT\_MIN(Q)

\ \ \ \ add u to S

\ \ \ \ for each v in G.Adj[u]:

\ \ \ \ \ \ Relax(u, v, w(u,v))
\end{code}

*\hspace{0em}由于优先队列存储,复杂度$O(|E|\log|V|)$

思考:是否对任何权值非负无向连通图,可以通过Dijkstra算法选取合适源点得到最小生成树?

[答案:否定,考虑$w_{ab}=w_{bc}=w_{cd}=2,w_{ac}=w_{bd}=3$。]

**\hspace{0em}给定两点间的最短路径时间复杂度与单源最短路径相同,因为必须找到所有单源最短路径才能断定最短

\

\textbf{所有点对间最短路径}

无负边:直接使用Dijkstra算法,理论复杂度可达$O(|V||E|\log|V|)$或$O(|V|^2\log|V|+|E||V|)$(取决于优先队列实现)。

\

化为矩阵乘法[\textbf{运算重载}技巧]

**技巧意义大于实际应用

当前最短路径矩阵D,记录每个结点的前驱矩阵Pi为$(\pi_{ij})$,$\pi_{ij}$表示i到j的最短路径中j的前驱结点。

*\hspace{0em}通过不断j = Pi[i][j]可递归找到最短路径

用邻接矩阵W表示所有边权,对角为0,不连接为无穷,递归寻找:

定义$l_{ij}^{(m)}$为i到j的途径至多m条边的最短路径长度,当m为1时即为W,由定义发现递推关系$l_{ij}^{(m)}=\min_k(l_{ik}^{(m-1)}+w_{kj})$,而最终所求的结果即为$l_{ij}^{(n-1)}$,利用动态规划即可求解。

*\hspace{0em}直接动态规划复杂度$O(|V|^2|E|)$或$O(|V|^4)$

优化:将每次的$L^{(m)}$视为矩阵L,重定义矩阵乘法:

原本乘法$c_{ij}=\sum_{i=1}^na_{ik}b{kj}$,将其中乘法重定义为加法,加法重定义为取最小值,则可类似矩阵乘法计算(核心:由于加法对取最小值可分配,仍然具有\textbf{结合律}),由于计算幂可以两两累加,效率能得到提升。

*\hspace{0em}复杂度可以为$|V|^3\log|V|$

\

\textbf{Floyd-Warshall算法}

考虑$l_{ij}^{(m)}$为i到j的顶点序号最大为m的最短路径,这时有递推$l_{ij}^{(m)}=\min\{l_{ij}^{(m-1)},l_{im}^{(m-1)}+l_{mj}^{(m-1)}\}$,从而仍然利用动态规划只需$O(|V|^3)$。

*\hspace{0em}Pi的确定:若递推中左侧较小则不变,右侧较小则更新成为$\pi_{mj}^{(m-1)}$

\begin{code}
def FLOYD\_WARSHALL(W):

\ \ n = rows(W)

\ \ D = W

\ \ for k = 1 to n:

\ \ \ \ for i = 1 to n, j = 1 to n:

\ \ \ \ \ \ d[i][j] = min(d[i][j], d[i][k] + d[k][j])

\ \ return D
\end{code}

*\hspace{0em}空间复杂度$\Theta(|V|^2)$,注意到更新顺序使得不需要新建D,原地迭代即可

*\hspace{0em}传递闭包:用或代替min,与代替+,01邻接矩阵,计算后为传递闭包

\

\textbf{稀疏图}Johnson算法[\textbf{权值重定义}技巧]

*\hspace{0em}更新所有边的权值使其均非负,然后对每点利用Dijkstra

定理:给每点x赋予权值h(x),边权w(u,v)更新为w'(u,v) = w(u,v) + h(u) - h(v),则更新后最短路径与更新前相同。

证明:由于路径中裂项相消,起点到终点的任何路径改变恒为h的差,因此最短不变。

*\hspace{0em}如何构造h?

加入新顶点s,到任何点都连有权为0的边,对点x,将s到x的最短路径值设为h(x)。由定义可发现h(x)小于等于0,当不存在负边时一定为0。

*\hspace{0em}这样赋予后利用三角不等式h(v) <= h(u) + w(u,v)可以证明新权值非负

思考:为何需要加入新顶点s?[未必存在某点能到达所有其他点]

于是,算法的过程为:加入s、计算单源最短路径、更新边、计算所有点间最短路径,二叉堆实现复杂度为$O(|V||E|\log|V|)$,稀疏图时优于Floyd-Warshall。

\

\subsection{串匹配算法}

串匹配问题:在文本[T, text]串中找到一个或所有模式[P, pattern]串

*\hspace{0em}找所有只需要实现找单个后不断右移

n代表T的长度,m代表P的长度,n一般远大于m。$\sigma,\Sigma,\Sigma^*$代表字符集大小、字符集与字符串[允许为空],$C_n$表示比较次数,一般作为衡量标准。

一些定义:若x存在分解wy,则称w为x的前缀,y为x的后缀,记作$w\sqsubset x,y\sqsupset x$。对P[1..m],记$P_k$代表其前k个字符,$T_k$同理。

后缀重叠引理:若$x\sqsupset z, y\sqsupset z$,则$|x|\le|y|\Rightarrow x\sqsupset y,|x|\ge|y|\Rightarrow y\sqsupset x,|x|=|y|\Rightarrow x = y$。

*\hspace{0em}算法基本原理为\textbf{滑动窗口机理},利用大小等于模式长度的窗口对文本串进行扫描,每次匹配后按需要右移窗口,直到超出文本右端。

\begin{enumerate}
    \item 从左到右\ Rabin-Karp, KMP
    \item 从右到左\ Boyer-Moore, Horspool
    \item 任意顺序\ Brute Force(原始算法)
\end{enumerate}

\

\textbf{Brute Force}

*\hspace{0em}每个位置检查,最坏情况$O(mn)$

\begin{code}
def BRUTE\_FORCE(T, P):

\ \ i = 0

\ \ while i <= n - m:

\ \ \ \ j = 0

\ \ \ \ while j < m and P[j+1] == T[i+j+1]: j++

\ \ \ \ if j == m: report match at i-j+1

\ \ \ \ i++
\end{code}

*\hspace{0em}如何避免重复计算的发生?[保存部分匹配信息]

\

\textbf{KMP算法}

*\hspace{0em}思想:当前指针只右移不倒退,遇到不成功匹配后充分利用前一部分信息。

先计算Next[j+1] = max(k+1,使得P[1..k]是P[1..j]的\textbf{后缀})

\begin{code}
def Next(P):

\ \ j = 0

\ \ for i = 1 to m:

\ \ \ \ Next[i] = j

\ \ \ \ while j > 0 and P[i] != P[j]:

\ \ \ \ \ \ j = Next[j]

\ \ \ \ j++
\end{code}

*\hspace{0em}复杂度$O(m)$

**\hspace{0em}书中定义的pi[i] = Next[i+1] - 1

\begin{code}
def KMP(T, P):

\ \ j = 1

\ \ for i = 1 to n:

\ \ \ \ while j < 0 and T[i] != O[j]:

\ \ \ \ \ \ j = Next[j]

\ \ \ \ if j == m: return i - m + 1

\ \ \ \ j++

\ \ return None
\end{code}

*\hspace{0em}总复杂度$O(m+n)$,当重复多时效果更好(如二进制文件)

\

\textbf{Boyer-Moore算法}

思路:每次自右向左匹配后缀,确定不匹配时最多能移动多少位,移动情况分为Gs[好后缀]与Bc[坏字符]两种。

\textbf{Horspool的简化}

*\hspace{0em}当字符的可能性相对模式长度很大时[如单词搜索],坏字符移动效果很好

坏字符移动:P的某个后缀和T某段匹配,更前一位并不匹配时,设此时T中对应字符为b,找到P中从右向左第一次出现b的位置,并移动使两个b对齐(若不出现,直接移动过T中b的位置)。

\begin{code}
def PRE\_BC(P, m, Bc):

\ \ for i = 1 to ASIZE: Bc[i] = m

\ \ for i = 1 to m-1: Bc[P[i]] = m - i

def HORSOPPL(P, T):

\ \ PRE\_BC(P, m, Bc)

\ \ i = m

\ \ while i <= n:

\ \ \ \ k = 1

\ \ \ \ while k <= m and P[m-k+1] == T[i-k+1]:

\ \ \ \ \ \ k++

\ \ \ \ if k == m: report match at i-m+1

\ \ \ \ else: i += BC[T[i]]
\end{code}

*\hspace{0em}最坏情况时间复杂度$O(mn)$,平均比较次数在$\big(\frac{n}{\sigma},\frac{2n}{\sigma+1}\big)$

\

\textbf{Rabin-Karp算法}

思路:利用哈希函数检测文本与模式中“相似”的部分,并进一步检查匹配。

先写出ord函数将字符集对应为0到sigma-1的数,再构造哈希表,一个长度为m的词w的哈希函数定义为:

$$\begin{cases}x=\mathrm{ord}(w)\\\mathrm{hash}(w)=\sum_ix_id^{m-i}\mod q\\\mathrm{rehash}(a, b, h)=(h-ad^{m-1})d+b\mod q\end{cases}$$

其中q是某个大素数,d为底数(一般可以取2)

\begin{code}
def KR(P, T):

\ \ y = d ** m-1; Phash = 0; Thash = 0

\ \ for i = 1 to m:

\ \ \ \ Phash = (Phash * d + ord(P[i])) % q

\ \ \ \ Thash = (Thash * d + ord(T[i])) % q

\ \ j = 1

\ \ while j <= n - m + 1:

\ \ \ \ if Phash == Thash and memcmp(P, T+j-1, m) == 0:

\ \ \ \ \ \ report match at j

\ \ \ \ Thash = ((Thash - y*ord(T[j])) * d + ord(T[j+m])) % q

\ \ \ \ j++
\end{code}

*\hspace{0em}当期望的有效偏移很少时复杂度$O(m+n)$

\end{document}