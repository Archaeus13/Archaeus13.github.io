\documentclass[a4paper,UTF8,fontset=windows]{ctexart}
\pagestyle{headings}
\title{\heiti 概率论\ 课堂笔记}
\author{原生生物}
\date{}
\setcounter{tocdepth}{2}
\setlength{\parindent}{0pt}
\usepackage{amsmath,amssymb,enumerate,geometry}
\geometry{left = 2.0cm, right = 2.0cm, top = 2.0cm, bottom = 2.0cm}
\ctexset{section={number=\zhnum{section}}}
\ctexset{subsection={name={\S},number=\arabic{section}.\arabic{subsection}}}
\newtheorem{thm}{定理}[section]
\newtheorem{exmp}{例}[section]
\newtheorem{defi}{定义}[section]
\everymath{\displaystyle}

\DeclareMathOperator{\Cov}{Cov}
\DeclareMathOperator{\Exp}{Exp}
\DeclareMathOperator{\GOE}{GOE}
\DeclareMathOperator{\Var}{Var}
\DeclareMathOperator{\diag}{diag}
\DeclareMathOperator{\tr}{tr}
\newcommand{\con}[1]{\stackrel{#1}{\longrightarrow}}
\begin{document}
\maketitle
\begin{center}
\Large 斗满人概,人满天概。——管子
\end{center}
\tableofcontents

\newpage

\section{概率空间与独立性}
\subsection{事件与概率}
\textbf{样本点}-具体结果\ \textbf{样本空间}$\Omega$-样本点的全体\ \textbf{事件}$A$-样本空间的子集

\begin{exmp}
掷硬币 $\Omega=\{H,T\},A=\{H\}$

电子自旋 $\Omega=\{\uparrow,\downarrow\},A=\{\uparrow\}$

掷骰子 $\Omega=\{1,2,3,4,5,6\},A=\{2,4,6\}$
\end{exmp}

*上方的例子中,样本点均只有有限多个。

\begin{exmp}
道琼斯指数\ 样本点-连续曲线$x(t),t\in[0,T]$\ 样本空间$\Omega=\{x(t)\in C[0,T]\}$
\end{exmp}

事件运算$\longleftrightarrow$集合运算

事件$A$发生$\longleftrightarrow$试验结果$\omega\in A$

$A=\varnothing$\textbf{不可能事件} $A=\Omega$\textbf{必然事件}

事件交、并、余$\longleftrightarrow A\cap B,A\cup B,A^\mathrm{c}$

事件的交-两事件均发生\ 事件的并-两事件至少发生一个\ 事件的余-事件未发生

*可记$A\cap B$为$AB$

$A$与$A^\mathrm{c}$称\textbf{对立事件}。

$A$发生则$B$亦发生$\longleftrightarrow A\subset B$

$A\cap B=\varnothing$时称$A,B$\textbf{不相容}。

*由此亦可定义$A_1,\dots,A_n,\dots$互不相容

~

问题:是否要将$\Omega$的所有子集定义为随机事件?

*良好定义的随机事件为$\Omega$的一个子集族,且至少要求对交、并、余三种运算\textbf{封闭}。

\begin{exmp}
掷硬币第一次正面向上的时刻 $\Omega=\{1,2,\dots\}$ 此时样本空间为无限集合,有限交并性质不足!
\end{exmp}

\begin{defi}
$\mathcal{F}$为$\Omega$某些子集构成的子集族,称其为事件域($\sigma$域),若:

1. $\Omega\in\mathcal{F}$

2. $A\in\mathcal{F}\implies A^\mathrm{c}\in\mathcal{F}$

3. $A_n\in\mathcal{F},n\in\mathbb{N}^*\implies\bigcup_{n=1}^{\infty}A_n\in\mathcal{F}$

并称$(\Omega,\mathcal{F})$为一个可测空间。
\end{defi}

\begin{exmp}
最大$\sigma$域,$\Omega$的幂集

最小$\mathcal{F}=\{\Omega,\varnothing\}$ 

中间域,如$A\neq\varnothing,\Omega$时$\mathcal{F}=\{\Omega,\varnothing,A,A^\mathrm{c}\}$
\end{exmp}

~

定义概率:直观想法-\textbf{频率稳定性}(重复试验后计数发生次数)

重复$N$次,$A$发生$N_A$次,经验表明$\lim_{N\to\infty}\frac{N_A}{N}=c$,记$c$为$P(A)$。明显性质:$P(\varnothing)=0,P(\Omega)=1$。

若$A\cap B=\varnothing$,$N_{A\cup B}=N_A+N_B\implies P(A\cup B)=P(A)+P(B)$

\begin{defi}
$P:\mathcal{F}\to\mathbb{R}$为$(\Omega,\mathcal{F})$上的概率测度,若:

1. 非负性 $P(A)\ge0$

2. 规范性 $P(\Omega)=1$

3. 可列可加性\ 若$\{A_n\}$互不相容,则$P\bigg(\bigcup_{n}A_n\bigg)=\sum_nP(A_n)$

并称$(\Omega,\mathcal{F},P)$为一个概率空间。
\end{defi}

\begin{exmp}
掷硬币 $\Omega=\{H,T\},\mathcal{F}=2^\Omega,P(\{H\})=p,P(\{T\})=q,p+q=1$
\end{exmp}

\begin{exmp}
$\Omega=\{1,2,3,4,5,6\},\mathcal{F}=2^\Omega$ 假定公平,则$P(A)=\frac{|A|}{6}$
\end{exmp}

样本点有限,且每个样本点等概率发生,则称\textbf{古典概型}。

\begin{thm}
概率测度基本性质

1. $P(A^\mathrm{c})=1-P(A)$

2. $A\subset B$时$P(B)=P(B\backslash A)+P(A)\ge P(A)$

3. $P(A\cup B)=P(B)+P(A)-P(AB)$

4. Jordan公式 $P\bigg(\bigcup_{i=1}^nA_i\bigg)=\sum_{k=1}^n\sum_{i_1<\dots<i_k}(-1)^{k-1}P(A_{i1}\dots A_{ik})$
\end{thm}

\begin{thm} 概率测度连续性
	
1. 单调增事件列$A_1\subset\cdots\subset A_n\subset\cdots$,记$A=\bigcup_{i=1}^{\infty}A_i=\lim_{n\to\infty}A_n$,则$P(A)=\lim_{n\to\infty}P(A_n)$。

2. 单调减事件列$B_1\supset\cdots\supset B_n\supset\cdots$,记$B=\bigcap_{i=1}^{\infty}B_i=\lim_{n\to\infty}B_n$,则$P(B)=\lim_{n\to\infty}P(B_n)$。
\end{thm}

证明:$A=A_1\cup(A_2\backslash A_1)\cup\cdots\cup(A_n\backslash A_{n-1})\cup\cdots$。此处的并均为无交并,因此可由定义与引理计算即得结果。对于$B$,取余即可化为$A$的情况。

\subsection{条件概率与独立性}
直观想法-重复试验,$B$发生$N_B$次,$B$发生条件下$A$发生次数$N_{AB}$,次数足够多时\textbf{条件概率}可看作$\frac{N_{AB}}{N_B}$。

\begin{defi} 条件概率

设$P(B)>0$,$B$发生条件下$A$的条件概率$P(A|B)=\frac{P(AB)}{P(B)}$

变形有乘法公式$P(AB)=P(B)P(A|B)$
\end{defi}

*$B_1,\dots,B_n$为$\Omega$的互不相容子集,且$\bigcup_iB_i=\Omega$,则称其为$\Omega$的一个\textbf{划分}。

\begin{thm} 全概率公式

$B_1,\dots,B_n$为$\Omega$的一个划分,$P(B_i)>0$,则$A=A\cap\Omega=\bigcup_i(B_i\cap A)\Rightarrow P(A)=\bigcup_iP(B_i)P(A|B_i)$。
\end{thm}

\begin{exmp}
坛子里有3白2红共5个球,每次无放回摸出一个球,$A=$\{第二次摸到白球\},按第一次抽到白或红划分可知$P(A)=\frac{3}{5}\cdot\frac{1}{2}+\frac{2}{5}\cdot\frac{3}{4}=\frac{3}{5}$(注意到,在每个轮次抽出白球的概率一致)。
\end{exmp}

\begin{thm} 贝叶斯公式

$A_1,\dots,A_n$为$\Omega$的一个划分,$P(A_i)>0$,则$P(B)>0$时$P(A_i|B)=\frac{P(A_i)P(B|A_i)}{\sum_jP(A_j)P(B|A_j)}$
\end{thm}

\begin{exmp} 发出s时,收到s概率为$0.8$,收到t概率为$0.2$;发出t时,收到s概率为$0.1$,收到t概率为$0.9$。且发出s概率为$0.6$,发出t概率为$0.4$。
	
收到s的情况下,发出s的概率为:$\frac{0.6\cdot0.8}{0.6\cdot0.8+0.4\cdot0.1}=0.923$

\end{exmp}

~

掷硬币,$B$代表第一次正面,$A$代表第二次正面,则$P(A|B)=P(A)$,即$P(AB)=P(A)P(B)$,由此引出定义:

\begin{defi} 独立性
	
称$A,B$独立,若$P(AB)=P(A)P(B)$

更一般,称$A_1,\dots,A_n$\emph{相互独立}是指$\forall2\le k\le n,i_1<\dots<i_k,P(\prod_{j=1}^kA_{i_k})=\prod_{j=1}^kP(A_{i_k})$;

\emph{两两独立}是指只需$k=2$时满足。
\end{defi}

两两独立与相互独立不同,举例如下:

\begin{exmp}
古典概型中,$\Omega=\{1,2,3,4\},A=\{1,3\},B=\{1,2\},C=\{1,4\}$。

计算可知$A,B,C$两两独立,但不相互独立。
\end{exmp}

\begin{thm}
若$A,B$独立,则$A$与$B^\mathrm{c}$,$A^\mathrm{c}$与$B$,$A^\mathrm{c}$与$B^\mathrm{c}$独立。更一般地,若一些事件相互独立,将其中部分改为其对立事件后仍然相互独立。
\end{thm}

证明:两事件时,由对称,只需证明$A$与$B^\mathrm{c}$独立。由$P(AB^\mathrm{c})+P(AB)=P(A)$可算出结果。多事件时类似两事件一个个调整即可。

\begin{exmp} “重复独立试验,小概率事件必发生”

记事件为$A$,$A_k=$\{第k次试验中$A$发生\},则$\bigcup_{k=1}^nA_k$表示\{前k次试验中$A$发生\}。

$P(\bigcup_{k=1}^nA_k)=1-P(\bigcap_{k=1}^nA_k^\mathrm{c})$,由独立性,此式为$1-\prod_{k=1}^n(1-P(A_k))$,由此知结果。
\end{exmp}

\subsection{概率模型}
\begin{exmp} 对称随机游走

赌徒财富$k$,庄家$N-k$,掷均匀硬币,正面则赌徒赢庄家1,否则庄家赢赌徒1,赌至一方输光,问赌徒输光概率。	
\end{exmp}

记赌徒初始为$k$且输光的事件为$A_k$,$B$为首局正面,由此$P(A_k)=P(B)P(A_k|B)+P(B^\mathrm{c})P(A_k|B^\mathrm{c})=\frac{P(A_{k-1})+P(A_{k+1})}{2}$,且$P(A_0)=1,P(A_N)=0$,由等差数列知$P(A_k)=\frac{N-k}{N}$。

~

\textbf{计数问题}

\begin{exmp}
坛子里有4白6红共10个球,随机取4个,求2白2红概率。
\end{exmp}

样本点数目$\mathrm{C}_{10}^4$,事件发生的样本点数目$\mathrm{C}_4^2\mathrm{C}_6^2$,结果为$\frac{3}{7}$。

*\textbf{古典概型}重要应用:\textbf{排列组合}计算样本点数目

经典问题:$n$个对象中选$m$个,问选法种数(是否可重复?是否考虑顺序?)。

有序不重复:$\mathrm{A}_n^m$
无序不重复:$\mathrm{C}_n^m$
有序可重复:$n^m$

无序可重复:插板法,看作$m$个小球$n$个盒子($n-1$个挡板),知结果为$\mathrm{C}_{n-1+m}^m$

\begin{exmp}
将$n$个小球投入到$N\geq n$个盒子中,投法等可能,求前n个盒子中各一个球的概率(球是否可分辨?盒子是否有容量限制?)。
\end{exmp}

(1) 球可区分,盒子无限制(麦克斯韦-玻尔兹曼统计):样本点个数$N^n$,合要求个数$n!$,概率$\frac{n!}{N^n}$

(2) 球不可区分,盒子无限制(玻色-爱因斯坦统计):化为无序可重复,样本点个数$\mathrm{C}_{n+N-1}^n$,合要求个数1,概率$\frac{1}{\mathrm{C}_{n+N-1}^n}$

(3) 球不可区分,盒子容量一(费米-狄拉克统计):样本点个数$\mathrm{C}_N^n$,合要求个数1,概率为$\frac{1}{\mathrm{C}_N^n}$

\begin{exmp} Polya模型
	
坛子里有一些球,$b$黑$r$红,先摸出一个记下颜色后放回,并且再放入$c$个同色球。记$B_n$表示第$n$次抽到黑球,求概率。
\end{exmp}

观察可发现,$n$次抽取抽中$k$次黑球,任意给定次序概率相同,为$D_k(b)=\frac{\prod_{i=0}^{k-1}(b+ic)\prod_{i=0}^{n-k-1}(r+ic)}{\prod_{i=0}^{n-1}b+r+ic}$,有$B_{n+1}=\sum_{k=0}^n\mathrm{C}_n^kD_k(b)\frac{b+kc}{b+r+nc}=\frac{b}{b+r}\sum_{k=0}^n\mathrm{C}_n^kD_k(b+r)$,而由概率含义$\sum_{k=0}^n\mathrm{C}_n^kD_k(b+r)$构成整个样本空间,因此为1,因此$B_{n+1}=\frac{b}{b+r}$。

*$c=-1$即为无放回,$c=0$即为有放回。

\section{随机变量与分布函数}
\subsection{随机变量}
$\mathcal{F}_1,\mathcal{F}_2$为$\Omega$上的$\sigma$域,可验证其交亦为$\sigma$域。更一般地,给定某指标集$I$,$\mathcal{F}_i,i\in I$的交集亦为$\sigma$域。

$\mathbb{R}$上Borel域定义为包含所有$(a,b],a,b\in\mathbb{R}$的\textbf{最小}$\sigma$域(最小定义:所有包含的取交),记为$B(\mathbb{R})$。

$\{b\}=\bigcap_n\left(b-\frac{1}{n},b\right]\in B(\mathbb{R})$,类似知$(a,b),[a,b],[a,b)\in B(\mathbb{R})$。

$\mathcal{B}(\mathbb{R}^n)$为包含所有左开右闭区间笛卡尔积形成的矩形的最小$\sigma$域,Borel域中的集合称Borel集。

\begin{defi} 随机变量、概率分布函数
	
$(\Omega,\mathcal{F},P)$中,称$X:\Omega\to\mathbb{R}$为一个随机变量,若$\forall x\in\mathbb{R}$,有$\{\omega\in\Omega:X(\omega)\le x\}\in\mathcal{F}$。

此时记后方集合为$\{X\le x\}$,称$F(x)=P(\{X\le x\})$为随机变量$X$的(概率)分布函数。
\end{defi}

\begin{exmp} 掷均匀硬币
	
$\Omega=\{H,T\},X:\Omega\to\mathbb{R},X(H)=1,X(T)=-1$

$P(\{X\le x\})=\begin{cases}1&x\ge1\\0.5&-1\le x<1\\0&x<-1\end{cases}$
\end{exmp}

\begin{thm} 分布函数$F(x)$性质

1. 单调增

2. 负无穷极限0,正无穷极限1

3. 右连续 $\lim_{\sigma\to0^+}F(x+\sigma)=F(x)$
\end{thm}

证明:

1. 利用包含关系说明。

2. 取一列数趋向正/负无穷,利用概率的极限等于极限的概率知结论。

3. 类似2,取子列说明。

注:

(1) 若某函数这三条性质,一定为某随机变量的概率分布函数,因此,一般将满足三条性质的函数称为分布函数。

(2) 另一种定义分布函数的方式:$G(x)=P(\{X<x\})$,此时其具有\textbf{左}连续性,一二两条不变。

(3) 分布函数\textbf{丢失}了关于样本空间的信息,与样本空间无关。

\begin{exmp}
若$X=c$概率为1,称$X$几乎处处常值,则$F(x)=\begin{cases}1&x\ge c\\0&x<c\end{cases}$
\end{exmp}

\begin{exmp} Bernoulli两点分布
	
若$P(X=1)=p,P(X=0)=q,p+q=1$,则$F(x)=\begin{cases}1&x\ge1\\q&0\le x<1\\0&x<0\end{cases}$

特别地,$A\in\mathcal{F}$,示性函数$I_A(\omega)=\begin{cases}1&\omega\in A\\0&\omega\notin A\end{cases}$亦为随机变量,满足Bernoulli两点分布,且有$P(I_A=1)=P(A)$。
\end{exmp}

\begin{thm} 随机变量性质

1. $P(X>x)=1-F(x)$

2. $P(x<X\le y)=F(y)-F(x)$

3. $P(X=x)=F(x)-F(x^-)$
\end{thm}

\begin{thm}
设$X$为$(\Omega,\mathcal{F},P)$上随机变量,则任意Borel集的原象为事件域中元素。
\end{thm}

证明:记$\mathcal{A}=\{A\subset\mathbb{R}:X^{-1}(A)\in\mathcal{F}\}$,由于$X^{-1}(A^\mathrm{c})=X^{-1}(A)^\mathrm{c},X^{-1}\left(\bigcup_nA_n\right)=\bigcup_nX^{-1}(A_n)$分别验证三条性质可知$\mathcal{A}$为$\mathbb{R}$上$\sigma$域,。因此,$(a,b]=(-\infty,b]-(-\infty,a]\in\mathcal{A}$,进一步可知$B(\mathbb{R})\subset A$,即可得证。

*随机变量相加后仍为随机变量

证明:令$r_n$为一切有理数的一个排列,证出$\{X+Y\le x\}=\bigcap_{n=1}^\infty(\{X\le r_n\}\cup\{Y\le x-r_n\})$即可说明结论。

右包含左易证,故只需证明$\omega$不属于左侧时亦不属于右侧。若其不属于左侧,取$m$使$X(\omega)>r_m>Y(\omega)$即发现其不属于右侧。

\subsection{随机向量}
\begin{defi} 离散型随机变量

随机变量$X$取值至多可列个$x_1,x_2,\dots$,则称$X$为离散型随机变量。

记$p_k=P(X=x_k)$,则$\{p_k\}$为$X$的分布列,此时分布函数$F(x)=\sum_{k: x_k\le x}p_k$在$x_k$处跳跃,又称\emph{原子分布}。
\end{defi}

\begin{defi} 连续型随机变量

若随机变量$X$的分布函数$F(x)=\int_{-\infty}^xf(u)\mathrm{d}u$,其中$f$非负可积,则称$X$为连续型随机变量,称$f$为$X$的\emph{密度函数}。
\end{defi}

注:

(1) 密度函数含义:当$x=x_0$为$f$连续点时,$\Delta x\to0$,则$P(x_0<X\le x_0+\Delta x)=f(x_0)\Delta x$。

(2) 密度函数改变有限多个值仍为密度函数。

(3) $P(X=a)\le\int_{a-1/n}^af(u)\mathrm{d}u\to0$,因此$X$在任意有限多个点取值概率为0。

(4) 若$F$连续且除去有限多个点外$F^\prime(x)$存在且连续,则$X$为连续型随机变量,且$F^\prime$可作为一个密度函数。

(5) $X$为连续型随机变量,则$F$绝对连续。

\begin{exmp} 钟表指针
	
$\Omega=[0,2\pi),\mathcal{F}=B(\mathbb{R})\cap\Omega,P(A)=\frac{|A|}{2\pi}$,$|A|$指勒贝格测度。

令$X(\omega)=\omega,Y(\omega)=\omega^2$,则

$F_X(x)=\begin{cases}x<0&0\\\frac{x}{2\pi}&x\in(0,2\pi]\\x\ge2\pi&1\end{cases},F_Y(x)=\begin{cases}x<0&0\\\frac{\sqrt{y}}{2\pi}&x\in(0,4\pi^2]\\x\ge4\pi^2&1\end{cases}$,求导知$f_X,f_Y$。
\end{exmp}

分布函数$F$性质:

(1) 单调$\to$不连续点至多可数

(2) \textbf{勒贝格分解} $F=c_1F_d+c_2F_c+c_3F_s$其中$c_i\ge0,\sum_ic_i=1$,$F_d$为离散型随机变量的分布函数,$F_c$为连续型随机变量的分布函数,$F_s$为奇异的。

~

\begin{defi} 随机向量

$X_1,\dots,X_n$为$(\Omega,\mathcal{F},P)$上的随机变量,称$\vec{X}=(X_1,\dots,X_n)$为$n$维随机向量,$F(x_1,\dots,x_n)=P(X_1\le x_1,\dots,X_n\le x_n)$为$\vec{X}$的联合分布函数。

离散型:$\vec{X}$取值至多可列多个,\emph{联合分布列}$f(x_1,\dots,x_n)=P(X_1=x_1,\dots,X_n=x_n)$。

连续型:$F(x_1,\dots,x_n)=\int_{-\infty}^{x_1}\cdots\int_{-\infty}^{x_n}f(u_1,\dots,u_n)\mathrm{d}u_1\dots\mathrm{d}u_n$,$f$非负可积,称$f$为$X$的\emph{联合密度函数}。
\end{defi}

\begin{thm}
考虑二维随机向量的联合分布函数$F(x,y)$:

1. $F(x,y)$关于$x,y$均单调增。

2. $F(x,y)$关于$x,y$均右连续。

3. $F(x,y)$在$x,y$趋近负无穷时极限均为0,$x,y$均趋近正无穷时极限为1。

4. $x_1\le x_2,y_1\le y_2$ 时$F(x_2,y_2)-F(x_1,y_2)-F(x_2,y_1)+F(x_1,y_1)=P(X\in(x_1,x_2],Y\in(y_1,y_2])\ge0$
\end{thm}

注:

(1) 取极限可发现4\textbf{蕴含}1,反之不然(举例:$F(x)=\begin{cases}1&x+y\ge0\\0&x+y<0\end{cases}$满足1,2,3但不满足4)。

(2) 若某二元函数满足2,3,4三条性质,一定为某随机向量的联合分布函数。

$F_X(x)=P(X\le x)=\lim_{y\to+\infty}F(x,y)$称为\textbf{边际分布}。

连续型随机变量$F_X(x)=\int_{-\infty}^x\int_{-\infty}^{+\infty}f(u,v)\mathrm{d}v\mathrm{d}u$,$f_X(x)=\int_{-\infty}^{+\infty}f(x,v)\mathrm{d}v$称为\textbf{边际密度}。

\begin{exmp} 三项分布

$\Omega=\{H,T,E\}$,均匀“三面硬币”,设扔$n$次后三种次数分别为$H_n,T_n,E_n$,有$H_n+E_n+T_n=n$,则

$P\big((H_n,T_n,E_n)=(h,t,e)\big)=\frac{n!}{h!t!e!}\frac{1}{3^n}$
\end{exmp}

\begin{exmp}
$G\subset\mathbb{R}^n$为有限区域,则联合密度函数$f(x_1,\dots,x_n)=\frac{1}{|G|},\vec{x}\in G$

特别地,$G=[0,1]^2$时,$f(x,y)=\begin{cases}1&(x,y)\in G\\0&(x,y)\notin G\end{cases}$
\end{exmp}

\section{离散型随机变量}
\subsection{分布列与独立性}
回顾:离散型随机变量$X$取值至多可列个$x_1,x_2,\dots$,记$p_k=P(X=x_k)$,则$\{p_k\}$为$X$的分布列。

\begin{exmp} 二项分布

$P(x=k)=\mathrm{C}_n^kp^kq^{n-k},p+q=1$时称$X$符合二项分布,记为$X\sim B(n,p)$。

背景:抛$n$次硬币,$X$为正面向上次数,
\end{exmp}

\begin{exmp} 几何分布

$P(X=k)=q^{k-1}p,p+q=1$时称$X$符合几何分布,此时$P(X>k)=q^k$。

背景:抛$n$次硬币,$X$为第一次正面向上时抛的次数。

几何分布具有\emph{无记忆性}:$P(X-m=k|X>m)=P(X=k)$。反之,若取值为$\mathbb{N}^*$的某随机变量满足无记忆性,即对任意$m,k$符合上式,则必须服从几何分布。
\end{exmp}

\begin{exmp} 泊松分布

$P(X=k)=\frac{\lambda^k}{k!}e^{-\lambda},\lambda>0$时称$X$符合泊松分布,记为$X\sim P(\lambda)$。

背景:网站访问量、百科新词条
\end{exmp}

*放射性粒子数:体积为$V$的小物块分为$n$等份,每一小块$\Delta v=\frac{V}{n}$,假设每一小块在7.5s内放出1个$\alpha$粒子的概率为$p=\mu\cdot\Delta v$,放出更多概率为0,且各小块放出与否相互独立。

分析:$n$块共放出$k$个概率符合二项分布,令$\lambda=\mu V$,则

$P(X=k)=\mathrm{C}_n^kp^k(1-p)^{n-k}=\frac{n\dots(n-k+1)}{k!}\left(\frac{\lambda}{n}\right)^k\left(1-\frac{\lambda}{n}\right)^{n-k}$,固定$k$,令$n$趋向无穷,此式极限即为$\frac{\lambda^k}{k!}e^{-\lambda}$。由此可知,二项分布可以\textbf{逼近}泊松分布。

~

\begin{defi} 独立性

若$\forall x,y\in\mathbb{R},P(X=x,Y=y)=P(X=x)P(Y=y)$,则称离散型随机变量$X,Y$独立。

更一般,称$X_1,\dots,X_n$相互独立,若$\forall x_i\in\mathbb{R},P(X_1=x_1,\dots,X_n=x_n)=P(X_1=x_1)\cdots P(X_n=x_n)$。
\end{defi}

\begin{thm}
离散型随机变量$X,Y$独立,当且仅当$P(X\le x,Y\le y)=P(X\le x)P(Y\le y)\Leftrightarrow F(x,y)=F_X(x)F_Y(y)$。
\end{thm}

证明:利用分布列$f(x,y)$与分布函数$F(x,y)$关系,利用求和可证明仅当,利用左极限可证明当。

\begin{exmp} 泊松翻转

抛均匀硬币1次,记$X,Y$为正反出现的次数,计算容易发现不独立。

抛$N$枚均匀硬币,$N\sim P(\lambda)$,计算$f(x,y)=P(X=x,Y=y,N=x+y)=\frac{\lambda^{x+y}}{(x+y)!}e^{-\lambda}\frac{\mathrm{C}_{x+y}^x}{2^{x+y}}$

$=\left(\frac{\lambda}{2}\right)^x\frac{e^{-\lambda/2}}{x!}\left(\frac{\lambda}{2}\right)^y\frac{e^{-\lambda/2}}{y!}$,注意到$f_X(x)=\sum_yf(x,y)=\left(\frac{\lambda}{2}\right)^x\frac{e^{-\lambda/2}}{x!}$,由此知$X,Y$独立。
\end{exmp}

\begin{thm} 
离散型随机变量$X,Y$独立,$g,h$是$\mathbb{R}$上的Borel可测函数,则$g(X),h(Y)$独立。
\end{thm}

证明:$P(g(X)=a,h(Y)=b)=P\bigg(\bigcup_{g(x)=a}\{X=x\},\bigcup_{h(y)=a}\{Y=y\}\bigg)$分解求和。

\subsection{数学期望}
\begin{defi} 数学期望

离散型随机变量$X$对应分布列$f$,$\sum_{x:f(x)>0}xf(x)$若\emph{绝对收敛},则称为$X$的数学期望,记为$E[X]$。
\end{defi}

*$E[x]=\sum_kx_kp_k$(原则上$x_i$互不相同,事实上相同不会影响计算)

\begin{thm} 佚名统计学家公式
	
$g$为$\mathbb{R}$上函数,$Y=g(X)$,$X$分布列为$f$,则$E[Y]=\sum_xg(x)f(x)$(假定右侧绝对收敛)。
\end{thm}

证明:考虑$Y$分布列即可。

\begin{defi} 数字特征

离散型随机变量$X$的$k$阶矩为$m_k=E[X^k]$,$k$阶中心矩$\sigma_k=E[(X-m_1)^k]$。

方差$\Var(X)$为二阶中心矩,$\Var(X)=\sum_x(x-m_1)^2f(x)=E[X^2]-2m_1E[X]+m_1^2=E[X^2]-E^2[X]$。

标准差定义为$\sqrt{\Var(X)}$
\end{defi}

\begin{exmp} Bernoulli分布
	
$P(X=1)=p,P(X=0)=q,p+q=1\Rightarrow E[X]=p,\Var(X)=p-p^2=pq$
\end{exmp}

\begin{exmp} 二项分$X\sim B(n,p)$
	
$E[X]=\sum_{k=0}^nk\mathrm{C}_n^kp^kq^{n-k}=np\sum_{k=0}^{n-1}\mathrm{C}_{n-1}^kp^kq^{n-1-k}=np$

$E[X(X-1)]=\sum_{k=0}^nk(k-1)\mathrm{C}_n^kp^kq^{n-k}=n(n-1)p^2$

$E[X^2]=np(np+q),\Var(X)=npq$
\end{exmp}

\begin{thm} 数学期望性质

1. 非负性:$X\ge0\Rightarrow E[X]\ge0$

2. 归一性:$E[1]=1$

3. 线性性:$E[aX+bY]=aE[X]+bE[Y]$

由此,$E$可以看作一个期望算子。
\end{thm}

线性性证明:令\textbf{示性函数}$A_x=\{X=x\}$,则$X=\sum_xxI_{A_x},E[X]=\sum_xxP(A_x)$,对$Y$用$B_y$类似处理,则$aX+bY=\sum_xxI_{A_x}+\sum_yyI_{B_y}=\sum_{Ax+By}I_{A_x}I_{B_y}$。

*观点:扩展至量子物理、\textbf{非线性}期望

\begin{thm}
$X,Y$独立且期望存在,$E[XY]=E[X]E[Y]$。
\end{thm}

证明:$E[X]=\sum_xxI_{A_x},E[Y]=\sum_yyI_{B_y}\Rightarrow E[X]E[Y]=\sum_{x,y}xyI_{A_xB_y}$。

\begin{thm} 方差性质

1. $\Var(aX+b)=a^2\Var(X)$

2. $\Var(X+Y)=\Var(X)+\Var(Y)+2(E(XY)-E(X)E(Y))$,$X,Y$独立时即可加。
\end{thm}

\begin{exmp} 期望不存在的例子

$P(X=x_k)=\frac{1}{2^k}$,$x_k=(-1)^k\frac{2^k}{k}$,不绝对收敛,$\sum_kx_kp_k=-\ln2$,而期望不存在。

若$P(X=x_k)=2^{k-1}$,可发现期望趋向于无穷。
\end{exmp}

\subsection{协方差}
\begin{exmp}
考虑一个随机的$n$元置换$\pi$,记不动点$\pi(x)=x$的个数为$N$,求分布列$P(N=r),r=0,1,\dots,n$。
	
记$A_k=\{k$为不动点$\}$,对应的示性函数为$I_k$,令$X=\sum_{i_1<\dots<i_r}^{i_{r+1}<\dots<i_n}I_{i_i}I_{i_2}\dots I_{i_r}(1-I_{i_{r+1}})\dots(1-I_{i_n})$,其中$i_k$为$1$到$n$的排列,则$X=I_{\{N=r\}}$。
	
	
$E[X]=\mathrm{C}_n^rE[I_1\dots I_r(1-I_{r+1})\dots(1-I_n)]=\mathrm{C}_n^r\sum_{s=0}^{n-r}(-1)^s\mathrm{C}_{n-r}^s\frac{(n-r-s)!}{n!}=\frac{1}{r!}\sum_{s=0}^{n-r}\frac{(-1)^s}{s!}$
\end{exmp}

\begin{exmp} 不计算分布列,求上例中的期望与方差。
	
$N=I_1+I_2+\dots+I_n$。则$E[N]=nE[I_1]=n\frac{(n-1)!}{n!}=1,E[N^2]=\sum_{i,j=1}^nE[I_iI_j]=nE[I_1^2]+n(n-1)E[I_1I_2]=1+1=2,\Var(N)=1$。
\end{exmp}

\begin{exmp} Erdos概率方法:正17边形染红5个顶点,证明存在7个相邻顶点中至少有3个红点。
	
建立模型,17个点中等概率随机取一个,令$I_k$为$k$为红色的示性函数,再令$X(k)=I_{k+1}+\dots+I_{k+7}$,则$E[X]=\frac{35}{17}>2$,由此$P(X>2)>0$,得证。
\end{exmp}

类似随机变量的情况,对随机向量有结论:$f(x,y)$为$(X,Y)$的联合分布列,$g:\mathbb{R}^2\to\mathbb{R}$的Borel可测函数,则$E[g(X,Y)]=\sum_{x,y}g(x,y)f(x,y)$(假定右侧绝对收敛)。

\begin{defi} 协方差与相关系数

协方差$\Cov(X,Y)=E[(X-E[X])(Y-E[Y])]=E[XY]-E[X]E[Y]$

相关系数$\rho(X,Y)=\frac{\Cov(X,Y)}{\sqrt{\Var(X)\Var(Y)}}$,为0时称两变量不相关。
\end{defi}

*对$n$维随机向量$\vec{X}=(X_1,\dots,X_n)$,协方差矩阵$\Sigma=(\sigma_{ij}),\sigma_{ij}=\Cov(X_i,X_j)$

性质:协方差矩阵对称且非负定。

非负定性证明:$\sum_{i,j}t_it_j\sigma_{ij}=\sum_{i,j}t_it_jE[(X_i-E[X_i])(X_j-E[X_j])]=E\left[\left(\sum_it_i(X_i-E[X_i])\right)^2\right]\ge0$

\begin{thm} 相关系数性质

1. $|\rho|\le1$

2. $X,Y$独立或不相关$\Leftrightarrow\rho=0$

3. $\rho=\pm1\Leftrightarrow P(Y=aX+b)=1$
\end{thm}

\begin{thm} Cauchy不等式

$(E[XY])^2\le E[X^2]E[Y^2]$,等号成立当且仅当存在不全为0实数$a,b$使得$P(aX=bY)=1$。
\end{thm}

证明:先利用佚名统计学家公式说明$E[X^2]=0$时可推出$P(X=0)=1$,从而计算得$E[XY]=0$;若$E[X^2]>0$,$E[(Y-tX)^2]=t^2E[X^2]-2tE[XY]+E[Y^2]\ge0$,利用判别式得证,取等时再利用$E[X^2]=0$的条件即可。

\begin{exmp} 多项分布

$\vec{X}=(X_1,\dots,X_r),P(X_1=k_1,\dots,X_r=k_r)=\frac{n!}{k_1!\dots k_r!}p_1^{k_1}\dots p_r^{k_r}$,其中$\sum_ip_i=1,\sum_ik_i=n$。

计算$\Cov(X_i,X_j),\rho(X_i,X_j)$。

$i\neq j$时$X_i\sim B(n,p_i),X_i+X_j\sim B(n,p_i+p_j)$,由此$E[X_i]=np_i,\Var(X_i)=np_i(1-p_i),\Var(X_i+X_j)=n(p_i+p_j)(1-p_i-p_j)$,利用定理3.6可知$\Cov(X,Y)=\frac{1}{2}(\Var(X+Y)-\Var(X)-\Var(Y))=-np_ip_j$,相关系数为$-\sqrt{\frac{p_ip_j}{(1-p_i)(1-p_j)}}$。
\end{exmp}

\subsection{条件分布与条件期望}
\begin{defi} 条件分布

$(X,Y)$为离散型随机向量,给定$X=x$,且$P(X=x)>0$,$Y$的条件分布列$f_{Y|X}(y|x)=P(Y=y|X=x)$,对应条件分布函数$F_{Y|X}(y|x)=P(Y\le y|X=x)$
\end{defi}

注:$\sum_yf_{Y|X}(y|x)=1$,其亦为一个分布函数。

$f_{Y|X}(y|x)=\frac{f(x,y)}{f_X(x)},f_X(x)>0$

\begin{defi} 条件期望

给定$X=x$下,$Y$的条件期望$\psi(x)=E[Y|X=x]=\sum_yyf_{Y|X}(y|x)$(假设其绝对收敛),并称$\psi(X)$为$Y$关于$X$的条件期望,记为$E[Y|X]$。
\end{defi}

\begin{thm} 全期望公式

$E[E[Y|X]]=E[Y]$ *另一种形式:$E[Y]=\sum_xf_X(x)E[Y|X=x]$
\end{thm}

证明:$E[\psi(X)]=\sum_x\psi(x)f_X(x)=\sum_xf_X(x)\sum_yyf_{Y|X}(y|x)=\sum_{x,y}yf(x,y)=\sum_yyf_Y(y)$,注意步骤中交换次序要求绝对收敛。

*类似可定义关于事件的条件期望$E[X|A]=E[X|I_A=1]$。

\begin{exmp} 多项分布的条件期望$i\neq j,E[X_j|X_i>0]$

利用全期望公式$E[X_j]=P(X_i=0)E[X_j|X_i=0]+P(X_i>0)E[X_j|X_i>0]$。又由于$P(X_j=k|X_i=0)=\frac{P(X_j=k,X_i=0)}{P(X_i=0)}=\mathrm{C}_n^k\left(\frac{p_j}{1-p_1}\right)^k\left(1-\frac{p_j}{1-p_1}\right)^{n-k}$,因此$E[X_j|X_i=0]=\frac{np_j}{1-p_i}$,进而算出$E[X_j|X_i>0]=\frac{np_j(1-(1-p_i)^{n-1})}{(1-(1-p_i)^n)}$
\end{exmp}

\begin{exmp}
鸟下$N$枚蛋,$N\sim P(\lambda)$,每颗蛋独立以概率$p$变为小鸟,记$K$为小鸟数,计算$E[K|N]$、$E[K]$、$E[N|K]$。

记$q=1-p$,由于$f_{K|N}(k|n)=\mathrm{C}_n^kp^kq^{n-k}$,因此$E[K|N=n]=np$,即$E[K|N]=Np$,进而$E[K]=E[E[K|N]]=pE[N]=p\lambda$。

而$f_{N|K}(n|k)=\frac{P(N=n,K=k)}{P(K=k)}=\frac{P(K=k|N=n)P(N=n)}{\sum_mP(K=k|N=m)P(N=m)}=\frac{(\lambda q)^{n-k}e^{-\lambda q}}{(n-k)!}$,由此$E[N|K=k]=\sum_{n\ge k}n\frac{(\lambda q)^{n-k}e^{-\lambda q}}{(n-k)!}=\sum_{n\ge 0}(n+k)\frac{(\lambda q)^ne^{-\lambda q}}{n!}=\lambda q+k$,即$E[N|K]=\lambda q+K$。
\end{exmp}

\begin{thm}
记$\psi(X)=E[Y|X]$,$g$保证所述期望均存在,则$E[\psi(X)g(X)]=E[Yg(X)]$。
\end{thm}

\subsection{随机游走}
$S_0=a,S_n=S_{n-1}+X_n=S_0+\sum_{i=1}^nX_i$,$\{X_i\}$相互独立且同分布,取值为$\pm1$,$P(X_i=1)=p,P(X_i=-1)=q,p+q=1$。

*直线上简单随机游走,当$p=\frac{1}{2}$时称为\textbf{对称的}。

\begin{exmp} 自由随机游走,$S_0=a$,求$P(S_n=b)$,已知$2|a+b+n$。

设向右游走次数为$r$,向左游走次数为$l$,则$\begin{cases}
r+l=n\\r-l=b-a\end{cases}\Rightarrow r=\frac{n+b-a}{2}$,因此$P(S_n=b)=\mathrm{C}_n^{(n+b-a)/2}p^{(n+b-a)/2}q^{(n-b+a)/2}$
\end{exmp}

\begin{thm} 简单随机游走性质
	
1. 空间齐性 $P(S_n=j+b|S_0=a+b)=P(S_n=j|S_0=a)$

2. 时间齐性 $P(S_{n+m}=j|S_m=a)=P(S_n=j|S_0=a)$

3. 马氏性 $P(S_{m+n}=j|S_0=j_0,\dots,S_m=j_m)=P(S_{m+n}=j|S_m=j_m)$

*要求等式两边有意义
\end{thm}

证明:利用$P(S_n=j|S_0=a)=P(\sum_{i=1}^nX_i=j-a|S_0=a)$转化即可计算得出结果。

~

*\textbf{轨道计数}

平面表示$\{(n,S_n):n=0,1,\dots\}$

记$N_n(a,b)$为$(0,a)\to(n,b)$轨道个数,$N_n^\circ(a,b)$为$(0,a)\to(n,b)$且访问$x$轴至少一次的轨道个数。

\begin{thm} 反射原理

若$a,b>0$,则$N_n^\circ(a,b)=N_n(-a,b)$。
\end{thm}

证明:寻找第一个交点,利用一一对应。

\begin{thm}
$N_n(a,b)=\mathrm{C}_n^{(n+b-a)/2}$
\end{thm}

~

*关心问题:\textbf{返回出发点}、\textbf{游走最远距离}、\textbf{首次击中某点}

\begin{thm} 投票定理

$b>0$,则$(0,0)\to(n,b)$不再过$x$轴的轨道个数为$(1,1)\to(n,b)$不过$x$轴的轨道个数,即$N_{n-1}(1,b)-N_{n-1}^\circ(1,b)=\frac{b}{n}N_n(0,b)$。
\end{thm}

\begin{exmp}
A得票$a$,B得票$b<a$,求A得票始终大于B的概率。

问题可化为$(0,0)$到$(a+b,a-b)$轨道中不再过$x$轴轨道数与总数之比,为$\frac{a-b}{a+b}$。
\end{exmp}

\begin{thm} 不返回出发点

$S_0=0,n\neq1$,则$P(S_1\dots S_n\neq 0,S_n=b)=\frac{|b|}{n}P(S_n=b)$。

由此可推得$P(S_1\dots S_n\neq0)=\frac{E[|S_n|]}{n}$
\end{thm}

证明:利用投票定理计算即可。

~

*记最到达的最右端$M_n=\max\{S_i:0\le i\le n\}$,$S_0=0\Rightarrow M_n\ge0$

\begin{thm} 最右端

$P(M_n\ge r,S_n=b)=\begin{cases}P(S_n=b)&b\ge  r\\\left(\frac{q}{p}\right)^{r-b}P(S_n=2r-b)&b<r\end{cases}$
\end{thm}

证明:考虑第一次到达$r$的点,反射其后的部分,即可与$S_n=2r-b$的轨道数量一一对应,而反射产生的概率差别为$\left(\frac{q}{p}\right)^{r-b}$。

*推论:$p=q$时可以计算出$P(M_n\ge r)=P(S_n=r)+2P(S_n\ge r+1)$。

~

\begin{thm} 首中时定理

$S_0=0$,时刻$n$首次击中$b$概率为$f_b(n)=\frac{|b|}{n}P(S_n=b)$。
\end{thm}

证明:不妨设$b>0$,注意条件等价于$M_{n-1}=S_{n-1}=b-1,S_n=b$即可算出结果。

\begin{thm} 反正弦律
	
对称随机游走,$S_0=0$,记$T_{2n}=\max\{0\le i\le2n:S_i=0\}$,则$P(T_{2n}=2k)=P(S_{2k}=0)P(S_{2n-2k}=0)$
\end{thm}

证明:利用$P(S_{2k+1}\dots S_{2n}\neq0|S_{2k}=0)=P(S_1\dots S_{2n-2k}\neq0|S_0=0)$。此外$P(S_1\dots S_{2m}\neq0)=P(S_2\dots S_{2m}\neq0|S_1=1)=P($第$2m$次后首次击中$-1)$,由此可计算出结论。

*此分布称为\textbf{反正弦律}

利用Stirling公式可计算$P(T_n\le2xn)\sim\frac{1}{n}\sum_{\frac{k}{n}<x}\frac{1}{\pi}\frac{1}{\sqrt{\frac{k}{n}(1-\frac{k}{n})}}\sim\int_0^x\frac{1}{\pi\sqrt{u(1-u)}}\mathrm{d}u=\frac{2}{\pi}\arcsin\sqrt{x}$,$\frac{T_{2n}}{2n}$渐近分布$\frac{2}{\pi}\arcsin\sqrt{x}$。

~

$\mathbb{Z}^d$上的随机游走:$S_n=S_{n-1}+X_n=S_0+\sum_{i=1}^nX_i$,向量$\{X_i\}$相互独立且同分布,仅有一个不为0且取值为$\pm1$的分量。

\begin{exmp} 平面上对称随机游走,求$P(S_{2n}=\mathbf{0})$。

考虑上下左右次数知结果为$\sum_{k=0}^n\frac{(2n)!}{k!k!(n-k)!(n-k)!}\frac{1}{4^{2n}}=C_{2n}^n\frac{1}{4^{2n}}\sum_{k=0}^n(\mathrm{C}_n^k)^2=\frac{1}{4^{2n}}(\mathrm{C}_{2n}^n)^2$。
\end{exmp}

\subsection{母函数}
*数列的母函数:$\{a_i,i\in\mathbb{N}\}$,母函数$G_a(s)=\sum_{i=0}^\infty a_is^i$,如$\mathrm{C}_n^i$对应$(1+s)^n$

\textbf{卷积}:$c_n=\sum_{k=0}^na_kb_{n-k}$,记为$c=a*b$,验证有$G_c(s)=G_a(s)G_b(s)$。

\begin{exmp} 随机游走$S_0=0$,求$S_{2n}=0,S_i\ge0$的轨道个数$C_n$。

设首次返回原点为$2k$时,则考虑第一步后可知从0到$2k$的轨道个数为$C_{k-1}$,因此有$C_n=\sum_{k=0}^{n-1}C_{k-1}C_{n-k}$,于是$G(s)-1=sG^2(s)$,考虑合理解知$G(s)=\frac{1-\sqrt{1-4s}}{2s}$,展开可得$C_n=\frac{(2n)!}{n!(n+1)!}$(\emph{卡特兰}数)。
\end{exmp}

~

\begin{defi} 非负整值随机变量的母函数

$G_X(s)=E[S^X]=\sum_{i=1}^\infty s^iP(X=i)=\sum_{i=1}^\infty f(i)s^i$
\end{defi}

*其收敛半径$R\ge1$,在收敛域内可微

*$G(0)=P(X=0),G(1)=1,f(i)=\frac{G^{(i)}(0)}{i!}$

*只要系数非负,$G(1)=1$,即可看作概率母函数

\begin{enumerate}
	\item 二项分布$B(n,p)$母函数$(ps+q)^n$
	\item 几何分布$P(X=i)=pq^{i-1}$母函数$\frac{ps}{1-qs}$
	\item 泊松分布$P(\lambda)$母函数$e^{\lambda(s-1)}$
\end{enumerate}

\begin{thm} 母函数与数字特征矩

1. $E[X]=G'(1)$

2. $E[x(x-1)\dots(x-k+1)]=G^{(k)}(1)$

3. $\Var(X)=G"(1)-G'(1)(1-G'(1))$

此处$G^{(k)}(1)$指$\lim_{x\to1^-}G^{(k)}(x)$,由阿贝尔定理此定义合理。
\end{thm}

证明:直接求导即可说明。

\begin{thm} 随机变量卷积

设非负整值随机变量$X_1,\dots,X_n$互相独立,$Y=\sum_{k=1}^nX_k$,则$G_Y=\prod_{k=1}^nG_{X_k}$。
\end{thm}

证明:利用卷积归纳即可。

\begin{thm} 复合分布
	
设$X_i$独立同分布,母函数$G_X$,$N$与$X_i$独立,母函数$G_N$,$Y=\sum_{k=1}^NX_k$母函数为$G_N(G_X)$。
\end{thm}

证明:利用全概率公式分解计算。

\begin{defi} 联合母函数

$X,Y$联合母函数$G(s,t)=E[s^xt^y]=\sum_{i,j}s^it^jP(X=i,Y=j)$。
\end{defi}

\begin{thm} $X,Y$独立等价于$G(s,t)=G_X(s)G_Y(t)$
\end{thm}

证明:比较系数。

\begin{exmp} 掷三颗均匀骰子,求点数和分布。

设每个骰子点数$X_i$,和为$Y$,则$G_Y(s)=G_X^3(s)=\frac{s^3(1-s^6)^3}{6^3(1-s)^3}$,考虑每项系数即为分布。
\end{exmp}

*二项分布\textbf{再生性}:独立变量$X_1\sim B(n_1,p),X_2\sim B(n_2,p)$,考虑母函数可发现$X_1+X_2\sim B(n_1+n_2,p)$。

泊松分布也有再生性:$X_1,X_2$独立泊松分布$P(\lambda_1),P(\lambda_2)$,其和分布为$P(\lambda_1+\lambda_2)$。

*离散随机变量的母函数是否可以定义为$G(s)=\sum_is^{x_i}P(X=x_i)$?其他类型随机变量呢?(第五章内容)

\section{连续型随机变量}
\subsection{独立性}
$X$连续型随机变量,有\textbf{密度函数}$f$,非负且在实轴上积分为1,分布函数$F(x)=\int_{-\infty}^xf(u)du$。

*$P(X=x)=0,P(a<X<b)=\int_a^bf(u)du$

*\textbf{均匀分布}$X\sim U[a,b]$

$f(x)=\frac{1}{b-a},x\in[a,b]$

*\textbf{指数分布}$X\sim \Exp(\lambda)$

$f(x)=\lambda\mathrm{e}^{-\lambda x},F(x)=1-\mathrm{e}^{-\lambda x}$

背景:$P(t\le X\le t+\Delta t|X>t)=\lambda\Delta t$微分方程(如半衰期等)

也具有\textbf{无记忆性}

*\textbf{正态分布}$X\sim N(\mu,\sigma^2)$

$f(x)=\frac{1}{\sqrt{2\pi\sigma^2}}\exp\left(-\frac{1}{2\sigma^2}(x-\mu)^2\right)$

$\mu$为对称轴且为最大值点

$\mu\pm\sigma$为拐点

背景:\textbf{随机误差}分布

Wigner\textbf{半圆律}

$f(x)=\frac{1}{2\pi\sigma^2}\sqrt{4\sigma^2-x^2}$

背景:随机矩阵、自由概率论(和正态分布同地位)

\begin{exmp}
半圆律中$P(X\in(0,\sigma))$

积分可得结果为$\frac{1}{6}+\frac{\sqrt3}{4\pi}$
\end{exmp}

\begin{exmp} $X\sim N(0,1)$,求$Y=X^2$密度函数

$P(Y\le y)=P(-\sqrt{y}<X<\sqrt{y})=\int_{-\sqrt{y}}^{\sqrt{y}}\frac{1}{\sqrt{2\pi}}e^{-x^2/2}\mathrm{d}x$,求导得密度函数为$\frac{\mathrm{e}^{-y/2}}{\sqrt{2y\pi}}$。
\end{exmp}

*正态分布常用$\phi(x)$表示密度函数,$\Phi(x)=\int_{-\infty}^x\phi(u)\mathrm{d}u$,$N(0,1)$称标准正态分布。

*$X\sim U(0,1)$,计算可得$Y=\Phi^{-1}(X)\sim N(0,1)$

~

\begin{defi} 一般随机变量的独立性
	
$X_1,\dots,X_n$满足$P(X_1\le x_1,\dots,X_n\le x_n)=\prod_{k=1}^nP(X_k\le x_k)$,则称随机变量相互独立。
\end{defi}

\begin{thm} 等价刻画
$X_1,\dots,X_n$相互独立等价于$\forall B_1,\dots,B_n\in B(\mathbb{R}),P(\forall i,X_i\in B_i)=\prod_{k=1}^nP(X_k\in B_k)$
\end{thm}

(证明须更多测度论知识)

\begin{thm}
设$g_1,\dots,g_n$均Borel可测,$X_1,\dots,X_n$独立,则$g_1(X_1),\dots,g_n(X_n)$独立。
\end{thm}

证明:利用Borel集原像为Borel集可由上一定理说明。

\begin{thm} 连续型随机变量的独立

$X_1,\dots,X_n$密度函数$f_1,\dots,f_n$,独立等价于$f_1\dots f_n$为联合密度函数。
\end{thm}

证明:利用多重积分性质计算。

*\textbf{卷积}:$X,Y$独立时,称$Z=X+Y$为其卷积,$f_Z=f_X*f_Y$

\begin{thm}
$(X,Y)$密度函数为$f(x,y)$,则$f_{X+Y}(x)=\int_\mathbb{R}f(x,z-x)\mathrm{d}x=\int_\mathbb{R}f(z-y,y)\mathrm{d}y$
\end{thm}

证明:利用二重积分变量代换公式计算。

*$X,Y$独立时,$f(x,y)=f_X(x)f_Y(y)$,由此$X+Y$密度函数成为两函数卷积

\begin{exmp} $X,Y$独立标准正态分布,求$X+Y$分布函数。

利用定理计算积分得$X+Y\sim N(0,2)$。
\end{exmp}

\subsection{期望}
\begin{defi} 连续型随机变量的期望

当$\int_\mathbb{R}xf(x)\mathrm{d}x$绝对收敛,定义其为$X$的数学期望$E[X]$。
\end{defi}

\begin{thm}
$X$有密度函数$f$,期望存在,则$E[X]=\int_0^\infty(1-F_X(x))\mathrm{d}x-\int_{-\infty}^0F_X(x)\mathrm{d}x$。
\end{thm}

证明:利用分部积分公式计算。

\begin{thm} 复合的期望
	
$g$为Borel可测函数,$X$与$g(X)$均为连续型随机变量,则$E[g(X)]=\int_\mathbb{R}g(x)f_X(x)\mathrm{d}x$
\end{thm}

证明:利用上一定理由$F$推导计算。

\begin{thm} 连续型随机向量情形

$g$是二元Borel可测函数,$X,Y$联合分布$f(x,y)$,$g(X,Y)$为连续型随机向量且期望存在,则$E[g(X,Y)]=\int_{\mathbb{R}^2}g(x,y)f(x,y)\mathrm{d}x\mathrm{d}y$。特别地,当$g=ax+by$时代入可发现$E[aX+bY]=aE[X]+bE[Y]$。
\end{thm}

*可用于计算\textbf{协方差}

矩$m_k=E[X^k]$,方差$\Var(X)=E[X^2]-E^2[X]$

协方差$\Cov(X,Y)=E[XY]-E[X]E[Y]$,相关系数$\rho(X,Y)=\frac{\Cov(X,Y)}{\sqrt{\Var(X)\Var(Y)}}$

\begin{thm} 柯西不等式

$(X,Y)$连续型随机向量,且$E[X^2],E[Y^2]$存在,则$(E[XY])^2\le E[X^2]E[Y^2]$
\end{thm}

证明:与离散情形相同构造。

*由此知$|\rho|\le1$,当$P(Y=aX)=1$时可取等。

\begin{exmp} 正态分布

积分计算期望:$\int_\mathbb{R}x\frac{1}{\sqrt{2\pi\sigma^2}}\exp\left(-\frac{1}{2\sigma^2}(x-\mu)^2\right)\mathrm{d}x=\int_\mathbb{R}(x-\mu)\frac{1}{\sqrt{2\pi\sigma^2}}\exp\left(-\frac{1}{2\sigma^2}(x-\mu)^2\right)\mathrm{d}x+\mu$

$=\int_\mathbb{R}x\frac{1}{\sqrt{2\pi\sigma^2}}\exp\left(-\frac{1}{2\sigma^2}x^2\right)\mathrm{d}x+\mu=\mu$。

同样直接代入计算可知方差为$\sigma^2$。
\end{exmp}

\begin{exmp} 柯西分布

$f(x)=\frac{1}{\pi(1+x^2)}$,积分可知期望不存在。
\end{exmp}

\begin{exmp} 二元正态分布

$f(x,y)=\frac{1}{\sqrt{2\pi(1-\rho^2)}}\exp\left(-\frac{x^2-2\rho xy+y^2}{2(1-\rho^2)}\right),\rho\in(-1,1)$

计算可知$f_X(x)=\frac{1}{\sqrt{2\pi}}\mathrm{e}^{-x^2/2}$,因此$X\sim N(0,1)$,由对称性知$Y$亦如此,$E[X]=E[Y]=0$。

$\Cov(X,Y)=E[XY]=\int_{\mathbb{R}^2}xyf(x,y)\mathrm{d}x\mathrm{d}y$将$xy$写为$x(y-\rho x)+\rho x^2$,可发现左侧项积分为0,考虑右侧按先$y$后$x$积分得$\rho$。

此时$\rho$即为相关系数,$\rho=0\Leftrightarrow X,Y$独立。
\end{exmp}

~

条件期望:直观上考虑$P(Y\le y|x<X\le x+\Delta x),\Delta x\to0$,可得$\frac{\int_{-\infty}^yf(x,u)\mathrm{d}u}{f_X(x)}$

\begin{defi} 设$(X,Y)$联合密度函数$f(x,y)$,$X$分布为$f_X(x)$

条件密度$f_{Y|X}(y|x)=\frac{f(x,y)}{f_X(x)}$关于$y$构成密度函数。

$F_{Y|X}(y|x)=\frac{\int_{-\infty}^yf(x,u)}{f_X(x)}\mathrm{d}u$

条件期望$E[Y|X=x]=\int_{\mathbb{R}}yf_{Y|X}(y|x)\mathrm{d}y$,由此有随机变量$E[Y|X]$。
\end{defi}

\begin{thm} 期望形式的全概率公式

$E[E[Y|X]]=E[Y]$
\end{thm}

\begin{exmp} 二元正态分布的条件期望

关于$y$的分布为$N(\rho x,1-\rho^2)$,由此$E[Y|X]=\rho X$。
\end{exmp}

\subsection{多元正态分布}
一般分析结论:随机向量$(X_1,X_2)$密度函数$f(x_1,x_2)$,映射$T:(X_1,X_2)\to(Y_1,Y_2)$,$T$为将$D\subset\mathbb{R}^2$映射到$R\subset\mathbb{R}^2$的一一映射,$T:(x_1,x_2)\to(y_1(x_1,x_2),y_2(x_1,x_2))$,设其逆映射$(x_1(y_1,y_2),x_2(y_1,y_2))$有连续偏导数。则$f_{Y_1,Y_2}(y_1,y_2)=f(x_1,x_2)|J|I_{T(D)}$,其中$J$是$(x_1,x_2)$的Jacobi行列式。

*证明:利用变量替换,考虑$(Y_1,Y_2)$取在Borel集$(-\infty,y_1]\times(-\infty,y_2]$中的概率。

(若在$P(x_1,x_2\in D_0)=1,D_0\subset D$上一一映射,结论仍成立)

\begin{exmp} $X,Y$独立$\sim N(0,1)$,$X=R\cos\Theta,Y=R\sin\Theta$,$R\ge0,\Theta\in[0,2\pi]$,求$R,\Theta$分布。

$f_{X,Y}(x,y)=\frac{1}{2\pi}\mathrm{e}^{-x^2/2-y^2/2}$,由此$f_{R,\Theta}(r,\theta)=\frac{1}{2\pi}r\mathrm{e}^{-r^2/2},r\ge0,\theta\in(0,2\pi]$,有$f_\Theta(\theta)=\frac{1}{2\pi},f_R(r)=r\mathrm{e}^{-r^2/2}$。
\end{exmp}

*$U_1,U_2$独立$\sim U(0,1)$,则$X=\sqrt{-2\ln{U_1}}\cos(2\pi U_2),Y=\sqrt{-2\ln{U_1}}\sin(2\pi U_2)$为独立标准正态分布。

一元正态分布$N(\mu,\sigma^2)$,密度函数$f(x)=\frac{1}{\sqrt{2\pi\sigma^2}}\exp\left(-\frac{1}{2\sigma^2}(x-\mu)^2\right)$

二元正态分布$f(x,y)=\frac{1}{\sqrt{2\pi(1-\rho^2)}}\exp\left(-\frac{x^2-2\rho xy+y^2}{2(1-\rho^2)}\right)$

一般情况:$C_n\mathrm{e}^{-Q(x_1,\dots,x_n)}$
二次型$Q=\sum_{i,j}x_ix_jA_{ij},A^T=A$,需\textbf{加条件}

\begin{defi} 多元正态分布
	
$\vec{X}=(X_1,\dots,X_n)$密度函数$f(\vec{x})=\frac{1}{\sqrt{(2\pi)^n|\Sigma|}}\exp(-\frac{1}{2}(\vec{x}-\vec{\mu})\Sigma^{-1}(\vec{x}-\vec{\mu})^T)$,$\Sigma=(\sigma_{ij})$正定,则记$\vec{X}\sim N(\vec{\mu},\Sigma)$

*$n=2$时可简化表达
\end{defi}

\begin{thm} 参数性质

设$\vec{X}\sim N(\vec{\mu},\Sigma)$,则

1. $E[\vec{X}]=\vec{\mu}$

2. $E[(\vec{X}-\vec{\mu})^T(\vec{X}-\vec{\mu})]=\Sigma$,即$\sigma_{ij}=\Cov(X_i,X_j)$
\end{thm}

证明:设$\Sigma=B^T\Lambda B$为正交相似对角化,分析计算。

\begin{thm} 线性变换下不变性

$\vec{X}\sim N(\vec{\mu},\Sigma)$,$D$为$n$阶可逆方阵,则$\vec{X}D\sim N(\vec{\mu}D,D^T\Sigma D)$。
\end{thm}

证明:对分量分别计算可知结论。

\begin{thm}
$\vec{X}\sim N(\vec{\mu},\Sigma)$,$\Sigma=\diag(\Sigma_1,\Sigma_2)$,两分块均为方阵,将向量与期望对应拆分为$\vec{X}_1,\vec{X}_2$与$\vec{\mu}_1,\vec{\mu}_2$,则$\vec{X}_1\sim N(\vec{\mu}_1,\Sigma_1),\vec{X}_1\sim N(\vec{\mu}_1,\Sigma_2)$。
\end{thm}

证明:分离变量得结果。

\begin{thm}
$\vec{X}\sim N(\vec{\mu},\Sigma)$,$\Sigma=\begin{pmatrix}\Sigma_{11}&\Sigma_{12}\\\Sigma_{21}&\Sigma_{22}\end{pmatrix}$,将随机向量与期望对应拆分为$\vec{X}_1,\vec{X}_2$与$\vec{\mu}_1,\vec{\mu}_2$,则$\vec{X}_1\sim N(\vec{\mu}_1,\Sigma_{11}),\vec{X}_1\sim N(\vec{\mu}_1,\Sigma_{22})$。
\end{thm}

证明:对角化后利用分部积分计算。

\begin{thm} 更广泛的线性变换下不变性
	
$\vec{X}\sim N(\vec{\mu},\Sigma)$,$D$为$n\times m$阶列满秩方阵,则$\vec{X}D\sim N(\vec{\mu}D,D^T\Sigma D)$。
\end{thm}

证明:与之前类似,拆分计算。

\begin{thm} 独立性

$\Sigma=\diag(\Sigma_1,\Sigma_2)$,各分量相互独立当且仅当协方差矩阵对角。
\end{thm}

证明:分离变量得结果。

~

*定义\textbf{均值}$\bar{X}=\frac{X_1+\dots+X_n}{n}$,\textbf{方差}$S^2=\frac{1}{n-1}\sum_{i=1}^n(X_i-\bar{X})^2$

\begin{defi} 卡方分布
	
当密度函数$f(x)=\frac{1}{2^{d/2}\Gamma(d/2)}x^{d/2-1}e^{-x/2},x>0$,称$X$服从$d$个自由度卡方分布,记为$X\sim\chi^2(d)$。
\end{defi}

\begin{thm} 卡方分布性质

$Y_1,\dots,Y_d$独立同分布$N(0,1)$,则$\sum_{i=1}^dY_i^2\sim\chi^2(d)$。
\end{thm}

证明:利用极坐标换元计算。

\begin{thm} 均值、方差性质

设$X_i$独立同分布$N(\mu,\sigma^2)$,则:

1. $\bar{X}\sim N(\mu,\sigma^2/n)$

2. $(n-1)/\sigma^2S^2\sim\chi^2(n-1)$

3. $\bar{X},S^2$独立
\end{thm}

\textbf{复平面}二维随机向量:$Z=X+Y\mathrm{i}$

$E[Z]=E[X]+\mathrm{i}E[Y]$

\textbf{复高斯分布}:$N_\mathbb{C}(\mu,\sigma^2),\mu\in\mathbb{C},\sigma>0$,联合密度$f(z)=\frac{1}{\pi\sigma^2}\exp\left(-\frac{1}{\sigma^2}|z-\mu|^2\right)$。

结论:$Z\sim N_\mathbb{C}(0,1)\Rightarrow E[Z^k\overline{Z^l}]=\begin{cases}k!&k=l\\0&k\neq l\end{cases}$

\section{中心极限定理}
\subsection{一般随机变量的期望}
1. 记号准备

对随机变量$X$与分布函数$F$,离散型/连续型具有分布列/分布函数$f$,从而对$xf(x)$求和/积分可得期望。而由于$\mathrm{d}F$分别为$F(x)-F(x^-)/f(x)\mathrm{d}x$,期望可统一写为$E[X]=\int_\mathbb{R}x\mathrm{d}F$,佚名统计学家公式也可写为$E[g(X)]=\int_\mathbb{R}g(x)\mathrm{d}F$。

~

2. 抽象积分

对一般$(\Omega,\mathcal{F},P)$,随机变量$X$与分布函数$F$,如何定义$E[X]$?

STEP 1 对\textbf{简单}(取有限个值)随机变量

可记为$X=\sum_{i=1}^nx_iI_{A_i}$,$A_1,\dots,A_n$为$\Omega$的划分,则$E[X]=\sum_{i=1}^nx_iP(A_i)$。

STEP 2 对\textbf{非负}随机变量

存在单调增的简单随机变量列,收敛到此随机变量:记$X_n=nI_{A_n}+\sum_{j=1}^{n2^n}\frac{j-1}{2^n}I_{A_{nj}}$,其中$A_{nj}=\{\frac{j-1}{2^n}\le X<\frac{j}{2^n}\},A_n=\{X\ge n\}$。

再说明若简单随机变量序列$X_n,Y_n$均单调增收敛至$X$,则期望的极限相同(此处相同包含正无穷,由此可由期望的极限定义$X$的期望)。

STEP 3 对\textbf{一般}随机变量

将一般随机变量$X$写为$X^+-X^-,X^+=\max\{X,0\},X^-=-\max\{-X,0\}$,当$E[X^+],E[X^-]$至少一个有限时,可定义$E[X]=E[X^+]-E[X^-]$。特别地,若两者都有限,则称$X$的\textbf{数学期望}存在。

*统一记号为$E[X]=\int_\Omega X\mathrm{d}P$或$\int_\Omega X(\omega)P(\mathrm{d}\omega)$

~

3 期望性质

非负性:$X\ge0\Rightarrow E[X]\ge0$\ (由定义过程知成立)

规范性:$c\in\mathbb{R}\Rightarrow E[c]=c$\ 

线性性:$E[aX+bY]=aE[X]+bE[Y]$

“\textbf{连续}性”:$X_n$趋向$X$(或偏移的概率为0),只要满足以下条件之一即有期望亦有极限:

\begin{enumerate}
	\item 单调收敛:$X_n$单调
	\item 控制收敛:$|X_n|\le Y,E[Y]<\infty$
	\item \textbf{有界收敛}:$|X_n|\le c,c\in\mathbb{R}$
\end{enumerate}

~

4 Lebesgue-Stieltjes积分

对随机变量$X$与分布函数$F$,引入$(\mathbb{R},R(\mathbb{R}))$上概率测度$\mu_F((a,b])=F(b)-F(a)$,则$(\mathbb{R},R(\mathbb{R}),\mu_F)$构成概率空间。对Borel可测函数$g$,有抽象积分$\int g\mathrm{d}F$,称为Lebesgue-Stieltjes积分。

有结论:$E[g(X)]=\int g\mathrm{d}F$。

证明:回到三步的定义方式\textbf{逐步证明}。

*此结论对多元也成立

*\textbf{Fatou引理}:$X_i$非负随机变量,则$E[\liminf_{n\to\infty}X_n]\le\liminf_{n\to\infty}E[X_n]$

~

5 独立随机变量乘积期望与期望乘积相同

\begin{thm} $X,Y$为独立随机变量,且期望与乘积的均存在,则$E[XY]=E[X]E[Y]$
\end{thm}

证明:

对简单随机变量:直接拆分计算即可。

对非负随机变量:注意到可取出两列递增的\textbf{独立}简单随机变量趋向$X_n$与$Y_n$,由此得证。

对一般随机变量:利用线性性拆分计算即可。

\subsection{特征函数}
*非负整值母函数定义:$G_X(s)=\sum_{k=0}^\infty P(X=k)s^k=E[s^X]$

当$s=\mathrm{e}^t$,有\textbf{矩母函数}$M_X(t)=E[\mathrm{e}^{tX}]$

* “\textbf{好}”的情形:存在0的邻域使矩母函数存在

* 不好的例子:$f(x)=\frac{1}{\pi(1+x^2)}$

\begin{defi}
$X,Y$为$(\Omega,\mathcal{F},P)$上随机变量,称$Z=X+Yi$为复随机变量。
\end{defi}

(1) 实质为二维随机向量

(2) $Z_1,Z_2$独立,指$(X_1,Y_1)$与$(X_2,Y_2)$独立,即$P(X_1\le x_1,Y_1\le y_1,X_2\le x_2,Y_2\le y_2)=P(X_1\le x_1,Y_1\le y_1)P(X_2\le x_2,Y_2\le y_2)$。

(3) $E[Z]=E[X]+\mathrm{i}E[Y]$

(4) $Z_i$相互独立时,$E[Z_1\dots Z_n]=E[Z_1]\dots E[Z_n]$。

\begin{defi} 特征函数

$\phi(t)=E[\mathrm{e}^{\mathrm{i}tX}]$,有时用$\phi_X(t)$表示。
\end{defi}

(1) $\phi(t)=E[\cos(tX)]+\mathrm{i}E[\sin(tX)]$

(2) 由于$|\mathrm{e}^{\mathrm{i}tx}|=1$,$\phi(t)$总存在。

(3) $\phi(t)=\int_\mathbb{R}\mathrm{e}^{\mathrm{i}tx}\mathrm{d}F(x)$,有分布函数时可写为$\int_\mathbb{R}\mathrm{e}^{\mathrm{i}tx}f(x)\mathrm{d}x$。

\begin{thm} 基本性质

1. $\phi(0)=1,|\phi(t)|\le1,\overline{\phi(t)}=\phi(-t)$

2. $\phi$在$\mathbb{R}$上一致连续。

3. \emph{非负定}性:$t_i\in\mathbb{R},z_i\in\mathbb{C}$,有$\sum_{j,k=1}^n\phi(t_j-t_k)z_j\overline{z_k}\ge0$。

\end{thm}

证明:

2. $|\phi(t+h)-\phi(t)|\le\int_\mathbb{R}{|\mathrm{e}^{\mathrm{i}tx}(\mathrm{e}^{\mathrm{i}hx}-1)|\mathrm{d}F}\le\int_\mathbb{R}{|\mathrm{e}^{\mathrm{i}hx}-1|\mathrm{d}F}$,有界收敛定理可知一致趋于0。

3. 原式$=E\Big[\big|\sum_jz_j\mathrm{e}^{\mathrm{i}t_jx}\big|^2\Big]\ge0$。

*满足定理5.2三条性质的函数必为某随机变量的特征函数

\begin{thm}
$E[|X|^k]<\infty$,则$\forall j<k,\phi^{(j)}(0)=\mathrm{i}^jE[X^j]$,进而$\phi(t)=\sum_{j=0}^k\frac{(\mathrm{i}t)^j}{j!}E[X^j]+o(t^k)$。
\end{thm}

证明:由题6.5.4知$E[|X|^j]<\infty$,由此积分与求导可交换,再由求导结果可知成立。

\begin{thm} 两变量特征函数关系

1. $Y=aX+b$时$\phi_Y(t)=\mathrm{e}^{\mathrm{i}bt}\phi_X(at)$。

2. $X,Y$独立时$\phi_{X+Y}(t)=\phi_X(t)\phi_Y(t)$。
\end{thm}

证明:直接计算即可。

~

\begin{defi} 多元特征函数

$\vec{X}=(X_1,\dots,X_n)$的特征函数为$\phi_{\vec{X}}(\vec{t})=E[\mathrm{e}^{\mathrm{i}\sum_jt_jX_j}]$。
\end{defi}

\begin{thm} 独立性

$X,Y$独立当且仅当$\phi_{X,Y}(s,t)=\phi_X(s)\phi_Y(t)$。
\end{thm}

证明:左推右可直接计算,右推左需要\textbf{反转公式}(见下节)。

~

\begin{exmp} Bernoulli分布

$\phi(t)=p\mathrm{e}^{\mathrm{i}t}+q$

由此亦可知二项分布母函数为$(p\mathrm{e}^{\mathrm{i}t}+q)^n$
\end{exmp}

*对非负整值随机变量,母函数$G$,则$G(\mathrm{e}^{\mathrm{i}t})$即为特征函数。

\begin{exmp} 指数分布

$f(x)=\lambda\mathrm{e}^{-\lambda x},x>0$

$\phi(t)=\frac{\lambda}{\lambda-\mathrm{i}t}$
\end{exmp}

证明:可分别计算实部虚部积分,亦可通过复分析直接计算。

\begin{exmp} 标准正态分布$N(0,1)$

$\phi(t)=\frac{1}{\sqrt{2\pi}}\int_\mathbb{R}\mathrm{e}^{itx-x^2/2}\mathrm{d}x=\mathrm{e}^{-t^2/2}$
\end{exmp}

证明:“物理方法”假设$\mathrm{i}t$为实数,再解析延拓;“数学方法”计算导数说明。

*$N(\mu,\sigma^2)$的分布函数为$\mathrm{e}^{\mathrm{i}\mu t-\sigma^2t^2/2}$

\begin{exmp} 多元正态分布$N(\vec{\mu},\Sigma)$

$\phi(\vec{t}\ )=\exp\left(\frac{\mathrm{i}\vec{\mu}\vec{t}\ ^T-\vec{t}\ \Sigma\vec{t}\ ^T}{2}\right)$
\end{exmp}

证明:设$Y=\vec{X}\cdot\vec{t}\ ^T$,将其化为一元正态分布的情况。

\begin{exmp} 均匀分布$U(-1,1)$

$\phi(t)=\frac{\sin{t}}{t}$
\end{exmp}

\subsection{反转与连续性定理}

\begin{thm} 反转公式

$-\infty<a<b<\infty,\frac{F(b)+F(b^-)}{2}-\frac{F(a)+F(a^-)}{2}=\lim\limits_{T\to\infty}\int_{-T}^T\frac{\mathrm{e}^{-\mathrm{i}at}-\mathrm{e}^{-\mathrm{i}bt}}{2\pi\mathrm{i}t}\phi(t)\mathrm{d}t$
\end{thm}

理解:类似\textbf{傅里叶反变换}

证明:记极限中的积分为$I_T$,$I_T=\int_{-T}^T\frac{\mathrm{e}^{-\mathrm{i}at}-\mathrm{e}^{-\mathrm{i}bt}}{2\pi\mathrm{i}t}\int_\mathbb{R}\mathrm{e}^{\mathrm{i}tx}\mathrm{d}F\mathrm{d}t$,由\textbf{Fubini定理}积分符号可交换,由积分区域对称可转化为$\int_\mathbb{R}g_T(x)\mathrm{d}F,g_T(x)=\frac{1}{\pi}\int_0^T\left(\frac{\sin{t(x-a)}}{t}-\frac{\sin{t(x-b)}}{t}\right)\mathrm{d}t$。由于$\int_0^{+\infty}\frac{\sin{tx}}{\pi t}=\begin{cases}1/2&x>0\\0&x=0\\-1/2&x<0\end{cases}$,$g_T(x)$有界,由控制收敛定理其极限为$\begin{cases}1&x\in(a,b)\\1/2&x=a,b\\0&$其他$\end{cases}$,因此结果为$P(X\in(a,b))+\frac{P(X=a)+P(X=b)}{2}$,即为左侧。

*推论:\textbf{特征函数相同即可知同分布}

证明:记$C_F$为$F$连续点,$\mathbb{R}\backslash C_F$至多可数。让$a,b\in C_F,a\to-\infty$,可唯一确定$F(b)$,由此连续点已唯一确定。再用连续点从右侧逼近可唯一确定不连续点处。

\begin{thm} 假设随机向量对任何长方体,落入其表面概率为$0$,则有:

$P(a_j<X_j\le b_j,j=1,\dots,n)=\lim\limits_{T_i\to\infty}\int_{-T_n}^{T_n}\dots\int_{-T_1}^{T_1}\prod_{j=1}^n\frac{\mathrm{e}^{-\mathrm{i}a_jt}-\mathrm{e}^{-\mathrm{i}b_jt}}{2\pi\mathrm{i}t_j}\phi(t_1,\dots,t_n)\mathrm{d}t_1\dots\mathrm{d}t_n$


\end{thm}

*由此即可说明特征函数可分离变量时随机变量独立

\begin{exmp} 求$\cos{t}$对应的分布函数。

可构造出$P(X=1)=P(X=-1)=\frac{1}{2}$,由此分布函数即为结果。
\end{exmp}

*对随机变量$X_n$,分布函数$F_n$,特征函数$\phi_n$,$F_n,\phi_n$收敛性关系?

\begin{exmp}
$X_n=\frac{1}{n}$,分布函数的极限为$\begin{cases}1&x>0\\0&x\le0\end{cases}$,不满足右连续性。
\end{exmp}

\begin{defi} 分布函数的收敛

$F,F_n$为分布函数,称$F_n$\emph{弱收敛}至$F$,若对$F$的任意连续点$x$,$\lim\limits_{n\to\infty}F_n(x)=F(x)$,记作$F_n\con{W}F$。
\end{defi}

\begin{thm} L\'evy-Cram\'er连续性定理

假设$F_n$为分布函数,对应特征函数$\phi_n$

1. 若$F_n\stackrel{W}{\to}F$,$F$为分布函数,对应特征函数$\phi$,则$\phi_n$内闭一致收敛到$\phi$。

2. 若$\phi_n$逐点收敛到$\phi$,$\phi$在$t=0$处连续,则$\phi(t)$为特征函数,其对应分布函数$F$,且$F_n\con{W}F$。
\end{thm}

\begin{exmp}
$X\sim U(-n,n)$,则$\phi_n(t)=\frac{\sin{nt}}{nt}$,特征函数极限为$\begin{cases}0&t\ne0\\1&t=0\end{cases}$,不满足连续性。
\end{exmp}

\subsection{极限定理}
1. 问题:研究$T_n=\frac{1}{B_n}\sum_{i=1}^n(X_i-a_i)$的极限性质

*$X_k$性质-独立同分布

*$a_k=E[X_k],B_n=c\sqrt{n}$

*研究不同收敛意义下极限

~

2. 大数定律(LLN)、中心极限定理(CLT)

\begin{defi} 弱收敛

若$F_{X_n}\con{W}F_X$,则称$X_n$\emph{依分布收敛}(弱收敛)到$X$,记为$X_n\con{D}X$。
\end{defi}

\begin{thm} 大数定律

$X_i$独立同分布,期望$\mu=E[X_i]$存在,令$S_n=\sum_{k=1}^nX_k$,则$\frac{S_n}{n}\con{D}\mu$,即$\frac{S_n}{n}-\mu\con{D}0$。
\end{thm}

证明:运用连续性定理,由$\phi_X'(0)=\mu\mathrm{i}$将$X_i$的分布函数在0处展开一次项即可计算得结果。

*$\frac{S_n}{n}-\mu$的无穷小阶数可推测为$\frac{1}{\sqrt{n}}$,这是由于$\Var(S_n)=n\Var(X_1)$,具体定理为:

\begin{thm} 中心极限定理

$X_i$独立同分布,期望$\mu=E[X_i]$,方差$\sigma^2=\Var(X_i),\sigma>0$存在,则$\frac{S_n-n\mu}{\sqrt{n\sigma^2}}\con{D}N(0,1)$。 
\end{thm}

证明:通过平移放缩可不妨设$\mu=0,\sigma=1$,再对$\frac{S_n}{\sqrt{n}}$展开估算。

\begin{exmp} 各次测量值独立同分布,方差为$4$,欲以$95\%$把握保证测量精度达$\pm0.5$,求最低测量次数。

$Z_n=\frac{S_n-n\mu}{2\sqrt{n}}\con{D}N(0,1)$,$P\bigg(\frac{S_n}{n}-\mu\bigg)\le0.5=P\bigg(|Z_n|\le\frac{\sqrt{n}}{4}\bigg)$,将$Z_n$视为正态分布,查表得至少需要$n=62$。
\end{exmp}

~

3. Lindeberg条件(处理独立但\textbf{未必同分布})

设$a_k=E[X_k],b_k^2=\Var(X_k),B_n^2=\sum_{k=1}^nb_k^2$,$F_k$为$X_k$分布函数。

L条件:$\forall\varepsilon>0,\lim\limits_{n\to\infty}\frac{1}{B_n^2}\sum_{k=1}^n\int_{|x-a_k|>\varepsilon B_n}(x-a_k)^2\mathrm{d}F_k=0$

\begin{thm} Lindeberg - Feller CLT

$X_i$相互独立,满足L条件,则$\frac{\sum_{k=1}^n(X_k-a_k)}{B_n}\con{D}N(0,1)$,且$\frac{\max_kb_k^2}{B_n^2}\con{}0$。

特别地,$X_i$相互独立,$a_i=0$,$E[|X_i|^3]<\infty$,且$\frac{1}{B_n^3}\sum_{k=1}^nE[|X_k|^3]\con{}0$,则$\frac{\sum_{k=1}^nX_k}{B_n}\con{D}N(0,1)$。
\end{thm}

*由$\int_{|x|>\varepsilon B_n}x^2\mathrm{d}F_k\le\frac{1}{\varepsilon B_n}\int_{|x|>\varepsilon B_n}|x|^3\mathrm{d}F_k$即可验证特殊情况。

注:

(1) L条件的概率意义:

$\sum_{k=1}^n\int_{|x-a_k|>\varepsilon B_n}\frac{(x-a_k)^2}{B_n^2}\mathrm{d}F_k\ge\varepsilon^2\sum_{k=1}^nP\bigg(\frac{|X_k-a_k|}{B_n}>\varepsilon\bigg)\ge\varepsilon^2P\bigg(\bigcup_{k=1}^n\bigg\{\frac{|X_k-a_k|}{B_n}>\varepsilon\bigg\}\bigg)$

$=\varepsilon^2P\bigg(\frac{\max_k|X_k-a_k|}{B_n}>\varepsilon\bigg)$,由此L条件可推出$P\bigg(\frac{\max_k|X_k-a_k|}{B_n}>\varepsilon\bigg)\con{}0$。

(2) $X_i$相互独立时,L条件增加$B_n\con{}\infty,\frac{\max_kb_k^2}{B_n^2}\con{}0$(Feller条件)即为CLT的\textbf{必要条件}。

(3) Lyapunov条件:$\exists\delta>0,E[|X_i-a_i|^{2+\delta}]<\infty$,且$\frac{1}{B_n^{2+\delta}}\sum_{k=1}^nE[|X_k-a_k|^{2+\delta}]\con{}0$

类似特殊情况的推导知Lyapunov条件可推出L条件,从而推出CLT。

(4) 利用$b_k^2=\int_\mathbb{R}(x-a_k)^2\mathrm{d}F_k$可拆分为两段,放缩知$\frac{b_k^2}{B_n^2}$在极限时一致$\le\varepsilon^2$,从而且$\frac{\max_kb_k^2}{B_n^2}\con{}0$。

~

4. 局部极限定理(LLT)

*\textbf{二项分布}的正态逼近

取$X_i$独立同分布$B(1,p)$,由再生性知$S_n\sim B(n,p)$,由此可进行估计:

\begin{thm} 二项分布的\emph{局部极限定理}

$p\in(0,1),q=1-p,x_k=\frac{k-np}{\sqrt{npq}}$,对$|x_k|\le A$,$n\to\infty$时一致地有$\mathrm{C}_n^kp^kq^{n-k}\sim\frac{1}{\sqrt{2\pi npq}}\mathrm{e}^{-x_k^2/2}$。
\end{thm}

证明:利用Stirling公式估算系数。

\begin{thm} 积分形式LLT

$S_n\sim B(n,p)$,则$P\bigg(a<\frac{S_n-np}{\sqrt{npq}}\le b\bigg)\con{}\frac{1}{\sqrt{2\pi}}\int_a^b\mathrm{e}^{-x^2/2}\mathrm{d}x$
\end{thm}

证明:取上一定理中$x_k$,有:

左$=\sum_{k:x_k\in(a,b]}\mathrm{C}_n^kp^kq^{n-k}\sim\sum_{k:x_k\in(a,b]}\frac{\mathrm{e}^{-x_k^2/2}}{\sqrt{2\pi npq}}=\sum_{k:x_k\in(a,b]}\frac{\mathrm{e}^{-x_k^2/2}}{\sqrt{2\pi}}(x_{k+1}-x_k)\con{}$右,最后一步是由于黎曼和极限为积分。

*$n$固定时,$x_k$与$k$一一对应,形成等距分划

~

5. 矩方法

$\gamma_k=\int_\mathbb{R}x^k\frac{1}{\sqrt{2\pi}}\mathrm{e}^{-x^2/2}\mathrm{d}x=\begin{cases}0&2\nmid k\\(k-1)!!&2\mid k\end{cases}$,可看作$1,2,\dots,k$两两配对的方式数。

\begin{thm}
$X_i$独立,且期望均为$0$,方差均为$1$,$\forall m\ge3,C_m=\sup_kE[|X_k|^m]<\infty$(\emph{一致有界高阶矩}),则$S_n=\sum_{i=1}^nX_i$有$E\bigg[\bigg(\frac{S_n}{\sqrt{n}}\bigg)^k\bigg]\con{}\gamma_k$,进而$\frac{S_n}{\sqrt{n}}\con{D}N(0,1)$。
\end{thm}

证明:$E\bigg[\bigg(\frac{S_n}{\sqrt{n}}\bigg)^k\bigg]=n^{-k/2}\sum_{i_1,\dots,i_k}E[X_{i_1}\dots X_{i_k}]$,其中非零项每个随机变量次数必至少为2。由于高阶矩一致有界,若选出的项小于$k/2$个,会在极限中趋于0,由此只有$k=2m$时可两两配对$i_1,\dots,i_{2m}$得出项,再由每种配对方式对应极限为$1$可算出结果。

*由矩的极限推出依分布收敛:

\begin{thm} \emph{矩收敛定理}

条件:

1. $k\in\mathbb{N},\gamma_{k,n}=\int x^k\mathrm{d}F_n$存在

2. $\forall k,\lim_{n\to\infty}\gamma_{k,n}=\gamma_k$

3. $\gamma_k=\int x^k\mathrm{d}F$,且满足\emph{Carleman条件}$\sum_{k=1}^\infty(\gamma_{2k})^{-1/(2k)}=\infty$

则$F_n\con{W}F$。
\end{thm}

\section{几种收敛}
\subsection{四种收敛方式}
\begin{defi} 假设$X,X_n$为$\{\Omega,\mathcal{F},P\}$上的随机变量,则:
\begin{enumerate}
	\item \emph{几乎处处收敛}(以概率$1$收敛):
	
	$P(\{\omega\in\Omega:X_n(\omega)-X(\omega)\con{}0\})=1$,
	
	记作$X_n\con{a.s.}X$。
	
	\item $\mathbf{r}$\emph{阶收敛}:
	
	$r\in\mathbb{N}^*,\forall n,E[|X_n|^r]<\infty$,且$E[|X_n-X|^r]\con{}0$,
	
	记作$X_n\con{r}X$。$r=1$时称\emph{平均收敛},$r=2$时称\emph{均方收敛}。
	
	\item \emph{依概率收敛}:
	
	$\forall\varepsilon>0,P(|X_n-X|>\epsilon)\con{}0$,
	
	记作$X_n\con{P}X$。
	
	\item \emph{依分布收敛}:
	
	$F_X$的连续点处$P(X_n\le x)\con{}P(X\le x)$,
	
	记作$X_n\con{D}X$。
\end{enumerate}
\end{defi}

*依分布收敛\textbf{与样本空间选择无关},具有特殊性

\begin{thm} 四种收敛的关系

$X_n\con{a.s.}X\Rightarrow X_n\con{P}X\Rightarrow X_n\con{D}X$

$r>s,X_n\con{r}X\Rightarrow X_n\con{s}X\Rightarrow X_n\con{P}X$

且反方向均无法推出,由此强弱有\emph{严格性}。
\end{thm}

定理证明拆分为以下:

\begin{enumerate}
	\item $X_n\con{P}X\Rightarrow X_n\con{D}X$
	
	$F_n(x)=P(X_n\le x,X\le x+\varepsilon)+P(X_n\le x,X>x+\varepsilon)\le F(x+\varepsilon)+P(|X-X_n|>\varepsilon)$,同理$F(x-\varepsilon)\le F_n(x)+P(|X-X_n|>\varepsilon)$,连续点处考虑$F_n(x)$上下极限可知结果。
	
	\begin{exmp} $X_n\con{D}X\nRightarrow X_n\con{P}X$
	
	$P(X=0)=P(X=1)=\frac{1}{2},Y=1-X,X_n=X$,则$X_n\con{D}Y$,但其他三种收敛均不成立。
	\end{exmp}
	
	\item $r>s,X_n\con{r}X\Rightarrow X_n\con{s}X$
	
	利用问题14.4.28(可由\textbf{H\"older不等式}证明)有$E[|X_n-X|^s]^{1/s}\le E[|X_n-X|^r]^{1/r}$,由此得结论。
	
	\item $X_n\con{1}X\Rightarrow X_n\con{P}X$
	
	考虑概率测度在$|X|\ge a$部分的积分得\textbf{Markov不等式}:$a>0,P(|X|>a)\le\frac{E[|X|]}{a}$,由此取$a=\varepsilon$知结论。
	
	*\textbf{Chebyshev不等式}:$a>0,P(|X-E[X]|>a)\le\frac{\Var(X)}{a^2}$
	
	\begin{exmp} $X_n\con{P}X\nRightarrow X_n\con{r}X$
	
	$\Omega=(0,1]$,$P$为Lebesgue测度,$X_n(\omega)=\begin{cases}n^{1/r}&\omega\in\bigg(0,\frac{1}{n}\bigg]\\[2ex]0&\omega\in\bigg(\frac{1}{n},1\bigg]\end{cases}$,$X=0$,计算可验证依概率收敛但不$r$阶收敛。
	\end{exmp}
	
	\item $X_n\con{a.s.}X\Rightarrow X_n\con{P}X$
	
	分析得$\{\omega\in\Omega:X_n(\omega)-X(\omega)\con{}0\}=\bigcap_{k=1}^\infty\bigcup_{m=1}^\infty\bigcap_{n=m}^\infty\{\omega\in\Omega:|X_n(\omega)-X(\omega)|\le\frac{1}{k}\}$,由此知$X_n\con{a.s.}X\Leftrightarrow P\bigg(\bigcup_{k=1}^\infty\bigcap_{m=1}^\infty\bigcup_{n=m}^\infty\{\omega\in\Omega:|X_n(\omega)-X(\omega)|>\frac{1}{k}\}\bigg)=0$,类似分析得其等价于$\forall\varepsilon>0,\lim\limits_{m\to\infty}P\bigg(\bigcup_{n=m}^\infty\{\omega\in\Omega:|X_n(\omega)-X(\omega)|>\varepsilon\}\bigg)=0$,由此得结论。
	
	\begin{exmp} $X_n\con{r}X\nRightarrow X_n\con{a.s.}X$,由此$X_n\con{P}X\nRightarrow X_n\con{a.s.}X$
	
	$X_i$相互独立,$X_n\sim B(1,\frac{1}{n})$,可验证任意$r$阶均收敛,但$P\bigg(\bigcup_{n=m}^\infty\{\omega\in\Omega:|X_n(\omega)-X(\omega)|>\varepsilon\}\bigg)=1$,故不几乎处处收敛。
	\end{exmp}
\end{enumerate}

\begin{thm} 反向的成立条件

1. 若$X_n\con{D}c\in\mathbb{R}$,则$X_n\con{P}c$。

2. 若$\exists k,P(|X_n|\le k)=1,X_n\con{P}X$,则$X_n\con{r}X$。

3. 若$\forall\varepsilon>0,\sum_{n=1}^\infty P(|X_n-X|>\varepsilon)<\infty$,则$X_n\con{a.s.}X$。
\end{thm}

证明:

1. 利用$P(|X_n-c|>\varepsilon)=P(X_n>c+\varepsilon)+P(X_n<c-\varepsilon)$,利用分布函数估计知结果。

2. 先拆分$X$为$|X-X_n|+|X_n|$估计出$X$有同样的界,再将差的期望拆分为$|X_n-X|\le\varepsilon$与$|X_n-X|>\varepsilon$的部分可知期望的极限。

3. 利用并的概率小于等于概率的和直接估算。

\begin{thm} 弱大数律

$X_i$独立同分布,期望$\mu=E[X_i]$存在,令$S_n=\sum_{k=1}^nX_k$,则$\frac{S_n}{n}\con{P}\mu$。
\end{thm}

证明:由大数定律与收敛于常随机变量得出。

\begin{thm} \emph{Skorokhodem表示定理}

设$X_n\con{D}X$,则存在$(\Omega,\mathcal{F},P)$,其上的$Y_n,Y$满足$Y_n$与$X_n$,$Y$与$X$同分布,$Y_n\con{a.s.}Y$。
\end{thm}

\begin{thm} 弱收敛性质

$X_n\con{D}X\Leftrightarrow\forall g\in C_b(\mathbb{R})$(有界连续函数),$E[g(X_n)]\con{}E[g(X)]$。
\end{thm}

证明:左推右由表示定理将$X_n,X$替换为$Y_n,Y$利用控制收敛定理可说明$E[g(Y_n)]\con{}E[g(Y)]$;右推左利用$P(X_n\le x)=E[I_{(-\infty,x]}(X_n)]$,以有界连续函数逼近知结论。

\subsection{重要结论}
1. \textbf{不等式}

*记$||X||_p=E[|X|^p]^{1/p},p\ge1$

H\"older不等式:$\frac{1}{p}+\frac{1}{q}=1,E[|XY|]\le||X||_p||Y||_q$

Minkowski不等式:$||X+Y||_p\le||X||_p+||Y||_p$

Markov不等式:$a>0,aP(|X|>a)\le E[|X|]$

Chebyshev不等式:$a>0,a^2P(|X-E[X]|>a)\le\Var(X)$

\begin{exmp} $\exists r>0,E[|X|^r]=0\Rightarrow P(X=0)=1$

由Markov不等式,$\forall\varepsilon,P(|X|\ge\varepsilon)\le\frac{E[|X|^r]}{\varepsilon^r}=0$,由此$P(|X|<\varepsilon)=1\Rightarrow P(|X|\le2\varepsilon)=1$,利用右连续性有结论。
\end{exmp}

~

2. 收敛

\begin{thm}
记$\square$为$a.s.$或$r$或$P$,有:

1. $X_n\con{\square}X,X_n\con{\square}Y\Rightarrow P(X=Y)=1$

2. $X_n\con{\square}X,Y_n\con{\square}Y\Rightarrow X_n+Y_n\con{\square}X+Y$

3. 前两条对依分布收敛一般不成立,但1.可以改为$X,Y$同分布
\end{thm}

证明:利用拆分与不等式放缩可验证成立,对依分布收敛,取$P(X=1)=P(X=-1)=\frac{1}{2},Y_n=X_n=Y=-X$即为前两条的反例,利用连续点处相等与单调右连续可知同分布。

~

3. \textbf{Borel - Cantelli引理}

事件列的\textbf{上下极限}:

$A_n$的上限事件:$\limsup_{n\to\infty}A_n=\bigcap_{n=1}^\infty\bigcup_{m=n}^\infty A_m$ ($A_n$中发生无穷多次的样本点),

$A_n$的下限事件:$\liminf_{n\to\infty}A_n=\bigcup_{n=1}^\infty\bigcap_{m=n}^\infty A_m$ ($A_n$中至多有限多次不发生的样本点)

*$A_n$的上限事件记为$\{A_n\ i.o.\}$。

\begin{thm} Borel - Cantelli引理

1. $\sum_{n=1}^\infty P(A_n)<\infty\Rightarrow P(A_n\ i.o.)=0$

2. $\sum_{n=1}^\infty P(A_n)=\infty$,$A_i$独立$\Rightarrow P(A_n\ i.o.)=1$
\end{thm}

证明:利用交、并、独立与概率的连续性放缩。

*由此可发现,若$A_i$独立,$P(A_n\ i.o.)$只能为0或1(\textbf{零一律})。

\begin{exmp} $X_i\sim\Exp(1)$独立同分布,则$P\bigg(\limsup_{n\to\infty}\frac{X_n}{\ln{n}}=1\bigg)=1$。

令$A_n=\bigg\{\frac{X_n}{\ln{n}}\ge 1+a\bigg\}$,则$P(A_n)=\frac{1}{n^{1+a}}$且相互独立。

可发现$a\in(-1,0]$时$P(A_n\ i.o.)=1$($A_n$几乎处处发生无穷多次),$a>0$时$P(A_n\ i.o.)=0$($A_n$几乎处处只发生有限多次),由此知结论。
\end{exmp}

\subsection{强大数律}
弱大数律:$X_i$独立同分布,期望$\mu=E[X_i]$存在,令$S_n=\sum_{k=1}^nX_k$,则$\frac{S_n}{n}\con{P}\mu$。

\begin{thm} 强大数律

$X_i$独立同分布,期望$\mu=E[X_i]$存在,$E[X_i^2]$存在,令$S_n=\sum_{k=1}^nX_k$,则$\frac{S_n}{n}\con{2}\mu,\frac{S_n}{n}\con{a.s.}\mu$。
\end{thm}

证明:对均方收敛,可由独立性直接计算$E\bigg[\bigg(\frac{S_n}{n}-\mu\bigg)^2\bigg]=\frac{\Var(X_1)}{n}\con{}0$。

对几乎处处收敛,先找几乎处处收敛的子列,再证明对非负的$X_k$成立,最后推至一般。

\begin{thm} \emph{柯尔莫哥洛夫强大数律}

$X_i$独立同分布,则$\frac{X_1+\dots+X_n}{n}\con{a.s.}\mu\Leftrightarrow E[|X_1|]$存在且$E[X_1]=\mu$。
\end{thm}

左推右证明:由于可拆分为两部分,不妨设随机变量均非负。

\qquad STEP 1 截尾:取$Y_n=X_nI_{X_n<n}$,拆分估计可知$\sum_{n=1}^\infty P(X_n\ne Y_n)=\sum_{n=1}^\infty P(X_n\ge n)<\infty$,因此$P(X_n\ne Y_n\ i.o.)=0$,因此$\frac{\sum_{k=1}^n(X_k-Y_k)}{n}\con{a.s.}0$,由此只需证结论对各阶矩存在的$Y_n$成立。

\qquad STEP 2 几乎处处收敛子列:对$\alpha>1$,令$\beta_k=\lceil\alpha^k\rceil$,则$\frac{\beta_{k+1}}{\beta_k}\con{}\alpha$,且存在$A$使$\forall m,\sum_{k=m}^\infty\frac{1}{\beta_k^2}\le\frac{A}{\beta_m^2}$。记$S_n'=\sum_{i=1}^nY_i$,利用二阶矩估计知$\sum_{n=1}^\infty P\bigg(\bigg|\frac{S'_{\beta_n}-E[S'_{\beta_n}]}{\beta_n}\bigg|>\varepsilon\bigg)<\infty$,由此$\bigg|\frac{S'_{\beta_n}-E[S'_{\beta_n}]}{\beta_n}\bigg|>\varepsilon\con{a.s.}0$,进而$\frac{S'_{\beta_n}}{\beta_n}\con{a.s.}\mu$。

\qquad STEP 3 收敛:由于$S_n'$单调增,$\beta_m\le n<\beta_{m+1}$有$\frac{\beta_m}{n}\cdot\frac{S'_{\beta_m}}{\beta_m}<\frac{S_n'}{n}<\frac{\beta_{m+1}}{n}\cdot\frac{S'_{\beta_{m+1}}}{\beta_{m+1}}$,取上下极限可估计得结论成立

*推论(\textbf{Borel强大数律}):试验中事件$A$发生概率$p$,$S_n$为$n$次独立重复试验中$A$发生次数,则$\frac{S_n}{n}\con{a.s.}p$。

\begin{thm} 重对数律

$X_i$独立同分布,期望$0$,方差$1$,则:

$P\bigg(\limsup_{n\to\infty}\frac{S_n}{\sqrt{2n\ln\ln n}}=1\bigg)=1$

$P\bigg(\liminf_{n\to\infty}\frac{S_n}{\sqrt{2n\ln\ln n}}=-1\bigg)=1$
\end{thm}

*意味着CLK不能加强到更强的收敛

\section{概率论外篇}
\subsection{信息熵}
记事件$E$发生概率$p=P(E)$,定义“惊奇程度”$S(p)$,基本要求:$S(1)=0$、$S(p)$严格单调减、关于$p$连续、$S(pq)=S(p)+S(q)$ (直观理解:独立事件引起惊奇程度为分别发生之和)。这些要求可确定:

\begin{thm} $S(p)=-c\ln{p},c>0$。
\end{thm}

证明:先考虑$p_0^{m/n}$,再由连续推到一切$p$。

\begin{defi} Shannon熵

离散型随机变量$X$,取不同点概率为$p_1,\dots,p_n,\dots$,定义$H(X)=-\sum_kp_k\ln{p_k}$。

联合熵:$H(x,y)=-\sum_{i,j}P(x_i,y_j)\ln{P(x_i,y_i)}$。

相对熵:$H_Y(X)=\sum_jH_{Y=y_j}(X)P_Y(y_j)$,其中$H_{Y=y_j}(X)=-\sum_iP(x_i|y_j)\ln{P(x_i|y_j)}$。
\end{defi}

注:$H_Y(X)=-\sum_{i,j}P(x_i,y_j)\ln\frac{P(x_i,y_j)}{P_Y(y_j)}$

\begin{thm}
$H(X,Y)=H(Y)+H_Y(X)$
\end{thm}

证明:直接计算知结论。

\begin{thm}
$H_Y(X)-H(X)\le0$
\end{thm}

证明:由$\ln{x}\le x-1$估计知结论。

*由\textbf{凸性}可知$P(X=x_i)=p_i,i=1,2,\dots,n$,$p_1=\dots=p_n=\frac{1}{n}$时$H(X)=\ln{n}$最大,即熵可代表\textbf{不确定程度}的大小。

~

\begin{defi} 连续型随机变量的熵
$X$连续,密度函数$f(x)$,则$H(X)=-\int_\mathbb{R}f(x)\ln{f(x)}\mathrm{d}x$。

联合熵:$H(X,Y)=-\int_{\mathbb{R}^2}f(x,y)\ln{f(x,y)}\mathrm{d}x\mathrm{d}y$。
\end{defi}

*\textbf{Gibbs不等式}(利用分析知识可证明):$u-u\ln{u}\le v-u\ln{v}$,积分得$\int_\mathbb{R}f(1-\ln{f})\mathrm{d}x\le\int_\mathbb{R}g(1-\ln{g})\mathrm{d}x$,再利用密度函数在实轴积分为1知$\int_\mathbb{R}-f\ln{f}\mathrm{d}x\le\int_\mathbb{R}-f\ln{g}\mathrm{d}x$。

\begin{thm} 熵最大的条件

令$D=\{x\in\mathbb{R}:f(x)\ne0\}$,则:

1. $D=\mathbb{R},E[X]=0,\Var(X)=1$时,正态分布$N(0,1)$熵最大,为$\ln\sqrt{2\pi\mathrm{e}}$。

2. $D=(0,\infty),E[X]=\frac{1}{\lambda}$时,指数分布$\Exp(\lambda)$熵最大,为$\ln\frac{\mathrm{e}}{\lambda}$。

3. $D=[0,a]$时,均匀分布$U(0,a)$熵最大,为$\ln{a}$。
\end{thm}

证明:分别取$g$为三种分布的密度函数,利用Gibbs不等式估算即可。

~

*\textbf{Boltzmann熵}:$S=k\ln\Omega$,$k=k_B$为玻尔兹曼常数,$\Omega$为微观状态数(类似离散型均匀分布时的情况)。

\subsection{Linderberg替换理论}
*L条件形式CLK:

设$a_k=E[X_k],b_k^2=\Var(X_k),B_n^2=\sum_{k=1}^nb_k^2$,$F_k$为$X_k$分布函数。

$\forall\varepsilon>0,\lim\limits_{n\to\infty}\frac{1}{B_n^2}\sum_{k=1}^n\int_{|x-a_k|>\varepsilon B_n}(x-a_k)^2\mathrm{d}F_k=0$,则$\frac{\sum_{k=1}^n(X_k-a_k)}{B_n}\con{D}N(0,1)$,且$\frac{\max_kb_k^2}{B_n^2}\con{}0$。

\begin{thm} $X_n\con{D}X$等价于下列条件之一:

1. $\forall g\in C_b(\mathbb{R})$(有界连续函数),$E[g(X_n)]\con{}E[g(X)]$。

2.	任意有界一致连续$g$,$E[g(X_n)]\con{}E[g(X)]$。

3. 给定$m\in\mathbb{N}$,$\forall g\in C_b(\mathbb{R}),g',g'',\dots,g^{(m)}\in C_b(\mathbb{R})$,$E[g(X_n)]\con{}E[g(X)]$。

4. $\varphi_{X_n}(t)\con{}\varphi_X(t)$(逐点收敛)。
\end{thm}

Linderberg思想:取至三阶导均有界连续的$g$,独立随机变量列$Y_n\sim N(0,b_n^2)$,与$X_n$亦独立。

定义$\zeta_{nk}=\sum_{1\le i<k}X_k+\sum_{k<i\le n}Y_i$,则$\zeta_{nn}+X_n=S_n,\zeta_{n1}+Y_1=B_n\cdot N(0,1)$,$\zeta_{nk}+X_k=\zeta_{n,k+1}+Y_{k+1}$,利用逐项相消,可将$X_i$\textbf{替换}为$Y_i$:

$E\left[g\left(\frac{S_n}{B_n}\right)\right]-E[g(Y)]=\sum_{k=1}^n\left(E\left[g\left(\frac{\zeta_{nk}+X_k}{B_n}\right)\right]-E\left[g\left(\frac{\zeta_{nk}+Y_k}{B_n}\right)\right]\right)$

再利用$\zeta_{nk},X_k,Y_k$独立泰勒展开估算$h(t)=\sup_x\{g(x+t)-g(x)-g'(x)t-\frac{1}{2}g''(x)t\}$,拆分证明。

\subsection{随机矩阵}
1. 起源

统计(样本协方差阵):

$X_k=\begin{pmatrix}X_{1k}&\cdots&X_{pk}\end{pmatrix}^T,X=\begin{pmatrix}X_1&\cdots&X_n\end{pmatrix}$为$p\times n$矩阵。

当$X_{ij}$独立同$N(0,1)$时,$X$的联合密度$f(X)=\frac{1}{\sqrt{2\pi}^{pn}}\exp\bigg(-\frac{1}{2}\tr(XX^T)\bigg)$

物理:波函数、随机矩阵模拟

~

2. 高斯正交系综(Gaussian Orthogonal Ensemble)

$X_n:\Omega\longrightarrow M_{n\times n}(\mathbb{R}),X_n(\omega)=(x_{ij}(\omega))_{i,j=1}^n$

$x_{ij}$独立同$N(0,\sigma^2)$,记$A_n=\frac{X_n+X_n^T}{2}$,计算可发现$a_{ii}\sim N(0,\sigma^2),a_{ij}\sim N\bigg(0,\frac{\sigma^2}{2}\bigg)(i\ne j)$,且$A_n$的上半三角部分独立,由此直接计算乘积可知$A_n$(上半三角)的分布:

$f(A_n)=2^{-n/2}(\pi\sigma^2)^{-n(n+1)/4}\exp\bigg(-\frac{1}{2\sigma^2}\tr A_n^2\bigg)$,记为$A_n\sim\GOE_n(\sigma)$

*正交变换下不变性:$Q$为正交阵,$B_n=Q^TA_nQ$,则$B_n\sim\GOE_n(\sigma)$

~

3. 半圆律

$X$分布函数$\omega(x)=\frac{1}{2\pi}\sqrt{4-x^2}$,记$\gamma_k=E[X^k]$,计算知其为$\begin{cases}0&k=2m+1\\\frac{1}{m+1}\mathrm{C}_{2m}^m&k=2m\end{cases}$。

实Wigner矩阵:$A_n=(a_{ij})$为实对称阵,$a_{ii}$独立与$Y$同分布,$a_{ij}(i>j)$独立与$z$同分布,$E[Y]=E[Z]=1,\Var(Z)=1,\Var(Y)<\infty$,$|Y|,|Z|$各高阶矩存在。

\begin{thm}
$k\in\mathbb{N},\frac{1}{n}E\bigg[\tr\bigg(\frac{A_n}{\sqrt{n}}\bigg)\bigg]\con{}\gamma_k$
\end{thm}

证明:左$=n^{-1-k/2}\sum_{i_1,\dots,i_k}E[a_{i_1i_2}a_{i_2i_3}\dots a_{i_ki_1}]$,类似定理5.14可证明不消失的项中必然每个不同的$a_{ij}$出现两次,进而说明$i_1,\dots,i_k$选取方法与$1,2,\dots,k$\textbf{不相交}(对任何$a<b<c<d$,不存在配对$(a,c),(b,d)$)的两两配对数($k=2m$时即为\textbf{卡特兰数}$C_m$)一一对应,再利用组合计算知结论。

~

4. Wishart矩阵模型

$X=(x_{ij})_{p\times n}$,矩阵元独立同$N(0,1)$,设$n-p=\alpha$固定,则$\frac{1}{p}E\bigg[\tr\bigg(\frac{1}{p}XX^T\bigg)^m\bigg]\con{}C_m$。

\end{document}