\documentclass[a4paper,UTF8]{ctexart}
\pagestyle{headings}
\title{\heiti 数学分析A3\ 课堂笔记}
\author{原生生物}
\date{}

\setcounter{tocdepth}{2}
\setlength{\parindent}{0pt}

\usepackage{amsmath,amssymb,enumerate,geometry,mathdots}

\geometry{left = 2.0cm, right = 2.0cm, top = 2.0cm, bottom = 2.0cm}

\ctexset{section={number=\zhnum{section}}}
\ctexset{subsection={name={\dag},number={}}}

\newtheorem{defi}{定义}[section]
\newtheorem{thm}{定理}[section]
\newtheorem{exmp}{例}[section]

\everymath{\displaystyle}
\begin{document}
\maketitle

\tableofcontents

\newpage

\section*{历史回顾}
\addcontentsline{toc}{section}{历史回顾}

\subsection{微积分}
在老师工作的基础下,为求函数在区间上的最值,\textbf{费马}构造差分,发明了导数。

三类初等函数微积分的历史:

\begin{enumerate}
	
\item 多项式函数

\textbf{牛顿}将函数(事实上此处应为初等函数)视为幂级数展开$f(x)=\sum_{n=0}^{\infty}a_nx^n$,又引入\textbf{二项式定理}计算出$x^n$的微分与积分,从而只要掌握多项式微积分即可算出任何初等函数微积分。

\textbf{莱布尼茨}将微积分看作满足一定计算规则的\textbf{计算方法}(Calculus) (如$(af+bg)^\prime=af^\prime+bg^\prime, (fg)^\prime=f^\prime g+fg^\prime, f(g(x))^\prime=f^\prime g(x)g^\prime(x)$,其中第二条被称为\textbf{莱布尼茨公式},推广到流形上有重要作用)。

*此处微分上的公式通过同积分即可推广到积分上,莱布尼茨公式对应分部积分。此外,通过复合函数求导公式可推出反函数求导公式。通过莱布尼茨公式可\textbf{递推}出多项式函数的微分。

\item 三角函数

\textbf{欧拉}用弧长定义弧度,进而定义角度,并给出了三角函数的定义与记号(重要极限:$\lim_{x\to0}\frac{\sin x}{x}=1$)。

*和角公式$\sin(\alpha+\beta)=\sin\alpha\cos\beta+\cos\alpha\sin\beta$等出发可进行平面坐标上的旋转,由此亦可计算三角函数的导数,再结合反函数求导公式可推出反三角函数的导数。

\item 对数函数、指数函数

对数函数出现先于指数函数,发明目的是将乘除变为加减($\log(a_1a_2)=\log(a_1)+\log(a_2)$)。

“等差数列”与“等比数列”之间的对应即为某种意义上的对数函数与指数函数在整数上的取值。为获取中间的值,需要编制\textbf{对数表}。

1617年,英国人\textbf{Briggs}编制了首张对数表(做法:通过二进制反复计算平方根逼近,组合出对应的小数次方)。

*计算平方根方式:先找到逼近的值,再通过$(x+\Delta x)^2\approx x^2+2x\Delta x$计算。

另一个重要极限:$\lim_{x\to 0}\frac{\log(1+t)}{t}$,化为计算$\left(1+\frac{1}{t}\right)^t$极限,由此出发定义\textbf{自然底数}$e$,进而得出对数函数的导数,而对数函数结合反函数求导公式可推出指数函数的导数。
\end{enumerate}

\textbf{费马}:极值点导数若存在,必然为0,新问题:函数是否存在极值?(涉及连续性理论与实数)

\textbf{牛顿-莱布尼茨公式}联系了微分与积分,从求导出发即可进行一些积分的计算。

\subsection{实数与连续性}
正整数的构造 - 表达整数(十进制) \textbf{九章算术}前

负整数的构造 \textbf{刘徽}之前

加入零\ 公元7世纪

*加法与乘法满足\textbf{基本运算律}(交换、结合、分配等)

分数的构造 - $(p,q)$的等价类\ 将\textbf{单位}变小,仍可满足基本规则

实数的构造 - 任意小数(单位\textbf{无穷}减小)

*事实上是将实数看成了有理数的\textbf{极限},由此可知仍满足基本规则

*实数可具有全序关系

实数的\textbf{完备性}(拓扑概念):任一柯西列($\{a_n\},\forall\varepsilon>0,\exists N,\forall n,m>N,|a_n-a_m|<\varepsilon$)必有极限

(证明:回顾实数完备性的六个等价定理)

*实数的其他构造方式:Dedekind分割、柯西列等价类等(注意\textbf{回到原始定义}证明的重要性)

由此出发可定义开集、闭集、聚点、连通性等(回顾A2中相关的点集拓扑基础)

*$\mathbb{R}$上的开集是可数个不交开区间的并(证明:对任何$a\in E$,考虑$\inf_tt<a,(t,a)\subset E$与$\sup_tt>a,(a,t)\subset E$即可)

连续函数等价定义:任意开集的\textbf{原象}是开集(可转化为$\varepsilon-\delta$语言,回顾连续等价条件)

连续函数性质:闭区间上有界、存在极值、介值定理(回顾连续相关性质)

\section{数项级数}
本质与数列等价(级数的\textbf{部分和}数列)

由此有直接的结论:若数项级数收敛,其\textbf{通项极限}必为0;级数增减\textbf{有限多}项不影响敛散性。

重要问题:判别敛散(判别法综合运用)。

\subsection{正项级数}
\begin{enumerate}
\item 积分判别法

原理:比较\textbf{面积}可证明,单调下降且极限为0的函数$f(x)$,$\sum_{n=1}^\infty f(n)$与$\int_1^\infty f(x)\mathrm{d}x$同敛散。

例:$\sum_{i=1}^\infty\frac{1}{n(\ln n)^p(\ln\ln n)^q}$收敛的条件为$p<1$或$p=1,q<1$(利用$n$为比任何$(\ln n)^\alpha$高阶的无穷大)。

\item 比较判别法

原理:正项级数对应\textbf{单调上升}数列,有界即收敛;若两不同正项级数的通项之比有界,则必然同敛散。

判别法化为\textbf{极限形式}:回忆A1,利用$\limsup_{n\to\infty}a_n=\lim_{n\to\infty}\sup_{m\ge n}\{a_m\}$,可将“存在无穷多个”与“至多有限多个”表示为上下极限。有时为方便使用,直接采取极限形式。

\item 柯西判别法

原理:与\textbf{等比数列}比较,考虑$\sqrt[n]a_n$与1的大小关系。

例:$\sum_{n=1}^\infty\frac{x^n}{n}$收敛的条件为$-1\le x<1$(柯西判别法处理$x$非负时情况,为负时须后续知识)。

\item 达朗贝尔判别法

原理:与\textbf{等比数列}比较,考虑相邻项之比与1的大小关系。

例:$\sum_{n=1}^\infty\frac{x^n}{n!}$必然收敛(此时用此判别法可规避斯特林公式)。

与柯西判别法关系:效果更弱,但有时好用。

\item 拉贝判别法

原理:与$\frac{1}{n^\sigma}$比较,考虑相邻项之比减一的无穷小情况。

\item $\triangle$高斯判别法

原理:与$\frac{1}{n(\ln n)^\sigma}$比较,考虑相邻项之比减一的无穷小情况。
\end{enumerate}

\subsection{任意项级数}
*由数项级数而定义:收敛、发散点集

\begin{enumerate}
\item 柯西收敛准则

原理:直接对部分和数列利用柯西收敛准则即得结果,是任意项级数收敛的\textbf{充分必要}条件。

例:若某级数所有项取绝对值后得到的正项级数收敛,则此级数必然收敛(此时称此级数\textbf{绝对收敛});反之,若级数收敛但取绝对值得到的级数不收敛(如$(-1)^n\frac{1}{n}$),则称此级数\textbf{条件收敛}。

\item 莱布尼茨判别法

原理:类似积分判别的证明方式,估算交错级数部分和不同子列。

\item 迪利克雷判别法

证明:利用\textbf{分部求和}公式改写和式,从而放缩知收敛。

作用:考虑乘积是否收敛时可拆分判定。

例:$\sum_{n=1}^\infty\frac{\cos nx}{n}$,$x=0$时发散,否则令$a_n=\cos nx,b_n=\frac{1}{n}$,可计算出$\sum_{n=1}^ka_n=\frac{\cos\frac{k+1}{2}x\sin\frac{k}{2}x}{2\sin\frac{x}{2}}$有界,因此原级数收敛。

\item 阿贝尔判别法

原理:变形,$\sum_{k=1}^na_kb_k=\sum_{k=1}^na_k(b_k-b)+b\sum_{k=1}^na_k$。

与迪利克雷判别法关系:互有强弱,根据具体情况运用。
\end{enumerate}

\textbf{黎曼定理}:条件收敛的特殊性质,安排顺序收敛到任意目标。

证明:取出其中的正项与负项,由条件知正项与负项的和均发散,从而可安排顺序构造出结果。

$\triangle$其他内容:绝对收敛可交换次序、级数相乘、无穷乘积(取$\ln$后化为求和,利用与1差距计算)

\section{函数项级数}
\subsection{重要问题}
若函数$S_(x)$为函数$S_n(x)$极限(可看作函数项级数$\sum_{n=1}^\infty u_n(x)$):

(由此出发须定义\textbf{一致收敛})

\begin{enumerate}
\item $S_n(x)$连续,极限是否连续?

反例:$S_n(x)=x^n,x\in[0,1]$

加条件:$S_n(x)$一致收敛,则成立。

证明:由定义估算。

\item $S_n(x)$可积,积分是否可与求和交换?

反例:$S_n(x)=2n^2x\mathrm{e}^{-n^2x^2},x\in(0,1)$

加条件:$S_n(x)$一致收敛,则成立。

证明:利用保连续,由不连续点零测集可数并零测可证明。

另一种“加条件”做法:\textbf{黎曼可积}拓展为\textbf{勒贝格可积}

\item $S_n(x)$可导且导数连续,求导是否可与求和交换?

反例:$S_n(x)=\frac{\sin nx}{\sqrt{x}},x\in\mathbb{R}$

加条件:$S_n^\prime(x)$一致收敛,且$S_n(x)$至少某一点收敛,则成立。

证明:利用保积分,使用牛顿-莱布尼茨公式证明。
\end{enumerate}

\subsection{一致收敛判据}
\begin{enumerate}
\item 柯西判别法

原理:类似柯西准则,为充要条件。

\item $\triangle$已知收敛结果$f$时,$f_n$在$I$上一致收敛等价于$\lim_{n\to\infty}\sup_{x\in I}|f_n(x)-f(x)|=0$。

证明:利用柯西准则。

\item 魏尔斯特拉斯判别法

原理:利用柯西判别法,与数列比较。

\item 迪利克雷判别法

原理:类似数列的迪利克雷判别法,注意利用\textbf{一致有界}条件。

例:$\sum_{n=1}^\infty\frac{\cos nx}{n},x\in[\delta,2\pi-\delta](0<\delta<\pi)$类似数列时利用迪利克雷判别法可判定一致收敛。

\item 阿贝尔判别法

原理:类似数列的阿贝尔判别法(注意此时无法由迪利克雷判别法直接推得)。

\item Dini定理

原理:类似证明闭集连续函数闭一致连续,利用有限覆盖,可控制全区间大小。
\end{enumerate}

*证明一致收敛技巧:\textbf{分段}估计(若在两段均一致收敛则并集仍一致收敛)、注意两个\textbf{充要}条件

例:$\sum_{n=1}^\infty(1-x)\frac{x^n}{1-x^{2n}}\sin(nx)$一致收敛,通过拆分为$[0,a]$(此段可放大为$a^n$)与$[a,1]$(此段$\sin(nx)$一致有界)两段可以说明。

*证明不一致收敛技巧:从原始\textbf{定义}出发、考虑边界处、利用\textbf{柯西准则}找反例

例:考虑$\sum_{n=1}^\infty\frac{\sin(nx)}{n}$。对任何$N$,可以取$x=\frac{1}{N+1}$,第$N+1$项到第$2N$项均大于$\frac{\sin1}{2N}$,因此和大于$\frac{\sin1}{2}$,由定义知其不一致收敛。

\subsection{幂级数}

(复变函数中有重要推广)

考虑$\sum_{n=0}^\infty a_nx^n$,利用柯西判别知$|x|<R=(\limsup_{n\to\infty}\sqrt[n]{|a_n|})^{-1}$时收敛,大于则发散(等于时无法确定)。

*定义满足这样条件的$R$为级数的\textbf{收敛半径}(复变函数中成为圆)。

*对任何$0<r<R$,$\sum_{n=0}^\infty a_nx^n$在$[-r,r]$上一致收敛(\textbf{内闭一致收敛})。

证明:将$x$放大为$r$,利用魏尔斯特拉斯判别法。

由此其满足之前所述的\textbf{保求和}、\textbf{保导数}、\textbf{保积分}等性质。

例:$\ln(x+1)=\int_0^x\frac{1}{t+1}\mathrm{d}t=\int_0^x\sum_{n=0}^\infty(-1)^nt^n\mathrm{d}t=\sum_{n=1}^\infty(-1)^{n-1}\frac{x^n}{n}$

*初等函数可用多项式逼近,因此其幂级数相当于\textbf{无穷阶泰勒展开},回忆\textbf{中值定理}所推导出的几种\textbf{余项}形式(拉格朗日余项、柯西余项、多重积分余项)。

*记忆基本初等函数的幂级数展开形式。

$\triangle$一般地说,无穷阶可导未必可以表示为幂级数展开,若可表示则称为\textbf{实解析函数}。

魏尔斯特拉斯逼近定理:闭区间上函数可用多项式一致逼近$\Leftrightarrow$函数连续。

证明:构造\textbf{伯恩斯坦多项式},观察其性质,并估算、控制误差。

$\triangle$ \textbf{Abel定理}与Tauber定理:判定幂级数在边界上的性质

\subsection{特殊例子}
*利用函数项级数可构造出一些特殊的映射

\begin{enumerate}
\item 存在处处连续,处处\textbf{不可微}的函数
	
最早构造-魏尔斯特拉斯利用\textbf{三角函数}级数
	
范德瓦尔登“化曲为直”,更直观构造。
	
证明:级数的每项都是连续函数,利用魏尔斯特拉斯判别法可知一致收敛,从而极限处处连续。计算导数可发现极限可写为某个$\pm1$组成的级数,由于通项不趋向0,级数不可能收敛,故处处不可微。

\item 填充正方形的曲线

存在线段到正方形的连续映射(\textbf{皮亚诺曲线})

证明:仍利用魏尔斯特拉斯判别法推出此映射连续。考虑正方形中某点的二进制表示,可构造合适的$t_0$收敛至此点。
\end{enumerate}

$\triangle$ 由于$\int_a^b f=\lim_{A\to b^-}\int_a^A f$ (此处$b$可为$\infty$),反常积分可与级数类似方法处理

\section{傅里叶分析}
\subsection{定义与计算}
关心重点:\textbf{周期函数}(周期足够大可逼近任何函数)

一般函数用级数$f(t)=\sum_{n=0}^\infty A_n\sin(n\omega t+\varphi_n)=\sum_{n=0}^\infty (a_n\sin(n\omega t)+b_n\cos(n\omega t))$逼近

标准展开形式:$f(t)=\frac{a_0}{2}+\sum_{n=1}^\infty (a_n\sin(n\omega t)+b_n\cos(n\omega t))$ (此处$f$以$2\pi$为周期)

性质:\textbf{正交性}

$\int_{-\pi}^\pi\cos(nx)\cos(mx)\mathrm{d}x=\int_{-\pi}^\pi\sin(nx)\sin(mx)\mathrm{d}x=\begin{cases}\pi&m=n\\0&m\neq n\end{cases},\int_{-\pi}^\pi\cos(nx)\sin(mx)\mathrm{d}x=0$

由此可计算系数:$\int_{-\pi}^\pi f(x)\cos(nx)\mathrm{d}x=\pi a_n,\int_{-\pi}^\pi f(x)\sin(nx)\mathrm{d}x=\pi b_n$ (注意对$a_0$亦成立)

例1:考虑函数$f(x)$在$[-\pi,\pi)$上为$x$,以$2\pi$为周期,计算可得$a_n=0,b_n=(-1)^{n-1}\frac{2}{n}$,可发现逼近的结果在$(-\pi,\pi)$上为$f(x)$,端点处为0,而$f(\pi)=-\pi$。

例2(\textbf{锯齿波}):考虑函数$f(x)$在$[-\pi,\pi)$上为$|x|$,以$2\pi$为周期,此时$b_n=0,a_n=\begin{cases}-\frac{4}{n^2\pi}&2\nmid n\\0&2\mid n\end{cases},a_0=\pi$。

(考虑边界处可得$\sum_{n=0}^\infty\frac{1}{(2n+1)^2}=\frac{\pi^2}{8}$)

例3:考虑函数$f(x)$在$[-\pi,\pi)$上为$x^2$,以$2\pi$为周期,此时$b_n=0,a_n=(-1)^n\frac{4}{n^2},a_0=\frac{2\pi^2}{3}$。

(考虑边界处可得$\sum_{n=1}^\infty\frac{1}{n^2}=\frac{\pi^2}{6}$)

例4:$f(x)=\cos(ax),a\notin\mathbb{Z},x\in(-\pi,\pi)$,计算可得$b_n=0,a_n=\frac{(-1)^n}{\pi}\frac{a^2}{a^2-n^2}\sin(a\pi),a_0=\frac{2}{a\pi}\sin(a\pi)$。

(考虑边界处可得$\sum_{n=1}^\infty\frac{(-1)^na^2}{\pi(a^2-n^2)}+1=\frac{a}{\sin(a\pi)}$)

*考虑边界处可发现\textbf{恒等式}

\subsection{敛散性判别}

*$f$分段可微,间断点有限,则傅里叶级数逐点收敛于$\begin{cases}f(x_0)&x_0$为连续点$\\\frac{1}{2}(f(x_0^-)+f(x_0^+))&x_0$为间断点$\end{cases}$

$\triangle$ 黎曼-勒贝格引理证明:等分区间,由定义估算。

收敛性证明:利用三角恒等式代换为\textbf{迪利克雷积分},将问题转化为积分的极限是否存在。

~

\textbf{平方可积}意义下的收敛性:若$\lim_{n\to\infty}\frac{1}{\pi}\int_{-\pi}^\pi|S_n-f|^2\mathrm{d}x=0$,则称为$S_n$积分意义下收敛于$f$。

(此处由完备性应采取\textbf{勒贝格积分},目前先以黎曼积分讨论)

$S_n=\frac{a_0}{2}+\sum_{k=1}^n(a_k\cos(kx)+b_k\sin(kx))$,计算得$\frac{1}{\pi}\int_{-\pi}^\pi|S_n-f|^2\mathrm{d}x=\frac{1}{\pi}\int_{-\pi}^\pi f-\frac{a_0^2}{2}-\sum_{k=1}^n(a_k^2+b_k^2)\ge0$,此即为\textbf{帕塞瓦尔不等式}。

由于$\frac{1}{\pi}\int_{-\pi}^\pi f-\frac{a_0^2}{2}-\sum_{k=1}^n(a_k^2+b_k^2)$单调有界,极限必然存在,因此积分意义下只需说明极限不能大于0。

定理:黎曼积分下,$f$连续可推出此式极限为0 (勒贝格积分下:只需$f$可积且平方可积)。

证明:令$T_n=\frac{\alpha_0}{2}+\sum_{k=1}^n(\alpha_k\cos(kx)+\beta_k\sin(kx))$,可发现$\frac{1}{\pi}\int_{-\pi}^\pi|T_n-f|^2\mathrm{d}x=\frac{1}{\pi}\int_{-\pi}^\pi|S_n-f|^2\mathrm{d}x+\frac{1}{\pi}\int_{-\pi}^\pi|S_n-T_n|^2\mathrm{d}x\ge\frac{1}{\pi}\int_{-\pi}^\pi|S_n-f|^2\mathrm{d}x$,也即$S_n$为$n$阶三角多项式下的\textbf{最佳逼近}。

由此,只要$f$可用三角多项式逼近(积分意义下),即可说明结论。

*任何连续函数$f$可用三角多项式逐点逼近。

引理:连续函数的傅里叶级数在Ces\`aro意义上一致收敛于原函数(利用迪利克雷积分计算)。

由引理出发可以直接构造出逐点逼近序列。

\subsection{傅里叶变换}
*由于三角函数求导周期性,傅里叶级数展开在\textbf{微分方程}中有重要应用。

例:热传导方程$\frac{\partial u}{\partial t}=\frac{\partial^2u}{\partial x^2},u(0,x)=\varphi(x)$,写出$u$对$x$的傅里叶级数可算出解的级数表示。

~

*从$[-l,l]$上展开拓展到$(-\infty,+\infty)$上展开,由此得出傅里叶\textbf{积分公式}。

由\textbf{绝对可积}出发可说明此积分在有限时的和,趋向无穷时的收敛性可由\textbf{Dini定理}判别。

*绝对可积与广义左右导数存在可推出傅里叶积分收敛结果

~

由此得出傅里叶\textbf{正弦}、\textbf{余弦}变换公式,写为复数形式即得\textbf{傅里叶变换}公式。

*傅里叶展开可看作\textbf{离散形式}的变换

*傅里叶变换可将\textbf{卷积}化为乘积

\section{含参变量积分}
*按照常义积分与反常积分分别讨论

\subsection{反常积分的一致收敛}
定义:本质与函数项级数一致收敛相同

判别:

\begin{enumerate}
\item $\triangle$ 关于$\eta(A)$的\textbf{充要条件}

与函数项级数时充要条件类似,可以此直接计算说明是否一致收敛

\item 柯西收敛原理

充要条件,常用于反例构造

\item 魏尔斯特拉斯判别法

与连续函数的无穷积分比较,柯西收敛原理说明

*条件\textbf{较强},难以使用

\item 迪利克雷判别法

利用积分\textbf{中值定理}分段放缩可说明

\item 阿贝尔判别法

利用\textbf{分部积分}计算

*当只有单个因子包含$u$时情况更加简化

\end{enumerate}

\subsection{重要问题}
*重点观察:关于$u$的函数$\varphi(u)=\int_a^bf(x,u)\mathrm{d}x$性质(可看作函数项级数处理)

\begin{enumerate}
\item $\varphi(u)$是否连续?

对常义积分:二元函数$f$\textbf{连续}时,$\varphi$必然连续。

$\triangle$ 此处事实上可弱化为$f$在区间上\textbf{可积}(难证,利用可积函数由连续函数\textbf{逼近}说明)。

对反常积分:添加一致收敛后连续成立(通过拆分逼近)。

\item $\varphi(u)$是否可导?

对常义积分:$f$与$\frac{\partial f}{\partial u}$存在且连续时,可由定义直接计算导数

*$a,b$为常数时,可直接交换求导,否则考虑\textbf{拆分为复合函数}可计算出导数

对反常积分:求导后连续且一致收敛可推出求导与积分可交换

\item $\varphi(u)$是否可积?
	
对常义积分:$f$连续时积分号可以交换顺序(A2知识)

对反常积分:添加一致收敛,拆分逼近知对$u$常义积分可与广义积分交换

$\triangle$ 对$u$广义积分时,首先需对$x,u$分别的广义积分均一致收敛,再添加\textbf{绝对可积}条件

$\triangle$ 非负时,利用Dini定理可弱化条件
\end{enumerate}

*积分计算

例:$I(\alpha)=\int_0^\infty\mathrm{e}^{-\alpha x}\frac{\sin{x}}{x}\mathrm{d}x$,分部可算得$I'(\alpha)=-\frac{1+I'(\alpha)}{\alpha^2}$,由此直接积分有$I(\alpha)=-\arctan\alpha+\frac{\pi}{2}$。特别地,$I(0)-\frac{\pi}{2}$。

\subsection{$\Gamma$函数与$\mathrm{B}$函数}
$\Gamma(s)=\int_0^\infty t^{s-1}e^{-t}\mathrm{d}t$

$s>0$时,$\Gamma(s)$收敛,可任意阶求导(\textbf{一致收敛}性)。

$\Gamma$函数性质:
\begin{enumerate}
	\item $S>0$上恒正且$\Gamma(1)=1$
	\item 分部积分可得递推$\Gamma(s+1)=s\Gamma(s)$,由此$\Gamma(n)=(n-1)!$
	\item 利用\textbf{赫尔德不等式}可说明$\ln\Gamma(s)$为凸函数
\end{enumerate}

$\triangle$真正的性质在\textbf{复变函数}中

*与之相对,由此三条可\textbf{唯一确定}出$\Gamma$函数

~

$\mathrm{B}(p,q)=\int_0^1t^{p-1}(1-t)^{q-1}\mathrm{d}t$

分部积分知有递推$\mathrm{B}(p+1,q)=\frac{p}{p+q}\mathrm{B}(p,q+1)$

两函数联系:$\mathrm{B}(p,q)=\frac{\Gamma(p)\Gamma(q)}{\Gamma(p+q)}$ (可通过分析三条性质或直接计算说明)

*可推出\textbf{斯特林公式}

\section{流形}
\subsection{反函数、隐函数}



\end{document}