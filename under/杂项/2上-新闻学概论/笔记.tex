\documentclass[a4paper,UTF8]{ctexart}
\title{\heiti 新闻学概论\ 课堂笔记}
\author{原生生物}
\date{}

\setcounter{tocdepth}{2}
\setlength{\parindent}{0pt}

\usepackage{enumerate,geometry,ulem}

\geometry{left = 2.0cm, right = 2.0cm, top = 2.0cm, bottom = 2.0cm}

\ctexset{section={name={第,章},number=\zhnum{section}}}
\ctexset{subsection={number=\zhnum{subsection}}}
\ctexset{subsubsection={name={\S},number=\arabic{subsection}.\arabic{subsubsection}}}

\newcommand{\To}{$\to$}
\begin{document}
\maketitle

\tableofcontents

\newpage
\section{绪论}
\subsection{新闻学与新闻工作}

新闻学分类:新闻理论、新闻史、新闻业务(采写编评、报刊发行)、媒体管理与经营

\textbf{课程内容 - 新闻理论、新闻业务(采访、写作)}

起源:从业务逐渐成为学科

~

新闻工作:多种学科(社会学、法学等)和实践经验的综合运用

核心:\textbf{分析}能力、\textbf{判断}能力(对写作能力要求未必高)

\subsection{世界各国新闻学主导理论}
\subsubsection{新闻媒介三大运行体系}
\begin{enumerate}
	\item 私有制为主体,完全商业化运作
	
	代表:美国
	
	\item 公私并举,双轨制运作
	
	代表:西欧各国
	
	\item  完全国有,有限商业运作
	
	代表:中国
\end{enumerate}

\subsubsection{四种主导性理论}
\begin{enumerate}
	\item 自由主义报刊理论
	
	起源于欧洲,盛行于北美
	
	\subitem 报刊独立自主,\textbf{不受政府干涉}
	\subitem 报刊拥有对政府的监督权(三权分立外第四种权力)
	\subitem “自我修正”理论
	\subitem 对事实的信念
	
	著作:密尔 \textbf{论自由}、弥尔顿 \textbf{论出版自由}
	
	局限性:带有\textbf{片面性}与\textbf{空想成分},理性至上被利润至上取代
	
	\item 社会责任论
	
	美国学者构建,被西方大多数国家接受
	
	\textbf{自律}(媒体自身)+\textbf{他律}(政府干涉)
	
	\subitem 就当日事件在赋予其意义的情境中真实、全面、智慧的报道
	\subitem 交换评论与批评的论坛
	\subitem 供社会各群体互相传递意见与态度的工具
	\subitem 呈现与阐明社会目标与价值观的方法
	\subitem 将新闻界提供的信息流、思想流和感情流送达每一个社会成员的途径
	
	允许政府\textbf{设立自己的媒介}
	
	最大成就:新闻从业人员的教育转向\textbf{社会责任}的培训
	
	\item 发展新闻学
	
	主要在发展中国家盛行
	
	\subitem 服从、服务、促进国家尤其\textbf{经济发展}
	\subitem 守望(传播国内外重大信息)
	\subitem 整合(缓和社会矛盾)
	\subitem 教育(教育大众遵纪手法,推广新技术)
	
	\item 党报理论
	
	新闻媒介是党和政务的\textbf{喉舌}(宣传工具)与\textbf{耳目}(意见传达到上层)
\end{enumerate}

中国:发展新闻学+党报理论

\subsection{中国新闻制度发展}
\subsubsection{发展历程}
\begin{enumerate}
	\item 单一报种时期\ 规模小\  “眼睛向上”\ “老干部满意”
	\item 多报种时期\ 快速发展\ “一只眼睛向上,一只眼睛向下”
	\item 厚报时期\ 竞争白热化\ “二次售卖”-产品与注意力(\textbf{广告})
	
	政府、百姓、广告商“三满意”
	
	同质化严重-理想“一城一报”
	
	\textbf{创新}极为重要(粗糙的创新大于技巧的圆满)
\end{enumerate}

标志意义事件:

1981 新闻多样化问题提出\ 功能单一\To 多样

1991 “周末版”盛行\ 传播者本位\To 受众本位

1993 “东方时空”播出\ 宣传本位\To 新闻本位

1997.1 广州日报报业集团成立\ 承认新闻\textbf{产业性}

*大部分报业集团只是\textbf{物合},没有做到\textbf{化合}(资源综合配置)

做得较好:解放日报报业集团、南方日报报业集团

~

报业\textbf{趋同}:

50-80初\ 机关报趋同(政策性)

80中-90中\ 晚报趋同(模仿性) (晚报时期大量重复建立,浪费资源)

90中起\ 市场报趋同(竞争性)

\subsubsection{中国新闻改革}
难点:

新闻体制(媒体带有机关、事业、企业属性)

时政报道(关注度低)

党报(与其他报差异消解)

*\textbf{大报小报化}:大报(如日报)符合党报,关注时政新闻,但因时政新闻关注度低,报道小报内容

\textbf{小报大报化}:晚报、都市报做大后追求时政新闻报道

小众(分众)传播逐渐趋同,\textbf{专业化}消解

~

新业态:

互联网时代\ 真vs假、全面vs碎片化、信息充足vs查证缺失 \textbf{挖掘事实真相}变为重要技能

全球化时代

风险社会时代(\textbf{不放心、不相信})

~

当前情势

党报 - 繁华背后存在危机

晚报 - 死而不僵

都市报 - 方兴未艾

城市报 - 先天不足、发育不良

行业报 - 专业领航、大有可为

社区报 - 小荷才露尖尖角

\subsubsection{新闻未来发展}
\begin{enumerate}
	\item 媒体\textbf{融合}与融合新闻
	
	传统媒体与现代媒体融合发展,建立立体多样的\textbf{现代传播体系}
	
	\item \textbf{互联网思维}应用于新闻业
	
	\subitem 传统媒体是否接受?
	\subitem 传统媒体与现代媒体是否存在本质差异?(点、面新闻与\textbf{纵深}树)
	
	创新样本:赫芬顿邮报(公民记者)、Buzzfeed
	
	\item 大数据对新闻业的影响
	
	描述、判断、预测、定制
	
	*新闻游戏化、\textbf{数据新闻}(数字是骨骼,设计是灵魂)、新闻可视化(过于抽象与复杂则不适合)
\end{enumerate}

\section{新闻学基础知识}
\subsection{新闻活动}
*人类求生存图发展的必要

原始社会\ 群居,借助火、甲骨文等

奴隶社会\ 注意奴隶举动,探听商业信息

封建社会\ 驰道、驿站\ 运物资、通情报

~

\textbf{报纸}的产生 16-18世纪\ 三个阶段:

\begin{enumerate}
	\item 手抄报新闻(第一批新闻工作者) 单份-多份粘贴到房间里(新闻房)-沿街叫卖-定制 (分众传播雏形)
	\item 新闻书的出现\ 铅字印刷,但周期很长 (大众传播雏形)
	\item 周刊、日报的勃兴 1609德国-观察周刊(最早周刊) 1633德国-莱比锡新闻(最早日报)
	
	1703英国-每日新闻(最早现代报纸) 新闻事业真正诞生
\end{enumerate}

\subsection{新闻定义}

新闻定义:\textbf{陆定一}-新近发生的事实的报道(形式)、新近发生的事实\textbf{变动}的信息(内容、本质)

从业人员品质\ 观察\textbf{敏锐}、\textbf{真实}报道

*新闻是一种\textbf{报道}

\textbf{丁伯铨}-对新近发生的或正在发生的受众欲知、未知而应知的事实的报道

两点补正:强调报道正发生事实、强调对事实的选择

~

报道、分析、判断的区别:

报道-对可查证事实的客观陈述(不允许“合理想象”)

分析-建立在事实基础上的,对事实的解释与未来趋势的预测

*新闻中应用\textbf{事实}而非观点来解释事实

判断-对利弊、荣辱、是非、善恶所作的结论

~

新闻工作中:

采访-\textbf{眼睛}比嘴巴重要(观察)

分析-考验思想、理论水平

*在新闻中应尽可能提供事实,也可作些分析,但尽可能避免判断,将\textbf{判断转化为报道}

*注意\textbf{事实判断}与\textbf{价值判断}的区别

\subsection{新闻本源与新闻来源}
新闻的本源(哲学层面上\ 新闻是什么):\ 事实(真正有用-事实的变动)

*\textbf{挖掘}有含义的变动

*新闻的\textbf{角度}非常重要(以小见大)  同题竞争中-求新求异

写新闻时的注意:变化是什么、变化的意义是什么、用什么角度去写

~

新闻的来源:操作层面上\ 新闻如何获得

新闻来源最基本的时\textbf{权威}、\textbf{可靠}

*报道新闻时要善于借助方方面面的权威,并\textbf{注明出处}(“深喉”)

~

新闻要素:人物Who 时间When 地点Where,事件What 原因Why 过程How

\subsection{新闻类别}
标准多内容分类、发生地分类、时间性分类等

*事件性质分类:硬新闻与软新闻

硬新闻:国计民生及切身利益的新闻

\textbf{“抢新闻”}、报道须尽可能迅速

软新闻:与切身利益无直接关系,富有人情味

晚报新闻多、\textbf{耐压}(谈资)、写作技巧要求高

\subsection{新闻特征}
\begin{enumerate}
	\item 具体性(事实性)
	\item \textbf{真实性}(\textbf{不能虚构}、加工、想象、弄虚作假)
	
	新闻与小说关系:
	
	观点一:两者相互矛盾(真实性冲突)
	
	观点二:两者不冲突,某些方面相得益彰(都要\textbf{深入了解}事实)
	
	~
	
	*报道的策划与“\sout{策划的报道}”
	
	新闻策划:对已发生或将发生的事实准备报道
	
	策划新闻:捏造未发生事实为新闻
	
	*报网联动(\textbf{竞合}传播)好处:延申记者触角、整合媒体资源、节省投入成本、\textbf{放大传播效应}
	
	~
	
	反面:新闻\textbf{失实}、虚假新闻泛滥(\textbf{网帖新闻化}与\textbf{新闻(媒体)网帖化})
	
	虚假新闻后果严重(“星球大战”广播剧)
	*规避隐性失实(以偏概全、缺乏具体调查)
	
	\item 时效性(时间新、事物新)
	\item 重要性
	\item 及时传播性
\end{enumerate}

\subsection{新闻中的意识形态}
传媒工作者对信息的搜集、\textbf{选择}、加工中,体现了价值判断体系从而折射\textbf{意识形态}性质。

\begin{enumerate}
	\item 通过媒介语言呈现(\textbf{经济学人}封面:文字表达中性化,图片补充文字内容,\textbf{间接性})
	\item 呈现事实以表现观点(\textbf{藏舌头})
\end{enumerate}

新闻的客观公正性:记者与媒体的\textbf{偏见}、\textbf{成见}

~

新闻媒体与社会公平:

\begin{enumerate}
	\item 观念层面(\textbf{共识}性理念)
	\item 制度层面(相对刚性的\textbf{制度约束})
	\item 实践层面
\end{enumerate}

~

*新闻中的\textbf{人文关怀}(对灾难、痛苦不应调侃、娱乐化)

\section{新闻理论}
\subsection{新闻与信息}
新闻中信息含义(狭义):能\textbf{消除}人们随机、\textbf{不确定}的东西

*对于决策非常重要

新闻报道信息时,新闻\textbf{前景}与背景信息组合得到价值导向(信息\textbf{组合}决定新闻含义)

信息特性:

\begin{enumerate}
	\item 共享性(使用不灭性) 信息与物质的区别,\textbf{无限传播}的前提
	\item 扩缩性\ 编辑部、受众需要比内容更决定文章长短
	\item 组合性\ 信息有机组合可以产生新的信息\ 新闻\textbf{背后}的新闻比新闻本身重要
	\item 运用的多角度性\ 存在不同层面、角度的看法
	\item 相对性\ 与人对外界信息\textbf{选择}性密切相关,要求新闻工作者了解受众需要
\end{enumerate}

*通讯与信息的扩缩性有关(讯最早指电报,需要尽量\textbf{简洁}),深度报道则应用信息的组合、多角度性。

*每一条新闻要\textbf{穷尽}一条信息(尽力消除不确定性)

对新闻工作者要求:分阶段连续报道、加强深度报道、加强综合评述

\subsection{新闻与宣传}
\textbf{拉斯韦尔}定义宣传:以操纵\textbf{表述}来影响人们行动的技巧

新闻:重信息、忌重复旧闻、重事实、重时效、信息对称、注重平衡原则

宣传:重形式符号、重复施教、重观点、重时机、信息不对称、注重某一方面

\begin{enumerate}
	\item 最大区别:新闻传播信息,宣传传播观念(\textbf{受着晓其事,传者扬其理})
	\item 出发点:新闻为受众,宣传为宣传者
	\item 归宿:新闻为发布信息,宣传为收买人心
	\item 传播方式:新闻为新鲜材料,宣传为旧材料
	\item 传播要求:新闻要求定量准确,宣传要求定性准确
\end{enumerate}

*新闻用\textbf{事实}包含了编者的观点,而不是直接表达观点

*新闻与宣传有交错,且存在此消彼长的相互矛盾\textbf{相互作用}中

~

广告(\textbf{商业}宣传)基本特点:\textbf{迎合}个人眼前需要

与宣传区别:更注重眼前利益

新闻、宣传、广告间有千丝万缕的\textbf{联系}

~

文艺要塑造形象,给人美与\textbf{感性}的享受

*文艺中往往渗透着各种宣传甚至广告的成分(\textbf{植入性广告}愈发频繁,直接、间接生效)

~

宣传要素:

\begin{enumerate}
	\item 宣传者\ 自己人(\textbf{亲和})、明星(\textbf{榜样})、中立者(\textbf{客观})
	\item 宣传对象\ 尊重宣传对象,了解特点、需要
	\item 宣传内容\ 三\textbf{真}:真理、真实、真话
	\item 宣传环境\ 开放式,须收集更多材料
	\item 宣传时机\ 针对\textbf{热点}、\textbf{焦点}
	\item 宣传艺术
	
	西方:美化、丑化、借助神明(偶像)、以百姓自居、洗牌作弊、号名随大流(乐队花车)
	
	中国:无我(不能明显呈现意图)、求同存异、适度(避免\textbf{过度诠释})、扮演角色
\end{enumerate}

\subsection{新闻与舆论}
舆论:特定时空对特定社会公共事务\textbf{公开表达}的\textbf{基本一致}的意见、态度。

\begin{enumerate}
	\item 公开性:公开讨论形成、公开表达
	\item 公共性:指向与目标的公共性
	\item 急迫性:近在眼前的问题
	\item 广泛性:存在与影响均广泛
	\item 评价性:是一种意见、判断
\end{enumerate}

*\textbf{控制}作用:监督、约束国家政权与\textbf{政府};鼓舞、约束\textbf{公众}行为

~

新闻媒介对舆论介入阶段:

\begin{enumerate}
	\item 反映并代表舆论(\textbf{民意})
	\item 引发舆论(\textbf{议程设置}:媒介通过选择、强调,建构一个\textbf{拟态环境})
	\item 引导舆论
\end{enumerate}

舆论的形成:

\begin{enumerate}
	\item 引发阶段
	\item 形成阶段(新闻与\textbf{网络}相互影响,加速形成)
	\item 影响扩大阶段
	\item 影响消退阶段
\end{enumerate}

~

*\textbf{谣言}的产生:信息不充分、对权威部门信息缺乏信任

以口头传播为主,大众传播次要,但后者影响更广泛、深刻

大众传播介入原因:无意为之的\textbf{失实报道}与有目的的操纵传播

对谣言的理解:肯定性理解(与心理相吻合、\textbf{盲从})、否定性理解(理性思考或已有成见)

谣言消除:自然消除、人为消除(\textbf{辟谣}):

	\ \ 攻击性谣言\ 清算消息来源
	
	\ \ 牢骚性谣言\ 培养公众理性(理性思考的舆论氛围)
	
	\ \ 误解性谣言\ 公布事实真相

~

舆论的影响力:对思想、态度、行为影响巨大

*应对舆论的时机重要,\textbf{观点爆炸性挤兑}(阐释的狂欢)

*媒介、舆论、政府\textbf{三角关系},互相监督

*\textbf{初始报道}对舆论产生非常重要

*\textbf{媒体审判}现象:新闻报道形成舆论压力,妨碍、影响司法独立与公正

发展过程:舆论开端(“自媒体”言论)、发展(主流媒体跟进)、膨胀(深度挖掘)、整合(趋于一致)、消散

*\textbf{权力异化}现象:媒体从业者受商业利益腐蚀

*舆论监督中需要注意之处:

\begin{enumerate}
	\item 真实性(报道热点与炒作热点、策划新闻与制造新闻)
	\item 手段(暗访的合理合法性、亲历式报道中的职业道德)
	\item 保密与隐私(国家机密、隐私权)
	\item 罪案新闻(突发性案件、弱势群体、大案侦破细节)
	\item 民族宗教、媚俗
\end{enumerate}

*\textbf{意见领袖}在舆论中的影响力

\subsection{新闻价值}
新闻事实和材料中具有的\textbf{价值要素}

*新闻事实角度、受众角度、新闻工作者角度

\begin{enumerate}
	\item 不变要素(所有新闻都有)
	\subitem 真实性\ 媒介现实不等于客观现实(\textbf{反映}、\textbf{建构}并存)、实践中的真实不同于\textbf{文学}真实
	\subitem 新鲜性
	\item 可变因素
	\subitem 重要性\ 与\textbf{趣味性}形成紧张的关系、\textbf{探照灯}理论(覆盖面不断扩大)
	\subitem 反常性
	\subitem 接近性\ 地理距离(本地化视角)、心理层面
	\subitem 显著性\ 人/事显著
	\subitem 趣味性
	\subitem 人情性
\end{enumerate}

*媒体处理的考量:工作需要、新闻价值、社会效果

*不同媒体的新闻价值取向也有所不同

*\textbf{开掘}新闻价值:探寻事实背后的意义

\subsection{马克思主义新闻观}
马克思主义新闻观:
\begin{enumerate}
	\item 要根据事实来描写事实,不能根据希望来描写事实。
	\item 报纸是社会舆论的纸币,具有流通和中介作用。
	\item 报纸是社会的耳目和社会的捍卫者。
	\item 报纸是对当权者的孜孜不倦的揭露者。
	\item 报纸是人民日常思想和感情的表达者,是人民千呼万应的喉舌。
	\item 报纸具有连植物也具有的内在规律性。
	\item 报纸作为一个整体处在一种有机的运动过程之中。
	\item \textbf{出版自由}是一种基本的自由,是实现其他自由的保证。
\end{enumerate}

\textbf{党报}思想:
\begin{enumerate}
	\item 党报党刊是党的重要思想武器和政治阵地,是党存在和发展的标志。
	\item 党报党刊必须遵守和阐述党的纲领和策略,按党的精神进行编辑工作。
	\item 党报党刊应当真正代表和捍卫无产阶级和人民大众的利益,成为他们自己的报纸。
	\item 党报党刊要成为党内批评的强大武器,敢于开展新闻批评是一个党有力量的表现。
	\item 党报党刊要处理好与党的领导机关的关系,在党的领导和监督下开展工作。
	\item 党组织要加强对党报党刊工作的领导和监督。
\end{enumerate}

~

毛泽东的新闻观:
\begin{enumerate}
	\item 论新闻、新闻业和新闻政策:斗争工具(新闻\textbf{事业}、新闻\textbf{产业}的差别)、注重真实、“\textbf{免疫论}”
	\item 论宣传:记者作为宣传者、重要性
	\item “舆论不一律”演进为“舆论一律”,趋于\textbf{稳定}
\end{enumerate}

~

刘少奇的认识:
\begin{enumerate}
	\item 党的传媒作用:报纸作为“桥梁”、“导线”(\textbf{双向})
	\item 记者的一项任务:考察党的政策
	\item 媒体正确把握党的政策(清理口号、和实际\textbf{保持距离})
	\item 新闻的客观、公正、真实、全面、坚持立场
	\item 改进社会主义新闻业的系统思考
\end{enumerate}

*首次提到调查报纸\textbf{受众}

~

江泽民新闻观:继承毛泽东、邓小平的新闻观,注重\textbf{以正确的舆论引导}

~

胡锦涛的新闻观:
\begin{enumerate}
	\item 按新闻\textbf{传播规律}办事
	\item 保障人民知情权、参与权、表达权、监督权
	\item 强调舆论引导
	\item 通过\textbf{互联网}了解民情汇聚民智
	\item 形成与国际地位相称的\textbf{外宣}舆论力量
	\item 改革会议报道
	\item 自主创新地发展信息产业
\end{enumerate}

~

习近平的宣传观与新闻观:
\begin{enumerate}
	\item 形成\textbf{立体多样、融合发展}的现代传播体系
	\item 遵循新闻传播规律和\textbf{新兴媒体}发展\textbf{规律}
	\item 宣传工作要注重\textbf{大局观}、顺势而为
	\item 反对形式主义
	\item 党性与人民性\textbf{一致}、统一
	\item 弘扬主旋律、传播正能量
	\item 宣传理念、手段与基层工作\textbf{创新}
	\item 把我新闻、宣传的\textbf{时、度、效},正确引导舆论
	\item 打造融通中外的新概念新范畴新表述
\end{enumerate}

\section{新闻写作基础}
\subsection{基本原则}

*用\textbf{事实}说话
\begin{enumerate}
	\item 选择典型事实
	\subitem 鲜活、读者位置
	\subitem 具有重要性、具体
	\subitem 接近性强化事实
	\subitem 有的放矢
    \item 运用场景再现
    \subitem 第一人称增加代入
    \subitem 发挥场景力量
    \item 运用背景材料
    \subitem 引导联想
    \subitem 善于对比
    \subitem 注释说明性-暗示记者观点
    \subitem 立体使用加大深度力度
    \item 借助直接引语
    \subitem 借被采访者之口
\end{enumerate}

*动态消息-新闻尽量\textbf{短}
\begin{enumerate}
	\item 一事一报法
	\item 浓缩事实法
	\item 剖璞献玉法
	\item 典型材料法
	\item 取其一角法
	\item 化整为零法
\end{enumerate}

\subsection{新闻文体}
三种分法:

消息、通讯、特写

特写归入通讯

新增新闻评论文体(深度报道)
\begin{enumerate}
	\item 消息\ 最直接、最简练
	\item 通讯\ 多种表达方式
	\item 特写\ 聚焦在某个东西
\end{enumerate}

*通讯的延伸-\textbf{深度报道}:解释性、调查性、深度预测

*消息与通讯比较:无导语/有导语、简单叙述/详细叙述、表达手法少/多

*网络媒体带来变革

\subsection{新闻标题}
\begin{enumerate}
	\item 主标题\ 最主要的题目
	\item 肩题(引题) 主题之前
	\item 副题(子题) 主题之后
\end{enumerate}

四要:准确直接、鲜明深刻、生动传神、醒目抢眼

四不要:煽情、哗众取宠、过于追求形式、执于一端

步骤:关键词-连成句子-压缩、改变形式

注重:抓住新闻重点、不歪曲事实、用词妥当

技巧:利用\textbf{反差}、捕捉\textbf{细节}、艰深内容\textbf{通俗化}、追求语言张力、巧设悬念

语言:活用动词、形象化、引题增强效果、广泛应用修辞

\subsection{导语}
由来:由电报不稳定,需要将全文精要浓缩

*“\textbf{倒金字塔}”结构

定义:以凝炼的语言把要旨放在开头

作用:\textbf{提纲挈领}

基本原则:言之有物(\textbf{实在}、\textbf{简练}、体现新闻事实本质)、言之有味(新颖多样、吸引读者注意)

演变:

第一代(二战前)\ “六要素”导语,具体完整但主次不分、重点不突出

第二代(至今)\ “部分要素”导语,将最重要、新鲜的内容突出在导语中(“\textbf{一语中的}”)

*第三代\ 丰富型导语,突出个性化,融入人类感受

本质:回答\textbf{最有意义}、\textbf{最有新闻价值}的要素

技巧:最好一句话、具体而不过于细节、有兴奋点、合理取舍事实、将重大新闻挂钩地方、调动读者直觉

语言:直截了当、生动、故事口吻、不应有多余字、变旧为新(不出现“昨天”)、奇景俏皮

*思考导语:最重要的事实?与谁有关?直接/延缓?是否有生动词语/短语可使用?主语/动词如何选择?

~

*\textbf{导语分类}

按要素分类:
\begin{enumerate}
	\item 人物导语\ 名人的名字或主要人物的形象
	\item 事件导语\ 描述事件要素
	\item 时间导语\ 纪念日等特殊时间
	\item 地点导语\ 与时间导语特点类似
	\item 原因导语\ 最有意思的角度、读者想了解
	\item 方式导语\ 与原因导语特点类似
\end{enumerate}

按软硬分类:硬导语-直接表达/软导语-轻松文风

应用手法分类:
\begin{enumerate}
	\item 直接型导语:事件本身关注度高,直接呈现
	\item 概括型导语:作为全篇消息梗概
	\item 描写型导语:简单勾勒事件,引发兴趣
	\item 对比型导语:通过对比营造反差感
	\item 问题型导语:引起兴趣(\textbf{并非调动读者大脑})
	\item 引语型导语:直接引语体现人物特质
	\item 引典/警句型导语:语言直接表达力量
	\item 比喻/虚拟/拟人/悬念型导语:修辞增强表达效果
	\item “你”导语:第二人称拉近距离
	\item 怪导语:悬念效果
\end{enumerate}

~

导语问题:

无新闻(不强调反常、变动)

会议型(未抓住重点,解决方法:\textbf{行动性}导语指出具体措施)

表彰型(普通读者不了解)

无故事型(表面化,解决方法:\textbf{轶事}导语)

概括/概念型(解决方法:用形象细节、\textbf{具体事例}代替)

淹没型(主要观点淹没于报道主题的某处,解决方法:一篇稿件讲透一个\textbf{最生动}的主题)

埋葬型(\textbf{最重要的事实}不加以呈现)

~

导语的注重与技巧

考虑:主句/从句开头、主动/被动语态、陈述/否定语气、短粹/重大事实

注重:观点带上来源、避免主观、避免非名人全名、使用生动词汇、强调反常、地方化、时新化

技巧:

\begin{enumerate}
	\item 一语破的\ 用最短的文字达到效果
	\item 设置悬念\ 软新闻使用,提升兴趣
	\item 欲擒故纵\ 先放开一步,利用对比效果
	\item 化静为动\ 用动态表现事件
	\item 拟人修辞\ 给人亲切感
	\item 数字对比\ 利用主要数字给读者难忘印象
	\item 速写勾勒\ 简单描述场景
	\item 巧用背景\ 增加背景加深表达效果
	\item 抑扬顿挫\ 多样表现,出现起伏
	\item 一张一驰\ 人之常情引起共鸣
	\item 以小见大\ 从个人小处着手
	\item 先声夺人\ 直接引语的利用
	\item 拉近时间\ 寻找新闻由头
	\item 有意重复\ 对特殊题材加强表达
	\item 化整为零\ 多要素时分段
\end{enumerate}

\subsection{新闻结构}
\begin{enumerate}
	\item \textbf{倒金字塔结构}
	
	按事实主次递减顺序排列
	
	特点:开头提出要点(\textbf{虎头}),中间陈述事实、补充细节,结尾总结、不加入新信息
	
	优点:对写者-迅速、对编者-有助于编发、对读者-方便取舍
	
	缺点:拘于一格、机械
	
	\item 金字塔结构(\textbf{积累兴趣})
	
	特点:由简入繁,高潮与结局在最后
	
	优点:时间顺序接受度高,富有吸引力
	
	\item 沙漏结构
	
	特点:倒金字塔叙述事实,金字塔叙述事件(适合时间结点标志明显)
	
	优点:读者清晰把握事件来龙去脉
	
	\item 板块组合结构(总-分)
	
	特点:各板块\textbf{并列}关系,服务于同一主题
	
	
\end{enumerate}

~

* “华尔街日报体”

开头:设置能引起阅读兴趣的焦点(\textbf{个人}、\textbf{具体事件})

“螺母”段:过渡到主体部分

主体:展开焦点,报道主题新闻事实

结尾:回到开始的焦点、照应开头

\subsection{新闻背景}
*消息 - 前景为主,背景辅助

作用:

\begin{enumerate}
	\item 分析解释,令新闻\textbf{通俗}易懂
	\item 揭示事物意义
	\item 对比衬托
	\item \textbf{暗示}不便表明的观点
	\item 增强可读性
	\item 介绍人物阐释\textbf{合理性}
	\item 累加事实拓宽读者视野
\end{enumerate}

文中位置:标题、导语(造成\textbf{反差}效果、科技新闻\textbf{落地})、导语后接背景、分散入主体中

\section{新闻采访}
历史:

五四时期\ 蔡元培校长成立新闻学研究会

抗战时期\ 战地记者开展新闻工作

改革开放\ 获得空前发展

\subsection{记者}

首次使用记者一词:《申报》

“\textbf{无冕之王}”:起源于泰晤士报,表明凌驾于社会之上的地位(\textbf{理想化},现实中会受各种制约)。

最主要工作:党和人民的耳目、喉舌,\textbf{信息中枢}

三大任务:新闻采访(大众传播、报道)、耳目喉舌(小众传播、内参)、联系通讯员

类型:

\begin{enumerate}
	\item 专业记者\ 专业报道某一领域
	\item 机动记者\ 任务不固定
	\item 特派记者\ 因采访任务需要特别派遣
	\item 特约记者\ 社、台外的工作人员应邀完成报道任务
\end{enumerate}

职业特征:四面八方、广交朋友、\textbf{杂家}

修养:政治、理论、知识、专业

作风:密切联系群众、深入调查研究(\textbf{客观事实}出发)、战斗敏捷(“\textbf{抢}”新闻)、不畏艰难

\subsection{采访}
作用:

“七分跑,三分写”

*新闻采访是新闻写作的前提和基础,新闻写作是采访的结果和结晶(锻炼记者)。

*发现和落实新闻线索、获取\textbf{第一手}材料(但第一手未必绝对正确,需要完善信息)

~

突发事件采访:

第一时间\ 最快速赶赴现场、补齐新闻要素(以\textbf{视觉采访}为主)

第二时间\ 寻找一切可能知情人,发掘全部事实(以提问为主以\textbf{详细了解})

第三时间\ 补充背景、事实,采访相关人士与事件\textbf{社会影响}

第四时间(\textbf{终结期}) 事件处理情况、\textbf{详细调查}、不同意见

~

采访分类:

直面采访\ 面对面,原则:\textbf{平等}对话、因人而异、因势利导)

书面采访\ 突破时空界限,给对方\textbf{充裕思考时间},作为补充采访(缺:不能得到及时答复)

体验式采访\ 记者参与被报道者的\textbf{实践}中

电话采访\ “\textbf{快餐式}采访”

网络采访\ 运用网络调查采访

~

“望”\ 视觉采访,抓住事物与\textbf{环境},敏锐洞察

“闻”\ 认真听情节、\textbf{细节},必要时向他人求证

“问”\ 善于提问,\textbf{避免啰嗦}而不得要领,慎用\textbf{封闭式问题},故事、细节、评价值得\textbf{追问}

“切”\ 抓住重点

~

\textbf{新闻线索}

“七新”:动向、事物、成就、经验、风尚、人物、问题

线索来源:重要文件、会议、新闻媒体、互联网、受众来稿

~

采访前准备:了解采访对象、了解新闻背景、拟定\textbf{采访计划}

“\textbf{偶然}”:从身边找新闻、把握拉网实际、软塞思路(逆向思维)

~

若干问题:
\begin{enumerate}
	\item 遭受拒绝或暴力
	\item 暗访、偷拍\ \textbf{隐形采访}(适当使用)
	\item 人文关怀
	\item 收费问题
	\item \textbf{霍桑效应}\ 人知道被观察时改变行为的倾向
\end{enumerate}

\section{各文体写作}
\subsection{消息写作}

两种模式:故事模式(运用讲故事的技巧)、信息模式(干脆提供事实信息)

分类:
\begin{enumerate}
	\item 事件性消息:事件有明显的“变”、典型性特点、原始\textbf{戏剧性}
	\subitem 交代清楚(重要细微部分、\textbf{具体}内容)
	\subitem 尊重客观事实
	\subitem 勿做表面文章(深层含义)
	\subitem 见事时亦要\textbf{见人}(以人为焦点)
	\subitem 慎重报道恶性事件(不要煽情、不要传播作案手段、保护隐私、关注\textbf{成因})
	\subitem 软消息写作(不必拘泥倒金字塔,注意故事模式可读性)
	
	\item 非事件性消息:\textbf{社会问题}、\textbf{社会现象}等,缺少具体发生发展过程
	
	*特殊价值:拓宽新闻面、媒体主体意识
	
	*类型:预测性、服务性等
	
	\subitem 以点带面(个别的事例表现整体、\textbf{共性})
	\subitem 对比凸显(将\textbf{不容易发觉}的事实变动展现出来)
	\subitem 量化集中(全局性数字反应问题、“\textbf{精确新闻}”)
	\subitem 找新闻由头(从旧闻找新闻,让报道有依据)
	
	\item 描写性消息:“\textbf{再现}”,满足读者心理需求
	
	*新闻素描(特写)、花絮两者有\textbf{正侧面}区分
	
	\item 分析性消息:有深度新闻,传播\textbf{观点}为主
\end{enumerate}

\subsection{故事化新闻(特稿)写作}
新闻\textbf{故事化}:用对话、描写、场景设置等细致展现情节,增强客观性与可读性,融入人情味

*软新闻使用故事化较多

“蛋炒饭”原则:事件无人报道(生米)时用\textbf{硬新闻}(白饭)的方式,有人报道后利用\textbf{故事化}(蛋炒饭)

\begin{enumerate}
	\item 设置引人入胜的\textbf{悬念}(生动、曲折)
	\item 让形象的\textbf{事实}说话(突出情节、细节)
	\item 以\textbf{人}为故事主体(挖掘韧性、人情的色彩)
	\item 多维例题的故事\textbf{叙事视角}(第一人称、公众代言人、第三者)
	\item 采编时注重新闻性(“找故事”)
\end{enumerate}

*与一般新闻区别:“见事也见人”,关注事件\textbf{发展过程}

*故事化与\textbf{娱乐化}:

相同:表现形式上强调故事性、情节性

差异:娱乐化片面追求趣味,哗众取宠

*区分高情感/低情感故事

*威胁:流于浅薄、屏蔽部分\textbf{缺乏曲折情节的重大事件}、损害真实性

\subsection{通讯写作}
*特色:感染力、文体较自由

*注意:真实性原则(禁止心理描写)、主题鲜明、结构完整(禁止\textbf{扩缩})、展现个人风格与社会人文

\begin{enumerate}
	\item 人物通讯
	
	主题:人物事迹(新闻人物、凡人奇事、冰点人物-\textbf{平民化}报道、反面人物)与时代精神
	
	避免:千人一面(\textbf{模板化})、写“完人”、片面报道(尤其对反面人物)
	
	\item 事件通讯\ 突发事件、社会影响较大的预知事件、喜闻乐见的小故事
	
	\item 风貌通讯
	
	\item 社会观察通讯
	
	主题:报道社会现象、剖析社会问题
	
	类型:话题型(热点话题)、课题型(长远意义、深入研究)、
	
	报道方式:展现型(新事物、\textbf{全景}扫描)、剖析型
	
	能力:线索追寻、细节感知、主题\textbf{挖掘}、材料\textbf{架构}
\end{enumerate}
\end{document}