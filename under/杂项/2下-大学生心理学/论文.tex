\documentclass[UTF8,a4paper,fontset=windows,11pt,openany]{ctexbook}
\title{\Huge\heiti 科大内卷调查报告}
\author{\Large 郑滕飞PB20000296}
\date{}

\usepackage{enumerate,geometry}

\geometry{left = 3.18cm, right = 3.18cm, top = 2.54cm, bottom = 2.54cm}

\begin{document}

\maketitle
\tableofcontents

\chapter{引言}
内卷,一个当代大学生一定不会陌生的话题。在中国科学技术大学(下称科大),哪怕是日常的聊天中,都常常用“卷”来评价大学的环境,用“卷王”来或褒义或贬义地评价其他同学。然而,随着内卷一词的语境逐渐泛化,这个词的用法似乎也在逐渐模糊。当我们习惯了用“卷”去描述我们对竞争的感受时,我们却越来越难以详细叙述当我们谈论内卷时,我们究竟在谈论什么。

我们究竟在谈论什么?

日渐习惯的学习带来的高压、不知何时由成绩产生的分层、应有却被剥夺的玩耍的快乐、对他人过度竞争的不满,抑或只是为了平息焦虑感而随口评价?

定义、来源、泛化、对比、现状、影响、加深、缓解,关于内卷,值得思考的方向实在太多。在第一章,我们先来看几个时常被讨论的话题:

\section{初中、高中、大学}
第一个经常出现的话题就是各个学习阶段的内卷。由于个人的经历存在时间与记忆的偏差问题,我调查了当前就读于初一、初三与大一的学生,并且询问他们关于竞争与内卷的看法。

初一的学生已经能感受到小升初升学存在的一定竞争压力。其中一位受访者表示,能鲜明感受到班级开展的抢答活动与升学的竞争不同。班级的抢答活动很开心,哪怕是没有拿到最终奖励也没有什么关系,但升学中却感受到了类似鄙视链的存在。当被问及是否觉得这样的好坏之分正常时,他明显表露出了纠结。他说,他以前觉得不应该存在这样的竞争,但看到竞争激烈的现状就让他很难受。当他询问父母时,父母的解释是,比起以前,我们已经处在了一个相对公平的环境中,所以要珍惜这样公平的机会。然而,这并没有解开他的纠结。与他的父母相比,另一位受访者的父母就直白很多。她说她的父母告诉她要卷就要卷到别人全面,可她的体验是“卷到最后还是比别人低”。我问的另一个问题是关于超前学习(例如小学去接触C语言编程)算不算是一种过度竞争。他们最后的回答是,或许父母想让孩子多学会一种技能,以在未来的竞争中更加强势,并不能简单判断是不是过度竞争。

采访到的初三学生则是向我详细描述了她和她母亲朋友的孩子的对比。她觉得她所在的学校由于有直升高中的通道,所以不算是卷,不过仍然有一些竞争产生的压力。她的父母的教育观念一直是让她开心就好,因此也没有被逼得太紧。与她相对,她母亲朋友的孩子所在的学校就非常辛苦。他们每天晚自习到九点,而他十点睡觉甚至被老师找了家长,原因是“你怎么能让孩子十点睡”,建议每天到十二点再睡觉。她自己觉得,她要是处在那样的环境,肯定会被压抑出病来。我后来问她,有没有想过,她之所以可以每天有比较多的放松时间可能是因为她比较聪明,不像有的人完成学习任务就需要很久了。她告诉我,她确实没有想过这些可能,对她来说,课内比较简单,只要保持正常的睡眠就能跟上课内的所有内容。

当我去问采访的大一学生是否能感受到高考内卷时,他们感到了鲜明的地域差异。有的地方,诸如苏南地区,对自己所在地区比较自信,因此家长普遍不会让自己孩子拼命卷。与之相对,苏北的高中压力就普遍很大,之前在网络上一度成为热词的“小镇做题家”往往也是来自这样的地方。而即使是所在地方比较轻松的受访者也承认,对高考来说,“卷”是有意义的。高考比较套路化且难度不算太高的考察方式,让刷题对成绩的帮助显著。

从这几段采访来看,到了每一个“临界”的阶段,无论是小升初、中考还是高考,大家总会感受到竞争的压力。而具体的压力大小则与家长的观念密切相关,家长的观念又受地区的影响非常大。在小学时,竞争偏向各种超前学习与解决困难的问题,到了初中高中,除了一些参加各学科竞赛的人,大部分人的主要竞争目标就是具体的标准化的考试了。与这些相比,大学的竞争有显著的不同:
\begin{enumerate}
    \item 复杂性。哪怕是在学习维度,科研、竞赛、课内成绩都有着不同的作用,更不用说其他方面的情况。除此以外,不同方向选择也会导致很大的差别。而大学的环境将所有这些人聚在一起竞争,导致每个人看到的都是十分复杂的竞争环境,不再是对单一化目标的追逐。
    \item 困难性。大学的课内内容比起之前来说显著困难,这就导致很多之前由工作量(刷题)可以解决的问题不能再通过工作量解决,也就逼迫大家找到适合的方式方法。注重技巧的另一个结果是,有的人并不用付出太多时间就能获得很好的结果,有的人则花了大量时间仍然无效,加重了心理上的落差。
    \item 非强制性。由于课内限制的减少,大学的很多学习过程不再被强制,这就对个人的自律和规划能力有了很大的需求。尤其是初高中时习惯了强制下学习的同学,到了大学可能会更加不适应自己规划的学习模式。当然,强制性的减少也导致了更难以在各种娱乐与学习间平衡时间。
\end{enumerate}

出于这些差异,大学竞争不能与之前的竞争作同类讨论,更不能简单判断竞争是否构成内卷。采访中大部分人都认同,高中时为了高考而进行大量重复的做题并没有实质性提升水平,因此属于内卷的范畴,但对于大学时何种行为才算内卷,不同的人有不同的观点。

\section{努力的困惑}
另一个时常被讨论的话题则是“努力”与“卷”的界限。这个话题又直接关系到“内卷”一词语义的泛化。

当我刚进入科大时,群里流传得是“科气凝聚”与“科气退散”两张表情包。“科气”这个词在当时表示过度重视成绩又压抑的环境引起的一系列怪象,比如在群里重复没有什么意义的吹捧发言,或是聊天中三句不离学习与GPA。后来,我渐渐在和大家的聊天中了解到了内卷一词,也渐渐开始用内卷来形容引起这一切不悦的罪魁祸首。可是,随着时间的发展,“卷”的形容范围变得越来越广。

根据实际体验,这样的范围扩大有两方面原因。一方面是善意的泛化,来源于是朋友之间调侃,在对方认真学习时戏称一句“卷王”;另一方面则是恶意的泛化,通过给人贴上卷王的标签以攻击别人。但当这样善意与恶意一同作用,称呼的使用范围逐渐扩大时,越来越多本来并不属于内卷语境的人被贴上了卷的标签。这带来了普通努力者的困惑——“我怎么就成卷王了”,也让“内卷”中的贬义渐渐消解了。

虽然在下一章中才会讨论内卷的具体内涵,有一点是确定的:作为一个贬义词,这个词语绝对不应该加于普通努力者的身上。当然,词语在生活中表示含义的逐渐发展是十分正常的,也在一定程度上不可避免的,因此也没有必要指责不出于恶意使用“内卷”的人。只不过,在使用时确实应该思考,所想表达的究竟有多少的贬义成分。同样,当接收到别人的“内卷”评价时,也不用急着怀疑自己,而可以先想想,这样的形容当中有多少是针对内卷的本义,有多少是泛化的结果。

\section{内卷是个伪命题?}
最后,则是关乎这篇调查报告存在意义的讨论——内卷是不是一个伪命题。

梁永安老师制作的一期视频\footnote{内卷是个伪命题!我们该如何走出亲手制造的内卷陷阱?【复旦梁永安】- https://b23.tv/8ybl92k}中,直指“内卷”并不客观存在,更多的是一种心理状态。然而,这个观点我并不认同。心理状态确实是内卷的重要组成部分,但这并不能消解内卷的客观存在性。在报告第三章中将会讨论内卷的成因,其中的绝大部分都是现实性的原因。简单来说,内卷形成的最本质原因就是资源与获取资源的路径有限时,随着竞争逐渐饱和,对等量的回报,将会需要越来越多的付出。梁老师认为,寻求新的资源与获得路径才是最本质的解决方式,否则就不应责怪内卷的环境。这个说法忽略了每个环境中存在的大多数普通人。普通人难以发现新的资源与路径,即使发现了可能,也承担不起贸然尝试的失败代价,因此只能在当前环境中付出更多,也就不得不陷入无止境的内卷当中。

另一个有趣的讨论则是来自国际华语辩论赛官方所制作的一期节目\footnote{只想躺平不内卷,是我错了吗?丨差不多得了03 胡渐彪 x 周玄毅 - https://b23.tv/kxRj4rd}里,两位辩手关于躺平与内卷展开了深刻的讨论:
\begin{enumerate}
    \item 做事是否应该兴趣驱动?躺平方认为觉得好玩就应该继续做下去,而内卷方则认为兴趣容易被磨灭,并不能保持长久可靠。在这一点上,关于今天讨论的话题,就采访到的同学的体验而言,内卷方的结论更加适用。不少同学直言,无论自己之前对专业有怎样的热情,只要和考试、GPA挂钩,这份热情往往会带上焦虑与纠结的成分,尤其是当考试的反馈并不好时,会让人更加怀疑自己是否真的喜欢这个专业。但是,躺平方的观点也并非没有道理。如果连兴趣都没有,想找到一个能保持长久可靠的理由只会更加困难。
    \item 关于躺平与内卷的内在逻辑,躺平方认为对现有竞争规则的天然性不满引发了对躺平的追求,内卷方认为无法从所做之事上获得快乐才带来内卷的焦虑。这点上,采访结果更支持躺平方的观点。原因就像上一点中的,所做事情上带来的快乐如果在竞争中没有达到预期,仍然会带来不小的焦虑,而哪怕达到预期,也会产生对今后能否保持这样的隐忧,从而被限制在了竞争规则中。有了这样的限制,反抗的心态也会自然出现。
    \item 关于心态与本能的交锋,躺平方的观点和上述相同,只要和考核挂钩就难以保持完全的动力,而内卷方此时强调必须找到能保持动力的事业才能做好。关于这点,内卷方的观点又较为理想化,理由一方面是上面提过的尝试成本,另一方面则是保持动力本就是一个理想的结果。采访中,大部分人总会有动力与没有动力呈周期性变化,那么,如何衡量这种“保持”呢?
    \item 关于躺平的结果。躺平方强调对自己更好,过让自己更舒服的生活,内卷方则认为躺平后则会将自己“卷”得越来越小,只有保持做某些事的状态才能不断接受新的外界观点来丰富自己。这点上,内卷方的观点是有道理的。身边的不想困于当前的学习考试中的人,只有很小一部分能按自己的想法去发展,大部分还是陷入了新的不知道要做什么的迷茫中。
\end{enumerate}

从这些讨论出,也可以看出一些内卷必然存在的理由:首先,在大学中,GPA仍然是十分重要(某种意义上首要)的标准,而每门课的成绩又关系到GPA,因此在考试的束缚下,一定会产生功利性的追求,兴趣也更加容易被磨灭。其次,由于竞争本身的复杂性与非强制性,真正有心脱离的人很难坚定向自己想要的方向发展(甚至连方向是什么都难以确定)。在这两点的限制下,大部分同学还是会无可避免地落入单元化的竞争中。大家对这样的竞争规则产生本能的不满,又无法改变或是找到更好的出路,焦虑也自然产生。科大的课程难度和同辈压力\footnote{指peer pressure,本报告翻译为同辈压力,在第三章中将详细阐述。}加深了这种焦虑感,相对单一的标准又让不符合的人很难在其他方向排解这种焦虑,最后的结果往往是向恶性循环的方向发展。

内卷有时就像电磁学中的场,并不能直接看到它的实体,却对其中的每个人都产生作用。事实上,采访中也并不是所有人都明显感受到了内卷的存在,不同人对内卷的感知程度存在很大的差异,这其中的原因将在第四章的时空对比中详细阐述。无论如何,确认了内卷的真实性后,便有了调查与研究的价值。接下来的部分中,将以定义、起因、现状、影响与可能解决方案的顺序尽力还原科大的真实内卷情况,并且尝试给身陷其中的我们找到一条出路。

\chapter{卷字的三种用法}
这一章中,我们主要讨论的问题是内卷的定义。在实际使用中,虽然有着语境泛化的种种影响,探究内卷——尤其是那种令人不快的内卷——的本义才能清晰看出问题的所在。很容易查到,内卷\footnote{此处英文原词为involution。}这个词最早的贴近现在语义的使用是被格尔茨\footnote{Clifford Geertz,美国著名人类学家。}用来形容农业。它形容的是一种模式达到了某种最终形态后既无法稳定,也无法转变为新的形态,只能在内部不断复杂化。而对于我们口中的“内卷”,针对更多的并不是模式,而是一种基于客观的感受。从这个角度来说,一个更合适的名词会是俚语rat race,直译为老鼠竞赛,形容的就是过度的竞争与它引起的种种现象。

然而,这样的词语替换还是不能准确揭示出内卷的本质。要想知道词语的含义,还是得回归语言与语境。经过观察和分类,“卷”一般可以分为三种使用模式:第一,针对个人,比如说某某人“卷王”,或者自嘲“卷不动了”;第二,针对集体,比如身为少年班学院同学,我经常能听到的一个评价是“少院太卷了”,或者转院同学经常感慨原来的院/新的院“卷”;第三,针对环境,比如我们说“现在计算机行业很卷”,或者“当教授很卷”,乃至“科大很卷”。这三种不同语境中的不同用法,就暗示了讨论内卷定义的三种进路:个人、集体、环境。

\section{卷王之路}
个人角度的内卷大概是大家提及最多的,却也是最容易收到泛化影响的。在现在的语境下,我们说某个人“卷王”非常少带着贬义。在采访中,大家现在使用“卷王”这个词一般都是去形容比较努力或者成绩比较好的同学,很少真的带着贬义。不过,为了考察这个词的内涵,必须先考察具有贬义的原始含义。

当我问及怎样的行为可以被含贬义地称为卷王时,一位采访对象提出了很有意思的观点:最重要的是动机。他认为,只从行为上有时并不容易辨别:如果一个人因为喜欢而疯狂地学习,不能称之为卷。这件事在采访中引起了共识,如果学习是为了知识素养本身,不论如何疯狂都不应该称为卷。

排除了知识性的学习,我们再来看工具性的学习。如果一个人学习是为了赚钱,于是特别拼命,能被称作卷王吗?这个话题似乎具有一些迷惑性,所以我们不妨来假设一个最极端的情况:\emph{某同学家境一直不太好,需要赚钱来补贴家用,于是TA疯狂学习、找实习,想要得到更好的就业机会。虽然TA对专业并没有什么爱,也对所学知识没有什么感受,但付出成倍的努力后,终于掌握了不错的技能,也达到了自己的目标。}在这个虚构的案例中,道德本能会告诉我们,这样的行为是励志的,而不是应该抨击的。于是,结论似乎成为了,只要让自己的学习与成绩能用上,无论具体的行为如何,都不应该称为卷。

那么,到底怎样的个体才能称为卷呢?采访中普遍的回答是:刷存在感、炫耀、引起反感等等才是卷的必要条件。也就是说,学习的同时去引起关注,以各种方式去显性或隐性地炫耀自己学习努力,才能称为卷王。值得注意的是,这里的“各种方式”未必就是直说自己努力、厉害,甚至可以是它的反面——反复强调自己的不足,以至把自己的长处都说成不足。这种奇怪的“炫耀”被称为“卖弱”,将在第五章中讨论细节。以上的解释看似合理,但其中仍然有一个逻辑的陷阱:在这个定义里,个人的内卷也就相当于努力再附加上努力的动机是炫耀,可努力本身总会是因为知识性或者工具性的目标,炫耀只是附带的。也就是说,当我们说一个人“卷”的时候,我们反感的行为只是对他人的恶意,而不是行为本身。对于炫耀的反感,某种意义上已经不在内卷所表示的过度竞争的本义中了。

在这类回答之外,还有一种回答:比如资料等不愿意分享,他人来问时假装不会,这些称得上负面的卷。关于这件事,我们来看一个真实的故事:\emph{20级某学院的某课一直以安排十分不合理出名,例如有一项在大家几乎什么也没学到时布置的大作业论文。大部分人选择在群里讨论,交流思路,但正当大家交流时,有一个匿名突然出来说,TA已经在淘宝上找到人代写了\footnote{代写行为自然是完全错误的,即使课程不合理也不能以这样的不诚信去对待。不过本报告讨论主题为内卷,因此不多叙述代写的问题所在。},花了几十块钱,结果好像还不错。当大家问TA淘宝上的人大概是什么思路的时候,TA表示,这肯定不能告诉你们,“不然这钱不就白花了吗”。}在这个案例发生的时候,不少人都指责当事人“卷”,事实上,这样的场景才贴近内卷的本义:本身使用了恶性行为参与竞争,且这样的恶性是以让自己获得优胜为目的的,如果大家想与之抗衡,就容易陷入更恶性的竞争中去。

结合上面的这些讨论,我们否定了几种可能:知识性的学习当然不能称为卷,工具性的学习也不是卷,而为了炫耀的努力所被反感的只是炫耀,而不是努力本身。与这些说法相对,只有竞争本身利用了恶性的方法,才是贬义的卷在个人上的内涵。事实上,如果想确定一些个人的行为是不是卷,不妨去想想“如果大家都这样,最后的结果是正面的还是负面的”。只要方法本身是良性的,最后至少可以有收获更多知识、技能的结果,而如果是恶性的方法,只会导致更恶性的滑坡。

\section{岛民}
个人进路的讨论暂时到此为止。虽然只是简单地讨论了个人内卷的几种说法,也不难看出,真正内卷的个人是不可能很多的。大部分被称为卷的人,几乎都只是普通的努力,最多在努力之上有些小小的虚荣而已。这或许也是“卷王”一词得到了如此广泛使用的原因。可是,身边的焦虑与真实的困境似乎又在暗示,问题不可能如此简单。于是,我们来看看定义在集体上的内卷。

关于集体的一个非常重要的误解就是,内卷的集体并不是集体的内卷。正如上一节说的,不管是怎样的集体,其中能称之为卷的个人总是少数。那么,为何有些集体会让我们感觉更“卷”呢?让我们来看这样一个故事:

\emph{在一座小岛上,住着一些世代以务农为生的岛民。他们能看到外面的世界,但因为知道自己种的也就是些普通作物,对那些大富大贵也没有什么羡慕,安于自己的生活。直到有一天,岛民甲突发奇想,联系到了一个外界的专家来勘察他们岛屿。甲意外地发现,他们岛上有种平时不被他们注意的植物,其实是种非常珍惜的药材,能卖出很高的价钱,高到让他的生活足以超出岛上的任何人。作为心地善良的岛民,甲并不想独享这个秘密,而是告诉了全岛。但大家经过调查后才意识到,原来只有很少的一部分地方适宜这种作物。在这些地方有耕地的岛民自然十分开心,将自己的地用来种植药草后,很快实现了致富。岛上的人民风淳朴,没有因为这样的差距而感到太多不满。然而,随着药草的生产扩大,与岛民合作的企业决定进一步开发,一方面展开更多勘探,尽可能覆盖种植,另一方面将生产机器逐步转移到岛上。于是,原本在耕作的农民发现,在部分人实现了致富后,他们的耕地反而受到了挤占。虽然企业给出了不菲的收购金,但这并不足以让他们长期生存,于是,有的人选择了加入企业,有的人则仍不死心,看着耕地越来越少、污染越来越重。长此以往,矛盾渐渐加深了。}\footnote{值得说明的是,本故事并非纯属虚构,例如林生祥的音乐《动身》背后的反五轻运动就是众多这样的故事中的一个。}

这个故事中的岛民,最终成为了一个内卷的集体。大家只能放弃之前的平静生活,进入焦虑与纠结中。然而,从岛民的角度而言,这几乎是一个理想的集体——没有真正内卷的个人,也没有过多的矛盾,每个人只是普通地想让自己的生活更好而已,却酿成了这样的后果。作为一个普通的集体,可能出现的问题只会更多:比如,如果甲没有那么心地善良,只是暗自收购、耕作,等到大家发现时已经占据了大部分了利益,那么问题更早就会爆发。再比如,如果岛民觉得种植药草的耕地分配不均并不公平,不允许部分人独占珍贵的耕地,冲突只会更早爆发。

某种意义上,在科大所感受到的那种集体的内卷,或许贴近这里的岛民的情况。没有那么多恶意,也不会大量充斥内卷的个体,大家只是努力让自己的生活变得更好,却在不知不觉间陷入了内卷的深渊。好在,努力让生活变得更好似乎也不是内卷的充分条件:\emph{某院中,大家普遍认为最卷的地方会是科技英才班,因为那里的竞争难度更高,也更为激烈。可事实上,英才班或许是最不卷的地方。那里学得东西确实难,但有非常良好的交流讨论氛围,关系也相对融洽,因此虽然竞争难、激烈,实际上构成了促进发展的良性循环。}

为了从岛民的故事中找到内卷的定义,我们不妨看看哪些条件是必要的。最重要的自然就是到岛屿的环境。岛屿就这么大,如果有地方用来开工厂,自然能种的地就会少。第二则是岛民甲的意外发现——稀有的药草成为了大家都觉得重要并且想获得的东西,也能给生活带来显著的差异。第三,适宜种药草的地方只有一小部分,不可能每个人都靠着药草发家致富。第四,食利的企业追求继续开发,扩大生产,运用了更多岛屿上的土地。对于淳朴的岛民来说,这四点只要任何一点没有满足,都不至于造成最后的结果。遗憾的是,四重条件看起来苛刻,现实中能满足这些条件的场合一点都不少。哪怕是在科大内,大到全校,小到某个具体的教学班,都能发现类似的场景。不过,为了通过这些条件给出内卷的明确定义,我们先把目光放到更大的地方:

\section{望洋兴叹}

\emph{我的一个朋友,在这段时间终于做出了决定:彻底放弃科研道路。其中一个原因是,科研道路比起理想、初心与情怀,更多的是致命的内卷。他用的配图中,有一张是PhD毕业后出路统计,其中只有不到百分之四能成为教授,大部分则是选择了转行。}

科研的“卷”在科大几乎是众所周知的,而这里的“卷”字则是用来形容整体环境的某个特征。在刚才英才班的例子中,另一个值得注意的事是,旁观的“卷”并不算是卷,只有亲身体验到的内卷才是真实的内卷,对环境同样也是如此。因此,想要知道环境中的内卷的含义,必须看怎样的人会深刻感受到环境中的内卷。对于科研来说,内卷体现在竞争非常激烈的“非升即走”,那么最能感受到的内卷的就是“走”或者担心自己会走的人。继续细究下去,所感受到的卷主要有两个来源,一个是感到自己的付出与收获完全不成正比,一个是在与他人对比中产生的落差。

说到这里,我们来试着把科研环境中的内卷和岛民的诸多条件对应起来。岛屿的大小有限,自然对应给科研能分配的资源(如教职的位置个数、经费等等)有限;重要并且是公共追求的东西,也就是科研成果;很小一部分地方可以种植药草,就对应着,就对应着由于背景、环境乃至个人等种种原因所导致的科研能力的差距;最后,企业对应的则是非升即走等规则本身决定了成果越好就能获得更多的资源。

当然,岛民的内卷与环境的内卷也有着一个显著差异。如果说岛民的竞争局限于那座小岛之上的话,环境中的内卷则是在面对一片汪洋大海。虽然人们会因为种种方式产生各样的联系,但总体来说,环境中的内卷是很难看到竞争的对象的。这实质上导致了更大的焦虑——无法看到竞争对象,也就无从知道“平均”或者“大部分人”的成果和进度,如果想要保险,只能无止境地提升自己的结果。这或许也是为什么在环境中的内卷往往更加隐蔽,却对个体有更大的负面影响。在科大中,也能看到集体与环境角度的内卷,比如每个教学班内的竞争,或是保研科大的竞争,都更接近于集体(由于可以较完全了解到整体的情况),但保研外校与出国的竞争就更接近环境中的内卷(由于很难及时判断出整体情况)。

经过对集体、环境内卷异同的分析,内卷的模式就已经较为清晰了。在资源有限的竞争中,少部分人获得资源的能力更强,因此得到了更好的反馈,从而占据了大部分的资源,而剩下的大部分人由于获取资源的能力较弱,只能得到更差的反馈,从而付出很多努力也只能从较小一部分的资源里分配,也就格外感受到了内卷。自然,少部分人看到大部分人体验到的内卷时,也会担心失去已获得的资源,于是,他们必须付出更多以稳固当前所得的资源。在这样的情况下,大部分人与小部分人的差距逐渐拉大,资源的分配也逐渐固化,可所需的努力却都在提升,整体的内卷就此形成。必须强调的是,虽然整体都困于内卷之中,小部分人仍然是受益者,但往往难以意识到自己因其而受益。事实上,之前所说的个人的内卷也可以套到这个模式之中,我们之所以去批判一个人的“卷”,就是因为他靠着自己获得资源的能力——这里往往是利用了负面的手段——去抢占本不应属于他的资源,加剧了整体的不均衡;而讨厌一个人“刷存在感”、“炫耀”,也是因为他在抢占大部分资源的同时没有考虑只能得到小部分资源的人的感受,甚至是或明或暗地嘲讽。由此,个人、集体、环境角度的内卷统一成了一个固定的模式,而我们口中的“卷”的真正内涵,或许就是表达对这种分配模式与占据大部分利益的人的不满。

\section{卷不动的是谁?}

经历了个人、集体、环境在不同语境下的分析,我们终于给了内卷一个相对具体的解释。如果用一句话来概况刚才得到的结果,也就是“在重要资源有限的环境中,小部分人由于获取资源的能力更强,而有意识或无意识地对大部分人形成了倾轧”\footnote{事实上,这样的倾轧往往是有意识但被刻意忽略的。在第五章中,会深入讨论此处的有意识与无意识。}。此外,岛民例子中的种种要素,也给我们下一章中讨论内卷的起因提供了基础。不过,在进入对起因的讨论前,我们最后来看看在之前的三部分中忽略的一个语境:“卷不动了”。在这一章的最开始时,我们把卷不动了归为个人角度对内卷的形容,然而,这里的个人好像与我们之前讨论的“卷王”并不一样——它指向一个独特的个体,也就是自身。

在说自己“卷不动了”的时候,我们显然不至于把自己放在了恶意内卷的位置,于是,这个词也不是去说个人的内卷,而是一种不想再参与集体/环境内卷的感叹。可在之前的分析中已经说明了,集体的内卷并不是某个个体的结果,而是资源有限与获取能力不均的矛盾。只要这样的矛盾不能解决,内卷就不会消除。如果从一开始就不在意这样的资源,根本就不会参与其中,而一旦参与了,也不可能轻易退出。因此,这样的感叹终究只能是发出的一点牢骚,牢骚过后仍然需要投身于“卷”中去。

可是,每个人都会发牢骚,但这不代表每个人感受到的压力就相同。更复杂的情况是,人不可能在任何方面都处于“小部分”\footnote{第五章中将解释,“小部分”甚至未必是具体的存在。},于是被倾轧与倾轧并存,更难找到真正的问题所在。在采访中,很多人反馈的一个感受是“累”,或许是因为学习忙碌,或许不是,但总觉得一直处在运转的状态中,却不知道这样的运转是为了什么。

事实上,正是因为这些原因,内卷问题才如此复杂:
\begin{enumerate}
    \item 在岛民的例子中已经说明了,内卷并不是某个人有错或是恶意才能产生的。在大家都没有什么恶意的情况下,想要改变就十分缺乏空间,也无从下手。
    \item 现实的人不可能只处在一个集体之中,也不可能只属于某一个环境。在不同的环境中,人的身份不断切换,而不同集体给出的体验杂糅在一起,在增添压力的同时难以分辨不同的来源——甚至可能是众多稻草一同压死了骆驼。
    \item 就算在同一个环境中,少部分与大部分也不是一定可以区分的,反而可以互相转化。因此,从少部分到大部分与从大部分到少部分所产生的不适与落差也成为了心态问题的一个主要部分。
    \item 大部分人想获得成效的过程都是艰难的,在对比与想成为少部分人的过程中会自然引发更多的问题与冲突。
\end{enumerate}

所以,卷不动的是谁呢?既是说着自己卷不动而无法离开的人,也是想说但说不清究竟为何会如此劳累的人。只要有内卷存在的地方,这样的人一定是大多数。

说到这里,关于内卷定义的讨论终于基本完成了。由于这样的定义是站在了较为广泛的角度,在具体的情况上,问题未必能完美符合岛民的每一个条件,但至少,倾轧一词可以作为内卷本质的接近描述。下一章,就将落到更具体的角度,来看一看科大内卷的源头。

\chapter{源}

探讨内卷的源头并不是一件容易的事。哪怕有了一个普遍层面的定义,想落到具体问题上也是困难的。在上一章中,我们已经说明了在现在的语境下内卷的含义,而这个含义与农业生产中的内卷的本义并不完全相同。农业生产中的内卷导致的结果,是所有人都必须无限增大工作量才能获得原本的收益,而大学的语境中,内卷却是和倾轧密切相关。某种意义上,高考的内卷实际上是比大学更忙的,但所产生的问题\footnote{此处只考虑卷本身所产生的问题(在高中,这往往被定性为努力,自身并非坏事),不包含附加因素。}却没有大学明显,这其中很重要的一部分原因就是,在高中并不会明显感觉到倾轧的存在。高考的模式下,更多的是一分耕耘一分收获,在每个人所看到的情况中,并没有显著的获取能力差别,也就更接近农业生产里内卷的含义。而到了大学,情况就开始复杂了:

\section{降至现实}

为了与岛民的例子对应,最先需要确定的就是那个有限而珍贵的资源。但只要对这个问题稍微细想,很容易就会发现,哪怕是资源都不是一个能轻易确定的东西。从最表面的角度来说,答案似乎应该是课程的成绩与GPA,但实际上,这只是资源的一部分。\emph{在我所在的一个数学学院闲聊群中,大家不时会聊一些自己感到焦虑的话题。其中出现过的话题不止有具体的学习压力,还有对出路、前途的担忧,比如联系实验室和导师做项目、出国/保研的竞争,乃至就业或者从事科研的收入问题。除此以外,还包括作息、时间分配、自制力与“来科大一年多了都没和女生说过话”。无独有偶,在另一个计算机学院的闲聊群中,出路、前途的担忧也是反复被提到的。}这些话题中的大部分背后,其实都暗藏着某种资源,或者是广义的资源\footnote{广义的资源未必有明显的有限性,但仍为一种可比较且普遍以多为好的指标,在后文将详细解释。}:
\begin{enumerate}
    \item 首先,还是让我们回到课内成绩与GPA上。这样一个“资源”非常符合岛民例子中的条件:优秀率的限制与课程的成绩分布注定大部分人难以取得足够理想的成绩,而无论是走哪条道路,课内的成绩都是基础,也就让它拥有了重要性。如果非要给课内成绩的有限与重要找一个起因,大概也只能归结为“由于一所大学的人数较多,必须有一个比较统一的评价标准,而反映各门学习、考试情况的GPA是一个相对客观的评价标准”。更准确来说,在生命的大部分阶段,我们都会接受各种指标的评价,而大学阶段最主要的评价来自于GPA,因此也成为了最普遍的内卷感来源。
    \item 正如大学竞争的复杂性,GPA并不是决定出路的唯一重要指标。就像刚才的群中所提到的,提前联系实验室、导师,或是为了就业而找实习,也是内卷感的来源。这些方向准备的重要性不言而喻,除了考研与保研本校以外,其它出路都需要这些准备。在采访中也发现,对这些东西的焦虑主要来自看到他人已经得到了挺好的机会,自己却连方向都没有确定。从资源的角度,这里的“有限性”更接近于“好的机会有限”——大家公认较好的机会只有少部分人可以获得。如果说这件事的起因,答案仍然是于评价的标准。对不同方向能力的要求所衍生出的不同评价标准,给这些机会施加了各种不同的限制。
    \item 对出路本身的焦虑,也是因为某种内卷感。理由与上一种情况类似,都是公认的好机会有限并且难以取得,因此“出路”也成为了一种可以争夺的资源。在上一章中也提到了,出路角度的内卷比起集体,更接近在环境中的内卷,因此,出路的内卷起因也是落在了环境整体的争端上。
\end{enumerate}

以上这些重要资源,其实有一个共同的特征,它们本质上都是社会竞争的某种下放。事实上,大学的内卷的一个很重要的起因就是社会竞争的下放。GPA的本质是学习能力考核的下放,而联系导师、实验室既代表着解决问题的能力,也代表某种意义上的人际能力。这些因素基本决定了出路的情况,也就是确定了某种更大角度的资源竞争力。由此出发,可以找到这些资源背后的共同资源——社会中的机会资源。正是因为社会中的机会资源有限,细化成了不同方向的竞争,这些竞争又下放到学校形成了更加现实的“指标”争夺。

与这样的竞争对比,群里讨论提到的诸如自我管理、社交这些角度虽然也可能带来焦虑,但很难形成内卷。一方面是因为,这很难有一个明确的指标,也并不会直接关系到社会中的机会分配,另一方面则是,这里并不存在能称之为“资源”的东西,而是单纯“能力”的区分。事实上,如果能力本身不关乎资源,比如对科大学习理工的同学来说的艺术、文学能力\footnote{此处指与专业无关且与社会分配基本无关。},即使有些人愿意去磨练能力、去竞争,也不会给人“卷”的感觉。而一旦这些能力关系到资源,哪怕只是一门选修课的成绩,也有可能“卷起来”。

\emph{在某群曾经讨论过一个奇怪的话题,到底是在选修课写其他科作业是卷,还是在选修课认真听课是卷?当时讨论的结果是,如果抱着纯粹功利的心态,这两者只是选择的不同,都可以是卷,反之,如果是以能力本身的态度,都很难称为卷。一般来说,在选修课认真听课是兴趣使然为主,因此写其他科作业是更“卷”的行为。}经过刚才的分析,可以看出这里得出的结论确实是有道理的。

\section{大小之辩}

在分析了科大内卷最大的客观原因,社会竞争的下放之后,我们也必须看到主观因素在内卷中的影响。在上一章中,我们初步给出了一个“少部分人获取资源能力更强而倾轧大部分人”的模型,但这个模型应用到实际中同样也是十分复杂的。先回到上一章中提到的,感受到内卷主要是由于付出收获不成正比与对比而形成的落差,但那段没有提到的是,到底怎样的人更符合这样的条件。

\emph{在采访中,当问及大家是否感受到内卷的时候,出现的回答大致有三种:第一种回答是,确实觉得感受到了挺多内卷,也因此有不小的压力和焦虑;第二种回答则是觉得,内卷的压力非常强烈,几乎到了难以控制的程度;第三种是,觉得自己不太参与内卷,因此感受不算强烈。事实上,成绩比较好的同学一般属于第一种回答,而第二种回答往往来自成绩居于中等的同学,他们在竞争更好成绩的过程中感受到了内卷的压力。与这些相对,对于并不去(主动或被迫)竞争好成绩的人来说,完成的任务与学到知识的感觉都能带来收获感。然而,在专业课上缺乏竞争好成绩能力的人往往都会经历一段痛苦的落差,在最终接受现实后才能达到不参与内卷的状态。}

从这三种并不一致的反应中可以看出,内卷带来的焦虑感并不会和处在小部分人还是大部分人的位置直接相关,而是基本被每个人看到的环境所确定,更何况,每个人的认知也存在差异。在采访中实际上看到的是,很多人都更倾向于将自己放在“不太行”,或者至少是“还不够”的位置,去和自己所在的某个群体中更小部分人去比较,而得出令人焦虑的结论。对于成绩较好的同学来说,至少相对不错的成绩能提供一种安心感,不至于沉浸到焦虑当中,但中等情况——或者说很大一部分人——就不具有这样安心的机会,只能更加沉浸入内卷。

这样的心理感受的起因,首先就是比较。追求更好、与更优秀的人去比较并不是错误,但如果比较的结果只让自己增加了焦虑与无力,觉得自己“永远不可能达到”,那就并没有意义了。再去追问比较的根源,或许来自偏向单一的价值取向与精神的匮乏。在采访中,不少同学都由于时间原因,在大学里很少真正静下心来读书、思考。此外,科大的课余生活本也相对单调,加上课业的繁重,更难有机会去争取多方向的价值。值得一提的是,悬殊的男女比例也不止有表面的影响。虽然这个问题并不是调查报告主要讨论的范围,但是事实上,采访中的不少同学也有注意到它所引起的社交环境的特殊性,与这样的环境对心理的一系列影响。事实上,就像之前提到的,对自我管理与社交的焦虑虽然并不是内卷的范畴,但是会加重内卷对心理的影响,带来更多的负面情绪,也会减少内卷之外的选择空间。

总结刚才的讨论,在客观层面,内卷最重要的起因就是社会竞争的下放。这其中更深层次的社会公平与分配公正问题非常复杂,但造成的结果便是,大学作为踏入社会的跳板,也承接了早期的社会竞争。在主观层面,价值取向偏向单一引起了固定角度的比较,而在这样的比较中感到挫败后,内卷的感受就逐渐变得强烈了。不过,讨论到这里,其实仍然忽略了一个十分关键的问题:

\section{由外而内?}

谈大学内卷,绕不开同辈压力的话题。在“比较”的说法中,这似乎更近于个人的主观与认识,可这明显与现实中的情况并不相符——同辈压力从来都不止是一个主观感受。\emph{谈及内卷的定义,很多同学本能地觉得,在一些不必要的地方花费过多的时间,尤其是功利性地花费时间,是一种内卷。例如,某门课并非本专业的专业技能相关,但是有一个加分的论文项目。这种时候,很多哪怕对这门课一点不感兴趣的同学都会选择写一篇论文交上去,甚至花上很多时间,只是为了得到加分。事实上,绝大部分同学都认为类似的事情是内卷的典型例子。}不过,在对内卷整体进行分析后,可以发现这件事并不是定义,而是内卷压力的一种转移。专业课上的获取资源能力是相对固定的,也较难改变,而在这些非专业的内容中,很多时候反而更接近“公平内卷”的模式,有一定的付出就会有一定的结果。因此,在更大角度对成绩的需求对应专业课中的能力差异,导致了这样一种看起来内卷的行为。

在采访中,科大的同辈压力很大程度来自于专业课上的能力差异。在相对困难程度越高的课上,这样的差异就越是明显。但是,其中存在一个悖论:相对困难程度越高的地方越不存在上一段中所说的“内卷”,却是有限资源下获取能力差异产生倾轧的集中体现。这似乎意味着,这篇报告中的内卷定义与我们的直觉是相悖的。事实上,第六章探讨解决方案时将会提到,直觉中的内卷作为压力的转移,是无法先于本质问题解决的。同辈压力的程度越大,才越意味着靠近内卷的本质。

由此,同辈压力是一个介于主观与客观之间的概念,既有进行比较的感受,也有获取能力差异的影响。在真实情况中,同辈压力的作用往往是由外而内的。正是因为看到了其他人的情况,才会自然地与自己比较,进而产生内卷感。

说到这里,上一节中忽略的关键问题就昭然若揭了:我们看到的情况,真的就是真实情况吗?\emph{在同学们交流时发现了一个很有趣的现象,在上一届的最终保研情况公布后,不少人都发现,自己预估的保研线过高了,实际上并没有那么夸张。最极端的例子是,或许是由于一些本可以到达保研线的人由于判断失误而没有报名,某院保研的最低成绩远比想象中要低\footnote{尽管如此,保研线逐年提升的趋势是真实存在的,不过影响原因较为复杂,本报告不作详细讨论。}。}这似乎暗示了,从看到的情况中估计是并不准确的,而是存在着一定程度的夸大。事实上,所见的失真不仅是内卷的起因,更是内卷造成的核心影响的结果之一,很多时候,内卷的结果都会成为内卷加重的诱因,从而形成恶性循环,更加难以摆脱,这在第五章中会继续讨论。

在这一章的结尾,让我们回到那个重要的客观原因——社会竞争的下放。社会机会的竞争,或者说环境中的内卷,也必然存在其起因。采访中有同学提到,社会中的内卷来源是一个关系到社会公平、分配公正与经济发展的复杂话题,只要这些东西得不到平衡,社会中的内卷就不会消失。某种意义上,这个角度可以作为内卷的根本原因,而上面讨论的则更接近直接原因。不过,由于根本原因自身的宏大而难以确定,报告中对后续影响与解决方案的讨论都止于在科大看到的直接情况。如果能有在更大范围调查的机会与能力,或许也会从根本原因的角度看到一些问题与可能的解决方案。

\chapter{见闻}

在上一章讨论完起因之后,自然就轮到了对内卷现状的调查。从各种意义上来说,内卷现状都是一个很大的主题,也有非常多不同的方向。这一章主要进行时间(不同届)与空间(不同院系)层面的讨论,并结合具体情况分析内卷的程度与特点。

在生活中,我们总是会产生一些朴素的印象,比如“你们这一届怎么这么卷”、“某某院就是比某某院卷”,但事实上,正如第二章所说,是否内卷最终还是要看感受内卷的人,靠特定事件产生的一些印象往往是不可靠的。更何况,哪怕是同样的事件,不同人的实际体验也并不相同。因此,想要获知真正的内卷情况,需要通过实地的采访和调查。总体来说,科大的环境确实趋于内卷,但在时间空间上也存在不小的差异,甚至习惯了某个集体中情况的人可能完全不理解另一个集体的情况。下面就先从时间说起,回答一个很令人关心的问题:我们是不是越来越卷了?

\section{阶段与压力}

在讨论“是不是越来越卷”的问题前,我们需要先做一个界定:从长远来看,近几年由于出国形势的变化,出国人数减少,保研竞争比起往届的确更加激烈。因此,我们主要考虑近几年的情况,从而减小不可抗客观环境的差异。不过,之前保研线的例子也能看出,客观环境的差异或许也并没有想象中那么大。为了更好地讨论,我们先将目光聚焦在当前大一到大三的三届中。

\emph{在以大二学生为主的数学学院闲聊群中,时常会感叹自己所在的班上有不少大一提前选课的同学,并且学得都还挺不错。对比大家在大一时的情况,似乎感觉到这届大一要比大二更“卷”。从整体上来说,也有不少同学觉得下一届比这一届的竞争更加激烈。不过,从采访到的大一大二同学来说,并没有什么这一届的竞争比上一届更加激烈的感受。}在刚才的这个例子中,其实就体现了我们感觉到越来越卷的一个核心原因,幸存者偏差\footnote{原指当获取资讯的渠道只来自幸存者时产生的偏差,后泛特定资讯来源引起的认知偏差。}。由于学习或课程而接触到的低年级学生往往都是竞争中完成出色的那一批人,自然就会对实际情况产生错误的估计。事实上,往往只有成为助教以后,才较有机会了解到真实的整体情况。除此之外,培养方案的变化也让这届数学学院大一下时课程减少,提前选课也更加方便。因此,从提前选课情况来判断竞争更加激烈是不准确的。

在采访中,不同届的竞争激烈程度并没有明显的差异,至少远没有院系间的差异来得明显。不过,竞争激烈情况的结论并不能直接说明内卷。正如之前的定义,内卷归根结底要看被倾轧者的实际体验,而关于这件事,采访中的大一同学体验似乎真的比大二同学更为强烈。例如,谈及成绩时,大一同学比大二同学总体传达了更多的焦虑感,而且更多表达与排名最前的几个“卷王”对比所感到的落差。从这样的视角看来,好像大一真的比大二更加“卷”了,不过,另一份采访记录反驳了这个观点。

\emph{在我大一下时,曾经针对“科大数学学院同学的压力与解决”作过采访,也保存了采访记录。对比当时的采访记录可以发现,在落差感、挫败、焦虑等角度和现在对大一同学的采访结果是很相似的。就连高中到大学心态的变化、对整体环境的看法上,也有很多相似的角度。}这份一年前的报告暗示了,不同届之间对内卷体验不同并不主要因为“越来越卷”,而是和大家所处的阶段有关。

结合之前对集体与环境中内卷的定义可以发现,阶段所引起的差异和内卷的形态也密切相关。更准确地说,随着年级的推进,大家所感受到的竞争范围逐渐扩大,从刚上大学时基本只盯着校内的课程到看到不同道路的差异、谋取其他方向的机会,内卷感的来源从班级内部的学习情况差异逐渐扩展到了大方向中竞争所带来的压力。在对大一同学的采访中,对成绩的焦虑主要是对出路选择的影响,而大二同学很多已经开始在有了大致出路的情况下准备细节。事实上,第二种情况所带来的焦虑并不比第一种少,只是因为在大环境中竞争而没有明确的指向而已。

当然,以上说的这些并没有否认各届之间确实存在差异。无论是竞争强度还是内卷情况,不同届都是有区别的。只不过,比起空间差异和阶段性,这样的区别并不大,且也没有充足证据表明“我们正在越来越卷”。反过来说,这其实也某种程度上说明了情况并没有越来越好转,仍然存在很多需要解决的问题。

\section{行路难}

说完了时间层面的差异,就到了大家最为关心的院系间差异。这一节中,主要是讨论自然科学类院系的情况,包括偏向理论科学的数学学院(数院)、物理学院(物院),与偏向实验科学的化学、生物等。

在上一章中提到过一个悖论,相对困难程度越高的地方越不存在大家所明显感知的内卷,但实质上的倾轧更容易发生。这个悖论放到院系层面仍然成立。相对来说,数院和物院的课程困难程度几乎是最高的,在大家的印象里也并不太内卷。然而,在不太内卷的表象下,数院与物院并没有看起来那么友善。\emph{在一年前的采访中,有一位同学在高中时成绩不错,也学习过一点数学竞赛的课程,开学考试成功进入了数院的英才班。可这位同学很快发现,大学的数学与相对熟悉的高中数竞完全不同,也并不适应。期中考试时,学分最高的数学分析班级均分只有50分左右,这位同学的分数则更低。在过大的落差感的影响下,挫败渐渐转为了无力与麻木。直到现在,这位同学的情况也没有发生什么变化。}这样的情况并不是个例。初到大学的迷茫、对课程的无所适从与困难的考试不仅会冲击同学的自信心,也会带来更多对比中的落差感与分化感。

实际上,正如内卷的模型,产生这样感觉的人是大多数,只有少部分人能在大学的学习中顺风顺水(而他们几乎都是有着良好竞赛基础的)。理论科学类专业的悲剧之处就在于,如此产生的落差与分化十分难以解决。从采访到的情况来看,很多差异都是来自于之前对于知识和方法的积累,因此改善它们的时间周期较长,过程也较为艰辛,再加上一些外在条件的阻碍,便更加显得无法解决。即使有一些改善的力量(譬如班主任的偶尔约谈),往往也会因为难以长时间维持而无法有良好的效果,更何况,内卷的负面影响也起到了反面的作用。

\emph{在一个课程群中,有时会讨论一些事实上有关后继课程的“升级”问题。不止一位同学反馈,在看到这些升级问题时,对比自己连课内都搞不懂,不止会产生落差,也会失去在群里提问的勇气。类似的观察也可以适用于匿名与实名:在允许匿名的课程群中,升级或是困难的问题往往更倾向于实名交流,而关于上课理解的一些则多是匿名发问。}在采访中,有一位同学将这种情况总结为一种实质上由学习情况产生的权力关系,我认为这个总结是颇恰当的,而这样的“权力关系”,也是理论科学类专业中内卷的核心体现,即忽视。

就像任何内卷的环境中那样,理论科学类专业也存在资源获取能力的两极分化。可是,或许由于两极分化过于明显而难解,或许由于落差引起的麻木,哪怕这种两极分化已经形成了明显的割裂,大家仍然倾向于认为这是正常的,并且,大家(即使是属于被倾轧者的大部分人)的目光也更多投向了前排的少数人。或许值得庆幸的是,这种被忽略虽然带来了如一年前采访时有同学提到的“淡漠感”,但并无太多的矛盾,气氛相对和缓——与这种淡漠相比,下一节中讨论的情况就更接近另外一个极端了。最终的结果是,在对核心竞争力的追求中,陷入实际的困境的大部分既没有在开始时拥有足够的预警与准备,也难以得到足够支撑至度过困境的帮助,在分化的路上渐行渐远。

对偏向实验科学的专业来说,这样的两极分化会比理论科学要轻一些,因此内卷也相对弱化。不过,相似的问题仍然存在着,也造成了一些负面的影响。虽然如此,由于实验科学类专业存在更多实验室等偏向应用、需要实际操作的方向与机会,较之理论科学也更易找到出路,社会竞争造成的焦虑感相对较轻,情况还是比起理论科学类专业好上不少。

\section{全新的冲击}

与理论科学类专业的内卷形势相比,网络安全、计算机这些大家普遍认为内卷的信息科学类专业中,内卷所引发的冲突就更加明显。在采访中,计算机科学学院(计科)经常给同学们留下矛盾冲突多的印象,本质上也是内卷引发的结果。一位同学提到,计科的矛盾冲突多从某个角度来说是因为相对简单:比起理论科学类专业来说,计科的专业课难度相对较低,但是有很多精细、严格的标准。例如,数院的考试结果往往是均分很低,等待老师调分来得到最终成绩,而计科的最终给分往往与严格比例计算后的结果相差不大,甚至可能由于优秀率的限制向下调整。因此,计科同学们需要去计较平时每一点分数的得失,这之中自然增加了发生矛盾的可能。并且,完成难度不高的情况下,区分度很容易沦为工作量的竞争,进一步导致了收获无法弥补付出,从而积累了矛盾。

当然,这样的矛盾与冲突不能直接证明内卷的程度,现实中,对内卷的感知是十分复杂的:\emph{在计科的采访结果中,绝大部分人都认为计科是一个内卷的环境。具体的原因有较为直接的,如大家比拼完成速度与工作量、竞争加分,也有从戾气乃至在QQ群里吵架这些间接结果中感受到内卷。尤其是转院进入计科的同学,由于对比而产生了更深的体会。不过,当问及内卷对个人的影响时,大家的体验就不一致了。一些同学感到内卷对自己的影响很大,不仅增添了焦虑,还导致付出更加难以得到收获,可也有不少同学觉得,自己不陷入争执中,学习情况也没有到达可以竞争加分的程度,事实上并没有参与内卷。不过,采访到的这些同学中只有一位同学觉得做好自己就满意了,不需要与他人比较,其他则是或多或少感受到了负面的情况,相对常见的回答包括迷茫、被推着走、劳累等。只是这些感受的来源并没有被归咎于内卷,而是往往被归因到了自己身上,比如自己懒惰、学得不好、不聪明,诸如此类。}

下一章中将会提到,这里的很多负面感受实质上都与内卷存在关联,乃至直接由内卷导致。此处举一个简单的例子:采访中,不少同学会由于QQ群里的冲突多而不愿在群中发言,在前述的归因自己的作用下,更加不愿意向他人求助,而总是倾向于自己解决问题——哪怕是已经卡了很久的情况下。然而,很多时候只需要一句提示或者简单看一眼,就可以节省极多的时间与工作量,也并不会消耗他人的多少精力。\emph{有一位同学在某次上机实验时,代码出现了问题。为了解决问题,他花费了一周时间,甚至用两个通宵整理出了全部的结构,仍然未能解决。最后,在实验课时助教的帮助下,他才找到了真正的问题所在。}这样的例子时常发生,而这整整一周的纠结其实并不能带给这位同学多少专业知识上的进展,只是浪费了时间而已。此时,内卷直接引起了交流的减少,进而产生了更多的负面影响。

至此,计科内卷的模型就可以大致确定了。能轻松完成普通内容的小部分人出于精细标准的考虑,为保险而完成超出要求的内容,于是在产生区分的需求下,整体标准也随之提高。可大部分人本来就需要尽力才能完成普通内容,在标准提高后事实上任务更重,却没有更多回报,甚至还可能降低回报,而小部分人也需要做出更多才能保险。对这样整体工作量提高的不满成为了矛盾的根源,发生的冲突或是指向提高的标准,或是指向想要保险而超额完成的人。更多时候,不满由于未得到释放而积累起来,导致了更多争执的爆发。本来这样的争执可以暴露不少真实的情况,但一方面问题很多时候并没有得到解决,一方面争执中本身也有很多情绪化的成分,最终冲突走向了常态化且无意义,真实的情况反而被隐藏,分化愈发严重。需要表达观点的人无法安全地表达观点,渐渐习惯了不再表达,交流的空间因而失去了交流的作用。

信息科学类专业中,类似计科的情况并不少,但具体情况也有一定差别。例如,计科由于人数较多,更接近环境中的内卷,对整体的情况难以把握。而在人数较少的专业(如网络空间安全专业),有时可以直接看到大家完成的多少、快慢。这样更加公开的竞争对有些人来说增加了把握,因此不用无限度增加工作量,但也可能导致矛盾的激化,产生负面的影响。

\section{焉知非福?}

分析完两种不同的内卷形态后,我们可以再回头看看“内卷是什么”的问题。采访中,也有出现内卷环境可以一定程度督促努力的说法,但是,从对两种形态的分析中即可以反驳这样的“正面效果”:除了直接以倾轧去理解外,两种形态的内卷有一个共同特点,也即同学内部产生的分化。

在第一种形态中,这样的分化主要体现为了漠视与疏离,而第二种形态中则主要体现为矛盾冲突,而最后的结果是相似的,也即我们越来越难以听到不同的声音,也越来越难以理解其他人的处境。\emph{我的一位朋友正在做一个低绩点群体相关的采访。在预调查过程中,她发现绩点偏低的同学往往都比较支持这样的采访,而觉得这样的采访没什么意义的声音大部分来自成绩不错的同学。}虽然预调查存在一些局限性,在我的采访中也发现,在是否愿意表达上,关于具体问题与具体情况存在不少差异\footnote{在第五章中将会提到,这与话语权密切相关。},不过,这样的情况仍然可以一定程度说明分化的存在性。

反过来说,凡是引起或者扩大这种分化的行为、事件,几乎都与内卷相关。不管是数院的困难课程还是计科的精细标准,实质上都增大了区分度,而精细标准很多时候更是为了增大区分度而设立,因此才导致了更多矛盾的产生。遗憾的是,分化并不能带来努力。区分度所带来的层次感与疏离感自然不是努力的理由,而为了在区分中达到更高位置所付出的时间与精力也很难给人努力感——事实上,之所以很多同学会将“在不必要的地方花费过多的时间”作为内卷的定义,将花很多时间去写其实并不想写的加分论文称为“卷论文”,正是因为很多时候这些行为只是为了区分度,于是既无法给自己努力感,也会让他人感受到内卷的压力。与之相对,在更为团结、区分更少的场合,则更会有努力的动机,也更容易获得收获感。

绝大部分情况下,分化与倾轧本就是相伴而生的。有了分化就有了高低,倾轧自然从中产生,而倾轧与被倾轧也会在不同情境下导致分化。由于下一章讨论内卷的影响时也常用到分化作为定义,这里值得再说说分化和普通的差异有何区别。在前面的段落中提到,分化的特征是“难以听到不同的声音,难以理解他人的处境”,也就意味着,存在了某种基于普通差异的隔离。正是这种隔离把交流与观察阻断在了局限的空间之内,让分割出的不同群体都因此受害。在最后一章试图给出一些内卷的解决方案时,集体层面首先需要解决的也是这样的分化、隔离。

虽然这一章其实也写到了一些内卷的影响,但都是站在较为宏观的角度,在下一章,将从个人的视角出发,去尽力还原每个人在内卷中的真实处境,也会进一步论证,无论看起来是“成功者”还是“失败者”,大家实际上都受内卷所害。

\chapter{大风吹}

草东乐队有一首歌,名叫《大风吹》\footnote{乐队全称草东没有派对,这首歌则获得了2017年金曲奖的年度歌曲奖。}。从歌词的字面意思来看,歌曲讲述的是一群孩子做游戏时比较着玩具的好坏,欺负融不进这个圈子的、没有买新玩具的孩子,而歌的内容也影射着更大层面的分化。这一章主要是讨论内卷的影响,以“大风吹”为题,既是想说内卷之风的猛烈与难以逃脱,也是引用这首歌来表达分化的可怕。

个人所受到内卷的影响,最明显也是最普遍的一条即为压力。即使没有内卷的环境,只要有了竞争和评判,自然就会感受到压力,区别只在于压力的大小与产生的效果,而内卷加剧了这样的压力,也让压力的作用更加明显。很多时候,我们所认为的“内卷的正面作用”实质上是压力的正面作用,比如督促进步、避免懈怠。而接下来要叙述的影响,有的与压力密切相关,有的则关联不大:

\section{囚笼}

说到压力,最容易想到的负面影响就是焦虑,而焦虑感也恰恰是内卷带来极大影响的方面。\emph{在采访中,大部分同学都反馈了各自不同的焦虑感,小到考试结果、课程成绩、看到同学努力后对比自身,大到未来出路、就业前景,种种东西都可能带来焦虑。而焦虑造成的结果有时也挺严重,除了对工作效率、时间规划的影响外,甚至可能直接导致失眠等生理作用,进一步影响学习生活。}在之前的部分中,曾经提到过“内卷感”一词,而能感受到的内卷中,焦虑的占比非常之高。

内卷造成焦虑感的来源是多方面的,既有大家可以看见的竞争强度所增加的焦虑,也有如上一章提到的忽视了部分人导致的所见失真。于是,这里又产生了一个悖论:付出的越多,就会有越高的预期收获,而这样的收获在内卷下难以保证,于是愈发焦虑。这个悖论其实解释了,为何看起来是作为小部分倾轧者的人感受到的内卷一点也不比其他人少——正是因为他们付出了更多,他们的预期成果在内卷中更加飘忽。然而,这部分人的焦虑感会导致他们进一步付出,也进一步加重了分化。\emph{提前选课就是这样一个例子。一般来说,学有余力的同学可以提前选修一些更高年级甚至研究生的课程,而在对出路的焦虑感下,哪怕并没有足够的喜欢,有余力的同学也会投入更多的时间去提前选更多、更难的课,直到自己承受的上限。于是,这部分同学与按部就班依照培养方案学习的同学差距更大,共同交流的课程则更少。}

另一方面,承受的上限总还是存在。在提前选课的例子中,当课程已经达到或超出承受上限时,除非能取得突破性进展\footnote{在真的到达上限时突破的情况在采访中也有遇到,但极为罕见。},否则强行给自己加码只会导致已有的学业都难以完成。由此,落差感自然产生,痛苦也随之而来。这样的痛苦属于每一个试图突破自身极限的人,无论处境如何、倾轧还是被倾轧。之所以说这是内卷的影响,是因为不内卷的状态下,不存在明显的分化,需要极限运转以图突破分化的人也不会占多数。可是,只要尝试了,就会发现,个人的上限成为了一个囚笼,限制了几乎所有向上的可能。

采访中的发现乐观与悲观并存:乐观处在于,真正达到个人上限的情况没有那么多,不少人都有过些突破自己曾经觉得难以突破的障碍的体验,但悲观在于,需要挑战自我时,由于内卷所带来的忽视而缺乏足够的支持与帮助,常会导致挑战失败,进而认定自己的上限不过如此。之前的焦虑感在反复尝试中转变成无力感,逐渐失去了挑战的勇气与决心。这时,一个虚拟的囚笼诞生了,明明再踏出一步就可以离开,实际却连迈开脚步也做不到。

这样的从焦虑向无力、痛苦的转变,在采访中并不少见,最后的结果常是同学所调侃的“摆烂”,或是“躺平”。没有前进的动力,也没有努力的理由,只好熬过一天是一天。遗憾的是,对大部分人,“摆烂”也并不能真的缓解焦虑感,在努力与“躺平”的交替下,既没有良好的成效,也没有获得轻松。

\section{矛头}

如果说刚才的焦虑与痛苦还只是针对自己,这一部分所叙述的影响就是针对他人了。在之前曾经说过,反复强调自己客观或主观的不足被称为“卖弱”,这种行为的起因可能是上一节中所说的焦虑感,也可能是出于恶意的炫耀,但无论是否有恶意,它都加重了内卷。原因在于,当处境不错的同学去渲染自己的“弱”时,忽略了那些情况并不如他的同学的感受,而看到这些话的人,在肯定或否定的过程中也会下意识忽略,这让分化变得更重,也在实质上构成了倾轧。

类似的情况其实并不少见,之前提到过的在群里讨论升级问题等都会产生一定的影响\footnote{这里需要强调,导致内卷加重并不能说明行为本身是错误的,对他人造成伤害的可能也不能直接归结为责任,在第六章中将会进行讨论。},而这些引起内卷加重的行为,事实上也是内卷导致的结果。以“卖弱”为例,焦虑感的来源已在上一节中叙述了,哪怕是恶意的炫耀,本质也是为了巩固自身在两极分化中走到上方的形象,都与内卷密切相关。

事情进一步发展后,引起内卷加重的行为也会加深每个人内心的不满。\emph{在采访中,同学们看到课程群讨论自己完全不理解的东西时,往往会感觉恐慌,也会提升压力,这些恐慌与压力在超过限度后,逐渐成为对讨论问题者的不满。而在“卖弱”的例子里,这样的不满更为强烈,对事的不满与对人的不满更是交杂在一起。再比如,少部分人为了确保结果而付出更多时,大家对标准提升的不满也很容易转化为对这些人的不满。}在上一章计科的例子中,因为计科所遇到的内卷更加具体,不满的积累也会更快;积累不满的人多了,冲突也就增多了;再加上不满的指向很可能是个人,冲突随即升级为了人身攻击,从此周而复始,“戾气重”的印象便这么产生了。

本质来看,积累的不满与戾气其实都是对分化的反抗。不愿意被分化、不愿意受倾轧、不愿意屈服于内卷引发的焦虑,这些情绪都可以成为击破内卷的矛头,但现实中,矛头却常指向了和自己一样的受害者。从这个角度看,分化就多了一层含义,并不只是倾轧所产生的两极分化,还有着平行的不同划分——矛头总需要释放,可内卷却不是可以轻易触碰、轻易解决的,在转向同学们后,攻击的方向自然产生了不同的划分。

不管是戾气,还是所谓的“鄙视链”,都是这样横向划分的体现,就像两极分化一样,横向的划分对内卷也只有加剧的作用。划分阻断了沟通,阻断了一起交流解决问题的可能,而放大了微小的矛盾,于是在需要发泄的时候,划分也让发泄更大概率呈现攻击的形式。

结合这一节与上一节的内容,不但能看到内卷的影响,而且能看到内卷本身是如何恶性循环的。从焦虑、无力到戾气、矛盾,内卷的影响看起来已经造成了非常严重的结果,然而,内卷最大的负面作用之一却很少被提及,哪怕被提及,也很少被认为是内卷的影响:

\section{在沉默中灭亡?}

在之前,我们对倾轧一直没有一个明确的定义,只是模糊地表示这是资源获取能力的差异所造成的一个并不正常的结果。在对现状进行分析后,即可以得出倾轧的主要内容:生存空间的挤占与话语权的缺失。

生存空间是一个很大的词,更小处来说,在竞争资源的模型之中,它代表着付出平均程度的努力所能获取到的资源。对生存空间的挤占也就意味着,平均程度的努力所能得到的资源更少,因此必须付出更多的努力来保证相同的资源。对生存空间的挤占往往会带来很明显的感知,我们平时抱怨的内卷也往往来源于此。前文定义倾轧时,一直用模糊的语气去说“大部分人”与“小部分人”,则正是因为在对生存空间倾轧的感知中,“小部分人”事实上是一个概念上的存在\footnote{这也解释了为何几乎所有人都能感知到内卷,因为每个人都有自己概念上的“少部分人”。}。

此处应当解释,概念上的存在并不代表其不存在,但采访中,对于感到自己“被卷死”的人来说,除了说出几个符号化为“内卷代表”的人以外,很难具体表明究竟是被什么人给“卷死”的。在不同地方见到的,在不同方向似乎有着远胜自己才能的人,被共同整合成了“少部分人”,成为了一个抽象的倾轧者。可概念存在的问题就在于,被整合为倾轧者的个体很可能并没有超出其他人,与之相对,哪怕仅在某个小处的表现也可能被整合进全局的倾轧者形象当中。

\emph{在采访中,不少同学曾经产生或一直存在自己“一无是处”的想法,认为自己在任何角度都比不过别人,甚至有同学因为这种想法所带来的悲观、焦虑心理产生过自杀的念头。这些想法的产生原因往往是与别人比较的落差感与努力看不出成效的无力感,例如有同学就提到自己认真复习很久也达不到别人一两天“速成”出来的成绩。}这里,别人所带来的压力之所以如此大,不小程度上即是因为上一段中说的整合,而采访中的描述则对应了内卷其中一个最大的负面影响:自我价值感的降低。

自我价值感的降低在科大十分常见,不少同学也意识到了它带来的影响,但似乎鲜有人用内卷对其归因。然而,身为被倾轧者的恐慌造成了对自身不利的社会比较,又在生存空间的互相挤占中被放大,恰恰是降低自我价值的重要因素。在下一章中会仔细讨论这种情况的应对方案,而此处更想要关注的是,为何我们格外容易整合出一个倾轧者,降低我们的自我价值。而这就涉及了前面提到的另一个关键词:话语权。

称呼其为话语权或许不尽准确,因为并没有任何明确的规范去禁止什么,可正如前文“在群里讨论升级问题”与“权力关系”等话题中提及的,我们的确不习惯(尤其是在实名发言时)在偏公开的场合表达自己的困难。一方面自然是这样的场合中对自我形象的顾虑,另一方面,其实在隐含了在某种权力关系中受到的倾轧——超前学习、优秀是“应该”的,而跟不上进度则是“当受指责”的。这样的心理暗示在集体中造成沉默的螺旋,于是愿意公开表达的、会被看到的便常是当前场合成为“小部分人”的存在,所见的失真由此产生。上面说的第二个方面,很大程度来自分化感,因为表达困难仿佛意味着承认自己当属分化中较低的位置,而分化感则又与内卷紧密相联,这样一来,话语权也与内卷有了直接关系。

在对倾轧的进一步讨论中,我们得到了内卷最大的负面影响——在分化感下,伴随着沉默引起的失真,我们整合出了“全能”的他人形象,并以此对照自己,将自身的价值贬至谷底。说这是最大的负面影响有三个方面理由:首先,前面的很多负面影响都可以由此产生,哪怕针对他人的矛头也会因为低自我价值感而更易出现;其次,诸如焦虑一类的负面影响在脱离当前环境时会慢慢淡化,可降低的自我价值想要找回却困难得多;最后,这也更容易在整个环境中传播,影响到更大的范围。

在下一章,我们将试着给内卷提供一些解决的方案。如果说集体层面上对抗内卷就是对抗分化,个人层面上对抗内卷的重要话题即是找回自我价值。这一章结尾,我们最后一次回看本章的题目,“大风吹”,从上方讨论来看,每个人都难免身陷风中,被吹散、牵引。然而,一棵树或许易被吹折,聚成防风林后即可有效削弱大风,对内卷,或许也同样如此。当每个人的力量都发挥出来时,内卷是否也就不再那么可怕了呢?

\chapter{战或逃}
战或逃反应\footnote{英文原名Fight-or-flight response,由美国心理学家Walter Cannon创建。}最初指遭遇危险时,身体为了保护自己或准备逃脱而进行的一系列准备,而现在,“战或逃”一词则引申为了面对危险场景的两种不同抉择:留下搏斗,抑或规避危险。在个人面对内卷时,往往亦存在类似的抉择,可很多时候情况似乎是“战斗无法取胜,逃却也不能彻底逃脱”,便只好束手无策。

经过前两章对现状与影响的讨论,我们已经有了充足的对抗内卷的理由,因此,我们也应当思考如何对抗这样一个抽象的存在。这一章讨论内卷的解决方案,会与定义相同的顺序,从上一段说的个人角度谈起,再到集体、环境,试着从不同层面给出不同的可能。当然,空谈解决方案似乎显得过于理想化,因此本章会有不少具体的例子,从正反两面加以佐证。

\section{藩篱中}

个人的角度,大家多少都会有些抵抗内卷负面影响的办法。\emph{在群里的讨论中,最常见到的说法是所谓的“适当摆烂”,或者说,意识到自己的极限和与他人的差距,进行取舍,以达到自己相对可以接受的情况。讨论中发现,这样的心态调整对不少同学都是有一定效果的,可以将没有尽头的焦虑感转化为更切实的对未来的考虑,从而尽早作出相应的准备。不过,这也并不总会造成正面作用,就像前文说过的例子,如果感受到的落差过大,容易直接失去动力。}在分析这个说法前,有一件事需要明确:虽然内卷对个人的最大负面影响即是自我价值感的降低,“适当摆烂”事实上应对的也是类似的场景,但并不是每个人都会在内卷中受到这样的影响。无论是其他人形容中的“卷”还是“躺平”,只要没有怀疑自我价值,某种意义上已经规避了不少内卷。反过来说,这个说法造成的正面作用有限,或许也恰是因为它并不能真正解决问题,而是试图适应、习惯已经降低的价值感。

\emph{可是,如果不接受上面的说法,似乎会导致更严重的后果。在讨论中,也有同学提到花费大量时间在薄弱环节,却没得到满意的结果,甚至顾此失彼的情况。此外,“意识到与他人的差距”之前,确实更容易感觉到焦虑。}所以,“适当摆烂”确实构成一种解决方案,只是有些角度值得加以改进。在上一章中提到,“别人”的形象是一个经过整合的倾轧者形象,如果能意识到这个形象中有将他人优点结合的成分,现实中让自己对比产生无力感的人并没有那么多,就能对心态进行调整。又比如,承认客观差距、作出取舍并不代表否定自己的一切,找到自己的长处或想做的而作出准备也能提升自我价值,哪怕暂时没有确定,也不要沉溺于比较之中,而应给自己留下放松的空间。

以上两段所提到的提升自我价值、调整心态,其实是老生常谈,也总被认为是“逃离内卷陷阱”的方式。可是,这些方法都依托着一个非常重要的前提:认知。前文说过,内卷在一些时候突出表现为分化,而分化带来的严重后果便是阻断了对实际情况的认知。当内卷在每个人身边筑起一道藩篱后,我们看到的世界已经与真实的世界相隔离了。如果不试着跨越这道藩篱,在藩篱内作出的调整与努力无法得到反馈,很容易沦为徒劳。

既然藩篱是由分化产生,它的反面就是团结,而想要达成团结,理解则是必要的前提。\emph{有不少同学都对互相理解持悲观态度,尤其在大家的经历、体验各不相同时。一个具有代表性的看法是,“体谅”是理解的最低形式,但当输出观点、道理与自以为是的正义占据上风时,连体谅都难以做到,更不用奢求进一步的理解了。}这个看法事实上很有道理。绝大部分情况下,我们只有通过交流才能理解与自己不同经历带来的想法,但随着互联网媒介的发展,交流似乎变得越发廉价,而理解却反而越发困难\footnote{这或许与快节奏对思考的抑制等因素有关,但不在本报告的研究范围内。}。所以,如果个人想要克服内卷,需要的不仅是交流,更是有效的交流。

\emph{在之前“卖弱”的例子里,我曾经在群聊中看到过被指责卖弱的人坦诚自己的担忧与感到不足的原因,并对给其他人带来的焦虑感道歉,而在“矛头”所引起的争执中,也有双方都意识到自己所正在指责的也是受害者,于是达成和解的场景。}这些场合下,交流带来了对其他人的理解,或者至少是“体谅”,因此成为有效的交流。认知是一个双向的过程,当有效交流促进理解的同时,藩篱也会被慢慢打破,他人的世界与自己的价值都将慢慢显露。反之,如果开始时就带有恶意,或是固执己见,不愿倾听,有效的交流就很难完成了。

综合这一节的内容,结论或许有些不可思议。从自我价值、心态出发,我们得到的个人层面解决内卷的本质方案并不是对“自己”做些什么,而是多与“他人”进行有效的交流。不过,考虑到不论是倾轧还是分化,都是涉及他人的过程,这个结论就不那么意外了。如果不与他人沟通,自己不管怎么调整心态,终不能改变分化的事实。

\section{本末}

说完个人层面的解决方案,自然又轮到了集体。此处,先以教学班为例:\emph{不论是通过采访还是直接浏览评价课程的“评课社区”,均能发现,让同学们感到内卷的课程往往都对应着老师、助教的一些行为。例如,设置加分的论文、汇报,规定一些过于细致的标准,不回答偏基础的问题,或是布置通过工作量决定成绩的任务。}就像之前的分析,这些行为让人感到内卷是由于它们加重了分化。然而,从集体角度而言,由于排名、区分在现实中存在,分化好像是无可避免的,不仅如此,标准的细致,加分的论文、汇报也是教学中正常的安排。那么,集体的内卷真的能通过老师、助教的努力缓解吗?

事实上,这节的标题已经给出了答案。在教学班的教学中,让同学们获得知识、技能是本,区分却是末,甚至根本不属于教育的意义。所以,对于某个安排,最重要的是它的目的——究竟是为了“本”,提升大家的能力,还是为了“末”,只作为产生分化的工具——而不同目的对内卷的影响截然不同。然而,现实情况的复杂性就在于,提升能力与产生分化往往并不能简单分离:

\emph{对于涉及实验的课程,在实验报告上设置区分往往会遭到诟病,因为在合格的基础上,实验报告的详尽程度几乎只取决于付出的时间,而与对内容的掌握无关。可是,在实验上设置区分的方式也存在很大的差别。这学期的一门课常被人诟病,正是由于实验量比起往届大幅增多,难度也有提升。尽管增多与提升的内容往往都与课程密切相关,实验文档的设计也尽力帮助理解,对相当一部分人来说,难以甚至无法完成实验却导致了获得的知识反而减少,更因成为区分的对象感到了更明显的内卷。这样的情况反馈给老师后,老师坚持实验的区分作用,不愿降低实验量,给更多人带来了焦虑——接近期末时段的实验,哪怕尽力完成也会挤占不少复习的时间。另一方面,有的课程却因为实验内容几乎没有区分,导致为了产生差异不得不临时改变标准,也遭到了诸多不满。}

这个实验设计上的困局是很多集体内卷困局的缩影,好在,不少课程的尝试中,我们可以找到一个减轻内卷的解决方案。首先,实验必须结合知识,区分度也应基本来源于对知识的掌握。其次,难度不应让相当一部分人无法以合理的工作量完成,这样只会徒增分化感。在这两点的基础上,对避免分化最重要的行动,应该是关注难以完成的同学。一方面在无法全部完成时也尽量保证能学习到东西(如提供题解),另一方面提供一定的补救措施(如延长时间或提供得分稍低的替代方案)。

刚才的解决方案里,也蕴含了安排者的角度如何减轻集体层面的内卷。第一,即使有产生分化的需要,也不应只为“末”作出安排。主干必须是更有意义的内容,区分也应取决于此。第二,考察实际情况以设置合理的进度,尽可能追求“本”。第三,当发现分化时,及时采取行动避免过度,例如根据当前进行调整,或提醒注意他人的感受。如果总结一个最重要的方案,其实就是分清本末、降低分化的上限。

对个人来说,靠自己改变集体或环境的内卷是几乎不可能的,可以采取的方案仍然是有效沟通。\emph{一位学生代表将对某门安排并不合理的课程的改革建议整理成了提案,并聚集了百余位附议人,通过系统将提案提交到了学研代会。之后,这份提案成为了优秀提案,得到了更大的关注。}这个例子里,有效的交流确实团结了力量,并且成功反馈了真实的诉求。除此之外,安排者角度的“实际情况”想要避免失真,也需要安排者与个人进行有效的沟通。

\section{敢问路在何方?}

关于环境中的内卷,第三章的结尾提到了其出于一个更复杂的本质原因:社会竞争的下放。这个问题的可能解决方案已经超出了本报告的调查范围,因此,除了上节末尾提到的反馈诉求外,我们主要关注个人与集体层面,不对环境中内卷的解决方案作过多的讨论。

作为结语前的最后一章,本章试着给出的解决方案依托之前对定义、起因、现状与影响的分析。很遗憾的是,上面两节虽然从理论上得出了一个解法,却还是过于理想化了。\emph{以个人角度的有效交流为例,采访中发现,它的难以达成并不是因为大家不愿意采取交流的手段。根本原因是,理解乃至体谅本就很困难,而连立场都不尽相同时,交流仿佛并不能拉近距离,只会引起争端。于是,在一次次尝试未果后,沉默与攻击渐渐替代了交流,成为应对问题的首选。在集体或环境的角度,安排者未必能意识到实际的情况,就算意识到,也未必愿意改变计划——就像那个获得优秀提案的提案,在得到关注后便好像销声匿迹了,并没有收到有实质意义的回复。反过来,安排者有心改变时,也常会碰到各种阻碍,乃至对动机的质疑。}

现实就是如此。即使我们通过分析获得了可能有用的办法,能以他们解决问题仍然是小概率事件。究其原因,内卷早就根深蒂固,几乎一切与资源相关的比较都很可能带来分化与倾轧。不过,既然失败才是常态,我们就更需要找到足以坚持的理由,而这些理由,也是本节想谈论的核心:
\begin{enumerate}
    \item 每次尝试都意味着希望,不管概率有多小,尝试至少代表了一点点成功的可能,而尝试的次数多了,这一点点可能也会被提升成更大的希望。
    \item 坚持亦存在意义,当尝试在坚持中不断延续时,就代表了一股鲜活的力量。待到合适的时机,或是改变的冲力足够时,这股力量即能爆发。\emph{同学坚持反馈的不合理课程或安排,从整体上来看,确实是在不断改进的。往往当有心改革的老师负责这样的课程时,结合之前的反馈可以更好找到改进的方向。}
    \item “解决”这个词中包含了一种暗示,就好像问题只有解决与未解决两种状态,然而让情况缓解是可以做到的。正如理解存在不同的程度,也有机会随着交流慢慢加深,对内卷的缓解亦会减轻它的负面影响。
    \item 哪怕以上这些都没有成效,以至现状几乎没有改变,还有最后一个理由,也就是尝试本身。我们至少尝试过,否则,连判断其不可能的资格都没有。
\end{enumerate}

本节名为“敢问路在何方”,而下一句词即是“路在脚下”\footnote{取自《敢问路在何方》歌词。}。无论这篇报告的分析是否正确,解法是否可行,至少有一件事是确定的:只要坚持寻找真正的道路,保存抗争的决心,内卷绝非不可名状的深渊,也绝非不可战胜的怪物。分化与倾轧筑成的藩篱隔绝了人与人的内心,但这也意味着,解决之道就藏在内心当中。

\chapter{结语}

写到这里,这篇调查报告终于接近尾声了。前六章中,内卷被具象为了分化与倾轧,也据此得到了它的种种性质,给出可能的解决模式。结语中,我们来解释最后一个,或许也是最重要的问题之一:内卷是否可以战胜?

就像前面所多次提到的,哪怕不去考虑那个概念中的“倾轧者”,分化也是十分真实地存在着,并且由于作出比较的需求,产生分化几乎是注定的。这样一来,上一章所提的“解决”又好像无法起效了,因为内部竞争的情况始终存在,就算达成理解好像也不能在这个意义上产生效果。

对这个问题,我想给出一个回答:内卷是可以战胜的,这是由于身为学生,我们的本质利益从来都是一致的。竞争等情况造成了分化,但将精力投入对分化的维持中是一个违反所有人本质利益的事。反之,从更大的角度,甚至是社会竞争的角度来看,团结与作为其基础的“理解”“体谅”才是长远中有利的选择。或许,这不止是内卷可以战胜的原因,也是我们应当反抗内卷的最重要理由。向来的不合与表面的利益冲突可能影响对真实情况的判断,但只有意识到大家都成为了受害者后,方能凝聚起每个人心中改变的动力。

所以,就用几句诗给这篇论文做个结尾吧:

\ \ \ \ “新的转机和闪闪星斗,

\ \ \ \ 正在缀满没有遮拦的天空。

\ \ \ \ 那是五千年的象形文字,

\ \ \ \ 那是未来人们凝视的眼睛。\footnote{出自北岛《回答》。}”

\newpage
\chapter*{致谢}

感谢徐晓飞老师的付出,也感谢前两个学期教授人文类课程的林爱兵老师与杨映秋老师,老师们提供的专业知识是能写出这篇调查报告的基础。感谢所有在采访、调查中贡献过观点,或是对本报告提出宝贵修改意见的同学们,是大家的想法让这篇报告得以完成。最后,感谢每一个为对抗内卷而付出努力的人,你们的努力让这些文字有了真实的意义,也给予了鼓舞人心的力量。

\end{document}